%% Generated by Sphinx.
\def\sphinxdocclass{jupyterBook}
\documentclass[letterpaper,10pt,english]{jupyterBook}
\ifdefined\pdfpxdimen
   \let\sphinxpxdimen\pdfpxdimen\else\newdimen\sphinxpxdimen
\fi \sphinxpxdimen=.75bp\relax
\ifdefined\pdfimageresolution
    \pdfimageresolution= \numexpr \dimexpr1in\relax/\sphinxpxdimen\relax
\fi
%% let collapsible pdf bookmarks panel have high depth per default
\PassOptionsToPackage{bookmarksdepth=5}{hyperref}
%% turn off hyperref patch of \index as sphinx.xdy xindy module takes care of
%% suitable \hyperpage mark-up, working around hyperref-xindy incompatibility
\PassOptionsToPackage{hyperindex=false}{hyperref}
%% memoir class requires extra handling
\makeatletter\@ifclassloaded{memoir}
{\ifdefined\memhyperindexfalse\memhyperindexfalse\fi}{}\makeatother

\PassOptionsToPackage{warn}{textcomp}

\catcode`^^^^00a0\active\protected\def^^^^00a0{\leavevmode\nobreak\ }
\usepackage{cmap}
\usepackage{fontspec}
\defaultfontfeatures[\rmfamily,\sffamily,\ttfamily]{}
\usepackage{amsmath,amssymb,amstext}
\usepackage{polyglossia}
\setmainlanguage{english}



\setmainfont{FreeSerif}[
  Extension      = .otf,
  UprightFont    = *,
  ItalicFont     = *Italic,
  BoldFont       = *Bold,
  BoldItalicFont = *BoldItalic
]
\setsansfont{FreeSans}[
  Extension      = .otf,
  UprightFont    = *,
  ItalicFont     = *Oblique,
  BoldFont       = *Bold,
  BoldItalicFont = *BoldOblique,
]
\setmonofont{FreeMono}[
  Extension      = .otf,
  UprightFont    = *,
  ItalicFont     = *Oblique,
  BoldFont       = *Bold,
  BoldItalicFont = *BoldOblique,
]



\usepackage[Bjarne]{fncychap}
\usepackage[,numfigreset=1,mathnumfig]{sphinx}

\fvset{fontsize=\small}
\usepackage{geometry}


% Include hyperref last.
\usepackage{hyperref}
% Fix anchor placement for figures with captions.
\usepackage{hypcap}% it must be loaded after hyperref.
% Set up styles of URL: it should be placed after hyperref.
\urlstyle{same}


\usepackage{sphinxmessages}



        % Start of preamble defined in sphinx-jupyterbook-latex %
         \usepackage[Latin,Greek]{ucharclasses}
        \usepackage{unicode-math}
        % fixing title of the toc
        \addto\captionsenglish{\renewcommand{\contentsname}{Contents}}
        \hypersetup{
            pdfencoding=auto,
            psdextra
        }
        % End of preamble defined in sphinx-jupyterbook-latex %
        

\title{Computer Science Capstone}
\date{Jan 19, 2024}
\release{}
\author{Dr.\@{} Jim Ashe}
\newcommand{\sphinxlogo}{\vbox{}}
\renewcommand{\releasename}{}
\makeindex
\begin{document}

\pagestyle{empty}
\sphinxmaketitle
\pagestyle{plain}
\sphinxtableofcontents
\pagestyle{normal}
\phantomsection\label{\detokenize{intro::doc}}




\begin{sphinxadmonition}{warning}{Warning:}
\sphinxAtStartPar
🚧 This site is under construction! Apologies in advance for typos, broken links, etc. Feel free to make recommendations and become a contributor! 👷🏽‍♀️

\sphinxAtStartPar
\sphinxstylestrong{Nothing on this website is an official WGU resource developed by the WGU product development.} See their COS page to review the official resources. To make comments or suggestions regarding those resources use your COS page \sphinxhref{https://ashejim.github.io/C964/support\_this\_page.html\#help-support-this-website}{‘Course Feedback’} link.
\end{sphinxadmonition}

\begin{sphinxadmonition}{warning}{Warning:}
\sphinxAtStartPar
On 9/25/2023, C964 was updated to a new version, (SIM3). Though the rubric and task directions were reworded, \sphinxstylestrong{the actual requirements and their assessment criteria are unchanged.}
\end{sphinxadmonition}





\sphinxAtStartPar
Welcome! For the Computer Science capstone project, you’ll develop and present a machine learning application solving a proposed problem. The problem, the solution, and the presentation as a final product are up to you! The capstone allows you to demonstrate the application of skills collected throughout the CS program. Most importantly, that crucial skill setting CS majors apart, learning and applying new things. You are a problem\sphinxhyphen{}solver; this is your opportunity to shine.

\sphinxAtStartPar
The capstone includes three parts:
\begin{enumerate}
\sphinxsetlistlabels{\arabic}{enumi}{enumii}{}{.}%
\item {} 
\sphinxAtStartPar
\sphinxstylestrong{Task 1:} Get course instructor topic approval \sphinxhyphen{}a preliminary step to ensure you starts in the right direction.

\item {} 
\sphinxAtStartPar
\sphinxstylestrong{Task 2 part C:} The “app.” Develop a working application of machine learning (ML).

\item {} 
\sphinxAtStartPar
\sphinxstylestrong{Task 2 parts D, A, \& B:} Documentation communicating your product’s value and development process to audiences of varying technical understanding.

\end{enumerate}

\begin{sphinxuseclass}{sd-sphinx-override}
\begin{sphinxuseclass}{sd-cards-carousel}
\begin{sphinxuseclass}{sd-card-cols-3}
\begin{sphinxuseclass}{sd-card}
\begin{sphinxuseclass}{sd-sphinx-override}
\begin{sphinxuseclass}{sd-m-3}
\begin{sphinxuseclass}{sd-shadow-sm}
\begin{sphinxuseclass}{sd-card-hover}
\begin{sphinxuseclass}{sd-card-header}
\begin{sphinxuseclass}{bg-light}
\begin{sphinxuseclass}{text-center}
\sphinxAtStartPar
\sphinxstylestrong{Task 1}

\end{sphinxuseclass}
\end{sphinxuseclass}
\end{sphinxuseclass}
\begin{sphinxuseclass}{sd-card-body}
\begin{sphinxuseclass}{text-center}
\noindent\sphinxincludegraphics[height=100\sphinxpxdimen]{{idea-b}.png}

\sphinxAtStartPar
Choose a topic and get approval.

\end{sphinxuseclass}
\end{sphinxuseclass}
\begin{sphinxuseclass}{sd-card-footer}
\sphinxAtStartPar
Task 1 details 

\end{sphinxuseclass}\sphinxhref{./task1.html}{}
\end{sphinxuseclass}
\end{sphinxuseclass}
\end{sphinxuseclass}
\end{sphinxuseclass}
\end{sphinxuseclass}
\begin{sphinxuseclass}{sd-card}
\begin{sphinxuseclass}{sd-sphinx-override}
\begin{sphinxuseclass}{sd-m-3}
\begin{sphinxuseclass}{sd-shadow-sm}
\begin{sphinxuseclass}{sd-card-hover}
\begin{sphinxuseclass}{sd-card-header}
\begin{sphinxuseclass}{bg-light}
\begin{sphinxuseclass}{text-center}
\sphinxAtStartPar
\sphinxstylestrong{Task 2 part C}

\end{sphinxuseclass}
\end{sphinxuseclass}
\end{sphinxuseclass}
\begin{sphinxuseclass}{sd-card-body}
\begin{sphinxuseclass}{text-center}
\noindent\sphinxincludegraphics[height=100\sphinxpxdimen]{{ml_process_summary}.png}

\sphinxAtStartPar
Develop a working application of machine learning (ML).

\end{sphinxuseclass}
\end{sphinxuseclass}
\begin{sphinxuseclass}{sd-card-footer}
\sphinxAtStartPar
Task 2 part C details 

\end{sphinxuseclass}\sphinxhref{./task2\_c/task2\_part\_c.html}{}
\end{sphinxuseclass}
\end{sphinxuseclass}
\end{sphinxuseclass}
\end{sphinxuseclass}
\end{sphinxuseclass}
\begin{sphinxuseclass}{sd-card}
\begin{sphinxuseclass}{sd-sphinx-override}
\begin{sphinxuseclass}{sd-m-3}
\begin{sphinxuseclass}{sd-shadow-sm}
\begin{sphinxuseclass}{sd-card-hover}
\begin{sphinxuseclass}{sd-card-header}
\begin{sphinxuseclass}{bg-light}
\begin{sphinxuseclass}{text-center}
\sphinxAtStartPar
\sphinxstylestrong{Task 2 parts A, B, \& D}

\end{sphinxuseclass}
\end{sphinxuseclass}
\end{sphinxuseclass}
\begin{sphinxuseclass}{sd-card-body}
\begin{sphinxuseclass}{text-center}
\noindent\sphinxincludegraphics[height=100\sphinxpxdimen]{{document-a}.jpg}

\sphinxAtStartPar
Present your product to audiences of varying technical understanding through documentation and visualizations.

\end{sphinxuseclass}
\end{sphinxuseclass}
\begin{sphinxuseclass}{sd-card-footer}
\sphinxAtStartPar
Task 3 details 

\end{sphinxuseclass}\sphinxhref{./task2\_doc/task2\_doc.html}{}
\end{sphinxuseclass}
\end{sphinxuseclass}
\end{sphinxuseclass}
\end{sphinxuseclass}
\end{sphinxuseclass}
\end{sphinxuseclass}
\end{sphinxuseclass}
\end{sphinxuseclass}
\begin{DUlineblock}{0em}
\item[] \sphinxstylestrong{\Large Start Here}
\end{DUlineblock}

\sphinxAtStartPar
First, understand the project’s requirements. What they are \sphinxhyphen{}and what they aren’t. Watch the following video:



\sphinxAtStartPar
And review the \DUrole{xref,myst}{What does the application need to do?} section and part C of the \sphinxhref{https://westerngovernorsuniversity-my.sharepoint.com/:w:/g/personal/jim\_ashe\_wgu\_edu/ERGxhsNfkbhEutlkXVFITMQBPOmWlkVx1p5H0UisvnBesg?rtime=3q\_Efs-u2kg}{Task 2 template}.

\sphinxAtStartPar
Like the \sphinxhref{https://ashejim.github.io/C964/task2\_c/task2\_part\_c.html\#can-i-use-my-c950-project-for-c964}{\sphinxstyleemphasis{C950 \sphinxhyphen{}Data Structures and Algorithms II} task} and \sphinxhref{https://ashejim.github.io/C964/task2\_c/task2\_part\_c.html\#can-i-use-my-c951-task-1-or-2-for-c964}{\sphinxstyleemphasis{C951 \sphinxhyphen{}Intro to AI}} tasks 1 and 2, this project consists of a working application, ({\hyperref[\detokenize{task2_c/task2_part_c:task2-part-c}]{\sphinxcrossref{\DUrole{std,std-ref}{Task 2 part C}}}}), and accompanying documentation, {\hyperref[\detokenize{task2_doc/task2_doc:task2-doc}]{\sphinxcrossref{\DUrole{std,std-ref}{Task 2 parts A, B, \& D}}}}. But because of the breadth of allowable topics, we want to ensure you start working in the right direction, and thus require all topics to be approved by your assigned course instructor, {\hyperref[\detokenize{task1:task1}]{\sphinxcrossref{\DUrole{std,std-ref}{Task 1}}}}. So your next step is choosing a topic and having it approved.

\sphinxstepscope


\chapter{Task 1: Topic Approval}
\label{\detokenize{task1:task-1-topic-approval}}\label{\detokenize{task1::doc}}\phantomsection\label{\detokenize{task1:task1}}





\section{Choosing a Topic}
\label{\detokenize{task1:choosing-a-topic}}\phantomsection\label{\detokenize{task1:task1-choosing-a-topic}}
\sphinxAtStartPar
The approval form ensures you start in the right direction before investing time and effort into {\hyperref[\detokenize{task2_c/task2_part_c:task2-part-c}]{\sphinxcrossref{\DUrole{std,std-ref}{task 2}}}}. Evaluators look for our (instructors’) signature, and we look for the following:
\begin{enumerate}
\sphinxsetlistlabels{\arabic}{enumi}{enumii}{}{.}%
\item {} 
\sphinxAtStartPar
An application of machine learning (ML).

\item {} 
\sphinxAtStartPar
An organizational need for your ML application to help solve.

\item {} 
\sphinxAtStartPar
A \sphinxstyleemphasis{basic outline} of your implementation plan.

\end{enumerate}

\begin{sphinxShadowBox}
\sphinxstylesidebartitle{What is \sphinxstyleemphasis{machine learning}?}

\sphinxAtStartPar
It’s a new field, and answers vary depending on who you ask. But for the purposes of this project, ML is the application of an algorithm to data. Also see {\hyperref[\detokenize{task2_c/task2_part_c:task1-faq-what-is-machine-learning}]{\sphinxcrossref{\DUrole{std,std-ref}{FAQ: What is ML?}}}}.
\end{sphinxShadowBox}

\sphinxAtStartPar
The “organizational need” requirement gives your project a purpose and audience. From an assessment perspective, it has little other value, so you need not worry about your project’s profitability or practical impact. Such criteria are \sphinxstyleemphasis{not} assessed. This is a computer science project, not a business or software project.


\subsection{Data}
\label{\detokenize{task1:data}}
\sphinxAtStartPar
ML applies an algorithm to data, and you can’t apply an algorithm to data \sphinxhyphen{}without data. So finding data should be an early step.

\sphinxAtStartPar
A machine learning application is the hard requirement of the capstone, but you can’t apply an algorithm to data \sphinxhyphen{}without data. Moreover, most datasets provide a question to answer, e.g., predictions, classifications, or recommendations for which an “organizational need” can always be found.
\begin{itemize}
\item {} 
\sphinxAtStartPar
\sphinxhref{https://www.kaggle.com/datasets}{\sphinxstylestrong{Kaggle.com}}

\item {} 
\sphinxAtStartPar
\sphinxhref{https://datasetsearch.research.google.com/}{Google Dataset Search}

\item {} 
\sphinxAtStartPar
\sphinxhref{https://data.gov/}{Data.gov}

\item {} 
\sphinxAtStartPar
More \sphinxhref{https://careerfoundry.com/en/blog/data-analytics/where-to-find-free-datasets/}{here} and \sphinxhref{https://medium.com/analytics-vidhya/top-100-open-source-datasets-for-data-science-cd5a8d67cc3d}{here}

\item {} 
\sphinxAtStartPar
Simulated data

\end{itemize}

\begin{sphinxadmonition}{note}{Note:}
\sphinxAtStartPar
\sphinxstyleemphasis{No minimal data complexity or processing is required.} Choosing data which needs less processing or simplifying a dataset (you don’t have to use it all) can make the project technically more accessible.
\end{sphinxadmonition}


\subsection{Machine Learning Algorithms}
\label{\detokenize{task1:machine-learning-algorithms}}
\begin{sphinxShadowBox}
\sphinxstylesidebartitle{\sphinxstylestrong{Classification \& Regression Overview}}


\end{sphinxShadowBox}


\subsubsection{Supervised Learning}
\label{\detokenize{task1:supervised-learning}}
\sphinxAtStartPar
If the answers for what you are trying to predict (the dependent variable; also called \sphinxstyleemphasis{tagged} or \sphinxstyleemphasis{labeled}) are in the data, then a \sphinxhref{https://scikit-learn.org/stable/supervised\_learning.html}{supervised method} is the way to go. Using the other features (independent variables) as input, a supervised algorithm can \sphinxstyleemphasis{train} a function to predict a dependent variable for new unseen dependent variables. For predicting numbers, e.g., profit, temperature, etc., a \sphinxstyleemphasis{regression} method should be used. For predicting categories, e.g., yes/no, blue/red, spam/ham, etc., a \sphinxstyleemphasis{classification} method should be used.
\begin{itemize}
\item {} 
\sphinxAtStartPar
\sphinxstyleemphasis{Regression} algorithms predict numbers. Examples include linear regression, polynomial regression, decision tree regression, random forest regression, and many more.

\end{itemize}


\begin{itemize}
\item {} 
\sphinxAtStartPar
\sphinxstyleemphasis{Classification} algorithms predict categories. Examples include logistic regression (see the margin), naive Bayes, support vector machine, decision tree, and many others.

\end{itemize}
\phantomsection\label{\detokenize{task1:task1-choosing-topic-logistic}}
\begin{sphinxShadowBox}
\sphinxstylesidebartitle{Logisitc regression: regression or classification?}

\sphinxAtStartPar
Really both, but most often, it’s used for classification. Logistic regression uses input variables to predict the \sphinxstyleemphasis{probability} of an outcome, a number between 0.0 and 1.0 \sphinxhyphen{}hence “regression.” However, using that probability to predict whether an outcome occurs (yes/no) is classification.
\end{sphinxShadowBox}


\subsubsection{Unsupervised Learning}
\label{\detokenize{task1:unsupervised-learning}}
\sphinxAtStartPar
If what you are trying to find is not directly in the data, then an \sphinxhref{https://scikit-learn.org/stable/unsupervised\_learning.html}{unsupervised method} might be used. Unsupervised algorithms identify patterns in the data. Common approaches include clustering (e.g., k\sphinxhyphen{}means, expectation\sphinxhyphen{}maximization distribution, and agglomerative hierarchical), dimensionality reduction (e.g., PCA and LDA), and anomaly detection (e.g., outlier factor and isolation forest).


\subsubsection{Reinforced Learning}
\label{\detokenize{task1:reinforced-learning}}
\sphinxAtStartPar
Sometimes solutions need to adapt to particular situations. That is, the algorithms need to learn how to make decisions. For example, a robot needs to navigate a never before seen maze. By awarding (or punishing) a robot’s state after decisions, an algorithm can be progressively trained to predict decisions maximizing results. Reinforced methods can be more challenging to develop than supervised or unsupervised methods.

\begin{sphinxadmonition}{note}{Note:}
\sphinxAtStartPar
Need help determining which particular algorithm will work best? That’s OK! You won’t know until you’ve spent time investigating and experimenting. But that’s not the point of Task 1. For now, you only need to identify the type, i.e., supervised, unsupervised, or reinforced. \sphinxstyleemphasis{Any ML algorithm} given in Task 1 appropriate for your data will be accepted, and you might do something different in task without needing to revise Task 1 (see the {\hyperref[\detokenize{task1:task1-faq-change-task1}]{\sphinxcrossref{\DUrole{std,std-ref}{FAQ}}}} below).
\end{sphinxadmonition}


\subsection{Where to start? Data or the ML algorithm?}
\label{\detokenize{task1:where-to-start-data-or-the-ml-algorithm}}
\sphinxAtStartPar
It’s best to consider both. In a “real\sphinxhyphen{}world” scenario, you would likely be given data to analyze and a problem to solve. So starting with the data, and then determining the proper ML tools is most natural. However, you get to choose the problem and hence the tools needed to solve it \sphinxhyphen{}provided you have data fitting that problem. For example, someone interested in image recognition might restrict their data search to sets of labeled images. To simplify coding steps, they might look at images with standard sizes or take a subset of the images limiting the number of classifications. At any point, the problem or chosen algorithm can be revised to better fit what you want or can do (even after graduation 😄). A suggested approach:
\begin{enumerate}
\sphinxsetlistlabels{\arabic}{enumi}{enumii}{}{.}%
\item {} 
\sphinxAtStartPar
Find some data to match your preferred ML method.

\item {} 
\sphinxAtStartPar
Find an ML algorithm you can apply to that data.

\item {} 
\sphinxAtStartPar
Formulate an “organizational need” helped by that application.

\end{enumerate}

\begin{sphinxShadowBox}
\sphinxstylesidebartitle{Unsure about which ML method to use?}

\sphinxAtStartPar
Look for data which has mutually exclusive categories which can used as a dependent variables,e.g., yes/no, alive/dead, blue/red/yellow, etc. Our \DUrole{xref,myst}{supervised classification example}, uses mutually exclusive flower types. This type of supervised classification method most easily fits all the rubric requirements.
\end{sphinxShadowBox}


\subsection{Examples}
\label{\detokenize{task1:examples}}\label{\detokenize{task1:task1-examples}}

\subsubsection{Topic Summary Examples}
\label{\detokenize{task1:topic-summary-examples}}
\begin{sphinxuseclass}{sd-tab-set}
\begin{sphinxuseclass}{sd-tab-item}\subsubsection*{Example 1}

\begin{sphinxuseclass}{sd-tab-content}\begin{quote}

\sphinxAtStartPar
\sphinxstylestrong{Data:} Petal dimensions and species of \sphinxhref{https://www.kaggle.com/datasets/uciml/iris}{Iris Samples}.

\sphinxAtStartPar
\sphinxstylestrong{ML application (non\sphinxhyphen{}descriptive method):} {\hyperref[\detokenize{task2_c/example_sup_class/sup_class_ex:sup-class-ex}]{\sphinxcrossref{\DUrole{std,std-ref}{Classify an Iris’s species using its petal dimensions to train a supervised calssification model, say SVN or Logistic regression}}}}.

\sphinxAtStartPar
\sphinxstylestrong{Descriptive method:} \DUrole{xref,myst}{Histograms and barplots} showing distributions of different flower features and confusion matrix illustrating the accuracy of the classification model.

\sphinxAtStartPar
\sphinxstylestrong{Organizational Need:} \sphinxstyleemphasis{Grow Fast Ferilizer Inc.} needs to help customers identify their flowers.
\end{quote}

\end{sphinxuseclass}
\end{sphinxuseclass}
\begin{sphinxuseclass}{sd-tab-item}\subsubsection*{Example 2}

\begin{sphinxuseclass}{sd-tab-content}\begin{quote}

\sphinxAtStartPar
\sphinxstylestrong{Data:} Housing data

\sphinxAtStartPar
\sphinxstylestrong{ML application (non\sphinxhyphen{}descriptive method):} Predict the selling price of a house using a supervised regression model, say random forest or linear regression.

\sphinxAtStartPar
\sphinxstylestrong{Descriptive method:} Histograms showing distributions of data features and scatterplots exploring possible data correlations.

\sphinxAtStartPar
\sphinxstylestrong{Organizational Need:} The realty firm, \sphinxstyleemphasis{We Sell em Fast!}, needs a tool to estimate housing prices.
\end{quote}

\end{sphinxuseclass}
\end{sphinxuseclass}
\begin{sphinxuseclass}{sd-tab-item}\subsubsection*{Example 3}

\begin{sphinxuseclass}{sd-tab-content}\begin{quote}

\sphinxAtStartPar
\sphinxstylestrong{Data:} Movies, e.g., budget, genre, starring actors, etc.

\sphinxAtStartPar
\sphinxstylestrong{ML application (descriptive method):} Use unsupervised clustering, say K\sphinxhyphen{}means, to identify groups of similar movies.

\sphinxAtStartPar
\sphinxstylestrong{Non\sphinxhyphen{}descriptive method:} Using the clusters and user input, recommend a movie.

\sphinxAtStartPar
\sphinxstylestrong{Organizational Need:} The streaming service, \sphinxstyleemphasis{InterWebFlixs}, needs to help users pick a movie.
\end{quote}

\end{sphinxuseclass}
\end{sphinxuseclass}
\begin{sphinxuseclass}{sd-tab-item}\subsubsection*{Example 4}

\begin{sphinxuseclass}{sd-tab-content}\begin{quote}

\sphinxAtStartPar
\sphinxstylestrong{Data:} Images of dogs and cats.

\sphinxAtStartPar
\sphinxstylestrong{ML application} (non\sphinxhyphen{}descriptive): Classify an image as a dog or cat using a supervised neural network (classification model), say CNN.

\sphinxAtStartPar
\sphinxstylestrong{Descriptive method:} Histograms showing distributions of image features and a confusion matrix illustrating the accuracy of the classification model.

\sphinxAtStartPar
\sphinxstylestrong{Organizational Need:} \sphinxstyleemphasis{We Love Pets Inc.} wants to use customer\sphinxhyphen{}uploaded images to market the correct type of pet food.
\end{quote}

\end{sphinxuseclass}
\end{sphinxuseclass}
\end{sphinxuseclass}

\subsubsection{Completed Examples}
\label{\detokenize{task1:completed-examples}}\phantomsection\label{\detokenize{task1:task1-examples-completed-examples}}
\sphinxAtStartPar
These are (slightly modified) examples of approved topics. All of which went on to become passing capstone projects.

\begin{sphinxuseclass}{sd-tab-set}
\begin{sphinxuseclass}{sd-tab-item}\subsubsection*{Completed Task 1 Ex. A}

\begin{sphinxuseclass}{sd-tab-content}\begin{quote}

\sphinxAtStartPar
\sphinxhref{https://github.com/ashejim/C964/blob/main/resources/example\_task1-a.pdf}{\sphinxincludegraphics{{example_task1-a}.png}}
\end{quote}

\sphinxAtStartPar
Also see: {\hyperref[\detokenize{task2_doc/task2_doc:task2-doc-examples}]{\sphinxcrossref{\DUrole{std,std-ref}{task 2 document examples}}}}

\end{sphinxuseclass}
\end{sphinxuseclass}
\begin{sphinxuseclass}{sd-tab-item}\subsubsection*{Completed Task 1 Ex. B}

\begin{sphinxuseclass}{sd-tab-content}\begin{quote}

\sphinxAtStartPar
\sphinxhref{https://github.com/ashejim/C964/blob/main/resources/example\_task1-b.pdf}{\sphinxincludegraphics{{example_task1-b}.png}}
\end{quote}

\sphinxAtStartPar
Also see: \sphinxhref{https://github.com/ashejim/C964/blob/main/resources/example\_task2-b.pdf}{task 2 example B}

\end{sphinxuseclass}
\end{sphinxuseclass}
\end{sphinxuseclass}

\paragraph{WGU Capstone Excellence Archive}
\label{\detokenize{task1:wgu-capstone-excellence-archive}}
\sphinxAtStartPar
The \sphinxhref{https://westerngovernorsuniversity.sharepoint.com/sites/capstonearchives/excellence/Pages/UndergraduateInformation.aspx}{Capstone Excellence Archive} includes a wide range of completed projects. Reviewing them may help get ideas, provide inspiration, and deepen understanding of the requirements. However, keep in mind that they all are \sphinxstyleemphasis{above and beyond} the requirements. Therefore, don’t use these as examples of what’s needed to meet the requirements. For more down\sphinxhyphen{}to\sphinxhyphen{}earth examples of what’s required, see the {\hyperref[\detokenize{task1:task1-examples-completed-examples}]{\sphinxcrossref{\DUrole{std,std-ref}{examples}}}} above.


\section{Topic Approval}
\label{\detokenize{task1:topic-approval}}\label{\detokenize{task1:task1-approval}}
\sphinxAtStartPar
Once you’ve decided on a topic, complete the approval form following the template below and \sphinxstyleemphasis{email it to your \sphinxhref{mailto:myC964capstoneinstructor@wgu.edu?cc=my\%20program\%20mentor\&subject=C964:\%20capstone\%20topic\%20approval}{C964 assigned instructor} for approval.}


\begin{quote}

\sphinxAtStartPar
\sphinxhref{https://westerngovernorsuniversity-my.sharepoint.com/:w:/g/personal/jim\_ashe\_wgu\_edu/EaH8yexFJjhDp5hnrcAZeKoB6XxU9r8Z5IH1QqVLmVu87w?e=OwRtpe}{Topic Approval Form Template}
\end{quote}

\begin{sphinxadmonition}{note}{Note:}
\sphinxAtStartPar
The approval form only needs to be a \sphinxstyleemphasis{rough} outline of a passing project. \sphinxstylestrong{Changes from task 1 to task 2 are allowed and expected.} Determining the finer details of a complex project takes time and effort, which you won’t invest until task 2.
\end{sphinxadmonition}

\sphinxAtStartPar
\sphinxstylestrong{Project topic and purpose:} Describe the problem (the “organizational need”) your project will solve.

\sphinxAtStartPar
\sphinxstylestrong{Non\sphinxhyphen{}descriptive method(s):}  These methods infer from the data, i.e., make predictions or prescriptions. Examples include classification models, regression, image recognition, etc. Typically, but not necessarily, this is where ML is applied.

\sphinxAtStartPar
\sphinxstylestrong{Descriptive methods:} These methods describe and help understand the data, e.g., mean, median, bar plot, scatterplot, k\sphinxhyphen{}means clustering, etc. Three visualizations are required. The visualizations are typically descriptive and can count as your descriptive method.

\begin{sphinxadmonition}{note}{Note:}
\sphinxAtStartPar
You must identify a machine learning application in either the descriptive or non\sphinxhyphen{}descriptive section. Most often, the non\sphinxhyphen{}descriptive method uses ML, e.g., a classification model allowing the user to provide input to an app that returns a prediction.
\end{sphinxadmonition}

\sphinxAtStartPar
Directly emailing your \sphinxhref{mailto:myC964capstoneinstructor@wgu.edu?cc=my\%20program\%20mentor\&subject=C964:\%20capstone\%20topic\%20approval}{assigned course instructor} (identified on your C964 COS page; or find them {\hyperref[\detokenize{ci_c964:ci-c964}]{\sphinxcrossref{\DUrole{std,std-ref}{here}}}}) is typically the fastest and best way to get a signature. Whether emailing \sphinxhref{mailto:ugcapstoneit@wgu.edu?cc=my\%20course\%20instructor\&subject=C964:\%20capstone\%20topic\%20approval}{ugcapstoneit@wgu.edu} or your CI directly, always practice professional communication:
\begin{itemize}
\item {} 
\sphinxAtStartPar
Use your WGU email (non\sphinxhyphen{}WGU emails may be categorized as spam).

\item {} 
\sphinxAtStartPar
Provide a subject, your capstone course (we support all IT college capstones), and your program mentor’s name (if not in your signature).

\item {} 
\sphinxAtStartPar
Clearly state your questions or requests.

\end{itemize}

\begin{sphinxadmonition}{warning}{Warning:}
\sphinxAtStartPar
The submitted topic approval form must be \sphinxstyleemphasis{signed by a {\hyperref[\detokenize{ci_c964:ci-c964}]{\sphinxcrossref{\DUrole{std,std-ref}{C964 course instructor}}}}}. Forms without a signature are automatically returned without further review.
\end{sphinxadmonition}


\section{Waiver Form}
\label{\detokenize{task1:waiver-form}}\label{\detokenize{task1:task1-waiver-form}}


\begin{sphinxadmonition}{note}{Note:}
\sphinxAtStartPar
the waiver form is \sphinxstylestrong{only} required if your project is based upon or includes restricted information. If no waiver form is submitted, Task 1 \sphinxstyleemphasis{B: Capstone Release Form}, passes automatically.
\end{sphinxadmonition}

\sphinxAtStartPar
In most cases, obtaining authorization can be avoided by fabricating or masking identifying information. But if you choose to move forward using restricted information, you must obtain documented permissions and submit them along with a waiver form to Assessments.
\begin{quote}

\sphinxAtStartPar
\sphinxhref{https://westerngovernorsuniversity-my.sharepoint.com/:w:/g/personal/jim\_ashe\_wgu\_edu/ESLuMNRuDjpCrKvqWaC6cywB4I97WEPdk5MRZRq4LfmFhQ}{\sphinxincludegraphics{{waiver}.png}}
\end{quote}


\section{FAQ}
\label{\detokenize{task1:faq}}\label{\detokenize{task1:task1-faq}}

\subsection{Do I need to set up an appointment to get approval?}
\label{\detokenize{task1:do-i-need-to-set-up-an-appointment-to-get-approval}}
\sphinxAtStartPar
No. Usually, students email the approval form to their {\hyperref[\detokenize{ci_c964:ci-c964}]{\sphinxcrossref{\DUrole{std,std-ref}{assigned instructor}}}}. We then sign it or follow up with questions. However, if you have questions about the requirements or need help choosing a topic, you are encouraged to set up an appointment. A 15\sphinxhyphen{}30 minute phone call can address most questions or concerns.


\subsection{I have questions. Should I email the question or set up an appointment?}
\label{\detokenize{task1:i-have-questions-should-i-email-the-question-or-set-up-an-appointment}}
\sphinxAtStartPar
It’s up to you. But for specific questions, emails are usually best and almost always faster. Appointments are great for general discussions or when you’re unsure what to ask. For technical coding questions or if you are unsure who to contact, see the {\hyperref[\detokenize{ci_page:cipage}]{\sphinxcrossref{\DUrole{std,std-ref}{course faculty page}}}}.


\subsection{What if I start working on task 2 and want to change things? Do I need to resubmit task 1?}
\label{\detokenize{task1:what-if-i-start-working-on-task-2-and-want-to-change-things-do-i-need-to-resubmit-task-1}}\label{\detokenize{task1:task1-faq-change-task1}}
\sphinxAtStartPar
No, not unless it’s an entirely different topic. Minor changes from task 1 to task 2 are expected and allowed \sphinxstyleemphasis{without updating the approval form}. Evaluators will not rigorously compare tasks 1 and 2. Task 2 is where the work is, and even with complete topic changes, at most, you’ll be asked to revise the approval form (if at all). So never let task 1 dictate what you do in task 2.


\subsection{Can I use projects from other WGU courses?}
\label{\detokenize{task1:can-i-use-projects-from-other-wgu-courses}}
\sphinxAtStartPar
You can use any of your work or academic projects (at WGU or elsewhere), provided no proprietary information is used without permission. Don’t worry about self\sphinxhyphen{}plagiarism, as the similarity check will identify and ignore it. Just as in reusing work projects, expect to modify and remold past academic assignments to meet the project requirements.




\subsection{Can I use my C950 project for C964?}
\label{\detokenize{task1:can-i-use-my-c950-project-for-c964}}\label{\detokenize{task1:task1-faq-can-i-use-my-c950}}
\sphinxAtStartPar
Yes. You can use any of your own academic or professional work for C964 including the C950 project (Data Structures \& Algorithms II). Though the document (Task 2 parts A, B, and D) will need some adjustment, the coding portion of C950 almost meets all the requirements of the C964 application (Task 2 part C) \sphinxhyphen{}it only lacks visualizations. Referring to the \sphinxhref{https://ashejim.github.io/C964/task2\_c/task2\_part\_c.html\#what-does-the-application-need-to-do}{Task 2 part C page}, the C964 application needs the following:
\begin{enumerate}
\sphinxsetlistlabels{\arabic}{enumi}{enumii}{}{.}%
\item {} 
\sphinxAtStartPar
\sphinxstylestrong{Data → ML model:} C950 applies a reinforced learning algorithm to the distance and package data.

\item {} 
\sphinxAtStartPar
\sphinxstylestrong{Accuracy Metric:} The total miles. The maximum allowed miles for C950 is 120, which WGU Assessment Department has already determined to be “efficient.”

\item {} 
\sphinxAtStartPar
\sphinxstylestrong{Visualizations:} This will need to be added, but any three pictures will meet the requirements.

\item {} 
\sphinxAtStartPar
\sphinxstylestrong{User Application:} The console user interface required for C950 allows the user to provide input and apply the algorithm toward solving the problem.

\end{enumerate}

\sphinxAtStartPar
So only the three images will need to be added. Furthermore, you are free to adjust the distance and package data as desired. For example, dropping some delivery restrictions requiring different trucks or certain packages to be delivered together, and will be easier to apply a more sophisticated algorithm. Say a \sphinxhref{https://en.wikipedia.org/wiki/Monte\_Carlo\_method}{Monte Carlo method} such as \sphinxhref{https://en.wikipedia.org/wiki/Simulated\_annealing}{Simulated Annealing}.



\sphinxAtStartPar
{[}\hyperlink{cite.resources:id2}{Wik18}{]}


\subsection{Can I use my C951 task 3? Should I use it?}
\label{\detokenize{task1:can-i-use-my-c951-task-3-should-i-use-it}}
\sphinxAtStartPar
You can reuse anything you’ve of your own academic or professional work, including copying verbatim from C951 task 3. If it’s convenient, feel free to do it. But at best, the time saved is little. At worst, you might get bogged down trying to work on two projects simultaneously and going with an unnecessarily complex C964 topic. If you have time, consider completing C964 first, as parts A and B of task 2 can always be used verbatim for task 3 of C951.

\sphinxAtStartPar
Here are some points to consider:
\begin{itemize}
\item {} 
\sphinxAtStartPar
C951.3 is just a written project, typically around five pages (I’m guessing; ask your C951 instructor), and can be completed in a single afternoon. Comparatively, C964 requires a working machine learning application and accompanying documentation, typically around 20 pages.

\item {} 
\sphinxAtStartPar
C951.3 only relates to parts A and B of C964.2. These parts are just a framework for providing a general audience and purpose for the ML application. If present, these parts almost always pass. Furthermore, they’ll have to be at least partially rewritten anyway. Parts C and D of C964 are what evaluators care about, but C951.3 has no corresponding parts C and D.

\item {} 
\sphinxAtStartPar
Rewriting C951.3 content for a different C964 topic takes little additional work.

\item {} 
\sphinxAtStartPar
As it’s just a written project, students often pick a complex topic for C951.3. But then they feel pressured to use the same complex topic for C964 and struggle with creating the app.

\item {} 
\sphinxAtStartPar
Trying to comprehend two projects at once is just more difficult.

\end{itemize}

\sphinxAtStartPar
Whatever you do for C964 can meet the requirements of C951 task 3. If you have plenty of time, completing C964 first might be the best option.


\subsection{The rubric refers to the task directions which seem to require redundant or unnecessary items. What do I actually need to do?}
\label{\detokenize{task1:the-rubric-refers-to-the-task-directions-which-seem-to-require-redundant-or-unnecessary-items-what-do-i-actually-need-to-do}}\label{\detokenize{task1:task1-faq-the-rubric-is-confusing}}
\sphinxAtStartPar
Your C964 course instructor team has collaborated with evaluators (thank you evaluator team!) to ensure the explanations on this webpage align with how the assessment’s requirements are interpreted by the evaluators. So if you find the official directions unclear, we advise following the directions on this webpage for both the application (part C) and the documentation (parts A, B, and D).

\sphinxAtStartPar
The official rubric and directions were written to map the project’s elements to specific competencies, and following the official directions \sphinxstyleemphasis{will} meet all the requirements. However, be aware that some items might not be applicable or be inherently met by other items. For example, in part C:
\begin{itemize}
\item {} 
\sphinxAtStartPar
The descriptive and nondescriptive requirement is met by the visualization and decision support functionality respectively.

\item {} 
\sphinxAtStartPar
Visualization functionality and monitoring tools are inherently part of tools used to create visualizations and code.

\item {} 
\sphinxAtStartPar
As any ML method is an algorithm, the requirements to implement both is redundant.

\item {} 
\sphinxAtStartPar
Etc.

\end{itemize}

\sphinxAtStartPar
Furthermore, some terminology is open to interpretation and nowhere rigorously defined. The official directions potentially have similar issues. So following the official directions without further guidance could result in overworking some requirements or misinterpreting others. For a more succinct outline of the Task 2 requirements see:
\begin{itemize}
\item {} 
\sphinxAtStartPar
\sphinxhref{https://ashejim.github.io/C964/task2\_c/task2\_part\_c.html\#what-does-the-application-need-to-do}{\sphinxstylestrong{Part C requirements}}

\item {} 
\sphinxAtStartPar
\sphinxhref{https://ashejim.github.io/C964/task2\_doc/task2\_doc.html\#task-2-the-documentation}{\sphinxstylestrong{Parts A, B, and D requirements}}

\end{itemize}

\sphinxAtStartPar
For the documentation, preserve the template’s section titles, and order, and submit all four parts as a single document (preferably a \sphinxcode{\sphinxupquote{.pdf}}). With a long, complicated document, aligning content to competencies presents a challenge. Don’t make things difficult for the evaluator by spreading the content over several documents in an unfamiliar format.

\sphinxAtStartPar
If anything needs further explanation, please ask us! \sphinxhref{https://ashejim.github.io/C964/ci\_c964.html\#c964-course-faculty}{Contact your C964 course instructor}.


\subsection{The official learning resource seems to include documentation items not included on this webpage. Which should I follow?}
\label{\detokenize{task1:the-official-learning-resource-seems-to-include-documentation-items-not-included-on-this-webpage-which-should-i-follow}}
\sphinxAtStartPar
Either will meet the requirements. However, the \sphinxhref{https://ashejim.github.io/C964/task2\_doc/task2\_doc.html\#task-2-the-documentation}{template on this webpage}, is more succinct and was developed in collaboration with the capstone evaluators (thank you evaluator team!) to specifically align with versions SIM3 and SIM2. Hence, you can be ensured that following this website’s template will meet all the requirements.

\sphinxAtStartPar
The content for the official learning resource (LR) was largely copied from an older version of this website which aligned to the \sphinxhref{https://westerngovernorsuniversity-my.sharepoint.com/:w:/g/personal/jim\_ashe\_wgu\_edu/ERGxhsNfkbhEutlkXVFITMQBPOmWlkVx1p5H0UisvnBesg}{previous template} written for SIM2. When C964 was updated to SIM3, we updated this website and the template accordingly. Hence, the discrepancy. As the actual requirements for SIM2 and SIM3 are the same, following the official LR or this website should be fine.


\subsection{Help! I’ve never coded a machine learning project. For C951 task 3, I only had to write about Machine Learning. Where do I learn this?}
\label{\detokenize{task1:help-i-ve-never-coded-a-machine-learning-project-for-c951-task-3-i-only-had-to-write-about-machine-learning-where-do-i-learn-this}}
\sphinxAtStartPar
WGU provides access to a very good \sphinxhref{https://lrps.wgu.edu/provision/386121824}{AI textbook which includes a Machine Learning} section. However, it is conceptually focused and includes very little application or practical examples. Furthermore, reading this text might require mathematics not provided in WGU’s BSCS curriculum.

\sphinxAtStartPar
If you have time, Udemy offers some ML courses{]}(\sphinxurl{https://wgu.udemy.com/course/machinelearning/learn/lecture/14473662\#overview}). Maybe the fastest way to get started is with the \sphinxhref{https://ashejim.github.io/C964/task2\_c/example\_sup\_class/sup\_class\_ex.html}{video and examples included on this website}. Though a minimally passing C950 project (applying a greedy algorithm to hand\sphinxhyphen{}picked truckloads) would not be consider ML by many, it meets the criteria for this project as it is an algorithm applied to data.


\subsection{Do I need an “electronic signature” as specified in the official rubric?}
\label{\detokenize{task1:do-i-need-an-electronic-signature-as-specified-in-the-official-rubric}}
\sphinxAtStartPar
No, you can type in your name, use a “fancy” font, or insert an image of your signature.


\subsection{What are the common reasons for task 1 being returned?}
\label{\detokenize{task1:what-are-the-common-reasons-for-task-1-being-returned}}\begin{enumerate}
\sphinxsetlistlabels{\arabic}{enumi}{enumii}{}{.}%
\item {} 
\sphinxAtStartPar
No instructor signature on the approval form. You need to send it to us and get a signature \sphinxstyleemphasis{before} submitting it to Assessments.

\item {} 
\sphinxAtStartPar
Both or neither box is marked on the waiver form. Mark one and only one box. See the {\hyperref[\detokenize{task1:task1-waiver-form}]{\sphinxcrossref{\DUrole{std,std-ref}{waiver form instructions}}}}

\end{enumerate}

\sphinxAtStartPar
Note, the waiver form is \sphinxstylestrong{only} required if your project is based upon or included restricted information. Task 1 \sphinxstyleemphasis{B: Capstone Release Form}, passes automatically if no waiver form is submitted, i.e., the waiver only needs to be submitted if it’s needed for permissions.


\subsection{How many attempts are allowed for each task?}
\label{\detokenize{task1:how-many-attempts-are-allowed-for-each-task}}
\sphinxAtStartPar
You have \sphinxstyleemphasis{unlimited} attempts for both tasks 1 and 2. However, incomplete submissions or submissions significantly falling short of the minimum requirements may be \sphinxstyleemphasis{locked} from further submissions without instructor approval. Furthermore, such submissions do not receive meaningful evaluator comments. It’s in your best interest to the complete minimal requirements to the best of your understanding.


\subsection{What is a descriptive method?}
\label{\detokenize{task1:what-is-a-descriptive-method}}
\sphinxAtStartPar
Anything that describes the data. Histograms, scatterplots, pie charts \sphinxhyphen{}all the familiar descriptive statistics techniques are included. ML methods such as k\sphinxhyphen{}means clustering can also be descriptive. Whether a method is descriptive or non\sphinxhyphen{}descriptive is determined by its use. For example, using a regression line to describe the relationship between variables is descriptive, but using the line to predict a variable or claim a correlation between the variables exist is inferential (non\sphinxhyphen{}descriptive).

\sphinxAtStartPar
Descriptive and non\sphinxhyphen{}descriptive methods do not need to be explicitly identified in task 2. The visualizations and requirements of the user application will satisfy these parts respectively. The distinction is only made in the approval form so we can easily identify where and how ML will be applied.


\subsection{What is a non\sphinxhyphen{}descriptive method?}
\label{\detokenize{task1:what-is-a-non-descriptive-method}}
\sphinxAtStartPar
Anything that infers from the data, e.g., making predictions, recommendations, identifying correlations, inferring from correlations, etc. Also, see the comments above.

\sphinxAtStartPar
Descriptive and non\sphinxhyphen{}descriptive methods do not need to be explicitly identified in task 2. The visualizations and requirements of the user application will satisfy these parts respectively. The distinction is only made in the approval form, so we can easily identify where and how ML will be applied.


\subsection{What is machine learning (ML)?}
\label{\detokenize{task1:what-is-machine-learning-ml}}
\sphinxAtStartPar
That depends on who you ask! But for this project, it is an algorithm applied to data.

\sphinxAtStartPar
For computer science, Machine learning is a subfield of artificial intelligence (a subfield of mathematics), broadly defined as the development of machines capable of self\sphinxhyphen{}adjusting behavior based on data. However, from the data science perspective, machine learning is generally defined as using algorithms to identify patterns, make predictions, etc., from data. That is, machine learning is a goal, not a technique. So, for example, a data scientist (and the evaluators) consider linear regression machine learning \sphinxhyphen{}when it’s used as a prediction model. However, a mathematician would politely (or not so politely) disagree with a 19th\sphinxhyphen{}century equation being classified as ML.


\subsection{Can I use libraries outside the standard (Python, Java, etc.) installation?}
\label{\detokenize{task1:can-i-use-libraries-outside-the-standard-python-java-etc-installation}}
\sphinxAtStartPar
Yes. Unlike C950 (Data Structures \& Algorithms II), you are allowed and encouraged to use outside libraries. All the major languages, but particularly Python, have a wide array of highly developed ML tools. The C964 capstone is about applying these tools \sphinxhyphen{}not their development.


\subsection{What language, libraries, and platforms should I use?}
\label{\detokenize{task1:what-language-libraries-and-platforms-should-i-use}}
\sphinxAtStartPar
You can use whatever you like. However, we recommend using Python and the \sphinxhref{https://scikit-learn.org/stable/}{scikit\sphinxhyphen{}learn} (aka sklearn) library. In addition to having an extensive collection of ML\sphinxhyphen{}specific tools and tutorials, WGU has better faculty support for these. Jupyter Notebook is a great place to start for the app front end. Passing applications are often submitted as the Notebook file (\sphinxcode{\sphinxupquote{.ipynb}}) and data files. Jupyter Notebooks are a great way to present code and information together, but they can also progressively be developed into a more polished product. Students are often tempted to use Java because of their JavaFX experience in software II (C195), but a GUI is not required, and Python is overall better suited for everything which is required. Furthermore, implementing a GUI with Python or Jupyter Notebooks is not difficult (see these {\hyperref[\detokenize{task2_c/example_sup_class/sup_class_ex-ui:sup-class-ex-user-interface}]{\sphinxcrossref{\DUrole{std,std-ref}{widget examples}}}}).

\sphinxAtStartPar
A development path might look like the following:
\begin{quote}

\sphinxAtStartPar
Python IDE → Jupyter notebook → notebook with widgets → hosted notebook with widgets → web app.
\end{quote}

\sphinxAtStartPar
Provided the \DUrole{xref,myst}{minimal app criteria} are met, submitting at any point along this path will pass part C.


\subsection{What sort of user interface do I need? Do I need a GUI?}
\label{\detokenize{task1:what-sort-of-user-interface-do-i-need-do-i-need-a-gui}}\label{\detokenize{task1:task1-faq-gui}}
\sphinxAtStartPar
No, a GUI is \sphinxstyleemphasis{not} required. Your app must be usable by the “client” to solve the proposed problem. If the evaluators can run your app as intended, playing the role of the “client,” following your \DUrole{xref,myst}{user guide}, then your app will be considered to have a user\sphinxhyphen{}friendly interface. This can be done through a GUI and widgets, but using the command line or reading user data from a local directory will also suffice.


\subsection{Are there length requirements for the documentation?}
\label{\detokenize{task1:are-there-length-requirements-for-the-documentation}}
\sphinxAtStartPar
No. What you see in the examples and template are just guidelines. The individual evaluator determines what qualifies as “sufficient detail,” which further varies depending on the project and writing style. If you feel you’ve met the requirements, simply move on to the next section. Upon submission, it will pass, or they will request more details. In the latter case, you can then focus on revising the more narrow scope as directed by the evaluator’s comments, which is generally more efficient than overworking the entire project.


\subsection{I’ve completed the task 2 documentation. Should I send it to my course instructor for review?}
\label{\detokenize{task1:i-ve-completed-the-task-2-documentation-should-i-send-it-to-my-course-instructor-for-review}}
\sphinxAtStartPar
If you have specific questions or concerns \sphinxhyphen{}yes. However, in most cases, it’s best just to submit. What suffices as “sufficient detail” is highly subjective. We can always tell you to add more, but if you’ve done your best to fulfill the requirements, submit it and let them tell you which (if any) parts need revision. At best, it passes; at worst, we address the issues cited by the evaluator \sphinxhyphen{}and then it passes. Responding to the more narrow focus of the evaluator’s comments is generally easier than overworking the entire project.

\sphinxAtStartPar
You have \sphinxstyleemphasis{unlimited} submissions but limited time. So typically this is the best and most efficient approach.


\subsection{I only have a Linux (or Mac) machine. Will evaluators be able to run my code?}
\label{\detokenize{task1:i-only-have-a-linux-or-mac-machine-will-evaluators-be-able-to-run-my-code}}
\sphinxAtStartPar
Technically (and unfortunately), we are a “Windows” university, and all submitted projects should be able to run in Windows. However, being Windows\sphinxhyphen{}compatible is \sphinxstyleemphasis{nowhere specifically required} in the C964 rubric, and doing so would be a little silly for a computer science program. That said, WGU evaluators are only issued Windows 10 machines, and they may have difficulty running a Linux, Android, or Mac app without special instructions. Therefore for non\sphinxhyphen{}Windows apps, we recommend providing explicit instructions in the {\hyperref[\detokenize{task2_doc/task2_doc_d:task2-doc-d-user-guide}]{\sphinxcrossref{\DUrole{std,std-ref}{user guide}}}} for a Windows 10 user to run your code, such as using a \sphinxhref{https://ubuntu.com/tutorials/how-to-run-ubuntu-desktop-on-a-virtual-machine-using-virtualbox\#1-overview}{virtual machine}, a remote machine, or using a \sphinxhref{https://ubuntu.com/tutorials/install-ubuntu-on-wsl2-on-windows-10\#1-overview}{Linux subsystem}.


\subsection{How complex does my data, algorithm, or model need to be?}
\label{\detokenize{task1:how-complex-does-my-data-algorithm-or-model-need-to-be}}
\sphinxAtStartPar
It must be complex enough to meet the needs of your project. There is no explicit minimal complexity for any of these items. However, the model must meet the needs of the “organizational need” and the data must be appropriate for developing the model which could indirectly impose a minimal complexity. For example, a supervised model requires two variables.


\subsection{Are there any restrictions on which datasets I can choose?}
\label{\detokenize{task1:are-there-any-restrictions-on-which-datasets-i-can-choose}}
\sphinxAtStartPar
Only that data must be legally available to use and share with evaluators. For example, using data belonging to a current employer would require submitting a \DUrole{xref,myst}{waiver form}.
\begin{itemize}
\item {} 
\sphinxAtStartPar
You \sphinxstyleemphasis{can} use any dataset found on \sphinxhref{https://www.kaggle.com/datasets}{kaggle.com}.

\item {} 
\sphinxAtStartPar
You \sphinxstyleemphasis{can} use simulated data.

\item {} 
\sphinxAtStartPar
You \sphinxstyleemphasis{can} use data used for previous projects (submitted by you or others).

\item {} 
\sphinxAtStartPar
You only need to apply for \sphinxhref{https://cm.wgu.edu/t5/Frequently-Asked-Questions/WGU-IRB-and-Human-Subject-Protections-FAQ/ta-p/2002}{IRB review} if you are \sphinxstyleemphasis{collecting} data involving human participants (this is very rare). Otherwise, your project is in IRB compliance.

\end{itemize}


\section{Questions, comments, or suggestions?}
\label{\detokenize{task1:questions-comments-or-suggestions}}


\sphinxstepscope


\chapter{Task 2: Part C \sphinxhyphen{}the application}
\label{\detokenize{task2_c/task2_part_c:task-2-part-c-the-application}}\label{\detokenize{task2_c/task2_part_c::doc}}\phantomsection\label{\detokenize{task2_c/task2_part_c:task2-part-c}}
\sphinxAtStartPar
Your software application (the “app”) is the entirety of part C of task 2. However, the Computer Science capstone is \sphinxstyleemphasis{not} a software project. Other than requiring you to apply machine learning (ML) to data, they will only assess \sphinxstyleemphasis{what} your application does \sphinxhyphen{}not its presentation.

\begin{sphinxShadowBox}
\sphinxstylesidebartitle{\sphinxstylestrong{What is an \sphinxstyleemphasis{application}?}}

\sphinxAtStartPar
An application (app) is simply software that performs specific tasks.
\end{sphinxShadowBox}

\begin{sphinxadmonition}{tip}{Tip:}
\sphinxAtStartPar
The official rubric and directions were written to map the project’s elements to specific competencies. However, to allow for a broad range of projects its language is necessarily also broad. For more specifics, we recommend referring to the guidelines on this webpage and the \sphinxhref{https://westerngovernorsuniversity-my.sharepoint.com/:w:/g/personal/jim\_ashe\_wgu\_edu/ERGxhsNfkbhEutlkXVFITMQBPOmWlkVx1p5H0UisvnBesg?rtime=3q\_Efs-u2kg}{Task 2 template}.
\end{sphinxadmonition}


\section{What does the application need to do?}
\label{\detokenize{task2_c/task2_part_c:what-does-the-application-need-to-do}}\label{\detokenize{task2_c/task2_part_c:task2-part-c-what-does-the-application-need-to-do}}

\phantomsection\label{\detokenize{task2_c/task2_part_c:id1}}
\sphinxAtStartPar
Your app must help the user solve the organizational problem from {\hyperref[\detokenize{task1:task1}]{\sphinxcrossref{\DUrole{std,std-ref}{task 1}}}} by doing the following:
\begin{enumerate}
\sphinxsetlistlabels{\arabic}{enumi}{enumii}{}{.}%
\item {} 
\sphinxAtStartPar
\sphinxstylestrong{Data → ML model:} Use data to develop a machine learning model.

\item {} 
\sphinxAtStartPar
\sphinxstylestrong{Accuracy Metric:} Provide an appropriate metric or plan for measuring the app’s accuracy.

\item {} 
\sphinxAtStartPar
\sphinxstylestrong{Visualizations:} Present 3 different pictures; images can code generated or static.

\item {} 
\sphinxAtStartPar
\sphinxstylestrong{User Application:} Provide a way for the “user” to use the ML model towards solving the problem.

\end{enumerate}



\begin{sphinxuseclass}{sd-tab-set}
\begin{sphinxuseclass}{sd-tab-item}\subsubsection*{Example 1}

\begin{sphinxuseclass}{sd-tab-content}
\sphinxAtStartPar
\sphinxstylestrong{The App:} A standalone \sphinxhref{https://jupyter.org}{Jupyter Notebook} file (.ipynb) which can predict Iris types (see the \DUrole{xref,myst}{example}).
\begin{quote}

\sphinxAtStartPar
\sphinxstylestrong{Data → ML model:} Labeled \sphinxhref{https://www.kaggle.com/datasets/uciml/iris}{Iris petal dimensions} train an {\hyperref[\detokenize{task2_c/example_sup_class/sup_class_ex-develop:sup-class-ex-develop}]{\sphinxcrossref{\DUrole{std,std-ref}{SVM classification model}}}}.

\sphinxAtStartPar
\sphinxstylestrong{Accuracy Metric:} Percent of correct predictions on {\hyperref[\detokenize{task2_c/example_sup_class/sup_class_ex-accuracy:sup-class-ex-accuracy}]{\sphinxcrossref{\DUrole{std,std-ref}{testing data}}}}.

\sphinxAtStartPar
\sphinxstylestrong{Visualizations:} {\hyperref[\detokenize{task2_c/example_sup_class/sup_class_ex-process:sup-class-ex-descriptive-methods-and-visualizations}]{\sphinxcrossref{\DUrole{std,std-ref}{Histograms}}}} showing distributions of different flower features and a {\hyperref[\detokenize{task2_c/example_sup_class/sup_class_ex-accuracy:sup-class-ex-accuracy-confusion-matrix}]{\sphinxcrossref{\DUrole{std,std-ref}{confusion matrix}}}} illustrating the accuracy of the classification model.

\sphinxAtStartPar
\sphinxstylestrong{User interface:} (New data → ML model → prediction) Following detailed instructions, the user can {\hyperref[\detokenize{task2_c/example_sup_class/sup_class_ex-ui:sup-class-ex-ui-code}]{\sphinxcrossref{\DUrole{std,std-ref}{input flower dimensions}}}}, run code in the Jupyter notebook, and the app returns a prediction helping customers identify their flowers.
\end{quote}

\end{sphinxuseclass}
\end{sphinxuseclass}
\begin{sphinxuseclass}{sd-tab-item}\subsubsection*{Example 2}

\begin{sphinxuseclass}{sd-tab-content}
\sphinxAtStartPar
\sphinxstylestrong{The App:} A standalone Pycharm (Python) project predicting house prices.
\begin{quote}

\sphinxAtStartPar
\sphinxstylestrong{Data → ML model:} Labeled housing data trains a multi\sphinxhyphen{}linear regression model to predict house prices.

\sphinxAtStartPar
\sphinxstylestrong{Accuracy Metric:} \sphinxhref{https://scikit-learn.org/stable/modules/generated/sklearn.metrics.mean\_squared\_error.html\#examples-using-sklearn-metrics-mean-squared-error}{Mean squared error} of the model on testing data.

\sphinxAtStartPar
\sphinxstylestrong{Visualizations:} Histograms showing distributions of data features and scatterplots demonstrating data correlations.

\sphinxAtStartPar
\sphinxstylestrong{User interface:} (New data → ML model → prediction) Via console the user can input unseen house data, and the model predicts a house’s price helping a realty firm make investment decisions.
\end{quote}

\end{sphinxuseclass}
\end{sphinxuseclass}
\begin{sphinxuseclass}{sd-tab-item}\subsubsection*{Example 3}

\begin{sphinxuseclass}{sd-tab-content}
\sphinxAtStartPar
\sphinxstylestrong{The App:} A web app (developed with \sphinxhref{https://pythonforundergradengineers.com/deploy-jupyter-notebook-voila-heroku.html}{Juptyer, voila, and deployed on Heroku} recommending movies.
\begin{quote}

\sphinxAtStartPar
\sphinxstylestrong{Data → ML model:} Using movie data, data is \sphinxhref{https://scikit-learn.org/stable/modules/generated/sklearn.feature\_extraction.text.CountVectorizer.html}{vectorized} so that \sphinxhref{https://scikit-learn.org/stable/modules/generated/sklearn.metrics.pairwise.cosine\_similarity.html}{Cosine similarity} can provide similarity scores between any two movies.

\sphinxAtStartPar
\sphinxstylestrong{Accuracy Metric:} \sphinxhref{https://scikit-learn.org/stable/modules/metrics.html\#cosine-similarity}{Cosine simliarity score} on test data, or a plan to measure the app’s acccuracy based on future user feedback,

\sphinxAtStartPar
\sphinxstylestrong{Visualizations:} Histograms showing distributions of movie features and samples of Cosine Similarity matrices.

\sphinxAtStartPar
\sphinxstylestrong{User interface:} (New data → ML model → prediction) Using a user\sphinxhyphen{}provided movie, the web app returns five recommendations for a user typed movie helping customers find movies to watch. See this \sphinxhref{https://github.com/Prajwal10031999/Movie-Recommendation-System-Using-Cosine-Similarity}{movie recommendation example}
\end{quote}

\end{sphinxuseclass}
\end{sphinxuseclass}
\begin{sphinxuseclass}{sd-tab-item}\subsubsection*{Example 4}

\begin{sphinxuseclass}{sd-tab-content}
\sphinxAtStartPar
\sphinxstylestrong{The App:} A hosted Jupyter notebook identifies images as a cat or dog.
\begin{quote}

\sphinxAtStartPar
\sphinxstylestrong{Data → ML model:} Labeled images are used to \sphinxhref{https://www.tensorflow.org/tutorials/images/classification}{train a neural network model to classify images} of dogs and cats.

\sphinxAtStartPar
\sphinxstylestrong{Accuracy Metric:} Percent of correct predictions on testing data.

\sphinxAtStartPar
\sphinxstylestrong{Visualizations:} Graphs of training and validation accuracy and loss, a confusion matrix showing model accuracy, and image examples.

\sphinxAtStartPar
\sphinxstylestrong{User interface:} (New data → ML model → prediction) Following detailed instructions, the users upload an image to Jupyter notebook folder deployed on \sphinxhref{https://datalore.jetbrains.com}{Datalore}, and run the app classifying the image as cat or dog.
\end{quote}

\end{sphinxuseclass}
\end{sphinxuseclass}
\end{sphinxuseclass}

\subsection{Data requirements}
\label{\detokenize{task2_c/task2_part_c:data-requirements}}\label{\detokenize{task2_c/task2_part_c:task2c-data-requirements}}\begin{itemize}
\item {} 
\sphinxAtStartPar
Your data must be sufficiently complex and processed for your ML algorithm to function as needed.

\item {} 
\sphinxAtStartPar
Evaluators must be able to access your data.

\end{itemize}

\sphinxAtStartPar
There are no specific requirements regarding the complexity or nature of your data. However, your application must work and fulfill the organizational need you’ll outline in the {\hyperref[\detokenize{task2_doc/task2_doc:task2-doc}]{\sphinxcrossref{\DUrole{std,std-ref}{task 2 documentation}}}}. Therefore, when choosing your raw data, consider carefully any additional time and effort necessary to prepare that data for use. As these extra steps are only needed in so far as the chosen problem and algorithm need them, you may wish to adjust your project or choose a different data set accordingly.

\sphinxAtStartPar
You do not need special permission to use any open\sphinxhyphen{}source dataset (see our {\hyperref[\detokenize{resources:resources-task1-data}]{\sphinxcrossref{\DUrole{std,std-ref}{list}}}}). Furthermore, data sets used in previous C964 projects are available for reuse (no list of previously used datasets exists). It’s only required that the data meet the needs of your project and be legally available to use and share with evaluators. For example, using data belonging to a current employer would require submitting a \DUrole{xref,myst}{waiver form}.
\begin{itemize}
\item {} 
\sphinxAtStartPar
You \sphinxstyleemphasis{can} use any dataset found on \sphinxhref{https://www.kaggle.com/datasets}{kaggle.com}.

\item {} 
\sphinxAtStartPar
You \sphinxstyleemphasis{can} use simulated data.

\item {} 
\sphinxAtStartPar
You \sphinxstyleemphasis{can} use data used for previous projects (submitted by you or others).

\item {} 
\sphinxAtStartPar
You only need to apply for \sphinxhref{https://cm.wgu.edu/t5/Frequently-Asked-Questions/WGU-IRB-and-Human-Subject-Protections-FAQ/ta-p/2002}{IRB review} if you are \sphinxstyleemphasis{collecting} data involving human participants (this is very rare). Otherwise, your project is in IRB compliance.

\end{itemize}


\subsection{Machine Learning requirements}
\label{\detokenize{task2_c/task2_part_c:machine-learning-requirements}}\label{\detokenize{task2_c/task2_part_c:task2-part-c-mlreqs}}\begin{itemize}
\item {} 
\sphinxAtStartPar
You must apply machine learning to data.

\end{itemize}

\sphinxAtStartPar
You are encouraged to use ML libraries. Provided the source code is available to evaluators, any language or library of your choosing is allowed. However, we do recommend and can give better support for Python. The \sphinxhref{https://scikit-learn.org/stable/}{scikit\sphinxhyphen{}learn} library is an excellent choice; it is diverse, robust, and has many supplementary resources to help you get started.


\subsection{Visualization requirements}
\label{\detokenize{task2_c/task2_part_c:visualization-requirements}}\label{\detokenize{task2_c/task2_part_c:task2c-visualreqs}}\begin{itemize}
\item {} 
\sphinxAtStartPar
You must have three different images helping describe your project.

\end{itemize}

\sphinxAtStartPar
The purpose of the visuals is to help the reader understand your project, but little is required other than the three images be unique and related to your project. The images can be generated by the code or inserted statically. The visualizations can describe the data or ML algorithm. They can be the same type describing different data subsets or of different types describing the same data. Examples include pie charts, histograms, scatterplots, confusion matrices, etc.

\sphinxAtStartPar
The visualizations are required to be part of the application. However, sometime this may not be ideal, say when the app is intended to customer facing. So it is allowable to submit the code providing the visualizations separate from the main application code.



\begin{sphinxuseclass}{sd-sphinx-override}
\begin{sphinxuseclass}{sd-cards-carousel}
\begin{sphinxuseclass}{sd-card-cols-4}
\begin{sphinxuseclass}{sd-card}
\begin{sphinxuseclass}{sd-sphinx-override}
\begin{sphinxuseclass}{sd-mb-3}
\begin{sphinxuseclass}{sd-shadow-sm}
\begin{sphinxuseclass}{sd-card-body}
\begin{sphinxuseclass}{text-center}
\sphinxAtStartPar
Scatterplot

\noindent\sphinxincludegraphics[height=100\sphinxpxdimen]{{visual12}.png}

\end{sphinxuseclass}
\end{sphinxuseclass}
\end{sphinxuseclass}
\end{sphinxuseclass}
\end{sphinxuseclass}
\end{sphinxuseclass}
\begin{sphinxuseclass}{sd-card}
\begin{sphinxuseclass}{sd-sphinx-override}
\begin{sphinxuseclass}{sd-mb-3}
\begin{sphinxuseclass}{sd-shadow-sm}
\begin{sphinxuseclass}{sd-card-body}
\begin{sphinxuseclass}{text-center}
\sphinxAtStartPar
Regression Line

\noindent\sphinxincludegraphics[height=100\sphinxpxdimen]{{visual0}.png}

\end{sphinxuseclass}
\end{sphinxuseclass}
\end{sphinxuseclass}
\end{sphinxuseclass}
\end{sphinxuseclass}
\end{sphinxuseclass}
\begin{sphinxuseclass}{sd-card}
\begin{sphinxuseclass}{sd-sphinx-override}
\begin{sphinxuseclass}{sd-mb-3}
\begin{sphinxuseclass}{sd-shadow-sm}
\begin{sphinxuseclass}{sd-card-body}
\begin{sphinxuseclass}{text-center}
\sphinxAtStartPar
Confusion Matrix

\noindent\sphinxincludegraphics[height=100\sphinxpxdimen]{{visual1}.jpg}

\end{sphinxuseclass}
\end{sphinxuseclass}
\end{sphinxuseclass}
\end{sphinxuseclass}
\end{sphinxuseclass}
\end{sphinxuseclass}
\begin{sphinxuseclass}{sd-card}
\begin{sphinxuseclass}{sd-sphinx-override}
\begin{sphinxuseclass}{sd-mb-3}
\begin{sphinxuseclass}{sd-shadow-sm}
\begin{sphinxuseclass}{sd-card-body}
\begin{sphinxuseclass}{text-center}
\sphinxAtStartPar
Histogram

\noindent\sphinxincludegraphics[height=100\sphinxpxdimen]{{visual11}.png}

\end{sphinxuseclass}
\end{sphinxuseclass}
\end{sphinxuseclass}
\end{sphinxuseclass}
\end{sphinxuseclass}
\end{sphinxuseclass}
\begin{sphinxuseclass}{sd-card}
\begin{sphinxuseclass}{sd-sphinx-override}
\begin{sphinxuseclass}{sd-mb-3}
\begin{sphinxuseclass}{sd-shadow-sm}
\begin{sphinxuseclass}{sd-card-body}
\begin{sphinxuseclass}{text-center}
\sphinxAtStartPar
Histograms

\noindent\sphinxincludegraphics[height=100\sphinxpxdimen]{{visual5}.png}

\end{sphinxuseclass}
\end{sphinxuseclass}
\end{sphinxuseclass}
\end{sphinxuseclass}
\end{sphinxuseclass}
\end{sphinxuseclass}
\begin{sphinxuseclass}{sd-card}
\begin{sphinxuseclass}{sd-sphinx-override}
\begin{sphinxuseclass}{sd-mb-3}
\begin{sphinxuseclass}{sd-shadow-sm}
\begin{sphinxuseclass}{sd-card-body}
\begin{sphinxuseclass}{text-center}
\sphinxAtStartPar
Pie\sphinxhyphen{}chart

\noindent\sphinxincludegraphics[height=100\sphinxpxdimen]{{visual6}.png}

\end{sphinxuseclass}
\end{sphinxuseclass}
\end{sphinxuseclass}
\end{sphinxuseclass}
\end{sphinxuseclass}
\end{sphinxuseclass}
\begin{sphinxuseclass}{sd-card}
\begin{sphinxuseclass}{sd-sphinx-override}
\begin{sphinxuseclass}{sd-mb-3}
\begin{sphinxuseclass}{sd-shadow-sm}
\begin{sphinxuseclass}{sd-card-body}
\begin{sphinxuseclass}{text-center}
\sphinxAtStartPar
Scatterplot Matrix

\noindent\sphinxincludegraphics[height=100\sphinxpxdimen]{{visual7}.png}

\end{sphinxuseclass}
\end{sphinxuseclass}
\end{sphinxuseclass}
\end{sphinxuseclass}
\end{sphinxuseclass}
\end{sphinxuseclass}
\begin{sphinxuseclass}{sd-card}
\begin{sphinxuseclass}{sd-sphinx-override}
\begin{sphinxuseclass}{sd-mb-3}
\begin{sphinxuseclass}{sd-shadow-sm}
\begin{sphinxuseclass}{sd-card-body}
\begin{sphinxuseclass}{text-center}
\sphinxAtStartPar
Correlation Matrix

\noindent\sphinxincludegraphics[height=100\sphinxpxdimen]{{visual13}.png}

\end{sphinxuseclass}
\end{sphinxuseclass}
\end{sphinxuseclass}
\end{sphinxuseclass}
\end{sphinxuseclass}
\end{sphinxuseclass}
\begin{sphinxuseclass}{sd-card}
\begin{sphinxuseclass}{sd-sphinx-override}
\begin{sphinxuseclass}{sd-mb-3}
\begin{sphinxuseclass}{sd-shadow-sm}
\begin{sphinxuseclass}{sd-card-body}
\begin{sphinxuseclass}{text-center}
\sphinxAtStartPar
Barplot

\noindent\sphinxincludegraphics[height=100\sphinxpxdimen]{{visual9}.png}

\end{sphinxuseclass}
\end{sphinxuseclass}
\end{sphinxuseclass}
\end{sphinxuseclass}
\end{sphinxuseclass}
\end{sphinxuseclass}
\begin{sphinxuseclass}{sd-card}
\begin{sphinxuseclass}{sd-sphinx-override}
\begin{sphinxuseclass}{sd-mb-3}
\begin{sphinxuseclass}{sd-shadow-sm}
\begin{sphinxuseclass}{sd-card-body}
\begin{sphinxuseclass}{text-center}
\sphinxAtStartPar
Barplot

\noindent\sphinxincludegraphics[height=100\sphinxpxdimen]{{visual8}.png}

\end{sphinxuseclass}
\end{sphinxuseclass}
\end{sphinxuseclass}
\end{sphinxuseclass}
\end{sphinxuseclass}
\end{sphinxuseclass}
\begin{sphinxuseclass}{sd-card}
\begin{sphinxuseclass}{sd-sphinx-override}
\begin{sphinxuseclass}{sd-mb-3}
\begin{sphinxuseclass}{sd-shadow-sm}
\begin{sphinxuseclass}{sd-card-body}
\begin{sphinxuseclass}{text-center}
\sphinxAtStartPar
Line plots

\noindent\sphinxincludegraphics[height=100\sphinxpxdimen]{{visual10}.png}

\end{sphinxuseclass}
\end{sphinxuseclass}
\end{sphinxuseclass}
\end{sphinxuseclass}
\end{sphinxuseclass}
\end{sphinxuseclass}
\begin{sphinxuseclass}{sd-card}
\begin{sphinxuseclass}{sd-sphinx-override}
\begin{sphinxuseclass}{sd-mb-3}
\begin{sphinxuseclass}{sd-shadow-sm}
\begin{sphinxuseclass}{sd-card-body}
\begin{sphinxuseclass}{text-center}
\sphinxAtStartPar
Model Explanation

\noindent\sphinxincludegraphics[height=100\sphinxpxdimen]{{visual4}.png}

\end{sphinxuseclass}
\end{sphinxuseclass}
\end{sphinxuseclass}
\end{sphinxuseclass}
\end{sphinxuseclass}
\end{sphinxuseclass}
\begin{sphinxuseclass}{sd-card}
\begin{sphinxuseclass}{sd-sphinx-override}
\begin{sphinxuseclass}{sd-mb-3}
\begin{sphinxuseclass}{sd-shadow-sm}
\begin{sphinxuseclass}{sd-card-body}
\begin{sphinxuseclass}{text-center}
\sphinxAtStartPar
Regression

\noindent\sphinxincludegraphics[height=100\sphinxpxdimen]{{visual2}.png}

\end{sphinxuseclass}
\end{sphinxuseclass}
\end{sphinxuseclass}
\end{sphinxuseclass}
\end{sphinxuseclass}
\end{sphinxuseclass}
\begin{sphinxuseclass}{sd-card}
\begin{sphinxuseclass}{sd-sphinx-override}
\begin{sphinxuseclass}{sd-mb-3}
\begin{sphinxuseclass}{sd-shadow-sm}
\begin{sphinxuseclass}{sd-card-body}
\begin{sphinxuseclass}{text-center}
\sphinxAtStartPar
Clustering

\noindent\sphinxincludegraphics[height=100\sphinxpxdimen]{{visual3}.png}

\end{sphinxuseclass}
\end{sphinxuseclass}
\end{sphinxuseclass}
\end{sphinxuseclass}
\end{sphinxuseclass}
\end{sphinxuseclass}
\end{sphinxuseclass}
\end{sphinxuseclass}
\end{sphinxuseclass}

\subsection{User interface requirements}
\label{\detokenize{task2_c/task2_part_c:user-interface-requirements}}\label{\detokenize{task2_c/task2_part_c:task2c-user-interface-requirements}}\begin{itemize}
\item {} 
\sphinxAtStartPar
You must provide a user\sphinxhyphen{}friendly interface by which the proposed client can use your application to help solve the problem.

\end{itemize}

\sphinxAtStartPar
Playing the role of the client, the evaluator will follow your {\hyperref[\detokenize{task2_doc/task2_doc_d:task2-doc-d-user-guide}]{\sphinxcrossref{\DUrole{std,std-ref}{user guide}}}} in part D of the documentation. To meet this requirement they should be able to do the following:
\begin{enumerate}
\sphinxsetlistlabels{\arabic}{enumi}{enumii}{}{.}%
\item {} 
\sphinxAtStartPar
\sphinxstylestrong{Run your application (user\sphinxhyphen{}friendly).} Your application will be considered “user\sphinxhyphen{}friendly” if the evaluator successfully executes and uses your application on a Windows 10 machine following your instructions. They can be instructed to download and install necessary dependencies or software.

\item {} 
\sphinxAtStartPar
\sphinxstylestrong{Use your application to solve the proposed problem as intended (interface).} Most often the interface requirement is met by having some way for the user to provide input and receive output. For example, a user provides weather conditions, and the app returns a prediction of popsicle sales. How the interface is implemented, whether it be widgets, uploaded data, or simple console input; is up to you.

\end{enumerate}

\sphinxAtStartPar
At a minimum, the interface must provide means for the user to provide input and receive feedback. Any method by which you can provide instructions is acceptable. For example:
\begin{quote}

\sphinxAtStartPar
\sphinxstyleemphasis{\sphinxstylestrong{…}}
\sphinxstyleemphasis{\sphinxstylestrong{Step 10:}} \sphinxstyleemphasis{Next, the user should type the following command into line 57:}

\begin{sphinxVerbatim}[commandchars=\\\{\}]
\PYG{n}{mymodel}\PYG{o}{.}\PYG{n}{predict}\PYG{p}{(}\PYG{p}{[}\PYG{p}{[}\PYG{n}{temperature}\PYG{p}{,} \PYG{n}{humidity}\PYG{p}{]}\PYG{p}{]}\PYG{p}{)}
\end{sphinxVerbatim}

\sphinxAtStartPar
\sphinxstyleemphasis{in place of ‘temperature’ and ‘humidity’, the user should type the temperature in Fahrenheit and humidity as a percentage for which they’d like a prediction, e.g.,}

\begin{sphinxVerbatim}[commandchars=\\\{\}]
\PYG{n}{mymodel}\PYG{o}{.}\PYG{n}{predict}\PYG{p}{(}\PYG{p}{[}\PYG{p}{[}\PYG{l+m+mi}{75}\PYG{p}{,} \PYG{l+m+mf}{24.5}\PYG{p}{]}\PYG{p}{]}\PYG{p}{)}
\end{sphinxVerbatim}

\sphinxAtStartPar
\sphinxstyleemphasis{for a temperature of 75 degrees and humidity of 24.5\%.}

\sphinxAtStartPar
\sphinxstyleemphasis{\sphinxstylestrong{Step 11}} \sphinxstyleemphasis{Run the code by pressing the ‘Run’ button in the Jupyter Notebook menu or pressing ‘Crtl+Enter’ to the left of block 57.}
\sphinxstyleemphasis{\sphinxstylestrong{…}}
\end{quote}

\sphinxAtStartPar
See the individual {\hyperref[\detokenize{task2_doc/task2_doc_d:task2-doc-d-user-guide-examples}]{\sphinxcrossref{\DUrole{std,std-ref}{example user guides}}}} and guides provided in the completed {\hyperref[\detokenize{task2_doc/task2_doc:task2-doc-examples}]{\sphinxcrossref{\DUrole{std,std-ref}{task 2 document examples}}}}.




\section{Coding}
\label{\detokenize{task2_c/task2_part_c:coding}}\label{\detokenize{task2_c/task2_part_c:task2-part-c-coding}}
\sphinxAtStartPar
Time to get to work.



\begin{sphinxShadowBox}
\sphinxstylesidebartitle{…but the requirments won’t change. }

\begin{center}
\noindent\sphinxincludegraphics[width=200\sphinxpxdimen]{{good_code}.png}
\captionof{figure}{Goode Code {[}\hyperlink{cite.resources:id4}{xkcda}{]}}\label{\detokenize{task2_c/task2_part_c:id5}}\end{center}
\end{sphinxShadowBox}

\begin{sphinxadmonition}{warning}{Warning:}
\sphinxAtStartPar
Following tutorials/examples is a great way to learn. But when it comes to writing your own code, \sphinxstyleemphasis{don’t \sphinxhref{https://www.youtube.com/watch?v=-wtzy1aqS9Q}{copy, paste, and \sphinxstylestrong{pray}}}. Instead, understand what it’s doing for each line of code and check that it runs as intended. Investing in the extra time will make you a better computer scientist and could save many frustrating hours.

\begin{figure}[H]
\centering
\capstart

\noindent\sphinxincludegraphics[height=150\sphinxpxdimen]{{code_quality}.png}
\caption{\sphinxstylestrong{Code Quality} {[}\hyperlink{cite.resources:id3}{xkcdb}{]}}\label{\detokenize{task2_c/task2_part_c:id6}}\end{figure}
\end{sphinxadmonition}

\sphinxAtStartPar
Start \sphinxstyleemphasis{slow}. You must learn and incorporate many small but probably new skills into a larger working app \sphinxhyphen{}data processing, data analysis, new libraries, and user interface. Learn one new skill, implement it, and check your code before moving on to the next step. Things will start slowly and expect to make mistakes, but things can progress quickly after the initial investment.


\subsection{Start small and build up}
\label{\detokenize{task2_c/task2_part_c:start-small-and-build-up}}
\begin{sphinxShadowBox}
\sphinxstylesidebartitle{Watch}


\end{sphinxShadowBox}

\sphinxAtStartPar
A suggested path:
\begin{enumerate}
\sphinxsetlistlabels{\arabic}{enumi}{enumii}{}{.}%
\item {} 
\sphinxAtStartPar
Import data and convert it to a data frame (using \sphinxhref{https://pandas.pydata.org/pandas-docs/stable}{Pandas}).

\item {} 
\sphinxAtStartPar
Explore the data and create some images.

\item {} 
\sphinxAtStartPar
Determine which ML algorithm to start with and choose a supporting library.

\item {} 
\sphinxAtStartPar
Read the library’s documentation and understand the expected data format and usage.

\item {} 
\sphinxAtStartPar
Apply the algorithm to the data, e.g., train it, and create a model.

\item {} 
\sphinxAtStartPar
Apply the model to new data, e.g., a single input.

\item {} 
\sphinxAtStartPar
Create a procedure for the user to apply the model, e.g., provide input.

\end{enumerate}

\sphinxAtStartPar
Add {\hyperref[\detokenize{task2_c/example_sup_class/sup_class_ex-process:sup-class-ex-descriptive-methods-and-visualizations}]{\sphinxcrossref{\DUrole{std,std-ref}{three images}}}}, and you have a passing part C after step 7. Then, as time allows, you can go back to step 2, improving the performance and presentation of your application until you are satisfied.

\sphinxAtStartPar
Jupyter Notebook is a great choice for the application’s development \sphinxstylestrong{and} front end. Passing applications often include only the Notebook (.ipynb) and data files. Jupyter Notebooks are a great way to present code and information together. Moreover, they can also be progressively developed into a more polished product. For example, a development path might look like the following:
\begin{itemize}
\item {} 
\sphinxAtStartPar
Python IDE → Jupyter Notebook → notebook with widgets → hosted notebook with widgets → web app.

\end{itemize}

\sphinxAtStartPar
Provided the \DUrole{xref,myst}{minimal app criteria} are met, submitting any point along this path will pass part C. You can use whatever language or libraries you like. However, we recommend Python. For ML libraries,  the \sphinxhref{https://scikit-learn.org/stable/}{scikit\sphinxhyphen{}learn} (aka sklearn) is a great choice. In addition to having an extensive collection of ML\sphinxhyphen{}specific tools and tutorials, WGU has better faculty support available for both Python and sklearn.


\section{Application Performance}
\label{\detokenize{task2_c/task2_part_c:application-performance}}\label{\detokenize{task2_c/task2_part_c:task2-part-c-application-performance}}
\sphinxAtStartPar
For \sphinxstyleemphasis{supervised methods}, you should use a metric to measure accuracy and help improve the model. Knowing which algorithm will perform best requires an understanding of the data and algorithms. However, using a metric, you can quickly compare and experiment with different approaches \sphinxhyphen{}usually by changing a few lines of code. Such experimentation can then lead to understanding.

\sphinxAtStartPar
Depending on the method, metrics might similarly be used for \sphinxstyleemphasis{unsupervised models}, such as \sphinxhref{https://scikit-learn.org/stable/auto\_examples/cluster/plot\_kmeans\_silhouette\_analysis.html}{Silhouette coefficients} for KMeans clustering. Alternatively (and typically), a future development plan for measuring the accuracy of your unsupervised method can be used.

\begin{sphinxShadowBox}
\sphinxstylesidebartitle{What is good accuracy?}

\sphinxAtStartPar
A good question. The answer subjectively depends on the data and project needs. A 5\% accuracy in predicting stoplights is not so good. However, it is \sphinxstyleemphasis{very} good if predicting lottery numbers.
\end{sphinxShadowBox}

\begin{sphinxadmonition}{note}{Note:}
\sphinxAtStartPar
There is \sphinxstylestrong{no} minimal accuracy requirement. At most, evaluators will assess the appropriateness of the metric (or planned metric).
\end{sphinxadmonition}

\sphinxAtStartPar
Measuring accuracy (or a plan to do so) will be discussed in detail in the \DUrole{xref,myst}{Accuracy Analysis} section of part D of your documentation.


\section{FAQ}
\label{\detokenize{task2_c/task2_part_c:faq}}\label{\detokenize{task2_c/task2_part_c:task2c-faq}}

\subsection{What are the most common reasons for task 2 part C (the app/code) being returned?}
\label{\detokenize{task2_c/task2_part_c:what-are-the-most-common-reasons-for-task-2-part-c-the-app-code-being-returned}}\begin{enumerate}
\sphinxsetlistlabels{\arabic}{enumi}{enumii}{}{.}%
\item {} 
\sphinxAtStartPar
Evaluators cannot get the code to run as intended. This usually happens because of an incomplete or incorrect {\hyperref[\detokenize{task2_doc/task2_doc_d:task2-doc-d-user-guide}]{\sphinxcrossref{\DUrole{std,std-ref}{User Guide}}}} or because evaluators can’t access shared links (check the permissions!).

\item {} 
\sphinxAtStartPar
Evaluators are not sure how the code is meant to be used by the “user.” Again, an incomplete or incorrect {\hyperref[\detokenize{task2_doc/task2_doc_d:task2-doc-d-user-guide}]{\sphinxcrossref{\DUrole{std,std-ref}{User Guide}}}} is usually the culprit. Adding an explicit example (including example user files when appropriate) helps avoid this issue.

\end{enumerate}


\subsection{I’ve completed the coding for task 2. Should I send it to my course instructor for review?}
\label{\detokenize{task2_c/task2_part_c:i-ve-completed-the-coding-for-task-2-should-i-send-it-to-my-course-instructor-for-review}}
\sphinxAtStartPar
If you have specific questions or concerns \sphinxhyphen{}yes. However, if the code runs and meets the \DUrole{xref,myst}{minimum app requirements} it’s usually best to move on to the {\hyperref[\detokenize{task2_doc/task2_doc:task2-doc}]{\sphinxcrossref{\DUrole{std,std-ref}{documentation}}}}. You can continue to tweak and improve the app while comfortably knowing what you have will pass. Recall, you have \sphinxstylestrong{unlimited} submissions. So for both the code and documentation, it’s usually best to submit it as soon as it’s ready and restrict revisions according to the evaluator’s feedback.

\sphinxAtStartPar
For getting help with task 2 part C, see the advice below.


\subsection{I need help with part C. Who do I contact?}
\label{\detokenize{task2_c/task2_part_c:i-need-help-with-part-c-who-do-i-contact}}\label{\detokenize{task2_c/task2_part_c:task2-part-c-faq-i-need-help-with-part-c-who-do-i-contact}}
\sphinxAtStartPar
That depends on what you need help with. For questions about the capstone, how to meet the requirements, evaluator comments, or how to best approach the project to meet your individual goals, contact your {\hyperref[\detokenize{ci_c964:ci-c964}]{\sphinxcrossref{\DUrole{std,std-ref}{assigned course instructor}}}} or the \sphinxhref{mailto:ugcapstoneit@wgu.edu?cc=your\%20assigned\%20CI\&subject=C964\%20capstone\%20question}{capstone team inbox} (this inbox supports all IT capstones). However, the capstone team supports all of the IT college capstone projects. As such, your assigned course instructor may not have the technical expertise to answer questions related to computer science or coding. This is particularly so with debugging code, given the wide range of approaches, languages, and libraries available for use.

\sphinxAtStartPar
For technical questions related to code or math, see the {\hyperref[\detokenize{ci_other:ci-other}]{\sphinxcrossref{\DUrole{std,std-ref}{BSCS, Software, and other Course Faculty}}}} page, and follow these {\hyperref[\detokenize{ci_other:ci-other-better-questions-get-better-answers}]{\sphinxcrossref{\DUrole{std,std-ref}{guidelines}}}}. Keep in mind, that while these faculty may be subject matter experts in their field, they do \sphinxstyleemphasis{not} necessarily support the capstone and so may not know the capstone requirements. Hence it is often best to contact your capstone instructor first, so you can appropriately limit the scope of your question(s). When contacting faculty on the {\hyperref[\detokenize{ci_other:ci-other}]{\sphinxcrossref{\DUrole{std,std-ref}{BSCS, Software, and other Course Faculty}}}} page, follow these {\hyperref[\detokenize{ci_other:ci-other-better-questions-get-better-answers}]{\sphinxcrossref{\DUrole{std,std-ref}{guidelines}}}}. Keep in mind, that non\sphinxhyphen{}capstone faculty love to help \sphinxstylestrong{but do so as a generosity.} Their priority is the students struggling in their supporting courses.

\sphinxAtStartPar
Remember, our job (as educators) is to help \sphinxstyleemphasis{you} fix your problem \sphinxhyphen{}not fix it for you.


\subsection{What if I start working on task 2 and want to change things? Do I need to resubmit task 1?}
\label{\detokenize{task2_c/task2_part_c:what-if-i-start-working-on-task-2-and-want-to-change-things-do-i-need-to-resubmit-task-1}}\label{\detokenize{task2_c/task2_part_c:task2-part-c-faq-what-if-i-start-working-on-task-2-and-want-to-change-things}}
\sphinxAtStartPar
No, not unless it’s an entirely different topic. Minor changes from task 1 to task 2 are expected and allowed \sphinxstyleemphasis{without updating the approval form}. Evaluators will not rigorously compare tasks 1 and 2 (if at all). Task 2 is where the work is, and even with complete topic changes, at most, you’ll only be asked to revise the approval form (if at all). So never let task 1 dictate what you do in task 2.


\subsection{How many attempts are allowed for each task?}
\label{\detokenize{task2_c/task2_part_c:how-many-attempts-are-allowed-for-each-task}}
\sphinxAtStartPar
You have \sphinxstyleemphasis{unlimited} attempts for both tasks 1 and 2. However, incomplete submissions or submissions significantly falling short of the minimum requirements may be \sphinxstyleemphasis{locked} from further submissions without instructor approval. Furthermore, such submissions do not receive meaningful evaluator comments.


\subsection{What is a descriptive method?}
\label{\detokenize{task2_c/task2_part_c:what-is-a-descriptive-method}}
\sphinxAtStartPar
Anything that describes the data. Histograms, scatterplots, pie charts \sphinxhyphen{}all the familiar descriptive statistics techniques are included. ML methods such as k\sphinxhyphen{}means clustering can also be descriptive. Whether a method is descriptive or non\sphinxhyphen{}descriptive is determined by its use. For example, using a regression line to describe the relationship between variables is descriptive, but using the line to predict a variable or claim a correlation between the variables exist is inferential (non\sphinxhyphen{}descriptive).

\sphinxAtStartPar
For task 2, you do not need to explicitly identify descriptive and non\sphinxhyphen{}descriptive methods. In almost all cases, the visualizations will satisfy the former and the user interface the latter.


\subsection{What is a non\sphinxhyphen{}descriptive method?}
\label{\detokenize{task2_c/task2_part_c:what-is-a-non-descriptive-method}}
\sphinxAtStartPar
Anything that infers from the data, e.g., making predictions, recommendations, identifying correlations, inferring from correlations, etc. Also, see the comments above.


\subsection{What is machine learning?}
\label{\detokenize{task2_c/task2_part_c:what-is-machine-learning}}\label{\detokenize{task2_c/task2_part_c:task1-faq-what-is-machine-learning}}
\sphinxAtStartPar
That depends on who you ask! But for this project, it is an algorithm applied to data.

\sphinxAtStartPar
For computer science, machine learning is a subfield of artificial intelligence (a subfield of mathematics), broadly defined as the development of machines capable of self\sphinxhyphen{}adjusting behavior based on data. However, from the data science perspective, machine learning is generally defined as using algorithms to identify patterns, make predictions, etc., from data. That is, machine learning is a goal, not a technique. So, for example, a data scientist (and the evaluators) consider linear regression machine learning \sphinxhyphen{}when it’s used as a prediction model. However, a mathematician would politely (or not so politely) disagree with a 19th\sphinxhyphen{}century equation being classified as ML.


\subsection{Can I use libraries outside the standard (Python, Java, etc.) installation?}
\label{\detokenize{task2_c/task2_part_c:can-i-use-libraries-outside-the-standard-python-java-etc-installation}}
\sphinxAtStartPar
Yes! Unlike C950 (Data Structures \& Algorithms II), you are allowed and encouraged to use outside libraries. All the major languages, but particularly Python, have a wide array of highly developed ML tools. The C964 capstone is about applying these tools \sphinxhyphen{}not their development.


\subsection{What language, libraries, and platforms should I use?}
\label{\detokenize{task2_c/task2_part_c:what-language-libraries-and-platforms-should-i-use}}
\sphinxAtStartPar
You can use whatever you like. However, we strongly recommend Python and the \sphinxhref{https://scikit-learn.org/stable/}{scikit\sphinxhyphen{}learn} (aka sklearn) library. In addition to having an extensive collection of ML\sphinxhyphen{}specific tools and tutorials, WGU has better faculty support for these. Jupyter Notebook (a browser\sphinxhyphen{}based IDE designed for this type of project) is a great place to start for the app’s development and front end. Passing applications are often submitted as the Notebook (.ipynb) and data files. Jupyter Notebooks are a great way to present code and information together, but they can also progressively be developed into a more polished product. Students are often tempted to use Jave because of their JavaFX experience in software II, but a GUI is not required, and Python is better suited.

\sphinxAtStartPar
A development path might look like the following:
\begin{quote}

\sphinxAtStartPar
Python IDE → Jupyter Notebook → notebook with widgets → hosted notebook with widgets → web app.
\end{quote}

\sphinxAtStartPar
Provided the \DUrole{xref,myst}{minimal app criteria} are met, submitting at any point along this path will pass part C.


\subsection{What sort of user interface do I need? Do I need a GUI?}
\label{\detokenize{task2_c/task2_part_c:what-sort-of-user-interface-do-i-need-do-i-need-a-gui}}
\sphinxAtStartPar
No, a GUI is \sphinxstyleemphasis{not} required. Your app must be usable by the “client” to solve the proposed problem. If the evaluators can run your app as intended, playing the role of the “client,” following your {\hyperref[\detokenize{task2_doc/task2_doc_d:task2-doc-d-user-guide}]{\sphinxcrossref{\DUrole{std,std-ref}{user guide}}}}, then your app will be considered to have a user\sphinxhyphen{}friendly interface. This can be done through a GUI and widgets, but using the command line or reading user data from a local directory will also suffice.




\subsection{My project exceeds the 200 MB limit. How can I submit it?}
\label{\detokenize{task2_c/task2_part_c:my-project-exceeds-the-200-mb-limit-how-can-i-submit-it}}\label{\detokenize{task2_c/task2_part_c:task2-part-c-faq-my-project-exceeds-the-200-mb-limit}}
\sphinxAtStartPar
This is covered in more detail {\hyperref[\detokenize{task2_doc/task2_doc_finish:task2-doc-finish-how-to-submit-code}]{\sphinxcrossref{\DUrole{std,std-ref}{here}}}}. As you might guess, this is a common issue. Many models and datasets will exceed the 200 MB upload limit. Evaluators need \sphinxstyleemphasis{access} to everything necessary to develop and run your project, and those files cannot be modifiable after being submitted.

\sphinxAtStartPar
For webpages or hosted Jupyter notebooks, submit the url link (provide the link and instructions for your app in your user guide of part D) but also submit the html or .ipynb files. For large models, provide instructions and code to locally build the model and how access to the data as described as above. If build times are long, you might provide a link to the completed model as a courtesy to the evaluator.

\sphinxAtStartPar
Always submit the data directly if it is less than 200 MB. For larger data sets not modifiable by the student (e.g., data from \sphinxhref{http://Kaggle.com}{Kaggle.com}, \sphinxhref{http://Data.gov}{Data.gov}, sklearn, etc.), it is acceptable to submit a link for the data source or import the data directly with your code. For large data sets modifiable by the student (say in a GitHub repo), then you should provide a subset of the data not exceeding the 200 MB limit. The project’s documentation and performance assessment should be based on the full data.

\sphinxAtStartPar
Never use Google Drive, as WGU policy forbids WGU employees from using it. Use of Google Colab is acceptable, but upload the source code as directed above.




\subsection{I only have a Linux (or Mac) machine. Will evaluators be able to run my code?}
\label{\detokenize{task2_c/task2_part_c:i-only-have-a-linux-or-mac-machine-will-evaluators-be-able-to-run-my-code}}\label{\detokenize{task2_c/task2_part_c:task2-part-c-faq-i-only-have-a-linux-or-mac-machine}}
\sphinxAtStartPar
Technically (and unfortunately), we are a “Windows” university, and all submitted projects should be able to run in Windows. However, being Windows\sphinxhyphen{}compatible is \sphinxstyleemphasis{nowhere specifically required} in the C964 rubric, and doing so would be a little silly for a computer science program. That said, WGU evaluators are only issued Windows 10 machines, and they may have difficulty running a Linux or Mac app without special instructions. Therefore, we recommend that your {\hyperref[\detokenize{task2_doc/task2_doc_d:task2-doc-d-user-guide}]{\sphinxcrossref{\DUrole{std,std-ref}{user guide}}}} provide explicit instructions for a Windows 10 user to run your code, such as using a \sphinxhref{https://ubuntu.com/tutorials/how-to-run-ubuntu-desktop-on-a-virtual-machine-using-virtualbox\#1-overview}{virtual machine}, a remote machine, or using a \sphinxhref{https://ubuntu.com/tutorials/install-ubuntu-on-wsl2-on-windows-10\#1-overview}{Linux subsystem}. Provide a note when submitting to Assessments and alert your course instructor.


\subsection{How complex does my data, algorithm, or model need to be?}
\label{\detokenize{task2_c/task2_part_c:how-complex-does-my-data-algorithm-or-model-need-to-be}}
\sphinxAtStartPar
It must be complex enough to meet the needs of your project. There is no explicit minimal complexity for any of these items. However, the model must meet the needs of the “organizational need” and the data must be appropriate for developing the model which could indirectly impose a minimal complexity. For example, a supervised model requires two variables.


\subsection{Are there any restrictions on which datasets I can choose?}
\label{\detokenize{task2_c/task2_part_c:are-there-any-restrictions-on-which-datasets-i-can-choose}}
\sphinxAtStartPar
Only that data must be legally available to use and share with evaluators. For example, using data belonging to a current employer would require submitting a \DUrole{xref,myst}{waiver form}.
\begin{itemize}
\item {} 
\sphinxAtStartPar
You \sphinxstyleemphasis{can} use any dataset found on \sphinxhref{https://www.kaggle.com/datasets}{kaggle.com}.

\item {} 
\sphinxAtStartPar
You \sphinxstyleemphasis{can} use simulated data.

\item {} 
\sphinxAtStartPar
You \sphinxstyleemphasis{can} use data used for previous projects (submitted by you or others).

\item {} 
\sphinxAtStartPar
You only need to apply for \sphinxhref{https://cm.wgu.edu/t5/Frequently-Asked-Questions/WGU-IRB-and-Human-Subject-Protections-FAQ/ta-p/2002}{IRB review} if you are \sphinxstyleemphasis{collecting} data involving human participants (this is very rare). Otherwise, your project is in IRB compliance.

\end{itemize}


\subsection{Can I use my C950 project for C964?}
\label{\detokenize{task2_c/task2_part_c:can-i-use-my-c950-project-for-c964}}
\sphinxAtStartPar
Yes. You can use any of your own academic or professional work for C964 including the C950 project (Data Structures \& Algorithms II). Though the document (Task 2 parts A, B, and D) will need some adjustment, the coding portion of C950 almost meets all the requirements of the C964 application (Task 2 part C) \sphinxhyphen{}it only lacks visualizations. Referring to the \sphinxhref{https://ashejim.github.io/C964/task2\_c/task2\_part\_c.html\#what-does-the-application-need-to-do}{Task 2 part C page}, the C964 application needs the following:
\begin{enumerate}
\sphinxsetlistlabels{\arabic}{enumi}{enumii}{}{.}%
\item {} 
\sphinxAtStartPar
\sphinxstylestrong{Data → ML model:} C950 applies a reinforced learning algorithm to the distance and package data.

\item {} 
\sphinxAtStartPar
\sphinxstylestrong{Accuracy Metric:} The total miles. The maximum allowed miles for C950 is 120, which WGU Assessment Department has already determined to be “efficient.”

\item {} 
\sphinxAtStartPar
\sphinxstylestrong{Visualizations:} This will need to be added, but any three pictures will meet the requirements.

\item {} 
\sphinxAtStartPar
\sphinxstylestrong{User Application:} The console user interface required for C950 allows the user to provide input and apply the algorithm toward solving the problem.

\end{enumerate}

\sphinxAtStartPar
So only the three images will need to be added. Furthermore, you are free to adjust the distance and package data as desired. For example, dropping some of the delivery restrictions requiring different trucks or certain packages to be delivered together, and will be easier to apply a more sophisticated algorithm. Say a \sphinxhref{https://en.wikipedia.org/wiki/Monte\_Carlo\_method}{Monte Carlo method} such as \sphinxhref{https://en.wikipedia.org/wiki/Simulated\_annealing}{Simulated Annealing}.



\sphinxAtStartPar
{[}\hyperlink{cite.resources:id2}{Wik18}{]}


\subsection{Can I use my C951 task 1 or 2 for C964?}
\label{\detokenize{task2_c/task2_part_c:can-i-use-my-c951-task-1-or-2-for-c964}}
\sphinxAtStartPar
Yes. You can use any of your own academic or professional work for C964 including the C950 project (Data Structures \& Algorithms II). However, C951 Tasks 1 (chatbot) and 2 (rescue robot) can be passed without applying a mathematically based algorithm which is necessary to pass C964 (no C951 task requires AI). While chatbots are often trained and improved with ML/AI methods, it’s difficult to acquire sufficient user data within students’ typical timeframe. An algorithm is easier to employ in C951 Task 2, but the application would be limited to the \sphinxhref{https://www.coppeliarobotics.com/coppeliaSim}{CoppeliaSim} simulation software which may make meeting the user application requirements of C964 difficult. Furthermore, \sphinxhref{https://www.coppeliarobotics.com/coppeliaSim}{CoppeliaSim}’s dependency on Lua makes the application of ML libraries problematic. For these reasons and others, we do not recommend using either of these projects for C964 \sphinxhyphen{}unless the C951 projects far exceed their requirements and meet the C964 requirements as is.


\subsection{Can I use my C951 task 3? Should I use it?}
\label{\detokenize{task2_c/task2_part_c:can-i-use-my-c951-task-3-should-i-use-it}}
\sphinxAtStartPar
You can reuse anything you’ve of your own academic or professional work, including copying verbatim from C951 task 3. If it’s convenient, feel free to do it. But at best, the time saved is little. At worst, you might get bogged down trying to work on two projects simultaneously and going with an unnecessarily complex C964 topic. If you have time, consider completing C964 first, as parts A and B of task 2 can always be used verbatim for task 3 of C951.

\sphinxAtStartPar
Here are some points to consider:
\begin{itemize}
\item {} 
\sphinxAtStartPar
C951.3 is just a written project, typically around five pages (I’m guessing; ask your C951 instructor), and can be completed in a single afternoon. Comparatively, C964 requires a working machine learning application and accompanying documentation, typically around 20 pages.

\item {} 
\sphinxAtStartPar
C951.3 only relates to parts A and B of C964.2. These parts are just a framework for providing a general audience and purpose for the ML application. If present, these parts almost always pass. Furthermore, they’ll have to be at least partially rewritten anyways. Parts C and D of C964 are what evaluators care about, but C951.3 has no corresponding parts C and D.

\item {} 
\sphinxAtStartPar
Rewriting C951.3 content for a different C964 topic takes little additional work.

\item {} 
\sphinxAtStartPar
As it’s just a written project, students often pick a complex topic for C951.3. But then they feel pressured to use the same complex topic for C964 and struggle with creating the app.

\item {} 
\sphinxAtStartPar
Trying to comprehend two projects at once is just more difficult.

\end{itemize}

\sphinxAtStartPar
Whatever you do for C964 can meet the requirements of C951 task 3. If you have plenty of time, completing C964 first might be the best option.


\subsection{The rubric is vague and refers to the task directions which seem to require many redundant or unnecessary items. What do I actually need to do?}
\label{\detokenize{task2_c/task2_part_c:the-rubric-is-vague-and-refers-to-the-task-directions-which-seem-to-require-many-redundant-or-unnecessary-items-what-do-i-actually-need-to-do}}
\sphinxAtStartPar
Your C964 course instructor team has collaborated with evaluators (thank you evaluator team!) to ensure the explanations on this webpage align with how the assessment’s requirements are interpreted by the evaluators. So if you find the official directions unclear, we advise following the directions on this webpage for both the application (part C) and the documentation (parts A, B, and D).

\sphinxAtStartPar
The official rubric and directions were written to map the project’s elements to specific competencies, and following the official directions will meet all the requirements. However, be aware that many items are redundant or inherently met by other items. For example, in part C:
\begin{itemize}
\item {} 
\sphinxAtStartPar
The descriptive and nondescriptive requirement is met by the visualization and decision support functionality respectively.

\item {} 
\sphinxAtStartPar
Visualization functionality and monitoring tools are inherently part of tools used to create visualizations and code.

\item {} 
\sphinxAtStartPar
As any ML method is an algorithm, the requirements to implement both is redundant.

\item {} 
\sphinxAtStartPar
Etc., etc.

\end{itemize}

\sphinxAtStartPar
Furthermore, some terminology is open to interpretation and nowhere rigorously defined. The official directions potentially have similar issues. So following the official directions without further guidance could result in overworking some requirements or misinterpreting others. For a more succinct outline of the Task 2 requirements see:
\begin{itemize}
\item {} 
\sphinxAtStartPar
\sphinxhref{https://ashejim.github.io/C964/task2\_c/task2\_part\_c.html\#what-does-the-application-need-to-do}{\sphinxstylestrong{Part C requirements}}

\item {} 
\sphinxAtStartPar
\sphinxhref{https://ashejim.github.io/C964/task2\_doc/task2\_doc.html\#task-2-the-documentation}{\sphinxstylestrong{Parts A, B, and D requirements}}

\end{itemize}

\sphinxAtStartPar
For the documentation, preserve the template’s section titles, and order, and submit all four parts as a single document (preferably a pdf). With a long, complicated document, aligning content to competencies presents a challenge. Don’t make things difficult for the evaluator by spreading the content over several documents in an unfamiliar format.

\sphinxAtStartPar
If anything needs further explanation, please ask us! \sphinxhref{https://ashejim.github.io/C964/ci\_c964.html\#c964-course-faculty}{Contact your C964 course instructor}.


\subsection{The official learning resource seems to include documentation items not included on this webpage. Which should I follow?}
\label{\detokenize{task2_c/task2_part_c:the-official-learning-resource-seems-to-include-documentation-items-not-included-on-this-webpage-which-should-i-follow}}
\sphinxAtStartPar
Either will meet the requirements. However, the \sphinxhref{https://ashejim.github.io/C964/task2\_doc/task2\_doc.html\#task-2-the-documentation}{template on this webpage}, is more succinct and was developed in collaboration with the capstone evaluators (thank you evaluator team!) to specifically align with versions SIM3 and SIM2. Hence, you can be ensured that following this website’s template will meet all the requirements.

\sphinxAtStartPar
The content for the official learning resource (LR) was largely copied from an older version of this website which aligned to the \sphinxhref{https://westerngovernorsuniversity-my.sharepoint.com/:w:/g/personal/jim\_ashe\_wgu\_edu/ERGxhsNfkbhEutlkXVFITMQBPOmWlkVx1p5H0UisvnBesg}{previous template} written for SIM2. When C964 was updated to SIM3, we updated this website and the template accordingly. Hence, the discrepancy. As the actual requirements for SIM2 and SIM3 are the same, following the official LR or this website should be fine.


\subsection{Help! I’ve never coded a machine learning project. For C951 task 3, I only had to write about Machine Learning. Where do I learn this?}
\label{\detokenize{task2_c/task2_part_c:help-i-ve-never-coded-a-machine-learning-project-for-c951-task-3-i-only-had-to-write-about-machine-learning-where-do-i-learn-this}}
\sphinxAtStartPar
WGU provides access to a very good \sphinxhref{https://lrps.wgu.edu/provision/386121824}{AI textbook which includes a Machine Learning} section. However, it is conceptually focused and includes very little application or practical examples. Furthermore, reading this text might require mathematics not provided in WGU’s BSCS curriculum.

\sphinxAtStartPar
If you have time, Udemy offers some ML courses{]}(\sphinxurl{https://wgu.udemy.com/course/machinelearning/learn/lecture/14473662\#overview}). Maybe the fastest way to get started is with the \sphinxhref{https://ashejim.github.io/C964/task2\_c/example\_sup\_class/sup\_class\_ex.html}{video and examples included on this website}. Though a minimally passing C950 project (applying a greedy algorithm to hand\sphinxhyphen{}picked truckloads) would not be consider ML by many, it meets the criteria for this project as it is an algorithm applied to data.


\section{Questions, comments, or suggestions?}
\label{\detokenize{task2_c/task2_part_c:questions-comments-or-suggestions}}


\sphinxstepscope



\sphinxAtStartPar
Running Python code requires a running Python kernel. Click the  –> \sphinxguilabel{Live Code} button above on this page to run the code below.

\begin{sphinxadmonition}{warning}{Warning:}
\sphinxAtStartPar
🚧 This site is under construction! As of now, the Python kernel may not run on the page or have very long wait times. Also, expect typos.👷🏽‍♀️
\end{sphinxadmonition}


\section{Example: Supervised Classification App}
\label{\detokenize{task2_c/example_sup_class/sup_class_ex:example-supervised-classification-app}}\label{\detokenize{task2_c/example_sup_class/sup_class_ex:sup-class-ex}}\label{\detokenize{task2_c/example_sup_class/sup_class_ex::doc}}
\sphinxAtStartPar
A supervised classification method fits the project requirements well and is so a good place to start. The nature of your Data and organizational needs dictate which methods you can use. So what type of data works with supervised classification methods?
\begin{itemize}
\item {} 
\sphinxAtStartPar
One of the features (columns) contains mutually exclusive \sphinxstyleemphasis{categories} you want to predict (the dependent variable).

\item {} 
\sphinxAtStartPar
At least one other feature (the independent variable(s)).

\end{itemize}

\begin{sphinxShadowBox}
\sphinxstylesidebartitle{}

\sphinxAtStartPar
Classifying non\sphinxhyphen{}mutually exclusive categories is called \sphinxstyleemphasis{multi\sphinxhyphen{}label} or \sphinxstyleemphasis{mult\sphinxhyphen{}output} classification. Not to be confused with \sphinxstyleemphasis{multiclass} classification presented in this example, multi\sphinxhyphen{}label classification requires different techniques, particularly with measuring accuracy. See \sphinxhref{https://www.geeksforgeeks.org/an-introduction-to-multilabel-classification/}{Introduction to Multi\sphinxhyphen{}label Classification} for more information.
\end{sphinxShadowBox}

\sphinxAtStartPar
This will be a simple example. Simple data. Simple model. Simple interface. However, it does demonstrate the minimum requirements for \DUrole{xref,myst}{part C}. We’ll also show how things can progressively be improved, building on the \sphinxstyleemphasis{working} code. Simple is a great place to start \sphinxhyphen{}scaling up is typically easier than going in the other direction.



\sphinxAtStartPar
Let’s look at the famous \sphinxhref{https://en.wikipedia.org/wiki/Iris\_flower\_data\_set}{Fisher’s Iris data set}:

\begin{sphinxuseclass}{cell}
\begin{sphinxuseclass}{tag_hide-input}\begin{sphinxVerbatimOutput}

\begin{sphinxuseclass}{cell_output}
\begin{sphinxVerbatim}[commandchars=\\\{\}]
     sepal\PYGZhy{}length  sepal\PYGZhy{}width  petal\PYGZhy{}length  petal\PYGZhy{}width            type
0             5.1          3.5           1.4          0.2     Iris\PYGZhy{}setosa
1             4.9          3.0           1.4          0.2     Iris\PYGZhy{}setosa
2             4.7          3.2           1.3          0.2     Iris\PYGZhy{}setosa
3             4.6          3.1           1.5          0.2     Iris\PYGZhy{}setosa
4             5.0          3.6           1.4          0.2     Iris\PYGZhy{}setosa
..            ...          ...           ...          ...             ...
145           6.7          3.0           5.2          2.3  Iris\PYGZhy{}virginica
146           6.3          2.5           5.0          1.9  Iris\PYGZhy{}virginica
147           6.5          3.0           5.2          2.0  Iris\PYGZhy{}virginica
148           6.2          3.4           5.4          2.3  Iris\PYGZhy{}virginica
149           5.9          3.0           5.1          1.8  Iris\PYGZhy{}virginica
\end{sphinxVerbatim}

\end{sphinxuseclass}\end{sphinxVerbatimOutput}

\end{sphinxuseclass}
\end{sphinxuseclass}
\sphinxAtStartPar
Though we described everything as “simple,” we’ll also see that this dataset is quite \sphinxstyleemphasis{rich} with angles to investigate. At this point, we have many options, but for a classification project we need a categorical feature as our dependent variable, and for this, we only have the choice: \sphinxstylestrong{type}.

\begin{sphinxuseclass}{cell}
\begin{sphinxuseclass}{tag_hide-input}\begin{sphinxVerbatimOutput}

\begin{sphinxuseclass}{cell_output}
\begin{sphinxVerbatim}[commandchars=\\\{\}]
\PYGZlt{}pandas.io.formats.style.Styler at 0x2b9dfceae90\PYGZgt{}
\end{sphinxVerbatim}

\end{sphinxuseclass}\end{sphinxVerbatimOutput}

\end{sphinxuseclass}
\end{sphinxuseclass}
\begin{sphinxShadowBox}
\sphinxstylesidebartitle{Watch}


\end{sphinxShadowBox}

\sphinxAtStartPar
The highlighted column, \sphinxstylestrong{type} provides a category to predict/classify (dependent variables), and the non\sphinxhyphen{}highlighted columns are something by which to make that prediction/classification (independent variables).

\sphinxstepscope


\subsection{Data Exploring and Processing}
\label{\detokenize{task2_c/example_sup_class/sup_class_ex-process:data-exploring-and-processing}}\label{\detokenize{task2_c/example_sup_class/sup_class_ex-process:sup-class-ex-data}}\label{\detokenize{task2_c/example_sup_class/sup_class_ex-process::doc}}
\sphinxAtStartPar
Here we’ll do two essential things.
\begin{enumerate}
\sphinxsetlistlabels{\arabic}{enumi}{enumii}{}{.}%
\item {} 
\sphinxAtStartPar
Process the data.

\item {} 
\sphinxAtStartPar
Analyze the data.

\end{enumerate}

\sphinxAtStartPar
Processing will involve importing, cleaning, and sorting raw data; preparing it to be analyzed. Exploring the data builds an understanding of it and the problem you’re trying to solve, equipping you to develop a machine\sphinxhyphen{}learning application.

\sphinxAtStartPar
Before doing anything with raw data, you must import it.

\begin{sphinxuseclass}{cell}\begin{sphinxVerbatimInput}

\begin{sphinxuseclass}{cell_input}
\begin{sphinxVerbatim}[commandchars=\\\{\}]
\PYG{c+c1}{\PYGZsh{} We\PYGZsq{}ll import libraries as needed, but when submitting, }
\PYG{c+c1}{\PYGZsh{} it\PYGZsq{}s best to have them all at the top.}
\PYG{k+kn}{import} \PYG{n+nn}{pandas} \PYG{k}{as} \PYG{n+nn}{pd}

\PYG{c+c1}{\PYGZsh{} Load this well\PYGZhy{}worn dataset:}
\PYG{n}{url} \PYG{o}{=} \PYG{l+s+s2}{\PYGZdq{}}\PYG{l+s+s2}{https://raw.githubusercontent.com/jbrownlee/Datasets/master/iris.csv}\PYG{l+s+s2}{\PYGZdq{}}
\PYG{n}{df} \PYG{o}{=} \PYG{n}{pd}\PYG{o}{.}\PYG{n}{read\PYGZus{}csv}\PYG{p}{(}\PYG{n}{url}\PYG{p}{)} \PYG{c+c1}{\PYGZsh{}read CSV into Python as a DataFrame}
\PYG{n}{df} \PYG{c+c1}{\PYGZsh{} displays the DataFrame}
\end{sphinxVerbatim}

\end{sphinxuseclass}\end{sphinxVerbatimInput}
\begin{sphinxVerbatimOutput}

\begin{sphinxuseclass}{cell_output}
\begin{sphinxVerbatim}[commandchars=\\\{\}]
     5.1  3.5  1.4  0.2     Iris\PYGZhy{}setosa
0    4.9  3.0  1.4  0.2     Iris\PYGZhy{}setosa
1    4.7  3.2  1.3  0.2     Iris\PYGZhy{}setosa
2    4.6  3.1  1.5  0.2     Iris\PYGZhy{}setosa
3    5.0  3.6  1.4  0.2     Iris\PYGZhy{}setosa
4    5.4  3.9  1.7  0.4     Iris\PYGZhy{}setosa
..   ...  ...  ...  ...             ...
144  6.7  3.0  5.2  2.3  Iris\PYGZhy{}virginica
145  6.3  2.5  5.0  1.9  Iris\PYGZhy{}virginica
146  6.5  3.0  5.2  2.0  Iris\PYGZhy{}virginica
147  6.2  3.4  5.4  2.3  Iris\PYGZhy{}virginica
148  5.9  3.0  5.1  1.8  Iris\PYGZhy{}virginica

[149 rows x 5 columns]
\end{sphinxVerbatim}

\end{sphinxuseclass}\end{sphinxVerbatimOutput}

\end{sphinxuseclass}
\sphinxAtStartPar
\sphinxstyleemphasis{Oops.}* The first row of data has been set as headers. Some data sets have headers already \sphinxhyphen{}this one doesn’t. How do we fix that? Google \sphinxhref{https://www.google.com/search?q=how+to+python+add+headers+to+dataframe\&rlz=1C1GCEA\_enUS995US997\&ei=7TuSY7TmGsyJggfflr3YCg\&ved=0ahUKEwj0kK\_X4er7AhXMhOAKHV9LD6sQ4dUDCA8\&uact=5\&oq=how+to+python+add+headers+to+dataframe\&gs\_lcp=Cgxnd3Mtd2l6LXNlcnAQAzIGCAAQCBAeMgUIABCGAzIFCAAQhgMyBQgAEIYDMgUIABCGAzoKCAAQRxDWBBCwAzoHCAAQgAQQDToICAAQCBAeEA06CAgAEAgQBxAeSgQIQRgASgQIRhgAULgCWMkIYMEfaAFwAXgAgAFPiAH1A5IBATeYAQCgAQHIAQjAAQE\&sclient=gws-wiz-serp}{“how to python add headers to dataframe”}. You’ll need to learn a lot of micro\sphinxhyphen{}skills \sphinxhyphen{}pick them up when needed. Reading up on the data set, \sphinxhref{https://en.wikipedia.org/wiki/Iris\_flower\_data\_set}{Iris flower data set}, we name the columns:

\begin{sphinxuseclass}{cell}\begin{sphinxVerbatimInput}

\begin{sphinxuseclass}{cell_input}
\begin{sphinxVerbatim}[commandchars=\\\{\}]
\PYG{n}{column\PYGZus{}names} \PYG{o}{=} \PYG{p}{[}\PYG{l+s+s1}{\PYGZsq{}}\PYG{l+s+s1}{sepal\PYGZhy{}length}\PYG{l+s+s1}{\PYGZsq{}}\PYG{p}{,} \PYG{l+s+s1}{\PYGZsq{}}\PYG{l+s+s1}{sepal\PYGZhy{}width}\PYG{l+s+s1}{\PYGZsq{}}\PYG{p}{,} \PYG{l+s+s1}{\PYGZsq{}}\PYG{l+s+s1}{petal\PYGZhy{}length}\PYG{l+s+s1}{\PYGZsq{}}\PYG{p}{,} \PYG{l+s+s1}{\PYGZsq{}}\PYG{l+s+s1}{petal\PYGZhy{}width}\PYG{l+s+s1}{\PYGZsq{}}\PYG{p}{,} \PYG{l+s+s1}{\PYGZsq{}}\PYG{l+s+s1}{type}\PYG{l+s+s1}{\PYGZsq{}}\PYG{p}{]}
\PYG{n}{df} \PYG{o}{=} \PYG{n}{pd}\PYG{o}{.}\PYG{n}{read\PYGZus{}csv}\PYG{p}{(}\PYG{n}{url}\PYG{p}{,} \PYG{n}{names} \PYG{o}{=} \PYG{n}{column\PYGZus{}names}\PYG{p}{)} \PYG{c+c1}{\PYGZsh{}read CSV into Python as a DataFrame}
\PYG{n}{df} \PYG{c+c1}{\PYGZsh{} displays the DataFrame}
\end{sphinxVerbatim}

\end{sphinxuseclass}\end{sphinxVerbatimInput}
\begin{sphinxVerbatimOutput}

\begin{sphinxuseclass}{cell_output}
\begin{sphinxVerbatim}[commandchars=\\\{\}]
     sepal\PYGZhy{}length  sepal\PYGZhy{}width  petal\PYGZhy{}length  petal\PYGZhy{}width            type
0             5.1          3.5           1.4          0.2     Iris\PYGZhy{}setosa
1             4.9          3.0           1.4          0.2     Iris\PYGZhy{}setosa
2             4.7          3.2           1.3          0.2     Iris\PYGZhy{}setosa
3             4.6          3.1           1.5          0.2     Iris\PYGZhy{}setosa
4             5.0          3.6           1.4          0.2     Iris\PYGZhy{}setosa
..            ...          ...           ...          ...             ...
145           6.7          3.0           5.2          2.3  Iris\PYGZhy{}virginica
146           6.3          2.5           5.0          1.9  Iris\PYGZhy{}virginica
147           6.5          3.0           5.2          2.0  Iris\PYGZhy{}virginica
148           6.2          3.4           5.4          2.3  Iris\PYGZhy{}virginica
149           5.9          3.0           5.1          1.8  Iris\PYGZhy{}virginica

[150 rows x 5 columns]
\end{sphinxVerbatim}

\end{sphinxuseclass}\end{sphinxVerbatimOutput}

\end{sphinxuseclass}
\sphinxAtStartPar
Supervised methods use “answers” in the data to supervise the model. Does our data contain the “answers”? If we want to predict the Iris ‘type,’ then yes. It’s the only categorical feature, so we’ll go with it. However, a supervised method could predict \sphinxstyleemphasis{any} of the features, e.g., ‘sepal\sphinxhyphen{}length.’ A supervised method can’t predict what it doesn’t have, say, plant height or petal color.

\begin{sphinxShadowBox}
\sphinxstylesidebartitle{}

\sphinxAtStartPar
What? That’s it for data processing? We didn’t do much becasue the data didn’t need much. Like a lot of data out there, it was (mostly) ready to go. No minimal processing is required. The project’s needs determine the required data processing, i.e., if it works, you’ve done enough. See {\hyperref[\detokenize{task2_c/task2_part_c:task2c-data-requirements}]{\sphinxcrossref{\DUrole{std,std-ref}{Data Requirements}}}}.
\end{sphinxShadowBox}

\sphinxAtStartPar
That’s all the processing needed for now.


\subsubsection{Descriptive Methods and Visualizations}
\label{\detokenize{task2_c/example_sup_class/sup_class_ex-process:descriptive-methods-and-visualizations}}\label{\detokenize{task2_c/example_sup_class/sup_class_ex-process:sup-class-ex-descriptive-methods-and-visualizations}}
\sphinxAtStartPar
Let’s explore the data. A good starting question: how many different Iris categories do we have?

\begin{sphinxuseclass}{cell}\begin{sphinxVerbatimInput}

\begin{sphinxuseclass}{cell_input}
\begin{sphinxVerbatim}[commandchars=\\\{\}]
\PYG{n}{num\PYGZus{}types} \PYG{o}{=} \PYG{n}{df}\PYG{o}{.}\PYG{n}{groupby}\PYG{p}{(}\PYG{n}{by}\PYG{o}{=}\PYG{l+s+s1}{\PYGZsq{}}\PYG{l+s+s1}{type}\PYG{l+s+s1}{\PYGZsq{}}\PYG{p}{)}\PYG{o}{.}\PYG{n}{size}\PYG{p}{(}\PYG{p}{)}\PYG{p}{;}
\PYG{n}{display}\PYG{p}{(}\PYG{n}{num\PYGZus{}types}\PYG{p}{)}\PYG{p}{;} 
\end{sphinxVerbatim}

\end{sphinxuseclass}\end{sphinxVerbatimInput}
\begin{sphinxVerbatimOutput}

\begin{sphinxuseclass}{cell_output}
\begin{sphinxVerbatim}[commandchars=\\\{\}]
type
Iris\PYGZhy{}setosa        50
Iris\PYGZhy{}versicolor    50
Iris\PYGZhy{}virginica     50
dtype: int64
\end{sphinxVerbatim}

\end{sphinxuseclass}\end{sphinxVerbatimOutput}

\end{sphinxuseclass}\phantomsection\label{\detokenize{task2_c/example_sup_class/sup_class_ex-process:sup-class-ex-descriptive-visuals}}
\sphinxAtStartPar
Let’s \sphinxstyleemphasis{visualize} that with a \sphinxstylestrong{bar plot}.



\begin{sphinxuseclass}{cell}
\begin{sphinxuseclass}{tag_remove-output}\begin{sphinxVerbatimInput}

\begin{sphinxuseclass}{cell_input}
\begin{sphinxVerbatim}[commandchars=\\\{\}]
\PYG{c+c1}{\PYGZsh{}Using Pandas plot.bar }
\PYG{n}{plot} \PYG{o}{=} \PYG{n}{num\PYGZus{}types}\PYG{o}{.}\PYG{n}{plot}\PYG{o}{.}\PYG{n}{bar}\PYG{p}{(}\PYG{n}{color}\PYG{o}{=}\PYG{p}{[}\PYG{l+s+s1}{\PYGZsq{}}\PYG{l+s+s1}{red}\PYG{l+s+s1}{\PYGZsq{}}\PYG{p}{,}\PYG{l+s+s1}{\PYGZsq{}}\PYG{l+s+s1}{blue}\PYG{l+s+s1}{\PYGZsq{}}\PYG{p}{,}\PYG{l+s+s1}{\PYGZsq{}}\PYG{l+s+s1}{green}\PYG{l+s+s1}{\PYGZsq{}}\PYG{p}{]}\PYG{p}{,}\PYG{n}{rot}\PYG{o}{=}\PYG{l+m+mi}{0}\PYG{p}{)}
\end{sphinxVerbatim}

\end{sphinxuseclass}\end{sphinxVerbatimInput}

\end{sphinxuseclass}
\end{sphinxuseclass}
\begin{sphinxuseclass}{cell}
\begin{sphinxuseclass}{tag_remove-input}\begin{sphinxVerbatimOutput}

\begin{sphinxuseclass}{cell_output}
\begin{sphinxVerbatim}[commandchars=\\\{\}]
\PYGZlt{}IPython.core.display.HTML object\PYGZgt{}
\end{sphinxVerbatim}

\end{sphinxuseclass}\end{sphinxVerbatimOutput}

\end{sphinxuseclass}
\end{sphinxuseclass}
\begin{sphinxShadowBox}
\sphinxstylesidebartitle{}

\sphinxAtStartPar
Want to do something similar but different? Go to the \sphinxhref{https://pandas.pydata.org/docs/reference/api/pandas.DataFrame.plot.bar.html}{libary’s docs}. You’ll see many libraries and functions each with lots of options. Don’t just copy, paste, and pray. Read the docs and understand the parameters.
\end{sphinxShadowBox}

\sphinxAtStartPar
Three evenly distributed categories. What about the distribution of the petal widths? As with most things in nature, we might expect it to be somewhat normal.

\phantomsection\label{\detokenize{task2_c/example_sup_class/sup_class_ex-process:sup-class-ex-descriptive-visuals-histograms}}
\begin{sphinxuseclass}{cell}
\begin{sphinxuseclass}{tag_remove-output}\begin{sphinxVerbatimInput}

\begin{sphinxuseclass}{cell_input}
\begin{sphinxVerbatim}[commandchars=\\\{\}]
\PYG{n}{hist\PYGZus{}petal\PYGZus{}lengths} \PYG{o}{=} \PYG{n}{df}\PYG{p}{[}\PYG{l+s+s1}{\PYGZsq{}}\PYG{l+s+s1}{petal\PYGZhy{}length}\PYG{l+s+s1}{\PYGZsq{}}\PYG{p}{]}\PYG{o}{.}\PYG{n}{hist}\PYG{p}{(}\PYG{n}{grid} \PYG{o}{=} \PYG{k+kc}{False}\PYG{p}{,}\PYG{n}{bins}\PYG{o}{=}\PYG{l+m+mi}{10}\PYG{p}{,} \PYG{n}{legend} \PYG{o}{=} \PYG{k+kc}{True}\PYG{p}{)}
\end{sphinxVerbatim}

\end{sphinxuseclass}\end{sphinxVerbatimInput}

\end{sphinxuseclass}
\end{sphinxuseclass}
\begin{sphinxuseclass}{cell}
\begin{sphinxuseclass}{tag_remove-input}\begin{sphinxVerbatimOutput}

\begin{sphinxuseclass}{cell_output}
\begin{sphinxVerbatim}[commandchars=\\\{\}]
\PYGZlt{}IPython.core.display.HTML object\PYGZgt{}
\end{sphinxVerbatim}

\end{sphinxuseclass}\end{sphinxVerbatimOutput}

\end{sphinxuseclass}
\end{sphinxuseclass}
\sphinxAtStartPar
Not so normal. However, we are looking at the petal lengths of all three types. So let’s look at a single type.

\begin{sphinxuseclass}{cell}
\begin{sphinxuseclass}{tag_remove-output}\begin{sphinxVerbatimInput}

\begin{sphinxuseclass}{cell_input}
\begin{sphinxVerbatim}[commandchars=\\\{\}]
\PYG{n}{df\PYGZus{}typeA} \PYG{o}{=} \PYG{n}{df}\PYG{p}{[}\PYG{n}{df}\PYG{p}{[}\PYG{l+s+s1}{\PYGZsq{}}\PYG{l+s+s1}{type}\PYG{l+s+s1}{\PYGZsq{}}\PYG{p}{]} \PYG{o}{==} \PYG{l+s+s1}{\PYGZsq{}}\PYG{l+s+s1}{Iris\PYGZhy{}setosa}\PYG{l+s+s1}{\PYGZsq{}}\PYG{p}{]}
\PYG{n}{df\PYGZus{}typeA}\PYG{p}{[}\PYG{l+s+s1}{\PYGZsq{}}\PYG{l+s+s1}{petal\PYGZhy{}length}\PYG{l+s+s1}{\PYGZsq{}}\PYG{p}{]}\PYG{o}{.}\PYG{n}{hist}\PYG{p}{(}\PYG{n}{grid} \PYG{o}{=} \PYG{k+kc}{False}\PYG{p}{,} \PYG{n}{color} \PYG{o}{=} \PYG{l+s+s1}{\PYGZsq{}}\PYG{l+s+s1}{red}\PYG{l+s+s1}{\PYGZsq{}}\PYG{p}{)}\PYG{p}{;}
\end{sphinxVerbatim}

\end{sphinxuseclass}\end{sphinxVerbatimInput}

\end{sphinxuseclass}
\end{sphinxuseclass}
\begin{sphinxuseclass}{cell}\begin{sphinxVerbatimInput}

\begin{sphinxuseclass}{cell_input}
\begin{sphinxVerbatim}[commandchars=\\\{\}]
\PYG{n}{df\PYGZus{}typeA}\PYG{p}{[}\PYG{l+s+s1}{\PYGZsq{}}\PYG{l+s+s1}{petal\PYGZhy{}length}\PYG{l+s+s1}{\PYGZsq{}}\PYG{p}{]}\PYG{o}{.}\PYG{n}{hist}\PYG{p}{(}\PYG{n}{grid} \PYG{o}{=} \PYG{k+kc}{False}\PYG{p}{,} \PYG{n}{color} \PYG{o}{=} \PYG{l+s+s1}{\PYGZsq{}}\PYG{l+s+s1}{red}\PYG{l+s+s1}{\PYGZsq{}}\PYG{p}{)}\PYG{p}{;}
\PYG{n}{plot\PYGZus{}with\PYGZus{}alt\PYGZus{}text}\PYG{p}{(}\PYG{l+s+s1}{\PYGZsq{}}\PYG{l+s+s1}{A histogram showing the distribution of petal\PYGZhy{}length for the Iris\PYGZhy{}setosa typer. There are 10 bins ranging 1\PYGZhy{}2 (x\PYGZhy{}axis).}\PYG{l+s+s1}{\PYGZsq{}} \PYGZbs{}
                   \PYG{o}{+} \PYG{l+s+s1}{\PYGZsq{}}\PYG{l+s+s1}{The y\PYGZhy{}axis is the number in each bin ranging 0\PYGZhy{}14.}\PYG{l+s+s1}{\PYGZsq{}} \PYGZbs{}
                   \PYG{o}{+} \PYG{l+s+s1}{\PYGZsq{}}\PYG{l+s+s1}{There is one grouping appearing normally distributed an x\PYGZhy{}range of approximately = [1.0, 1.95] and a y\PYGZhy{}range of approximately = [0, 14].}\PYG{l+s+s1}{\PYGZsq{}}\PYG{p}{)}
\end{sphinxVerbatim}

\end{sphinxuseclass}\end{sphinxVerbatimInput}
\begin{sphinxVerbatimOutput}

\begin{sphinxuseclass}{cell_output}
\begin{sphinxVerbatim}[commandchars=\\\{\}]
\PYGZlt{}IPython.core.display.HTML object\PYGZgt{}
\end{sphinxVerbatim}

\end{sphinxuseclass}\end{sphinxVerbatimOutput}

\end{sphinxuseclass}
\sphinxAtStartPar
Evaluators don’t assess aesthetics, but Pandas’ visualizations are limited compared to others. Below we get a much better picture of what’s going on, and we clearly see the three different types.

\sphinxAtStartPar
A \sphinxstylestrong{histogram} to visualize distributions:

\begin{sphinxuseclass}{cell}
\begin{sphinxuseclass}{tag_remove-output}\begin{sphinxVerbatimInput}

\begin{sphinxuseclass}{cell_input}
\begin{sphinxVerbatim}[commandchars=\\\{\}]
\PYG{k+kn}{import} \PYG{n+nn}{matplotlib}\PYG{n+nn}{.}\PYG{n+nn}{pyplot} \PYG{k}{as} \PYG{n+nn}{plt}
\PYG{k+kn}{import} \PYG{n+nn}{seaborn} \PYG{k}{as} \PYG{n+nn}{sns}

\PYG{n}{sns}\PYG{o}{.}\PYG{n}{histplot}\PYG{p}{(}\PYG{n}{df}\PYG{p}{,}\PYG{n}{x}\PYG{o}{=}\PYG{l+s+s1}{\PYGZsq{}}\PYG{l+s+s1}{petal\PYGZhy{}length}\PYG{l+s+s1}{\PYGZsq{}}\PYG{p}{,} \PYG{n}{hue}\PYG{o}{=}\PYG{l+s+s1}{\PYGZsq{}}\PYG{l+s+s1}{type}\PYG{l+s+s1}{\PYGZsq{}}\PYG{p}{,} \PYG{n}{kde}\PYG{o}{=}\PYG{k+kc}{True}\PYG{p}{,} \PYG{n}{bins} \PYG{o}{=}\PYG{l+m+mi}{30}\PYG{p}{)}\PYG{p}{;}
\end{sphinxVerbatim}

\end{sphinxuseclass}\end{sphinxVerbatimInput}

\end{sphinxuseclass}
\end{sphinxuseclass}
\begin{sphinxuseclass}{cell}
\begin{sphinxuseclass}{tag_remove-input}\begin{sphinxVerbatimOutput}

\begin{sphinxuseclass}{cell_output}
\begin{sphinxVerbatim}[commandchars=\\\{\}]
\PYGZlt{}IPython.core.display.HTML object\PYGZgt{}
\end{sphinxVerbatim}

\end{sphinxuseclass}\end{sphinxVerbatimOutput}

\end{sphinxuseclass}
\end{sphinxuseclass}
\sphinxAtStartPar
A \sphinxstylestrong{scatterplot} to visualize popssible correlations:

\begin{sphinxuseclass}{cell}
\begin{sphinxuseclass}{tag_remove-output}\begin{sphinxVerbatimInput}

\begin{sphinxuseclass}{cell_input}
\begin{sphinxVerbatim}[commandchars=\\\{\}]
\PYG{c+c1}{\PYGZsh{}https://seaborn.pydata.org/generated/seaborn.lmplot.html}
\PYG{n}{sns}\PYG{o}{.}\PYG{n}{lmplot}\PYG{p}{(}\PYG{n}{x}\PYG{o}{=}\PYG{l+s+s1}{\PYGZsq{}}\PYG{l+s+s1}{sepal\PYGZhy{}length}\PYG{l+s+s1}{\PYGZsq{}}\PYG{p}{,} \PYG{n}{y}\PYG{o}{=}\PYG{l+s+s1}{\PYGZsq{}}\PYG{l+s+s1}{sepal\PYGZhy{}width}\PYG{l+s+s1}{\PYGZsq{}}\PYG{p}{,} \PYG{n}{data}\PYG{o}{=}\PYG{n}{df}\PYG{p}{,} \PYG{n}{fit\PYGZus{}reg}\PYG{o}{=}\PYG{k+kc}{False}\PYG{p}{,} \PYG{n}{hue}\PYG{o}{=}\PYG{l+s+s1}{\PYGZsq{}}\PYG{l+s+s1}{type}\PYG{l+s+s1}{\PYGZsq{}}\PYG{p}{)}
\PYG{n}{plt}\PYG{o}{.}\PYG{n}{show}\PYG{p}{(}\PYG{p}{)}
\end{sphinxVerbatim}

\end{sphinxuseclass}\end{sphinxVerbatimInput}

\end{sphinxuseclass}
\end{sphinxuseclass}
\begin{sphinxuseclass}{cell}
\begin{sphinxuseclass}{tag_remove-input}\begin{sphinxVerbatimOutput}

\begin{sphinxuseclass}{cell_output}
\begin{sphinxVerbatim}[commandchars=\\\{\}]
\PYGZlt{}IPython.core.display.HTML object\PYGZgt{}
\end{sphinxVerbatim}

\end{sphinxuseclass}\end{sphinxVerbatimOutput}

\end{sphinxuseclass}
\end{sphinxuseclass}
\sphinxAtStartPar
A \sphinxstylestrong{correlogram} to visualize distributions and correlations of and between multiple variables.

\begin{sphinxuseclass}{cell}
\begin{sphinxuseclass}{tag_remove-output}\begin{sphinxVerbatimInput}

\begin{sphinxuseclass}{cell_input}
\begin{sphinxVerbatim}[commandchars=\\\{\}]
\PYG{c+c1}{\PYGZsh{}correlogram. https://seaborn.pydata.org/generated/seaborn.pairplot.html\PYGZsh{}seaborn.pairplot}
\PYG{n}{sns}\PYG{o}{.}\PYG{n}{pairplot}\PYG{p}{(}\PYG{n}{df}\PYG{p}{,} \PYG{n}{hue}\PYG{o}{=}\PYG{l+s+s1}{\PYGZsq{}}\PYG{l+s+s1}{type}\PYG{l+s+s1}{\PYGZsq{}}\PYG{p}{)}
\PYG{n}{plt}\PYG{o}{.}\PYG{n}{show}\PYG{p}{(}\PYG{p}{)}
\end{sphinxVerbatim}

\end{sphinxuseclass}\end{sphinxVerbatimInput}

\end{sphinxuseclass}
\end{sphinxuseclass}
\begin{sphinxuseclass}{cell}
\begin{sphinxuseclass}{tag_remove-input}\begin{sphinxVerbatimOutput}

\begin{sphinxuseclass}{cell_output}
\begin{sphinxVerbatim}[commandchars=\\\{\}]
\PYGZlt{}IPython.core.display.HTML object\PYGZgt{}
\end{sphinxVerbatim}

\end{sphinxuseclass}\end{sphinxVerbatimOutput}

\end{sphinxuseclass}
\end{sphinxuseclass}
\sphinxAtStartPar
Each image is a descriptive method \sphinxstyleemphasis{and} a visualization (\(\geq3\) meets the requirements). And here are some non\sphinxhyphen{}visual descriptions of the data:

\begin{sphinxuseclass}{cell}\begin{sphinxVerbatimInput}

\begin{sphinxuseclass}{cell_input}
\begin{sphinxVerbatim}[commandchars=\\\{\}]
\PYG{n}{df}\PYG{o}{.}\PYG{n}{describe}\PYG{p}{(}\PYG{n}{include}\PYG{o}{=}\PYG{l+s+s1}{\PYGZsq{}}\PYG{l+s+s1}{all}\PYG{l+s+s1}{\PYGZsq{}}\PYG{p}{)}
\end{sphinxVerbatim}

\end{sphinxuseclass}\end{sphinxVerbatimInput}
\begin{sphinxVerbatimOutput}

\begin{sphinxuseclass}{cell_output}
\begin{sphinxVerbatim}[commandchars=\\\{\}]
        sepal\PYGZhy{}length  sepal\PYGZhy{}width  petal\PYGZhy{}length  petal\PYGZhy{}width         type
count     150.000000   150.000000    150.000000   150.000000          150
unique           NaN          NaN           NaN          NaN            3
top              NaN          NaN           NaN          NaN  Iris\PYGZhy{}setosa
freq             NaN          NaN           NaN          NaN           50
mean        5.843333     3.054000      3.758667     1.198667          NaN
std         0.828066     0.433594      1.764420     0.763161          NaN
min         4.300000     2.000000      1.000000     0.100000          NaN
25\PYGZpc{}         5.100000     2.800000      1.600000     0.300000          NaN
50\PYGZpc{}         5.800000     3.000000      4.350000     1.300000          NaN
75\PYGZpc{}         6.400000     3.300000      5.100000     1.800000          NaN
max         7.900000     4.400000      6.900000     2.500000          NaN
\end{sphinxVerbatim}

\end{sphinxuseclass}\end{sphinxVerbatimOutput}

\end{sphinxuseclass}
\sphinxAtStartPar
Play around \sphinxhyphen{}\sphinxstyleemphasis{explore}.

\sphinxstepscope


\subsection{Model Development}
\label{\detokenize{task2_c/example_sup_class/sup_class_ex-develop:model-development}}\label{\detokenize{task2_c/example_sup_class/sup_class_ex-develop:sup-class-ex-develop}}\label{\detokenize{task2_c/example_sup_class/sup_class_ex-develop::doc}}
\sphinxAtStartPar
Supervised algorithms use inputs (independent variables) and labeled outputs (dependent variable \sphinxhyphen{}the “answers”) to create a model that can measure its performance and learn over time. Splitting the data into independent and dependent variables, we have the following:

\begin{sphinxuseclass}{cell}\begin{sphinxVerbatimInput}

\begin{sphinxuseclass}{cell_input}
\begin{sphinxVerbatim}[commandchars=\\\{\}]
\PYG{c+c1}{\PYGZsh{}Note: we only repeat this step from before, because this is a new .ipyb page.}
\PYG{c+c1}{\PYGZsh{}   it only needs to be executed once per file. }
  
\PYG{c+c1}{\PYGZsh{}We\PYGZsq{}ll import libraries as needed, but when submitting, having them all at the top is best practice}
\PYG{k+kn}{import} \PYG{n+nn}{pandas} \PYG{k}{as} \PYG{n+nn}{pd}

\PYG{c+c1}{\PYGZsh{} Reloading the dataset}
\PYG{n}{url} \PYG{o}{=} \PYG{l+s+s2}{\PYGZdq{}}\PYG{l+s+s2}{https://raw.githubusercontent.com/jbrownlee/Datasets/master/iris.csv}\PYG{l+s+s2}{\PYGZdq{}}
\PYG{n}{column\PYGZus{}names} \PYG{o}{=} \PYG{p}{[}\PYG{l+s+s1}{\PYGZsq{}}\PYG{l+s+s1}{sepal\PYGZhy{}length}\PYG{l+s+s1}{\PYGZsq{}}\PYG{p}{,} \PYG{l+s+s1}{\PYGZsq{}}\PYG{l+s+s1}{sepal\PYGZhy{}width}\PYG{l+s+s1}{\PYGZsq{}}\PYG{p}{,} \PYG{l+s+s1}{\PYGZsq{}}\PYG{l+s+s1}{petal\PYGZhy{}length}\PYG{l+s+s1}{\PYGZsq{}}\PYG{p}{,} \PYG{l+s+s1}{\PYGZsq{}}\PYG{l+s+s1}{petal\PYGZhy{}width}\PYG{l+s+s1}{\PYGZsq{}}\PYG{p}{,} \PYG{l+s+s1}{\PYGZsq{}}\PYG{l+s+s1}{type}\PYG{l+s+s1}{\PYGZsq{}}\PYG{p}{]}
\PYG{n}{df} \PYG{o}{=} \PYG{n}{pd}\PYG{o}{.}\PYG{n}{read\PYGZus{}csv}\PYG{p}{(}\PYG{n}{url}\PYG{p}{,} \PYG{n}{names} \PYG{o}{=} \PYG{n}{column\PYGZus{}names}\PYG{p}{)} \PYG{c+c1}{\PYGZsh{}read CSV into Python as a dataframe}

\PYG{n}{X} \PYG{o}{=} \PYG{n}{df}\PYG{o}{.}\PYG{n}{drop}\PYG{p}{(}\PYG{n}{columns}\PYG{o}{=}\PYG{p}{[}\PYG{l+s+s1}{\PYGZsq{}}\PYG{l+s+s1}{type}\PYG{l+s+s1}{\PYGZsq{}}\PYG{p}{]}\PYG{p}{)} \PYG{c+c1}{\PYGZsh{}indpendent variables}
\PYG{n}{y} \PYG{o}{=} \PYG{n}{df}\PYG{p}{[}\PYG{p}{[}\PYG{l+s+s1}{\PYGZsq{}}\PYG{l+s+s1}{type}\PYG{l+s+s1}{\PYGZsq{}}\PYG{p}{]}\PYG{p}{]}\PYG{o}{.}\PYG{n}{copy}\PYG{p}{(}\PYG{p}{)} \PYG{c+c1}{\PYGZsh{}dependent variables}
\end{sphinxVerbatim}

\end{sphinxuseclass}\end{sphinxVerbatimInput}

\end{sphinxuseclass}
\begin{sphinxadmonition}{note}{Note:}
\sphinxAtStartPar
The focus of \DUrole{xref,myst}{Task 2 part D \sphinxstyleemphasis{Data Product}} will be the what, how, and why of your model’s development.
\end{sphinxadmonition}


\subsubsection{Training the Model}
\label{\detokenize{task2_c/example_sup_class/sup_class_ex-develop:training-the-model}}\label{\detokenize{task2_c/example_sup_class/sup_class_ex-develop:sup-class-ex-develop-train}}
\begin{sphinxShadowBox}
\sphinxstylesidebartitle{}

\sphinxAtStartPar
A model can learn the details and noise particular to the training data so well that it doesn’t perform well on new data. This is called \sphinxhref{https://en.wikipedia.org/wiki/Overfitting}{\sphinxstyleemphasis{overfitting}}. Overcomplicated non\sphinxhyphen{}linear and nonparametric models are more susceptible to this. The term \sphinxstyleemphasis{overtraining} can be used synonymously or to mean too much training causing overfitting.
\end{sphinxShadowBox}

\sphinxAtStartPar
Studying for a test when you have all the answers beforehand will likely yield a good grade. But how well would that grade measure understanding of material outside those answers? Similarly, supervised methods tend to perform well when tested on their training data, but you want your model to perform well on \sphinxstyleemphasis{unseen} data. So while it’s not required, separating data used to train and test the model (validation) is good practice. Furthermore, it provides content for part D of the documentation.

\sphinxAtStartPar
Fortunately, most libraries have built\sphinxhyphen{}in functions for this. Here we’ll stick with \sphinxhref{https://scikit-learn.org/stable/modules/generated/sklearn.model\_selection.train\_test\_split.html}{scikit\sphinxhyphen{}learn aka sklearn} validation processes We’ll need to randomly split the data into independent (input values) and dependent (output, i.e., the answers) variables. For now, we’ll keep things as DataFrames, but later convert them to 2\sphinxhyphen{}d arrays

\begin{sphinxuseclass}{cell}\begin{sphinxVerbatimInput}

\begin{sphinxuseclass}{cell_input}
\begin{sphinxVerbatim}[commandchars=\\\{\}]
\PYG{k+kn}{import} \PYG{n+nn}{numpy} \PYG{k}{as} \PYG{n+nn}{np}
\PYG{k+kn}{from} \PYG{n+nn}{sklearn}\PYG{n+nn}{.}\PYG{n+nn}{model\PYGZus{}selection} \PYG{k+kn}{import} \PYG{n}{train\PYGZus{}test\PYGZus{}split}

\PYG{c+c1}{\PYGZsh{}split the variable sets into training and testing subsets}
\PYG{n}{X\PYGZus{}train}\PYG{p}{,} \PYG{n}{X\PYGZus{}test}\PYG{p}{,} \PYG{n}{y\PYGZus{}train}\PYG{p}{,} \PYG{n}{y\PYGZus{}test} \PYG{o}{=} \PYG{n}{train\PYGZus{}test\PYGZus{}split}\PYG{p}{(}\PYG{n}{X}\PYG{p}{,} \PYG{n}{y}\PYG{p}{,} \PYG{n}{test\PYGZus{}size}\PYG{o}{=}\PYG{l+m+mf}{0.333}\PYG{p}{,} \PYG{n}{random\PYGZus{}state}\PYG{o}{=}\PYG{l+m+mi}{41}\PYG{p}{)}
\end{sphinxVerbatim}

\end{sphinxuseclass}\end{sphinxVerbatimInput}

\end{sphinxuseclass}
\begin{sphinxuseclass}{cell}
\begin{sphinxuseclass}{tag_hide-input}\begin{sphinxVerbatimOutput}

\begin{sphinxuseclass}{cell_output}
\end{sphinxuseclass}\end{sphinxVerbatimOutput}

\end{sphinxuseclass}
\end{sphinxuseclass}
\sphinxAtStartPar
Read the \DUrole{xref,myst}{docs}! By default \sphinxcode{\sphinxupquote{train\_test\_split}}, \sphinxhref{https://engineering.mit.edu/engage/ask-an-engineer/can-a-computer-generate-a-truly-random-number/}{“randomly”} splits the sets. Setting the seed (or state) with \sphinxcode{\sphinxupquote{random\_state}} controls the experiments. See \sphinxhref{https://datascience.stackexchange.com/questions/78109/should-you-use-random-state-or-random-seed-in-machine-learning-models}{should you use a random seed?}.

\sphinxAtStartPar
We can now train a model using the independent (usually denoted \sphinxcode{\sphinxupquote{X}}) and dependent variables (usually denoted \sphinxcode{\sphinxupquote{y}}) from the training data. Sklearn has a deep \sphinxhref{https://scikit-learn.org/stable/supervised\_learning.html}{supervised learning library}. Note that many of these models (including SVM) have both classification and regression extensions.

\begin{sphinxuseclass}{cell}\begin{sphinxVerbatimInput}

\begin{sphinxuseclass}{cell_input}
\begin{sphinxVerbatim}[commandchars=\\\{\}]
\PYG{k+kn}{from} \PYG{n+nn}{sklearn} \PYG{k+kn}{import} \PYG{n}{svm}

\PYG{n}{svm\PYGZus{}model} \PYG{o}{=} \PYG{n}{svm}\PYG{o}{.}\PYG{n}{SVC}\PYG{p}{(}\PYG{n}{gamma}\PYG{o}{=}\PYG{l+s+s1}{\PYGZsq{}}\PYG{l+s+s1}{scale}\PYG{l+s+s1}{\PYGZsq{}}\PYG{p}{,} \PYG{n}{C}\PYG{o}{=}\PYG{l+m+mi}{1}\PYG{p}{)} \PYG{c+c1}{\PYGZsh{}Creates a svm model object. Mote, \PYGZsq{}scale\PYGZsq{} and 1.0 are gamma and C\PYGZsq{}s respective defaults }
\PYG{n}{svm\PYGZus{}model}\PYG{o}{.}\PYG{n}{fit}\PYG{p}{(}\PYG{n}{X\PYGZus{}train}\PYG{p}{,}\PYG{n}{y\PYGZus{}train}\PYG{p}{)}
\end{sphinxVerbatim}

\end{sphinxuseclass}\end{sphinxVerbatimInput}
\begin{sphinxVerbatimOutput}

\begin{sphinxuseclass}{cell_output}
\begin{sphinxVerbatim}[commandchars=\\\{\}]
C:\PYGZbs{}Users\PYGZbs{}ashej\PYGZbs{}.virtualenvs\PYGZbs{}jupyter\PYGZhy{}books\PYGZhy{}WZpnkDri\PYGZbs{}Lib\PYGZbs{}site\PYGZhy{}packages\PYGZbs{}sklearn\PYGZbs{}utils\PYGZbs{}validation.py:1141: DataConversionWarning: A column\PYGZhy{}vector y was passed when a 1d array was expected. Please change the shape of y to (n\PYGZus{}samples, ), for example using ravel().
  y = column\PYGZus{}or\PYGZus{}1d(y, warn=True)
\end{sphinxVerbatim}

\begin{sphinxVerbatim}[commandchars=\\\{\}]
SVC(C=1)
\end{sphinxVerbatim}

\end{sphinxuseclass}\end{sphinxVerbatimOutput}

\end{sphinxuseclass}
\sphinxAtStartPar
What’s with the warning? \sphinxcode{\sphinxupquote{DataConversionWarning: A column\sphinxhyphen{}vector y was passed when a 1d array was expected.}}

\sphinxAtStartPar
Looking at the \sphinxcode{\sphinxupquote{sklearn.svm.SVC}} \sphinxhref{https://scikit-learn.org/stable/modules/generated/sklearn.svm.SVC.html\#sklearn.svm.SVC.fit}{docs for the \sphinxcode{\sphinxupquote{fit}} function}, a 1d array was expected for the \sphinxcode{\sphinxupquote{y}}, but we gave it a DataFrame. This is a warning \sphinxhyphen{}not an error, and the model appears to work. However, it’s best practice to clean warnings up when possible, and in this case, it’s an easy fix.

\begin{sphinxShadowBox}
\sphinxstylesidebartitle{}

\sphinxAtStartPar
\sphinxhref{https://scikit-learn.org/stable/auto\_examples/svm/plot\_rbf\_parameters.html}{What’s that \sphinxcode{\sphinxupquote{gamma}} and \sphinxcode{\sphinxupquote{C}} for?}, read the docs!
\end{sphinxShadowBox}

\begin{sphinxuseclass}{cell}\begin{sphinxVerbatimInput}

\begin{sphinxuseclass}{cell_input}
\begin{sphinxVerbatim}[commandchars=\\\{\}]
\PYG{n}{y\PYGZus{}train\PYGZus{}array}\PYG{p}{,} \PYG{n}{y\PYGZus{}test\PYGZus{}array} \PYG{o}{=} \PYG{n}{y\PYGZus{}train}\PYG{p}{[}\PYG{l+s+s1}{\PYGZsq{}}\PYG{l+s+s1}{type}\PYG{l+s+s1}{\PYGZsq{}}\PYG{p}{]}\PYG{o}{.}\PYG{n}{values}\PYG{p}{,} \PYG{n}{y\PYGZus{}test}\PYG{p}{[}\PYG{l+s+s1}{\PYGZsq{}}\PYG{l+s+s1}{type}\PYG{l+s+s1}{\PYGZsq{}}\PYG{p}{]}\PYG{o}{.}\PYG{n}{values}
\PYG{n}{svm\PYGZus{}model}\PYG{o}{.}\PYG{n}{fit}\PYG{p}{(}\PYG{n}{X\PYGZus{}train}\PYG{p}{,}\PYG{n}{y\PYGZus{}train\PYGZus{}array}\PYG{p}{)} 
\PYG{c+c1}{\PYGZsh{}Note: you can also use \PYGZsq{}svm\PYGZus{}model.fit(X\PYGZus{}train,y\PYGZus{}train\PYGZus{}array.ravel())\PYGZsq{} }
\end{sphinxVerbatim}

\end{sphinxuseclass}\end{sphinxVerbatimInput}
\begin{sphinxVerbatimOutput}

\begin{sphinxuseclass}{cell_output}
\begin{sphinxVerbatim}[commandchars=\\\{\}]
SVC(C=1)
\end{sphinxVerbatim}

\end{sphinxuseclass}\end{sphinxVerbatimOutput}

\end{sphinxuseclass}

\subsubsection{Applying the Model}
\label{\detokenize{task2_c/example_sup_class/sup_class_ex-develop:applying-the-model}}
\sphinxAtStartPar
Now we’ve trained the model (without warnings)! What does that mean? Sklearn’s SVM algorithm creates an equation representing the relationship between the variables.
\begin{equation*}
\begin{split}F_{\text{predict}}(X)=\text{prediction(s)}\end{split}
\end{equation*}
\sphinxAtStartPar
For example, the Iris at index \sphinxcode{\sphinxupquote{82}} has the values:

\begin{sphinxuseclass}{cell}
\begin{sphinxuseclass}{tag_hide-input}\begin{sphinxVerbatimOutput}

\begin{sphinxuseclass}{cell_output}
\begin{sphinxVerbatim}[commandchars=\\\{\}]
    sepal\PYGZhy{}length  sepal\PYGZhy{}width  petal\PYGZhy{}length  petal\PYGZhy{}width
82           5.8          2.7           3.9          1.2
\end{sphinxVerbatim}

\end{sphinxuseclass}\end{sphinxVerbatimOutput}

\end{sphinxuseclass}
\end{sphinxuseclass}
\sphinxAtStartPar
And \sphinxcode{\sphinxupquote{svm\_model.predict(X\_train.loc{[}{[}82{]}{]})}} inputs the flower dimensions into the prediction function:
\begin{equation*}
\begin{split}F_{\text{predict}}(5.8, 2.7, 3.9, 1.2)=\text{Iris-versicolor}\end{split}
\end{equation*}
\begin{sphinxuseclass}{cell}\begin{sphinxVerbatimInput}

\begin{sphinxuseclass}{cell_input}
\begin{sphinxVerbatim}[commandchars=\\\{\}]
\PYG{n+nb}{print}\PYG{p}{(}\PYG{n}{svm\PYGZus{}model}\PYG{o}{.}\PYG{n}{predict}\PYG{p}{(}\PYG{n}{X\PYGZus{}train}\PYG{o}{.}\PYG{n}{loc}\PYG{p}{[}\PYG{p}{[}\PYG{l+m+mi}{82}\PYG{p}{]}\PYG{p}{]}\PYG{p}{)}\PYG{p}{)}
\PYG{c+c1}{\PYGZsh{}Alternatively: svm\PYGZus{}model.predict([[5.8, 2.7, 3.9, 1.2]])}
\end{sphinxVerbatim}

\end{sphinxuseclass}\end{sphinxVerbatimInput}
\begin{sphinxVerbatimOutput}

\begin{sphinxuseclass}{cell_output}
\begin{sphinxVerbatim}[commandchars=\\\{\}]
[\PYGZsq{}Iris\PYGZhy{}versicolor\PYGZsq{}]
\end{sphinxVerbatim}

\end{sphinxuseclass}\end{sphinxVerbatimOutput}

\end{sphinxuseclass}
\sphinxAtStartPar
Which in this example turns out to be correct:

\begin{sphinxuseclass}{cell}
\begin{sphinxuseclass}{tag_hide-input}\begin{sphinxVerbatimOutput}

\begin{sphinxuseclass}{cell_output}
\begin{sphinxVerbatim}[commandchars=\\\{\}]
    sepal\PYGZhy{}length  sepal\PYGZhy{}width  petal\PYGZhy{}length  petal\PYGZhy{}width             type
82           5.8          2.7           3.9          1.2  Iris\PYGZhy{}versicolor
\end{sphinxVerbatim}

\end{sphinxuseclass}\end{sphinxVerbatimOutput}

\end{sphinxuseclass}
\end{sphinxuseclass}
\sphinxAtStartPar
Applying the prediction function to the entire dataset, we get a prediction for each flower:

\begin{sphinxuseclass}{cell}
\begin{sphinxuseclass}{tag_output_scroll}\begin{sphinxVerbatimInput}

\begin{sphinxuseclass}{cell_input}
\begin{sphinxVerbatim}[commandchars=\\\{\}]
\PYG{n}{predictions} \PYG{o}{=} \PYG{n}{svm\PYGZus{}model}\PYG{o}{.}\PYG{n}{predict}\PYG{p}{(}\PYG{n}{X}\PYG{p}{)}
\PYG{n+nb}{print}\PYG{p}{(}\PYG{n}{predictions}\PYG{p}{)}
\end{sphinxVerbatim}

\end{sphinxuseclass}\end{sphinxVerbatimInput}
\begin{sphinxVerbatimOutput}

\begin{sphinxuseclass}{cell_output}
\begin{sphinxVerbatim}[commandchars=\\\{\}]
[\PYGZsq{}Iris\PYGZhy{}setosa\PYGZsq{} \PYGZsq{}Iris\PYGZhy{}setosa\PYGZsq{} \PYGZsq{}Iris\PYGZhy{}setosa\PYGZsq{} \PYGZsq{}Iris\PYGZhy{}setosa\PYGZsq{} \PYGZsq{}Iris\PYGZhy{}setosa\PYGZsq{}
 \PYGZsq{}Iris\PYGZhy{}setosa\PYGZsq{} \PYGZsq{}Iris\PYGZhy{}setosa\PYGZsq{} \PYGZsq{}Iris\PYGZhy{}setosa\PYGZsq{} \PYGZsq{}Iris\PYGZhy{}setosa\PYGZsq{} \PYGZsq{}Iris\PYGZhy{}setosa\PYGZsq{}
 \PYGZsq{}Iris\PYGZhy{}setosa\PYGZsq{} \PYGZsq{}Iris\PYGZhy{}setosa\PYGZsq{} \PYGZsq{}Iris\PYGZhy{}setosa\PYGZsq{} \PYGZsq{}Iris\PYGZhy{}setosa\PYGZsq{} \PYGZsq{}Iris\PYGZhy{}setosa\PYGZsq{}
 \PYGZsq{}Iris\PYGZhy{}setosa\PYGZsq{} \PYGZsq{}Iris\PYGZhy{}setosa\PYGZsq{} \PYGZsq{}Iris\PYGZhy{}setosa\PYGZsq{} \PYGZsq{}Iris\PYGZhy{}setosa\PYGZsq{} \PYGZsq{}Iris\PYGZhy{}setosa\PYGZsq{}
 \PYGZsq{}Iris\PYGZhy{}setosa\PYGZsq{} \PYGZsq{}Iris\PYGZhy{}setosa\PYGZsq{} \PYGZsq{}Iris\PYGZhy{}setosa\PYGZsq{} \PYGZsq{}Iris\PYGZhy{}setosa\PYGZsq{} \PYGZsq{}Iris\PYGZhy{}setosa\PYGZsq{}
 \PYGZsq{}Iris\PYGZhy{}setosa\PYGZsq{} \PYGZsq{}Iris\PYGZhy{}setosa\PYGZsq{} \PYGZsq{}Iris\PYGZhy{}setosa\PYGZsq{} \PYGZsq{}Iris\PYGZhy{}setosa\PYGZsq{} \PYGZsq{}Iris\PYGZhy{}setosa\PYGZsq{}
 \PYGZsq{}Iris\PYGZhy{}setosa\PYGZsq{} \PYGZsq{}Iris\PYGZhy{}setosa\PYGZsq{} \PYGZsq{}Iris\PYGZhy{}setosa\PYGZsq{} \PYGZsq{}Iris\PYGZhy{}setosa\PYGZsq{} \PYGZsq{}Iris\PYGZhy{}setosa\PYGZsq{}
 \PYGZsq{}Iris\PYGZhy{}setosa\PYGZsq{} \PYGZsq{}Iris\PYGZhy{}setosa\PYGZsq{} \PYGZsq{}Iris\PYGZhy{}setosa\PYGZsq{} \PYGZsq{}Iris\PYGZhy{}setosa\PYGZsq{} \PYGZsq{}Iris\PYGZhy{}setosa\PYGZsq{}
 \PYGZsq{}Iris\PYGZhy{}setosa\PYGZsq{} \PYGZsq{}Iris\PYGZhy{}setosa\PYGZsq{} \PYGZsq{}Iris\PYGZhy{}setosa\PYGZsq{} \PYGZsq{}Iris\PYGZhy{}setosa\PYGZsq{} \PYGZsq{}Iris\PYGZhy{}setosa\PYGZsq{}
 \PYGZsq{}Iris\PYGZhy{}setosa\PYGZsq{} \PYGZsq{}Iris\PYGZhy{}setosa\PYGZsq{} \PYGZsq{}Iris\PYGZhy{}setosa\PYGZsq{} \PYGZsq{}Iris\PYGZhy{}setosa\PYGZsq{} \PYGZsq{}Iris\PYGZhy{}setosa\PYGZsq{}
 \PYGZsq{}Iris\PYGZhy{}versicolor\PYGZsq{} \PYGZsq{}Iris\PYGZhy{}versicolor\PYGZsq{} \PYGZsq{}Iris\PYGZhy{}versicolor\PYGZsq{} \PYGZsq{}Iris\PYGZhy{}versicolor\PYGZsq{}
 \PYGZsq{}Iris\PYGZhy{}versicolor\PYGZsq{} \PYGZsq{}Iris\PYGZhy{}versicolor\PYGZsq{} \PYGZsq{}Iris\PYGZhy{}versicolor\PYGZsq{} \PYGZsq{}Iris\PYGZhy{}versicolor\PYGZsq{}
 \PYGZsq{}Iris\PYGZhy{}versicolor\PYGZsq{} \PYGZsq{}Iris\PYGZhy{}versicolor\PYGZsq{} \PYGZsq{}Iris\PYGZhy{}versicolor\PYGZsq{} \PYGZsq{}Iris\PYGZhy{}versicolor\PYGZsq{}
 \PYGZsq{}Iris\PYGZhy{}versicolor\PYGZsq{} \PYGZsq{}Iris\PYGZhy{}versicolor\PYGZsq{} \PYGZsq{}Iris\PYGZhy{}versicolor\PYGZsq{} \PYGZsq{}Iris\PYGZhy{}versicolor\PYGZsq{}
 \PYGZsq{}Iris\PYGZhy{}versicolor\PYGZsq{} \PYGZsq{}Iris\PYGZhy{}versicolor\PYGZsq{} \PYGZsq{}Iris\PYGZhy{}versicolor\PYGZsq{} \PYGZsq{}Iris\PYGZhy{}versicolor\PYGZsq{}
 \PYGZsq{}Iris\PYGZhy{}virginica\PYGZsq{} \PYGZsq{}Iris\PYGZhy{}versicolor\PYGZsq{} \PYGZsq{}Iris\PYGZhy{}versicolor\PYGZsq{} \PYGZsq{}Iris\PYGZhy{}versicolor\PYGZsq{}
 \PYGZsq{}Iris\PYGZhy{}versicolor\PYGZsq{} \PYGZsq{}Iris\PYGZhy{}versicolor\PYGZsq{} \PYGZsq{}Iris\PYGZhy{}versicolor\PYGZsq{} \PYGZsq{}Iris\PYGZhy{}virginica\PYGZsq{}
 \PYGZsq{}Iris\PYGZhy{}versicolor\PYGZsq{} \PYGZsq{}Iris\PYGZhy{}versicolor\PYGZsq{} \PYGZsq{}Iris\PYGZhy{}versicolor\PYGZsq{} \PYGZsq{}Iris\PYGZhy{}versicolor\PYGZsq{}
 \PYGZsq{}Iris\PYGZhy{}versicolor\PYGZsq{} \PYGZsq{}Iris\PYGZhy{}virginica\PYGZsq{} \PYGZsq{}Iris\PYGZhy{}versicolor\PYGZsq{} \PYGZsq{}Iris\PYGZhy{}versicolor\PYGZsq{}
 \PYGZsq{}Iris\PYGZhy{}versicolor\PYGZsq{} \PYGZsq{}Iris\PYGZhy{}versicolor\PYGZsq{} \PYGZsq{}Iris\PYGZhy{}versicolor\PYGZsq{} \PYGZsq{}Iris\PYGZhy{}versicolor\PYGZsq{}
 \PYGZsq{}Iris\PYGZhy{}versicolor\PYGZsq{} \PYGZsq{}Iris\PYGZhy{}versicolor\PYGZsq{} \PYGZsq{}Iris\PYGZhy{}versicolor\PYGZsq{} \PYGZsq{}Iris\PYGZhy{}versicolor\PYGZsq{}
 \PYGZsq{}Iris\PYGZhy{}versicolor\PYGZsq{} \PYGZsq{}Iris\PYGZhy{}versicolor\PYGZsq{} \PYGZsq{}Iris\PYGZhy{}versicolor\PYGZsq{} \PYGZsq{}Iris\PYGZhy{}versicolor\PYGZsq{}
 \PYGZsq{}Iris\PYGZhy{}versicolor\PYGZsq{} \PYGZsq{}Iris\PYGZhy{}versicolor\PYGZsq{} \PYGZsq{}Iris\PYGZhy{}virginica\PYGZsq{} \PYGZsq{}Iris\PYGZhy{}virginica\PYGZsq{}
 \PYGZsq{}Iris\PYGZhy{}virginica\PYGZsq{} \PYGZsq{}Iris\PYGZhy{}virginica\PYGZsq{} \PYGZsq{}Iris\PYGZhy{}virginica\PYGZsq{} \PYGZsq{}Iris\PYGZhy{}virginica\PYGZsq{}
 \PYGZsq{}Iris\PYGZhy{}versicolor\PYGZsq{} \PYGZsq{}Iris\PYGZhy{}virginica\PYGZsq{} \PYGZsq{}Iris\PYGZhy{}virginica\PYGZsq{} \PYGZsq{}Iris\PYGZhy{}virginica\PYGZsq{}
 \PYGZsq{}Iris\PYGZhy{}virginica\PYGZsq{} \PYGZsq{}Iris\PYGZhy{}virginica\PYGZsq{} \PYGZsq{}Iris\PYGZhy{}virginica\PYGZsq{} \PYGZsq{}Iris\PYGZhy{}virginica\PYGZsq{}
 \PYGZsq{}Iris\PYGZhy{}virginica\PYGZsq{} \PYGZsq{}Iris\PYGZhy{}virginica\PYGZsq{} \PYGZsq{}Iris\PYGZhy{}virginica\PYGZsq{} \PYGZsq{}Iris\PYGZhy{}virginica\PYGZsq{}
 \PYGZsq{}Iris\PYGZhy{}virginica\PYGZsq{} \PYGZsq{}Iris\PYGZhy{}virginica\PYGZsq{} \PYGZsq{}Iris\PYGZhy{}virginica\PYGZsq{} \PYGZsq{}Iris\PYGZhy{}virginica\PYGZsq{}
 \PYGZsq{}Iris\PYGZhy{}virginica\PYGZsq{} \PYGZsq{}Iris\PYGZhy{}virginica\PYGZsq{} \PYGZsq{}Iris\PYGZhy{}virginica\PYGZsq{} \PYGZsq{}Iris\PYGZhy{}virginica\PYGZsq{}
 \PYGZsq{}Iris\PYGZhy{}virginica\PYGZsq{} \PYGZsq{}Iris\PYGZhy{}virginica\PYGZsq{} \PYGZsq{}Iris\PYGZhy{}virginica\PYGZsq{} \PYGZsq{}Iris\PYGZhy{}virginica\PYGZsq{}
 \PYGZsq{}Iris\PYGZhy{}virginica\PYGZsq{} \PYGZsq{}Iris\PYGZhy{}virginica\PYGZsq{} \PYGZsq{}Iris\PYGZhy{}virginica\PYGZsq{} \PYGZsq{}Iris\PYGZhy{}virginica\PYGZsq{}
 \PYGZsq{}Iris\PYGZhy{}virginica\PYGZsq{} \PYGZsq{}Iris\PYGZhy{}virginica\PYGZsq{} \PYGZsq{}Iris\PYGZhy{}virginica\PYGZsq{} \PYGZsq{}Iris\PYGZhy{}virginica\PYGZsq{}
 \PYGZsq{}Iris\PYGZhy{}virginica\PYGZsq{} \PYGZsq{}Iris\PYGZhy{}virginica\PYGZsq{} \PYGZsq{}Iris\PYGZhy{}virginica\PYGZsq{} \PYGZsq{}Iris\PYGZhy{}virginica\PYGZsq{}
 \PYGZsq{}Iris\PYGZhy{}virginica\PYGZsq{} \PYGZsq{}Iris\PYGZhy{}virginica\PYGZsq{} \PYGZsq{}Iris\PYGZhy{}virginica\PYGZsq{} \PYGZsq{}Iris\PYGZhy{}virginica\PYGZsq{}
 \PYGZsq{}Iris\PYGZhy{}virginica\PYGZsq{} \PYGZsq{}Iris\PYGZhy{}virginica\PYGZsq{} \PYGZsq{}Iris\PYGZhy{}virginica\PYGZsq{} \PYGZsq{}Iris\PYGZhy{}virginica\PYGZsq{}]
\end{sphinxVerbatim}

\end{sphinxuseclass}\end{sphinxVerbatimOutput}

\end{sphinxuseclass}
\end{sphinxuseclass}
\sphinxAtStartPar
But how good are these predictions? Answering that question is our next step.

\sphinxstepscope


\subsection{Accuracy Analysis (for classification)}
\label{\detokenize{task2_c/example_sup_class/sup_class_ex-accuracy:accuracy-analysis-for-classification}}\label{\detokenize{task2_c/example_sup_class/sup_class_ex-accuracy:sup-class-ex-accuracy}}\label{\detokenize{task2_c/example_sup_class/sup_class_ex-accuracy::doc}}
\sphinxAtStartPar
In the \DUrole{xref,myst}{Accuracy Analysis section of part D} of your project’s documentation, you will need to define and discuss the metric for measuring the success of your application’s algorithm.  The metric for measuring a supervised classification model’s accuracy is straightforward. We use \sphinxhref{https://scikit-learn.org/stable/modules/generated/sklearn.metrics.accuracy\_score.html\#sklearn.metrics.accuracy\_score}{the ratio of correct to total predictions}:
\begin{equation*}
\begin{split}\text{Accuracy}=\frac{\text{correct predictions}}{\text{total predictions}}\end{split}
\end{equation*}
\sphinxAtStartPar
Most libraries have builtins for this; see \sphinxhref{https://scikit-learn.org/stable/modules/model\_evaluation.html}{sklearn metrics}.

\begin{sphinxuseclass}{cell}
\begin{sphinxuseclass}{tag_hide-input}
\end{sphinxuseclass}
\end{sphinxuseclass}
\begin{sphinxuseclass}{cell}\begin{sphinxVerbatimInput}

\begin{sphinxuseclass}{cell_input}
\begin{sphinxVerbatim}[commandchars=\\\{\}]
\PYG{k+kn}{from} \PYG{n+nn}{sklearn} \PYG{k+kn}{import} \PYG{n}{metrics}
\PYG{n}{score} \PYG{o}{=} \PYG{n}{metrics}\PYG{o}{.}\PYG{n}{accuracy\PYGZus{}score}\PYG{p}{(}\PYG{n}{y\PYGZus{}train}\PYG{p}{,} \PYG{n}{predictions}\PYG{p}{)}
\PYG{n}{score}
\end{sphinxVerbatim}

\end{sphinxuseclass}\end{sphinxVerbatimInput}
\begin{sphinxVerbatimOutput}

\begin{sphinxuseclass}{cell_output}
\begin{sphinxVerbatim}[commandchars=\\\{\}]
0.99
\end{sphinxVerbatim}

\end{sphinxuseclass}\end{sphinxVerbatimOutput}

\end{sphinxuseclass}
\sphinxAtStartPar
99\% is pretty good (actually too good), but we tested the model using the \sphinxstyleemphasis{same} data used to train the model.
\begin{quote}

\sphinxAtStartPar
svm\_model.fit(X\_train,y\_train\_array)
\end{quote}

\sphinxAtStartPar
Testing with the training data is \sphinxstyleemphasis{not} good practice. Recall the \sphinxstyleemphasis{test} data was set aside for this purpose.

\begin{sphinxuseclass}{cell}\begin{sphinxVerbatimInput}

\begin{sphinxuseclass}{cell_input}
\begin{sphinxVerbatim}[commandchars=\\\{\}]
\PYG{c+c1}{\PYGZsh{}y\PYGZus{}test\PYGZus{}array = y\PYGZus{}test[\PYGZsq{}type\PYGZsq{}].values \PYGZsh{}Converts the dataframe to an array.}
\PYG{c+c1}{\PYGZsh{}predictions using test data}
\PYG{n}{predictions\PYGZus{}test} \PYG{o}{=} \PYG{n}{svm\PYGZus{}model}\PYG{o}{.}\PYG{n}{predict}\PYG{p}{(}\PYG{n}{X\PYGZus{}test}\PYG{p}{)}
\PYG{n}{score2} \PYG{o}{=} \PYG{n}{metrics}\PYG{o}{.}\PYG{n}{accuracy\PYGZus{}score}\PYG{p}{(}\PYG{n}{y\PYGZus{}test}\PYG{p}{,}\PYG{n}{predictions\PYGZus{}test}\PYG{p}{)}
\PYG{n}{score2}
\end{sphinxVerbatim}

\end{sphinxuseclass}\end{sphinxVerbatimInput}
\begin{sphinxVerbatimOutput}

\begin{sphinxuseclass}{cell_output}
\begin{sphinxVerbatim}[commandchars=\\\{\}]
0.94
\end{sphinxVerbatim}

\end{sphinxuseclass}\end{sphinxVerbatimOutput}

\end{sphinxuseclass}
\sphinxAtStartPar
Using the test data we set aside, \(94\%\) of the predictions are correct.

\phantomsection\label{\detokenize{task2_c/example_sup_class/sup_class_ex-accuracy:sup-class-ex-accuracy-confusion-matrix}}
\sphinxAtStartPar
A \sphinxhref{https://scikit-learn.org/stable/modules/generated/sklearn.metrics.ConfusionMatrixDisplay.html\#sklearn.metrics.ConfusionMatrixDisplay.plot}{\sphinxstyleemphasis{confusion matrix}} further breaks down the predictions by categories, helping develop better models and providing another visualization.

\begin{sphinxShadowBox}
\sphinxstylesidebartitle{Why is it called a confusion matrix?}

\sphinxAtStartPar
As it makes things less confusing, it would seem to be a misnomer. The name comes from making it easier to see whether the system is confusing two categories (i.e., commonly mislabeling one as another). Another (maybe less confusing) name is an \sphinxstyleemphasis{error matrix}.
\end{sphinxShadowBox}

\begin{sphinxuseclass}{cell}
\begin{sphinxuseclass}{tag_remove-output}\begin{sphinxVerbatimInput}

\begin{sphinxuseclass}{cell_input}
\begin{sphinxVerbatim}[commandchars=\\\{\}]
\PYG{k+kn}{from} \PYG{n+nn}{sklearn}\PYG{n+nn}{.}\PYG{n+nn}{metrics} \PYG{k+kn}{import} \PYG{n}{ConfusionMatrixDisplay}

\PYG{n}{ConfusionMatrixDisplay}\PYG{o}{.}\PYG{n}{from\PYGZus{}estimator}\PYG{p}{(}\PYG{n}{svm\PYGZus{}model}\PYG{p}{,} \PYG{n}{X\PYGZus{}test}\PYG{p}{,} \PYG{n}{y\PYGZus{}test}\PYG{p}{)}\PYG{p}{;}

\PYG{c+c1}{\PYGZsh{} cm = metrics.confusion\PYGZus{}matrix(y\PYGZus{}test, predictions\PYGZus{}test, labels=svm\PYGZus{}model.classes\PYGZus{})}
\PYG{c+c1}{\PYGZsh{} disp = metrics.ConfusionMatrixDisplay(confusion\PYGZus{}matrix=cm, display\PYGZus{}labels=svm\PYGZus{}model.classes\PYGZus{})}
\PYG{c+c1}{\PYGZsh{} disp.plot();}
\end{sphinxVerbatim}

\end{sphinxuseclass}\end{sphinxVerbatimInput}

\end{sphinxuseclass}
\end{sphinxuseclass}
\begin{sphinxuseclass}{cell}
\begin{sphinxuseclass}{tag_remove-input}\begin{sphinxVerbatimOutput}

\begin{sphinxuseclass}{cell_output}
\begin{sphinxVerbatim}[commandchars=\\\{\}]
\PYGZlt{}IPython.core.display.HTML object\PYGZgt{}
\end{sphinxVerbatim}

\end{sphinxuseclass}\end{sphinxVerbatimOutput}

\end{sphinxuseclass}
\end{sphinxuseclass}
\sphinxAtStartPar
94\%, which still seems fairly good ({\hyperref[\detokenize{task2_c/task2_part_c:task2-part-c-application-performance}]{\sphinxcrossref{\DUrole{std,std-ref}{but what is “good” accuracy?}}}}), but if selecting the test data randomly (try \sphinxcode{\sphinxupquote{random\_state=42}}), accuracy may \sphinxstyleemphasis{improve} on the test data because the set is relatively small and the model is fairly accurate. Using these results, the model can be further refined. However, continually tweaking parameters according to the test data results means we are back to studying from the answers, i.e., reintroducing the risk of overfitting. To deal with this, a \sphinxstyleemphasis{third set} can be withheld, called a “validation set,” to analyze the final results.

\sphinxAtStartPar
But Partitioning available data into three sets reduces the available data for training the model, making results more dependent on the random selection of training, testing, and validation sets. \sphinxhref{https://scikit-learn.org/stable/modules/cross\_validation.html}{Cross\sphinxhyphen{}validation} addresses this issue by resampling the data. Again, this is optional but could be very useful, particularly for small data sets.

\begin{sphinxuseclass}{cell}\begin{sphinxVerbatimInput}

\begin{sphinxuseclass}{cell_input}
\begin{sphinxVerbatim}[commandchars=\\\{\}]
\PYG{k+kn}{from} \PYG{n+nn}{sklearn}\PYG{n+nn}{.}\PYG{n+nn}{model\PYGZus{}selection} \PYG{k+kn}{import} \PYG{n}{KFold}\PYG{p}{,} \PYG{n}{cross\PYGZus{}val\PYGZus{}score}
\PYG{n}{k\PYGZus{}folds} \PYG{o}{=} \PYG{n}{KFold}\PYG{p}{(}\PYG{n}{n\PYGZus{}splits} \PYG{o}{=} \PYG{l+m+mi}{5}\PYG{p}{,} \PYG{n}{shuffle}\PYG{o}{=}\PYG{k+kc}{True}\PYG{p}{)}
\PYG{c+c1}{\PYGZsh{} The number of folds determines the test/train split for each iteration. }
\PYG{c+c1}{\PYGZsh{} So 5 folds has 5 different mutually exclusive training sets. }
\PYG{c+c1}{\PYGZsh{} That\PYGZsq{}s a 1 to 4 (or .20 to .80) testing/training split for each of the 5 iterations.}

\PYG{n}{scores} \PYG{o}{=} \PYG{n}{cross\PYGZus{}val\PYGZus{}score}\PYG{p}{(}\PYG{n}{svm\PYGZus{}model}\PYG{p}{,} \PYG{n}{X}\PYG{p}{,} \PYG{n}{y}\PYG{p}{)}
\PYG{c+c1}{\PYGZsh{} This shows the average score. Print \PYGZsq{}scores\PYGZsq{} to see an array of individual iteration scores.}
\PYG{n+nb}{print}\PYG{p}{(}\PYG{l+s+s2}{\PYGZdq{}}\PYG{l+s+s2}{Average Score: }\PYG{l+s+s2}{\PYGZdq{}}\PYG{p}{,} \PYG{n}{scores}\PYG{o}{.}\PYG{n}{mean}\PYG{p}{(}\PYG{p}{)}\PYG{p}{)}
\end{sphinxVerbatim}

\end{sphinxuseclass}\end{sphinxVerbatimInput}
\begin{sphinxVerbatimOutput}

\begin{sphinxuseclass}{cell_output}
\begin{sphinxVerbatim}[commandchars=\\\{\}]
Average Score:  0.9666666666666666
\end{sphinxVerbatim}

\end{sphinxuseclass}\end{sphinxVerbatimOutput}

\end{sphinxuseclass}

\subsubsection{More testing and development}
\label{\detokenize{task2_c/example_sup_class/sup_class_ex-accuracy:more-testing-and-development}}
\sphinxAtStartPar
Now we can further develop the model until our heart’s content. Refine the model through a cyclic process guided by knowledge and experimentation. Research, try different algorithms, and adjust. They’ve built the libraries for this so that the additional coding effort will be lite. These steps are optional and not required, but this is where things become more exciting and challenging.

\sphinxAtStartPar
Machine learning is a mix of art and science, requiring a balance of knowledge, intuition, and lots of \sphinxstyleemphasis{experimentation}. Research, play around, tweak, and constantly re\sphinxhyphen{}run code.


\subsubsection{Logistic Regression}
\label{\detokenize{task2_c/example_sup_class/sup_class_ex-accuracy:logistic-regression}}
\sphinxAtStartPar
Logistic regression predicts the probability of something being in a category (hence its “regression” name). That probability indicates whether it’s in that category, e.g., \(.65 > .50 \Rightarrow \text{yes}\)  (so it’s also a classification method). We’ll use \sphinxhref{https://scikit-learn.org/stable/modules/generated/sklearn.linear\_model.LogisticRegression.html}{sklearn logistic regression} to do the latter.

\begin{sphinxShadowBox}
\sphinxstylesidebartitle{Logisitc regression or classification?}

\sphinxAtStartPar
Really both, but most often, it’s used for classification. Logistic regression uses input variables to predict the \sphinxstyleemphasis{probability} of an outcome, a number between 0.0 and 1.0 \sphinxhyphen{}hence “regression.” However, using that probability to predict whether an outcome occurs (yes/no) is classification.
\end{sphinxShadowBox}

\begin{sphinxuseclass}{cell}\begin{sphinxVerbatimInput}

\begin{sphinxuseclass}{cell_input}
\begin{sphinxVerbatim}[commandchars=\\\{\}]
\PYG{k+kn}{from} \PYG{n+nn}{sklearn}\PYG{n+nn}{.}\PYG{n+nn}{linear\PYGZus{}model} \PYG{k+kn}{import} \PYG{n}{LogisticRegression}
\PYG{n}{log\PYGZus{}model} \PYG{o}{=} \PYG{n}{LogisticRegression}\PYG{p}{(}\PYG{n}{random\PYGZus{}state}\PYG{o}{=}\PYG{l+m+mi}{0}\PYG{p}{)}\PYG{o}{.}\PYG{n}{fit}\PYG{p}{(}\PYG{n}{X\PYGZus{}train}\PYG{p}{,} \PYG{n}{y\PYGZus{}train}\PYG{p}{)}
\end{sphinxVerbatim}

\end{sphinxuseclass}\end{sphinxVerbatimInput}

\end{sphinxuseclass}
\sphinxAtStartPar
That’s right, a new model in just two lines of code. This is typical if you stay within the same library. From here we can test, improve, and compare.

\begin{sphinxuseclass}{cell}\begin{sphinxVerbatimInput}

\begin{sphinxuseclass}{cell_input}
\begin{sphinxVerbatim}[commandchars=\\\{\}]
\PYG{n}{predictions\PYGZus{}log} \PYG{o}{=} \PYG{n}{log\PYGZus{}model}\PYG{o}{.}\PYG{n}{predict}\PYG{p}{(}\PYG{n}{X\PYGZus{}test}\PYG{p}{)}
\PYG{n}{score} \PYG{o}{=} \PYG{n}{metrics}\PYG{o}{.}\PYG{n}{accuracy\PYGZus{}score}\PYG{p}{(}\PYG{n}{y\PYGZus{}test}\PYG{p}{,} \PYG{n}{predictions\PYGZus{}log}\PYG{p}{)}
\PYG{n}{score}
\end{sphinxVerbatim}

\end{sphinxuseclass}\end{sphinxVerbatimInput}
\begin{sphinxVerbatimOutput}

\begin{sphinxuseclass}{cell_output}
\begin{sphinxVerbatim}[commandchars=\\\{\}]
0.92
\end{sphinxVerbatim}

\end{sphinxuseclass}\end{sphinxVerbatimOutput}

\end{sphinxuseclass}


\sphinxstepscope

\begin{sphinxuseclass}{cell}
\begin{sphinxuseclass}{tag_hide-input}\begin{sphinxVerbatimOutput}

\begin{sphinxuseclass}{cell_output}
\begin{sphinxVerbatim}[commandchars=\\\{\}]
SVC(C=1)
\end{sphinxVerbatim}

\end{sphinxuseclass}\end{sphinxVerbatimOutput}

\end{sphinxuseclass}
\end{sphinxuseclass}

\subsection{User Interface}
\label{\detokenize{task2_c/example_sup_class/sup_class_ex-ui:user-interface}}\label{\detokenize{task2_c/example_sup_class/sup_class_ex-ui:sup-class-ex-user-interface}}\label{\detokenize{task2_c/example_sup_class/sup_class_ex-ui::doc}}
\sphinxAtStartPar
We’ve made an application. Now we need a way for the user to apply it. There are \sphinxstyleemphasis{no specific requirements for how this must be done.} Following the \DUrole{xref,myst}{User Guide} in your documentation, the evaluator must be able to get it to work and meet the needs of the problem described in the documentation. We’ll present a few options. Remember that simpler interfaces need a more detailed \DUrole{xref,myst}{User Guide}.


\subsubsection{User inputs and runs code}
\label{\detokenize{task2_c/example_sup_class/sup_class_ex-ui:user-inputs-and-runs-code}}\label{\detokenize{task2_c/example_sup_class/sup_class_ex-ui:sup-class-ex-ui-code}}
\sphinxAtStartPar
The user can be instructed to input and run code. If you do this, provide explicit instructions and provide an example that can be copied and pasted.
\begin{quote}

\sphinxAtStartPar
To make a prediction for varaibles x1, x2, x3 and x4, type

\sphinxAtStartPar
\sphinxcode{\sphinxupquote{print(svm\_model.predict({[}{[}x1,x2,x3,x4{]}{]}))}}

\sphinxAtStartPar
into the code cell below and press the ‘Run’ button in the menu.
Example:
\end{quote}

\begin{sphinxuseclass}{cell}\begin{sphinxVerbatimInput}

\begin{sphinxuseclass}{cell_input}
\begin{sphinxVerbatim}[commandchars=\\\{\}]
\PYG{c+c1}{\PYGZsh{} Note that the model was trained with X = df.drop(columns=[\PYGZsq{}type\PYGZsq{}]).values}
\PYG{n+nb}{print}\PYG{p}{(}\PYG{n}{svm\PYGZus{}model}\PYG{o}{.}\PYG{n}{predict}\PYG{p}{(}\PYG{p}{[}\PYG{p}{[}\PYG{l+m+mi}{5}\PYG{p}{,} \PYG{l+m+mi}{4}\PYG{p}{,} \PYG{l+m+mi}{1}\PYG{p}{,} \PYG{l+m+mf}{.5}\PYG{p}{]}\PYG{p}{]}\PYG{p}{)}\PYG{p}{)}
\end{sphinxVerbatim}

\end{sphinxuseclass}\end{sphinxVerbatimInput}
\begin{sphinxVerbatimOutput}

\begin{sphinxuseclass}{cell_output}
\begin{sphinxVerbatim}[commandchars=\\\{\}]
[\PYGZsq{}Iris\PYGZhy{}setosa\PYGZsq{}]
\end{sphinxVerbatim}

\end{sphinxuseclass}\end{sphinxVerbatimOutput}

\end{sphinxuseclass}
\begin{sphinxadmonition}{note}{Note:}
\sphinxAtStartPar
When making predictions, the input should look \sphinxstyleemphasis{exactly} like the data the model as trained with. In this example, ‘svm\_model’ was trained with a 2d\sphinxhyphen{}array. Hence the need for the double brackets, \sphinxcode{\sphinxupquote{{[}{[}5,4,1,.5{]}{]}}} to avoid a \sphinxcode{\sphinxupquote{ValueError: Expected 2D array, got 1D array instead}} and converting \sphinxcode{\sphinxupquote{X}} with \sphinxcode{\sphinxupquote{.values}} to avoid a \sphinxcode{\sphinxupquote{UserWarning}} for missing feature names.
\end{sphinxadmonition}


\subsubsection{User inputs from Widget text boxes \& buttons}
\label{\detokenize{task2_c/example_sup_class/sup_class_ex-ui:user-inputs-from-widget-text-boxes-buttons}}
\sphinxAtStartPar
Using Jupyter Widgets, a more user\sphinxhyphen{}friendly (and less error\sphinxhyphen{}prone) interface can be implemented.



\begin{sphinxadmonition}{warning}{Warning:}
\sphinxAtStartPar
Running Python code requires a running Python kernel. Click the  –> \sphinxguilabel{Live Code} button above on this page, and run the code below. 🚧 This site is under construction! As of now, the Python kernel may run slowly or not at all.👷🏽‍♀️
\end{sphinxadmonition}

\sphinxAtStartPar
Click the  –> \sphinxguilabel{Live Code} button above on this page, and run the code below.

\sphinxAtStartPar
Recall, the feature names:

\begin{sphinxuseclass}{cell}
\begin{sphinxuseclass}{tag_hide-input}\begin{sphinxVerbatimOutput}

\begin{sphinxuseclass}{cell_output}
\begin{sphinxVerbatim}[commandchars=\\\{\}]
[\PYGZsq{}sepal\PYGZhy{}length\PYGZsq{}, \PYGZsq{}sepal\PYGZhy{}width\PYGZsq{}, \PYGZsq{}petal\PYGZhy{}length\PYGZsq{}, \PYGZsq{}petal\PYGZhy{}width\PYGZsq{}, \PYGZsq{}type\PYGZsq{}]
\end{sphinxVerbatim}

\end{sphinxuseclass}\end{sphinxVerbatimOutput}

\end{sphinxuseclass}
\end{sphinxuseclass}
\begin{sphinxuseclass}{cell}
\begin{sphinxuseclass}{tag_thebe-init}\begin{sphinxVerbatimInput}

\begin{sphinxuseclass}{cell_input}
\begin{sphinxVerbatim}[commandchars=\\\{\}]
\PYG{k+kn}{from} \PYG{n+nn}{ipywidgets} \PYG{k+kn}{import} \PYG{n}{widgets}

\PYG{c+c1}{\PYGZsh{}The text boxes where the user can input values.}
\PYG{n}{sl\PYGZus{}widget} \PYG{o}{=} \PYG{n}{widgets}\PYG{o}{.}\PYG{n}{FloatText}\PYG{p}{(}\PYG{n}{description}\PYG{o}{=}\PYG{l+s+s1}{\PYGZsq{}}\PYG{l+s+s1}{sepal L:}\PYG{l+s+s1}{\PYGZsq{}}\PYG{p}{,} \PYG{n}{value}\PYG{o}{=}\PYG{l+s+s1}{\PYGZsq{}}\PYG{l+s+s1}{0}\PYG{l+s+s1}{\PYGZsq{}}\PYG{p}{)}
\PYG{n}{sw\PYGZus{}widget} \PYG{o}{=} \PYG{n}{widgets}\PYG{o}{.}\PYG{n}{FloatText}\PYG{p}{(}\PYG{n}{description}\PYG{o}{=}\PYG{l+s+s1}{\PYGZsq{}}\PYG{l+s+s1}{sepal W:}\PYG{l+s+s1}{\PYGZsq{}}\PYG{p}{,} \PYG{n}{value}\PYG{o}{=}\PYG{l+s+s1}{\PYGZsq{}}\PYG{l+s+s1}{0}\PYG{l+s+s1}{\PYGZsq{}}\PYG{p}{)}
\PYG{n}{pl\PYGZus{}widget} \PYG{o}{=} \PYG{n}{widgets}\PYG{o}{.}\PYG{n}{FloatText}\PYG{p}{(}\PYG{n}{description}\PYG{o}{=}\PYG{l+s+s1}{\PYGZsq{}}\PYG{l+s+s1}{petal L:}\PYG{l+s+s1}{\PYGZsq{}}\PYG{p}{,} \PYG{n}{value}\PYG{o}{=}\PYG{l+s+s1}{\PYGZsq{}}\PYG{l+s+s1}{0}\PYG{l+s+s1}{\PYGZsq{}}\PYG{p}{)}
\PYG{n}{pw\PYGZus{}widget} \PYG{o}{=} \PYG{n}{widgets}\PYG{o}{.}\PYG{n}{FloatText}\PYG{p}{(}\PYG{n}{description}\PYG{o}{=}\PYG{l+s+s1}{\PYGZsq{}}\PYG{l+s+s1}{petal W:}\PYG{l+s+s1}{\PYGZsq{}}\PYG{p}{,} \PYG{n}{value}\PYG{o}{=}\PYG{l+s+s1}{\PYGZsq{}}\PYG{l+s+s1}{0}\PYG{l+s+s1}{\PYGZsq{}}\PYG{p}{)}

\PYG{c+c1}{\PYGZsh{}A button for the user to get predictions using input valus. }
\PYG{n}{button\PYGZus{}predict} \PYG{o}{=} \PYG{n}{widgets}\PYG{o}{.}\PYG{n}{Button}\PYG{p}{(} \PYG{n}{description}\PYG{o}{=}\PYG{l+s+s1}{\PYGZsq{}}\PYG{l+s+s1}{Predict}\PYG{l+s+s1}{\PYGZsq{}} \PYG{p}{)}
\PYG{n}{button\PYGZus{}ouput} \PYG{o}{=} \PYG{n}{widgets}\PYG{o}{.}\PYG{n}{Label}\PYG{p}{(}\PYG{n}{value}\PYG{o}{=}\PYG{l+s+s1}{\PYGZsq{}}\PYG{l+s+s1}{Enter values and press the }\PYG{l+s+se}{\PYGZbs{}\PYGZdq{}}\PYG{l+s+s1}{Predict}\PYG{l+s+se}{\PYGZbs{}\PYGZdq{}}\PYG{l+s+s1}{ button.}\PYG{l+s+s1}{\PYGZsq{}} \PYG{p}{)}

\PYG{c+c1}{\PYGZsh{}Defines what happens when you click the button }
\PYG{k}{def} \PYG{n+nf}{on\PYGZus{}click\PYGZus{}predict}\PYG{p}{(}\PYG{n}{b}\PYG{p}{)}\PYG{p}{:}
    \PYG{n}{predicition} \PYG{o}{=} \PYG{n}{svm\PYGZus{}model}\PYG{o}{.}\PYG{n}{predict}\PYG{p}{(}\PYG{p}{[}\PYG{p}{[}
        \PYG{n}{sl\PYGZus{}widget}\PYG{o}{.}\PYG{n}{value}\PYG{p}{,} \PYG{n}{sw\PYGZus{}widget}\PYG{o}{.}\PYG{n}{value}\PYG{p}{,} \PYG{n}{pl\PYGZus{}widget}\PYG{o}{.}\PYG{n}{value}\PYG{p}{,} \PYG{n}{pw\PYGZus{}widget}\PYG{o}{.}\PYG{n}{value}\PYG{p}{]}\PYG{p}{]}\PYG{p}{)}
    \PYG{n}{button\PYGZus{}ouput}\PYG{o}{.}\PYG{n}{value}\PYG{o}{=}\PYG{l+s+s1}{\PYGZsq{}}\PYG{l+s+s1}{Prediction = }\PYG{l+s+s1}{\PYGZsq{}}\PYG{o}{+} \PYG{n+nb}{str}\PYG{p}{(}\PYG{n}{predicition}\PYG{p}{[}\PYG{l+m+mi}{0}\PYG{p}{]}\PYG{p}{)}
\PYG{n}{button\PYGZus{}predict}\PYG{o}{.}\PYG{n}{on\PYGZus{}click}\PYG{p}{(}\PYG{n}{on\PYGZus{}click\PYGZus{}predict}\PYG{p}{)}

\PYG{c+c1}{\PYGZsh{}Displays the text boxes and buttons inside a VBox }
\PYG{n}{vb}\PYG{o}{=}\PYG{n}{widgets}\PYG{o}{.}\PYG{n}{VBox}\PYG{p}{(}\PYG{p}{[}\PYG{n}{sl\PYGZus{}widget}\PYG{p}{,} \PYG{n}{sw\PYGZus{}widget}\PYG{p}{,} \PYG{n}{pl\PYGZus{}widget}\PYG{p}{,} \PYG{n}{pw\PYGZus{}widget}\PYG{p}{,} \PYG{n}{button\PYGZus{}predict}\PYG{p}{,}\PYG{n}{button\PYGZus{}ouput}\PYG{p}{]}\PYG{p}{)}
\PYG{n+nb}{print}\PYG{p}{(}\PYG{l+s+s1}{\PYGZsq{}}\PYG{l+s+se}{\PYGZbs{}033}\PYG{l+s+s1}{[1m}\PYG{l+s+s1}{\PYGZsq{}} \PYG{o}{+} \PYG{l+s+s1}{\PYGZsq{}}\PYG{l+s+s1}{Enter values in cm and make a prediction}\PYG{l+s+s1}{\PYGZsq{}} \PYG{o}{+} \PYG{l+s+s1}{\PYGZsq{}}\PYG{l+s+se}{\PYGZbs{}033}\PYG{l+s+s1}{[0m}\PYG{l+s+s1}{\PYGZsq{}}\PYG{p}{)}
\PYG{n}{display}\PYG{p}{(}\PYG{n}{vb}\PYG{p}{)}

\PYG{c+c1}{\PYGZsh{} According to the widget docs, }
\PYG{c+c1}{\PYGZsh{} https://ipywidgets.readthedocs.io/en/7.6.3/examples/Widget\PYGZpc{}20Styling.html}
\PYG{c+c1}{\PYGZsh{} you cannot adjust the description length. For adjusting widget display behavior, }
\PYG{c+c1}{\PYGZsh{} you can use a labeled HBox contained in the VBox.}
\end{sphinxVerbatim}

\end{sphinxuseclass}\end{sphinxVerbatimInput}
\begin{sphinxVerbatimOutput}

\begin{sphinxuseclass}{cell_output}
\begin{sphinxVerbatim}[commandchars=\\\{\}]
\PYG{Color+ColorBold}{Enter values in cm and make a prediction}
\end{sphinxVerbatim}

\begin{sphinxVerbatim}[commandchars=\\\{\}]
VBox(children=(FloatText(value=0.0, description=\PYGZsq{}sepal L:\PYGZsq{}), FloatText(value=0.0, description=\PYGZsq{}sepal W:\PYGZsq{}), Flo…
\end{sphinxVerbatim}

\end{sphinxuseclass}\end{sphinxVerbatimOutput}

\end{sphinxuseclass}
\end{sphinxuseclass}

\subsubsection{User inputs with Widget sliders}
\label{\detokenize{task2_c/example_sup_class/sup_class_ex-ui:user-inputs-with-widget-sliders}}
\sphinxAtStartPar
Sliders provide a user\sphinxhyphen{}friendly experience that can easily be modified to control input range and increments. While sliders might not be the best choice here (text entry might be easier for selecting precise values), we’ll present an example as, in many cases, sliders work great.

\sphinxAtStartPar
Implementation is almost identical to that of text entry. Reviewing the \DUrole{xref,myst}{data’s statistics}, we set the sliders’ ranges to capture approximately 95\% of the flower’s parameter values:

\sphinxAtStartPar
\(\text{range}= \text{mean}\pm 2(\text{standard deviation})\)

\sphinxAtStartPar
Assuming it’s normally distributed (it’s close enough). Capturing 99.7\% of the data using 3 standard deviations might’ve been better \sphinxhyphen{}but you get the idea.

\sphinxAtStartPar
For example,

\begin{sphinxShadowBox}
\sphinxstylesidebartitle{}

\sphinxAtStartPar
See \sphinxhref{https://ipywidgets.readthedocs.io/en/7.6.3/examples/Widget\%20List.html\#FloatSlider}{widget slider docs} for more details and options.
\end{sphinxShadowBox}

\begin{sphinxuseclass}{cell}\begin{sphinxVerbatimInput}

\begin{sphinxuseclass}{cell_input}
\begin{sphinxVerbatim}[commandchars=\\\{\}]
\PYG{n}{df}\PYG{o}{.}\PYG{n}{describe}\PYG{p}{(}\PYG{p}{)}
\end{sphinxVerbatim}

\end{sphinxuseclass}\end{sphinxVerbatimInput}
\begin{sphinxVerbatimOutput}

\begin{sphinxuseclass}{cell_output}
\begin{sphinxVerbatim}[commandchars=\\\{\}]
       sepal\PYGZhy{}length  sepal\PYGZhy{}width  petal\PYGZhy{}length  petal\PYGZhy{}width
count    150.000000   150.000000    150.000000   150.000000
mean       5.843333     3.054000      3.758667     1.198667
std        0.828066     0.433594      1.764420     0.763161
min        4.300000     2.000000      1.000000     0.100000
25\PYGZpc{}        5.100000     2.800000      1.600000     0.300000
50\PYGZpc{}        5.800000     3.000000      4.350000     1.300000
75\PYGZpc{}        6.400000     3.300000      5.100000     1.800000
max        7.900000     4.400000      6.900000     2.500000
\end{sphinxVerbatim}

\end{sphinxuseclass}\end{sphinxVerbatimOutput}

\end{sphinxuseclass}
\begin{sphinxuseclass}{cell}\begin{sphinxVerbatimInput}

\begin{sphinxuseclass}{cell_input}
\begin{sphinxVerbatim}[commandchars=\\\{\}]
\PYG{n}{feature} \PYG{o}{=} \PYG{l+s+s1}{\PYGZsq{}}\PYG{l+s+s1}{petal\PYGZhy{}width}\PYG{l+s+s1}{\PYGZsq{}}
\PYG{n}{r\PYGZus{}max} \PYG{o}{=} \PYG{n+nb}{str}\PYG{p}{(}\PYG{n}{df}\PYG{p}{[}\PYG{n}{feature}\PYG{p}{]}\PYG{o}{.}\PYG{n}{describe}\PYG{p}{(}\PYG{p}{)}\PYG{p}{[}\PYG{l+s+s1}{\PYGZsq{}}\PYG{l+s+s1}{mean}\PYG{l+s+s1}{\PYGZsq{}}\PYG{p}{]}\PYG{o}{+}\PYG{l+m+mi}{2}\PYG{o}{*}\PYG{n}{df}\PYG{p}{[}\PYG{n}{feature}\PYG{p}{]}\PYG{o}{.}\PYG{n}{describe}\PYG{p}{(}\PYG{p}{)}\PYG{p}{[}\PYG{l+s+s1}{\PYGZsq{}}\PYG{l+s+s1}{std}\PYG{l+s+s1}{\PYGZsq{}}\PYG{p}{]}\PYG{p}{)}
\PYG{n}{r\PYGZus{}min} \PYG{o}{=} \PYG{n+nb}{str}\PYG{p}{(}\PYG{n}{df}\PYG{p}{[}\PYG{n}{feature}\PYG{p}{]}\PYG{o}{.}\PYG{n}{describe}\PYG{p}{(}\PYG{p}{)}\PYG{p}{[}\PYG{l+s+s1}{\PYGZsq{}}\PYG{l+s+s1}{mean}\PYG{l+s+s1}{\PYGZsq{}}\PYG{p}{]}\PYG{o}{\PYGZhy{}}\PYG{l+m+mi}{2}\PYG{o}{*}\PYG{n}{df}\PYG{p}{[}\PYG{n}{feature}\PYG{p}{]}\PYG{o}{.}\PYG{n}{describe}\PYG{p}{(}\PYG{p}{)}\PYG{p}{[}\PYG{l+s+s1}{\PYGZsq{}}\PYG{l+s+s1}{std}\PYG{l+s+s1}{\PYGZsq{}}\PYG{p}{]}\PYG{p}{)}

\PYG{n+nb}{print}\PYG{p}{(}\PYG{l+s+s1}{\PYGZsq{}}\PYG{l+s+s1}{min=}\PYG{l+s+s1}{\PYGZsq{}}\PYG{o}{+}\PYG{n}{r\PYGZus{}min} \PYG{o}{+}\PYG{l+s+s1}{\PYGZsq{}}\PYG{l+s+s1}{, min=}\PYG{l+s+s1}{\PYGZsq{}}\PYG{o}{+}\PYG{n}{r\PYGZus{}max}\PYG{p}{)}
\end{sphinxVerbatim}

\end{sphinxuseclass}\end{sphinxVerbatimInput}
\begin{sphinxVerbatimOutput}

\begin{sphinxuseclass}{cell_output}
\begin{sphinxVerbatim}[commandchars=\\\{\}]
min=\PYGZhy{}0.3276548167350155, min=2.724988150068349
\end{sphinxVerbatim}

\end{sphinxuseclass}\end{sphinxVerbatimOutput}

\end{sphinxuseclass}
\sphinxAtStartPar
Similarly, finding ranges for each independent variable, the sliders are set up.

\begin{sphinxuseclass}{cell}\begin{sphinxVerbatimInput}

\begin{sphinxuseclass}{cell_input}
\begin{sphinxVerbatim}[commandchars=\\\{\}]
\PYG{c+c1}{\PYGZsh{}The sliders where the user can input values. Min and max are set by using the complete datasets\PYGZsq{} }
\PYG{n}{sl\PYGZus{}widget} \PYG{o}{=} \PYG{n}{widgets}\PYG{o}{.}\PYG{n}{FloatSlider}\PYG{p}{(}\PYG{n}{description}\PYG{o}{=}\PYG{l+s+s1}{\PYGZsq{}}\PYG{l+s+s1}{sepal L:}\PYG{l+s+s1}{\PYGZsq{}}\PYG{p}{,}\PYG{n+nb}{min}\PYG{o}{=}\PYG{l+m+mf}{4.19}\PYG{p}{,} \PYG{n+nb}{max}\PYG{o}{=}\PYG{l+m+mf}{7.4}\PYG{p}{)}
\PYG{n}{sw\PYGZus{}widget} \PYG{o}{=} \PYG{n}{widgets}\PYG{o}{.}\PYG{n}{FloatSlider}\PYG{p}{(}\PYG{n}{description}\PYG{o}{=}\PYG{l+s+s1}{\PYGZsq{}}\PYG{l+s+s1}{sepal W:}\PYG{l+s+s1}{\PYGZsq{}}\PYG{p}{,} \PYG{n+nb}{min}\PYG{o}{=}\PYG{l+m+mf}{2.19}\PYG{p}{,} \PYG{n+nb}{max}\PYG{o}{=}\PYG{l+m+mf}{3.9}\PYG{p}{)}
\PYG{n}{pl\PYGZus{}widget} \PYG{o}{=} \PYG{n}{widgets}\PYG{o}{.}\PYG{n}{FloatSlider}\PYG{p}{(}\PYG{n}{description}\PYG{o}{=}\PYG{l+s+s1}{\PYGZsq{}}\PYG{l+s+s1}{petal L:}\PYG{l+s+s1}{\PYGZsq{}}\PYG{p}{,} \PYG{n+nb}{min}\PYG{o}{=}\PYG{l+m+mf}{0.23}\PYG{p}{,} \PYG{n+nb}{max}\PYG{o}{=}\PYG{l+m+mf}{7.29}\PYG{p}{)}
\PYG{n}{pw\PYGZus{}widget} \PYG{o}{=} \PYG{n}{widgets}\PYG{o}{.}\PYG{n}{FloatSlider}\PYG{p}{(}\PYG{n}{description}\PYG{o}{=}\PYG{l+s+s1}{\PYGZsq{}}\PYG{l+s+s1}{petal W:}\PYG{l+s+s1}{\PYGZsq{}}\PYG{p}{,} \PYG{n+nb}{min}\PYG{o}{=}\PYG{l+m+mf}{0.0}\PYG{p}{,} \PYG{n+nb}{max}\PYG{o}{=}\PYG{l+m+mf}{2.72}\PYG{p}{)}

\PYG{c+c1}{\PYGZsh{}A button for the user to get predictions using input valus. }
\PYG{n}{button\PYGZus{}predict} \PYG{o}{=} \PYG{n}{widgets}\PYG{o}{.}\PYG{n}{Button}\PYG{p}{(} \PYG{n}{description}\PYG{o}{=}\PYG{l+s+s1}{\PYGZsq{}}\PYG{l+s+s1}{Predict}\PYG{l+s+s1}{\PYGZsq{}} \PYG{p}{)}
\PYG{n}{button\PYGZus{}ouput} \PYG{o}{=} \PYG{n}{widgets}\PYG{o}{.}\PYG{n}{Label}\PYG{p}{(}\PYG{n}{value}\PYG{o}{=}\PYG{l+s+s1}{\PYGZsq{}}\PYG{l+s+s1}{Enter values and press the }\PYG{l+s+se}{\PYGZbs{}\PYGZdq{}}\PYG{l+s+s1}{Predict}\PYG{l+s+se}{\PYGZbs{}\PYGZdq{}}\PYG{l+s+s1}{ button.}\PYG{l+s+s1}{\PYGZsq{}} \PYG{p}{)}

\PYG{c+c1}{\PYGZsh{}Defines what happens when you click the button }
\PYG{k}{def} \PYG{n+nf}{on\PYGZus{}click\PYGZus{}predict}\PYG{p}{(}\PYG{n}{b}\PYG{p}{)}\PYG{p}{:}
    \PYG{n}{predicition} \PYG{o}{=} \PYG{n}{svm\PYGZus{}model}\PYG{o}{.}\PYG{n}{predict}\PYG{p}{(}\PYG{p}{[}\PYG{p}{[}
        \PYG{n}{sl\PYGZus{}widget}\PYG{o}{.}\PYG{n}{value}\PYG{p}{,} \PYG{n}{sw\PYGZus{}widget}\PYG{o}{.}\PYG{n}{value}\PYG{p}{,} \PYG{n}{pl\PYGZus{}widget}\PYG{o}{.}\PYG{n}{value}\PYG{p}{,} \PYG{n}{pw\PYGZus{}widget}\PYG{o}{.}\PYG{n}{value}\PYG{p}{]}\PYG{p}{]}\PYG{p}{)}
    \PYG{n}{button\PYGZus{}ouput}\PYG{o}{.}\PYG{n}{value}\PYG{o}{=}\PYG{l+s+s1}{\PYGZsq{}}\PYG{l+s+s1}{Prediction = }\PYG{l+s+s1}{\PYGZsq{}} \PYG{o}{+} \PYG{n+nb}{str}\PYG{p}{(}\PYG{n}{predicition}\PYG{p}{[}\PYG{l+m+mi}{0}\PYG{p}{]}\PYG{p}{)}
\PYG{n}{button\PYGZus{}predict}\PYG{o}{.}\PYG{n}{on\PYGZus{}click}\PYG{p}{(}\PYG{n}{on\PYGZus{}click\PYGZus{}predict}\PYG{p}{)}

\PYG{c+c1}{\PYGZsh{}Displays the text boxes and button inside a VBox }
\PYG{n}{vb}\PYG{o}{=}\PYG{n}{widgets}\PYG{o}{.}\PYG{n}{VBox}\PYG{p}{(}\PYG{p}{[}\PYG{n}{sl\PYGZus{}widget}\PYG{p}{,} \PYG{n}{sw\PYGZus{}widget}\PYG{p}{,} \PYG{n}{pl\PYGZus{}widget}\PYG{p}{,} \PYG{n}{pw\PYGZus{}widget}\PYG{p}{,} \PYG{n}{button\PYGZus{}predict}\PYG{p}{,}\PYG{n}{button\PYGZus{}ouput}\PYG{p}{]}\PYG{p}{)}
\PYG{n+nb}{print}\PYG{p}{(}\PYG{l+s+s1}{\PYGZsq{}}\PYG{l+s+se}{\PYGZbs{}033}\PYG{l+s+s1}{[1m}\PYG{l+s+s1}{\PYGZsq{}} \PYG{o}{+} \PYG{l+s+s1}{\PYGZsq{}}\PYG{l+s+s1}{Enter parameter values (in cm) and make a prediction:}\PYG{l+s+s1}{\PYGZsq{}} \PYG{o}{+} \PYG{l+s+s1}{\PYGZsq{}}\PYG{l+s+se}{\PYGZbs{}033}\PYG{l+s+s1}{[0m}\PYG{l+s+s1}{\PYGZsq{}}\PYG{p}{)}
\PYG{n}{display}\PYG{p}{(}\PYG{n}{vb}\PYG{p}{)}
\end{sphinxVerbatim}

\end{sphinxuseclass}\end{sphinxVerbatimInput}
\begin{sphinxVerbatimOutput}

\begin{sphinxuseclass}{cell_output}
\begin{sphinxVerbatim}[commandchars=\\\{\}]
\PYG{Color+ColorBold}{Enter parameter values (in cm) and make a prediction:}
\end{sphinxVerbatim}

\begin{sphinxVerbatim}[commandchars=\\\{\}]
VBox(children=(FloatSlider(value=4.19, description=\PYGZsq{}sepal L:\PYGZsq{}, max=7.4, min=4.19), FloatSlider(value=2.19, des…
\end{sphinxVerbatim}

\end{sphinxuseclass}\end{sphinxVerbatimOutput}

\end{sphinxuseclass}
\sphinxAtStartPar
To automatically update values from a Widget, see \sphinxhref{https://discourse.jupyter.org/t/is-it-possible-to-get-the-current-value-of-a-widget-slider-from-a-function-without-using-multithreading/15524}{get the current value of a widget} and \sphinxhref{https://stackoverflow.com/questions/54412449/ipywidgets-automatically-update-variable-and-run-code-after-altering-widget-val\#:~:text=import\%20ipywidgets\%20as\%20widgets\%20class\%20Updated:\%20def\%20\_\_init\_\_,(v):\%20update\_class.update\%20(v)\%20slider.observe\%20(on\_change,\%20names=\%27value\%27)\%20display\%20(slider)}{automatically run code after altering widgets}.


\subsubsection{Other Input Methods}
\label{\detokenize{task2_c/example_sup_class/sup_class_ex-ui:other-input-methods}}
\sphinxAtStartPar
What user interface approaches are allowed? Anything the evaluator (playing the role of the client) can get to work following your instructions. Consider user\sphinxhyphen{}friendliness when choosing your interface. With four independent variables, four input boxes or sliders work fine. But what if your model uses 400 variables? Or say the client needed to classify not one but hundreds of flowers? In such cases, the user could be directed to copy or upload their data, say a .xlsx or .csv file, to a location the model can retrieve and analyze. Whatever method you choose, don’t make things difficult for the evaluator. Provide explicit instructions, examples, or (when appropriate) example data files.

\sphinxstepscope


\section{Example: Supervised Regression App}
\label{\detokenize{task2_c/example_sup_reg/sup_reg_ex:example-supervised-regression-app}}\label{\detokenize{task2_c/example_sup_reg/sup_reg_ex:sup-reg-ex}}\label{\detokenize{task2_c/example_sup_reg/sup_reg_ex::doc}}
\sphinxAtStartPar
To predict a number for a feature contained in the data, use a supervised \sphinxstyleemphasis{regression} method (but not {\hyperref[\detokenize{task1:task1-choosing-topic-logistic}]{\sphinxcrossref{\DUrole{std,std-ref}{logistic regression}}}}).

\sphinxAtStartPar
For this example, we’ll slightly modify the {\hyperref[\detokenize{task2_c/example_sup_class/sup_class_ex:sup-class-ex}]{\sphinxcrossref{\DUrole{std,std-ref}{previous example}}}}. Instead of predicting the category \sphinxstyleemphasis{type}, we’ll predict the number \sphinxstyleemphasis{sepal\sphinxhyphen{}length}.

\begin{sphinxuseclass}{cell}
\begin{sphinxuseclass}{tag_hide-input}\begin{sphinxVerbatimOutput}

\begin{sphinxuseclass}{cell_output}
\begin{sphinxVerbatim}[commandchars=\\\{\}]
\PYGZlt{}pandas.io.formats.style.Styler at 0x2077f899f50\PYGZgt{}
\end{sphinxVerbatim}

\end{sphinxuseclass}\end{sphinxVerbatimOutput}

\end{sphinxuseclass}
\end{sphinxuseclass}
\sphinxAtStartPar
The highlighted numbers, ‘sepal\sphinxhyphen{}length,’ provides something to predict (dependent variables), and the non\sphinxhyphen{}highlighted columns are something by which to make that prediction (independent variables). The big differences from the previous example are as follows:
\begin{itemize}
\item {} 
\sphinxAtStartPar
\sphinxstylestrong{Data processing} (maybe) if we choose to include \sphinxstyleemphasis{type} as an independent variable, it’ll need to be converted from categorical data into numbers the model can use.

\item {} 
\sphinxAtStartPar
\sphinxstylestrong{Model Development} As we’ll be predicting a number, a \sphinxstyleemphasis{regression} method will be used instead of a classification method.

\item {} 
\sphinxAtStartPar
\sphinxstylestrong{Accuracy Metric} instead of a simple percentage, we’ll need a measurement of how close the data fits the model. e.g., mean squared error.

\end{itemize}


\subsection{Data Exploring and Processing}
\label{\detokenize{task2_c/example_sup_reg/sup_reg_ex:data-exploring-and-processing}}
\sphinxAtStartPar
As the data is identical, this step will be similar to what was done in the {\hyperref[\detokenize{task2_c/example_sup_class/sup_class_ex-process:sup-class-ex-data}]{\sphinxcrossref{\DUrole{std,std-ref}{previous example}}}}; please refer to it. Focusing on the \sphinxstyleemphasis{sepal length}, we can certainly see patterns:

\begin{sphinxuseclass}{cell}
\begin{sphinxuseclass}{tag_hide-input}\begin{sphinxVerbatimOutput}

\begin{sphinxuseclass}{cell_output}
\noindent\sphinxincludegraphics{{4d03f8457c1206e4e557a2bbf1b9ff17643e34b0e31874e2afd7355079f7a7b9}.png}

\end{sphinxuseclass}\end{sphinxVerbatimOutput}

\end{sphinxuseclass}
\end{sphinxuseclass}
\sphinxstepscope


\subsection{Regression Model Development (part 1)}
\label{\detokenize{task2_c/example_sup_reg/sup_reg_ex_develop:regression-model-development-part-1}}\label{\detokenize{task2_c/example_sup_reg/sup_reg_ex_develop:sup-reg-ex-develop}}\label{\detokenize{task2_c/example_sup_reg/sup_reg_ex_develop::doc}}
\sphinxAtStartPar
Supervised algorithms use inputs (independent variables) and labeled outputs (dependent variable \sphinxhyphen{}the “answers”) to create a model that can measure its performance and learn over time. Splitting the data into independent and dependent variables, we have the following (again, this will be very similar to the {\hyperref[\detokenize{task2_c/example_sup_class/sup_class_ex-develop:sup-class-ex-develop}]{\sphinxcrossref{\DUrole{std,std-ref}{previous example}}}}):

\begin{sphinxuseclass}{cell}
\begin{sphinxuseclass}{tag_hide-input}
\end{sphinxuseclass}
\end{sphinxuseclass}

\subsubsection{Train Model}
\label{\detokenize{task2_c/example_sup_reg/sup_reg_ex_develop:train-model}}\label{\detokenize{task2_c/example_sup_reg/sup_reg_ex_develop:sup-reg-ex-develop-train}}
\sphinxAtStartPar
Recall, splitting the data into training and testing sets is not required, but it is good practice. Furthermore, it provides content for part D. As with the {\hyperref[\detokenize{task2_c/example_sup_class/sup_class_ex-develop:sup-class-ex-develop}]{\sphinxcrossref{\DUrole{std,std-ref}{previous example}}}}, we’ll use \sphinxhref{https://scikit-learn.org/stable/modules/generated/sklearn.model\_selection.train\_test\_split.html}{scikit\sphinxhyphen{}learn aka sklearn} built\sphinxhyphen{}ins for this.

\begin{sphinxuseclass}{cell}\begin{sphinxVerbatimInput}

\begin{sphinxuseclass}{cell_input}
\begin{sphinxVerbatim}[commandchars=\\\{\}]
\PYG{k+kn}{import} \PYG{n+nn}{numpy} \PYG{k}{as} \PYG{n+nn}{np}
\PYG{k+kn}{from} \PYG{n+nn}{sklearn}\PYG{n+nn}{.}\PYG{n+nn}{model\PYGZus{}selection} \PYG{k+kn}{import} \PYG{n}{train\PYGZus{}test\PYGZus{}split}

\PYG{c+c1}{\PYGZsh{}split the variable sets into training and testing subsets}
\PYG{n}{X\PYGZus{}train}\PYG{p}{,} \PYG{n}{X\PYGZus{}test}\PYG{p}{,} \PYG{n}{y\PYGZus{}train}\PYG{p}{,} \PYG{n}{y\PYGZus{}test} \PYG{o}{=} \PYG{n}{train\PYGZus{}test\PYGZus{}split}\PYG{p}{(}\PYG{n}{X}\PYG{p}{,} \PYG{n}{y}\PYG{p}{,} \PYG{n}{test\PYGZus{}size}\PYG{o}{=}\PYG{l+m+mf}{0.333}\PYG{p}{,} \PYG{n}{random\PYGZus{}state}\PYG{o}{=}\PYG{l+m+mi}{41}\PYG{p}{)}
\end{sphinxVerbatim}

\end{sphinxuseclass}\end{sphinxVerbatimInput}

\end{sphinxuseclass}
\begin{sphinxuseclass}{cell}
\begin{sphinxuseclass}{tag_hide-input}\begin{sphinxVerbatimOutput}

\begin{sphinxuseclass}{cell_output}
\end{sphinxuseclass}\end{sphinxVerbatimOutput}

\end{sphinxuseclass}
\end{sphinxuseclass}
\sphinxAtStartPar
Review sklearn’s nice \sphinxhref{https://scikit-learn.org/stable/supervised\_learning.html}{supervised learning library}. Note that many of these models have both classification and regression extensions.

\begin{sphinxuseclass}{cell}\begin{sphinxVerbatimInput}

\begin{sphinxuseclass}{cell_input}
\begin{sphinxVerbatim}[commandchars=\\\{\}]
\PYG{k+kn}{from} \PYG{n+nn}{sklearn}\PYG{n+nn}{.}\PYG{n+nn}{linear\PYGZus{}model} \PYG{k+kn}{import} \PYG{n}{LinearRegression} 
\end{sphinxVerbatim}

\end{sphinxuseclass}\end{sphinxVerbatimInput}

\end{sphinxuseclass}
\sphinxAtStartPar
Our data is mostly quantitative and the scatterplots indicate some linear relations between variables. So linear regression isn’t a bad place to start. Once we’ve trained and tested a linear regression model, we’ll easily be able to experiment with different algorithms.

\begin{sphinxShadowBox}
\sphinxstylesidebartitle{Is linear regression ML?}

\sphinxAtStartPar
It depends on who you ask. Google “Is linear regression machine learning?” and you’ll see some interesting (and entertaining) discussion. For the capstone, it applies an algorithm to data so the answer is \sphinxhyphen{}yes.
\end{sphinxShadowBox}

\begin{sphinxuseclass}{cell}
\begin{sphinxuseclass}{tag_scroll-output}\begin{sphinxVerbatimInput}

\begin{sphinxuseclass}{cell_input}
\begin{sphinxVerbatim}[commandchars=\\\{\}]
\PYG{n}{linear\PYGZus{}reg\PYGZus{}model} \PYG{o}{=} \PYG{n}{LinearRegression}\PYG{p}{(}\PYG{p}{)}
\PYG{n}{linear\PYGZus{}reg\PYGZus{}model}\PYG{o}{.}\PYG{n}{fit}\PYG{p}{(}\PYG{n}{X\PYGZus{}train}\PYG{p}{,}\PYG{n}{y\PYGZus{}train}\PYG{p}{)}
\end{sphinxVerbatim}

\end{sphinxuseclass}\end{sphinxVerbatimInput}
\begin{sphinxVerbatimOutput}

\begin{sphinxuseclass}{cell_output}
\begin{sphinxVerbatim}[commandchars=\\\{\}]
\PYG{g+gt}{\PYGZhy{}\PYGZhy{}\PYGZhy{}\PYGZhy{}\PYGZhy{}\PYGZhy{}\PYGZhy{}\PYGZhy{}\PYGZhy{}\PYGZhy{}\PYGZhy{}\PYGZhy{}\PYGZhy{}\PYGZhy{}\PYGZhy{}\PYGZhy{}\PYGZhy{}\PYGZhy{}\PYGZhy{}\PYGZhy{}\PYGZhy{}\PYGZhy{}\PYGZhy{}\PYGZhy{}\PYGZhy{}\PYGZhy{}\PYGZhy{}\PYGZhy{}\PYGZhy{}\PYGZhy{}\PYGZhy{}\PYGZhy{}\PYGZhy{}\PYGZhy{}\PYGZhy{}\PYGZhy{}\PYGZhy{}\PYGZhy{}\PYGZhy{}\PYGZhy{}\PYGZhy{}\PYGZhy{}\PYGZhy{}\PYGZhy{}\PYGZhy{}\PYGZhy{}\PYGZhy{}\PYGZhy{}\PYGZhy{}\PYGZhy{}\PYGZhy{}\PYGZhy{}\PYGZhy{}\PYGZhy{}\PYGZhy{}\PYGZhy{}\PYGZhy{}\PYGZhy{}\PYGZhy{}\PYGZhy{}\PYGZhy{}\PYGZhy{}\PYGZhy{}\PYGZhy{}\PYGZhy{}\PYGZhy{}\PYGZhy{}\PYGZhy{}\PYGZhy{}\PYGZhy{}\PYGZhy{}\PYGZhy{}\PYGZhy{}\PYGZhy{}\PYGZhy{}}
\PYG{n+ne}{ValueError}\PYG{g+gWhitespace}{                                }Traceback (most recent call last)
\PYG{n}{Cell} \PYG{n}{In}\PYG{p}{[}\PYG{l+m+mi}{5}\PYG{p}{]}\PYG{p}{,} \PYG{n}{line} \PYG{l+m+mi}{2}
\PYG{g+gWhitespace}{      }\PYG{l+m+mi}{1} \PYG{n}{linear\PYGZus{}reg\PYGZus{}model} \PYG{o}{=} \PYG{n}{LinearRegression}\PYG{p}{(}\PYG{p}{)}
\PYG{n+ne}{\PYGZhy{}\PYGZhy{}\PYGZhy{}\PYGZhy{}\PYGZgt{} }\PYG{l+m+mi}{2} \PYG{n}{linear\PYGZus{}reg\PYGZus{}model}\PYG{o}{.}\PYG{n}{fit}\PYG{p}{(}\PYG{n}{X\PYGZus{}train}\PYG{p}{,}\PYG{n}{y\PYGZus{}train}\PYG{p}{)}

\PYG{n+nn}{File \PYGZti{}\PYGZbs{}.virtualenvs\PYGZbs{}jupyter\PYGZhy{}books\PYGZhy{}WZpnkDri\PYGZbs{}Lib\PYGZbs{}site\PYGZhy{}packages\PYGZbs{}sklearn\PYGZbs{}linear\PYGZus{}model\PYGZbs{}\PYGZus{}base.py:649,} in \PYG{n+ni}{LinearRegression.fit}\PYG{n+nt}{(self, X, y, sample\PYGZus{}weight)}
\PYG{g+gWhitespace}{    }\PYG{l+m+mi}{645} \PYG{n}{n\PYGZus{}jobs\PYGZus{}} \PYG{o}{=} \PYG{n+nb+bp}{self}\PYG{o}{.}\PYG{n}{n\PYGZus{}jobs}
\PYG{g+gWhitespace}{    }\PYG{l+m+mi}{647} \PYG{n}{accept\PYGZus{}sparse} \PYG{o}{=} \PYG{k+kc}{False} \PYG{k}{if} \PYG{n+nb+bp}{self}\PYG{o}{.}\PYG{n}{positive} \PYG{k}{else} \PYG{p}{[}\PYG{l+s+s2}{\PYGZdq{}}\PYG{l+s+s2}{csr}\PYG{l+s+s2}{\PYGZdq{}}\PYG{p}{,} \PYG{l+s+s2}{\PYGZdq{}}\PYG{l+s+s2}{csc}\PYG{l+s+s2}{\PYGZdq{}}\PYG{p}{,} \PYG{l+s+s2}{\PYGZdq{}}\PYG{l+s+s2}{coo}\PYG{l+s+s2}{\PYGZdq{}}\PYG{p}{]}
\PYG{n+ne}{\PYGZhy{}\PYGZhy{}\PYGZgt{} }\PYG{l+m+mi}{649} \PYG{n}{X}\PYG{p}{,} \PYG{n}{y} \PYG{o}{=} \PYG{n+nb+bp}{self}\PYG{o}{.}\PYG{n}{\PYGZus{}validate\PYGZus{}data}\PYG{p}{(}
\PYG{g+gWhitespace}{    }\PYG{l+m+mi}{650}     \PYG{n}{X}\PYG{p}{,} \PYG{n}{y}\PYG{p}{,} \PYG{n}{accept\PYGZus{}sparse}\PYG{o}{=}\PYG{n}{accept\PYGZus{}sparse}\PYG{p}{,} \PYG{n}{y\PYGZus{}numeric}\PYG{o}{=}\PYG{k+kc}{True}\PYG{p}{,} \PYG{n}{multi\PYGZus{}output}\PYG{o}{=}\PYG{k+kc}{True}
\PYG{g+gWhitespace}{    }\PYG{l+m+mi}{651} \PYG{p}{)}
\PYG{g+gWhitespace}{    }\PYG{l+m+mi}{653} \PYG{n}{sample\PYGZus{}weight} \PYG{o}{=} \PYG{n}{\PYGZus{}check\PYGZus{}sample\PYGZus{}weight}\PYG{p}{(}
\PYG{g+gWhitespace}{    }\PYG{l+m+mi}{654}     \PYG{n}{sample\PYGZus{}weight}\PYG{p}{,} \PYG{n}{X}\PYG{p}{,} \PYG{n}{dtype}\PYG{o}{=}\PYG{n}{X}\PYG{o}{.}\PYG{n}{dtype}\PYG{p}{,} \PYG{n}{only\PYGZus{}non\PYGZus{}negative}\PYG{o}{=}\PYG{k+kc}{True}
\PYG{g+gWhitespace}{    }\PYG{l+m+mi}{655} \PYG{p}{)}
\PYG{g+gWhitespace}{    }\PYG{l+m+mi}{657} \PYG{n}{X}\PYG{p}{,} \PYG{n}{y}\PYG{p}{,} \PYG{n}{X\PYGZus{}offset}\PYG{p}{,} \PYG{n}{y\PYGZus{}offset}\PYG{p}{,} \PYG{n}{X\PYGZus{}scale} \PYG{o}{=} \PYG{n}{\PYGZus{}preprocess\PYGZus{}data}\PYG{p}{(}
\PYG{g+gWhitespace}{    }\PYG{l+m+mi}{658}     \PYG{n}{X}\PYG{p}{,}
\PYG{g+gWhitespace}{    }\PYG{l+m+mi}{659}     \PYG{n}{y}\PYG{p}{,}
   \PYG{p}{(}\PYG{o}{.}\PYG{o}{.}\PYG{o}{.}\PYG{p}{)}
\PYG{g+gWhitespace}{    }\PYG{l+m+mi}{662}     \PYG{n}{sample\PYGZus{}weight}\PYG{o}{=}\PYG{n}{sample\PYGZus{}weight}\PYG{p}{,}
\PYG{g+gWhitespace}{    }\PYG{l+m+mi}{663} \PYG{p}{)}

\PYG{n+nn}{File \PYGZti{}\PYGZbs{}.virtualenvs\PYGZbs{}jupyter\PYGZhy{}books\PYGZhy{}WZpnkDri\PYGZbs{}Lib\PYGZbs{}site\PYGZhy{}packages\PYGZbs{}sklearn\PYGZbs{}base.py:554,} in \PYG{n+ni}{BaseEstimator.\PYGZus{}validate\PYGZus{}data}\PYG{n+nt}{(self, X, y, reset, validate\PYGZus{}separately, **check\PYGZus{}params)}
\PYG{g+gWhitespace}{    }\PYG{l+m+mi}{552}         \PYG{n}{y} \PYG{o}{=} \PYG{n}{check\PYGZus{}array}\PYG{p}{(}\PYG{n}{y}\PYG{p}{,} \PYG{n}{input\PYGZus{}name}\PYG{o}{=}\PYG{l+s+s2}{\PYGZdq{}}\PYG{l+s+s2}{y}\PYG{l+s+s2}{\PYGZdq{}}\PYG{p}{,} \PYG{o}{*}\PYG{o}{*}\PYG{n}{check\PYGZus{}y\PYGZus{}params}\PYG{p}{)}
\PYG{g+gWhitespace}{    }\PYG{l+m+mi}{553}     \PYG{k}{else}\PYG{p}{:}
\PYG{n+ne}{\PYGZhy{}\PYGZhy{}\PYGZgt{} }\PYG{l+m+mi}{554}         \PYG{n}{X}\PYG{p}{,} \PYG{n}{y} \PYG{o}{=} \PYG{n}{check\PYGZus{}X\PYGZus{}y}\PYG{p}{(}\PYG{n}{X}\PYG{p}{,} \PYG{n}{y}\PYG{p}{,} \PYG{o}{*}\PYG{o}{*}\PYG{n}{check\PYGZus{}params}\PYG{p}{)}
\PYG{g+gWhitespace}{    }\PYG{l+m+mi}{555}     \PYG{n}{out} \PYG{o}{=} \PYG{n}{X}\PYG{p}{,} \PYG{n}{y}
\PYG{g+gWhitespace}{    }\PYG{l+m+mi}{557} \PYG{k}{if} \PYG{o+ow}{not} \PYG{n}{no\PYGZus{}val\PYGZus{}X} \PYG{o+ow}{and} \PYG{n}{check\PYGZus{}params}\PYG{o}{.}\PYG{n}{get}\PYG{p}{(}\PYG{l+s+s2}{\PYGZdq{}}\PYG{l+s+s2}{ensure\PYGZus{}2d}\PYG{l+s+s2}{\PYGZdq{}}\PYG{p}{,} \PYG{k+kc}{True}\PYG{p}{)}\PYG{p}{:}

\PYG{n+nn}{File \PYGZti{}\PYGZbs{}.virtualenvs\PYGZbs{}jupyter\PYGZhy{}books\PYGZhy{}WZpnkDri\PYGZbs{}Lib\PYGZbs{}site\PYGZhy{}packages\PYGZbs{}sklearn\PYGZbs{}utils\PYGZbs{}validation.py:1104,} in \PYG{n+ni}{check\PYGZus{}X\PYGZus{}y}\PYG{n+nt}{(X, y, accept\PYGZus{}sparse, accept\PYGZus{}large\PYGZus{}sparse, dtype, order, copy, force\PYGZus{}all\PYGZus{}finite, ensure\PYGZus{}2d, allow\PYGZus{}nd, multi\PYGZus{}output, ensure\PYGZus{}min\PYGZus{}samples, ensure\PYGZus{}min\PYGZus{}features, y\PYGZus{}numeric, estimator)}
\PYG{g+gWhitespace}{   }\PYG{l+m+mi}{1099}         \PYG{n}{estimator\PYGZus{}name} \PYG{o}{=} \PYG{n}{\PYGZus{}check\PYGZus{}estimator\PYGZus{}name}\PYG{p}{(}\PYG{n}{estimator}\PYG{p}{)}
\PYG{g+gWhitespace}{   }\PYG{l+m+mi}{1100}     \PYG{k}{raise} \PYG{n+ne}{ValueError}\PYG{p}{(}
\PYG{g+gWhitespace}{   }\PYG{l+m+mi}{1101}         \PYG{l+s+sa}{f}\PYG{l+s+s2}{\PYGZdq{}}\PYG{l+s+si}{\PYGZob{}}\PYG{n}{estimator\PYGZus{}name}\PYG{l+s+si}{\PYGZcb{}}\PYG{l+s+s2}{ requires y to be passed, but the target y is None}\PYG{l+s+s2}{\PYGZdq{}}
\PYG{g+gWhitespace}{   }\PYG{l+m+mi}{1102}     \PYG{p}{)}
\PYG{n+ne}{\PYGZhy{}\PYGZgt{} }\PYG{l+m+mi}{1104} \PYG{n}{X} \PYG{o}{=} \PYG{n}{check\PYGZus{}array}\PYG{p}{(}
\PYG{g+gWhitespace}{   }\PYG{l+m+mi}{1105}     \PYG{n}{X}\PYG{p}{,}
\PYG{g+gWhitespace}{   }\PYG{l+m+mi}{1106}     \PYG{n}{accept\PYGZus{}sparse}\PYG{o}{=}\PYG{n}{accept\PYGZus{}sparse}\PYG{p}{,}
\PYG{g+gWhitespace}{   }\PYG{l+m+mi}{1107}     \PYG{n}{accept\PYGZus{}large\PYGZus{}sparse}\PYG{o}{=}\PYG{n}{accept\PYGZus{}large\PYGZus{}sparse}\PYG{p}{,}
\PYG{g+gWhitespace}{   }\PYG{l+m+mi}{1108}     \PYG{n}{dtype}\PYG{o}{=}\PYG{n}{dtype}\PYG{p}{,}
\PYG{g+gWhitespace}{   }\PYG{l+m+mi}{1109}     \PYG{n}{order}\PYG{o}{=}\PYG{n}{order}\PYG{p}{,}
\PYG{g+gWhitespace}{   }\PYG{l+m+mi}{1110}     \PYG{n}{copy}\PYG{o}{=}\PYG{n}{copy}\PYG{p}{,}
\PYG{g+gWhitespace}{   }\PYG{l+m+mi}{1111}     \PYG{n}{force\PYGZus{}all\PYGZus{}finite}\PYG{o}{=}\PYG{n}{force\PYGZus{}all\PYGZus{}finite}\PYG{p}{,}
\PYG{g+gWhitespace}{   }\PYG{l+m+mi}{1112}     \PYG{n}{ensure\PYGZus{}2d}\PYG{o}{=}\PYG{n}{ensure\PYGZus{}2d}\PYG{p}{,}
\PYG{g+gWhitespace}{   }\PYG{l+m+mi}{1113}     \PYG{n}{allow\PYGZus{}nd}\PYG{o}{=}\PYG{n}{allow\PYGZus{}nd}\PYG{p}{,}
\PYG{g+gWhitespace}{   }\PYG{l+m+mi}{1114}     \PYG{n}{ensure\PYGZus{}min\PYGZus{}samples}\PYG{o}{=}\PYG{n}{ensure\PYGZus{}min\PYGZus{}samples}\PYG{p}{,}
\PYG{g+gWhitespace}{   }\PYG{l+m+mi}{1115}     \PYG{n}{ensure\PYGZus{}min\PYGZus{}features}\PYG{o}{=}\PYG{n}{ensure\PYGZus{}min\PYGZus{}features}\PYG{p}{,}
\PYG{g+gWhitespace}{   }\PYG{l+m+mi}{1116}     \PYG{n}{estimator}\PYG{o}{=}\PYG{n}{estimator}\PYG{p}{,}
\PYG{g+gWhitespace}{   }\PYG{l+m+mi}{1117}     \PYG{n}{input\PYGZus{}name}\PYG{o}{=}\PYG{l+s+s2}{\PYGZdq{}}\PYG{l+s+s2}{X}\PYG{l+s+s2}{\PYGZdq{}}\PYG{p}{,}
\PYG{g+gWhitespace}{   }\PYG{l+m+mi}{1118} \PYG{p}{)}
\PYG{g+gWhitespace}{   }\PYG{l+m+mi}{1120} \PYG{n}{y} \PYG{o}{=} \PYG{n}{\PYGZus{}check\PYGZus{}y}\PYG{p}{(}\PYG{n}{y}\PYG{p}{,} \PYG{n}{multi\PYGZus{}output}\PYG{o}{=}\PYG{n}{multi\PYGZus{}output}\PYG{p}{,} \PYG{n}{y\PYGZus{}numeric}\PYG{o}{=}\PYG{n}{y\PYGZus{}numeric}\PYG{p}{,} \PYG{n}{estimator}\PYG{o}{=}\PYG{n}{estimator}\PYG{p}{)}
\PYG{g+gWhitespace}{   }\PYG{l+m+mi}{1122} \PYG{n}{check\PYGZus{}consistent\PYGZus{}length}\PYG{p}{(}\PYG{n}{X}\PYG{p}{,} \PYG{n}{y}\PYG{p}{)}

\PYG{n+nn}{File \PYGZti{}\PYGZbs{}.virtualenvs\PYGZbs{}jupyter\PYGZhy{}books\PYGZhy{}WZpnkDri\PYGZbs{}Lib\PYGZbs{}site\PYGZhy{}packages\PYGZbs{}sklearn\PYGZbs{}utils\PYGZbs{}validation.py:877,} in \PYG{n+ni}{check\PYGZus{}array}\PYG{n+nt}{(array, accept\PYGZus{}sparse, accept\PYGZus{}large\PYGZus{}sparse, dtype, order, copy, force\PYGZus{}all\PYGZus{}finite, ensure\PYGZus{}2d, allow\PYGZus{}nd, ensure\PYGZus{}min\PYGZus{}samples, ensure\PYGZus{}min\PYGZus{}features, estimator, input\PYGZus{}name)}
\PYG{g+gWhitespace}{    }\PYG{l+m+mi}{875}         \PYG{n}{array} \PYG{o}{=} \PYG{n}{xp}\PYG{o}{.}\PYG{n}{astype}\PYG{p}{(}\PYG{n}{array}\PYG{p}{,} \PYG{n}{dtype}\PYG{p}{,} \PYG{n}{copy}\PYG{o}{=}\PYG{k+kc}{False}\PYG{p}{)}
\PYG{g+gWhitespace}{    }\PYG{l+m+mi}{876}     \PYG{k}{else}\PYG{p}{:}
\PYG{n+ne}{\PYGZhy{}\PYGZhy{}\PYGZgt{} }\PYG{l+m+mi}{877}         \PYG{n}{array} \PYG{o}{=} \PYG{n}{\PYGZus{}asarray\PYGZus{}with\PYGZus{}order}\PYG{p}{(}\PYG{n}{array}\PYG{p}{,} \PYG{n}{order}\PYG{o}{=}\PYG{n}{order}\PYG{p}{,} \PYG{n}{dtype}\PYG{o}{=}\PYG{n}{dtype}\PYG{p}{,} \PYG{n}{xp}\PYG{o}{=}\PYG{n}{xp}\PYG{p}{)}
\PYG{g+gWhitespace}{    }\PYG{l+m+mi}{878} \PYG{k}{except} \PYG{n}{ComplexWarning} \PYG{k}{as} \PYG{n}{complex\PYGZus{}warning}\PYG{p}{:}
\PYG{g+gWhitespace}{    }\PYG{l+m+mi}{879}     \PYG{k}{raise} \PYG{n+ne}{ValueError}\PYG{p}{(}
\PYG{g+gWhitespace}{    }\PYG{l+m+mi}{880}         \PYG{l+s+s2}{\PYGZdq{}}\PYG{l+s+s2}{Complex data not supported}\PYG{l+s+se}{\PYGZbs{}n}\PYG{l+s+si}{\PYGZob{}\PYGZcb{}}\PYG{l+s+se}{\PYGZbs{}n}\PYG{l+s+s2}{\PYGZdq{}}\PYG{o}{.}\PYG{n}{format}\PYG{p}{(}\PYG{n}{array}\PYG{p}{)}
\PYG{g+gWhitespace}{    }\PYG{l+m+mi}{881}     \PYG{p}{)} \PYG{k+kn}{from} \PYG{n+nn}{complex\PYGZus{}warning}

\PYG{n+nn}{File \PYGZti{}\PYGZbs{}.virtualenvs\PYGZbs{}jupyter\PYGZhy{}books\PYGZhy{}WZpnkDri\PYGZbs{}Lib\PYGZbs{}site\PYGZhy{}packages\PYGZbs{}sklearn\PYGZbs{}utils\PYGZbs{}\PYGZus{}array\PYGZus{}api.py:185,} in \PYG{n+ni}{\PYGZus{}asarray\PYGZus{}with\PYGZus{}order}\PYG{n+nt}{(array, dtype, order, copy, xp)}
\PYG{g+gWhitespace}{    }\PYG{l+m+mi}{182}     \PYG{n}{xp}\PYG{p}{,} \PYG{n}{\PYGZus{}} \PYG{o}{=} \PYG{n}{get\PYGZus{}namespace}\PYG{p}{(}\PYG{n}{array}\PYG{p}{)}
\PYG{g+gWhitespace}{    }\PYG{l+m+mi}{183} \PYG{k}{if} \PYG{n}{xp}\PYG{o}{.}\PYG{n+nv+vm}{\PYGZus{}\PYGZus{}name\PYGZus{}\PYGZus{}} \PYG{o+ow}{in} \PYG{p}{\PYGZob{}}\PYG{l+s+s2}{\PYGZdq{}}\PYG{l+s+s2}{numpy}\PYG{l+s+s2}{\PYGZdq{}}\PYG{p}{,} \PYG{l+s+s2}{\PYGZdq{}}\PYG{l+s+s2}{numpy.array\PYGZus{}api}\PYG{l+s+s2}{\PYGZdq{}}\PYG{p}{\PYGZcb{}}\PYG{p}{:}
\PYG{g+gWhitespace}{    }\PYG{l+m+mi}{184}     \PYG{c+c1}{\PYGZsh{} Use NumPy API to support order}
\PYG{n+ne}{\PYGZhy{}\PYGZhy{}\PYGZgt{} }\PYG{l+m+mi}{185}     \PYG{n}{array} \PYG{o}{=} \PYG{n}{numpy}\PYG{o}{.}\PYG{n}{asarray}\PYG{p}{(}\PYG{n}{array}\PYG{p}{,} \PYG{n}{order}\PYG{o}{=}\PYG{n}{order}\PYG{p}{,} \PYG{n}{dtype}\PYG{o}{=}\PYG{n}{dtype}\PYG{p}{)}
\PYG{g+gWhitespace}{    }\PYG{l+m+mi}{186}     \PYG{k}{return} \PYG{n}{xp}\PYG{o}{.}\PYG{n}{asarray}\PYG{p}{(}\PYG{n}{array}\PYG{p}{,} \PYG{n}{copy}\PYG{o}{=}\PYG{n}{copy}\PYG{p}{)}
\PYG{g+gWhitespace}{    }\PYG{l+m+mi}{187} \PYG{k}{else}\PYG{p}{:}

\PYG{n+nn}{File \PYGZti{}\PYGZbs{}.virtualenvs\PYGZbs{}jupyter\PYGZhy{}books\PYGZhy{}WZpnkDri\PYGZbs{}Lib\PYGZbs{}site\PYGZhy{}packages\PYGZbs{}pandas\PYGZbs{}core\PYGZbs{}generic.py:2070,} in \PYG{n+ni}{NDFrame.\PYGZus{}\PYGZus{}array\PYGZus{}\PYGZus{}}\PYG{n+nt}{(self, dtype)}
\PYG{g+gWhitespace}{   }\PYG{l+m+mi}{2069} \PYG{k}{def} \PYG{n+nf}{\PYGZus{}\PYGZus{}array\PYGZus{}\PYGZus{}}\PYG{p}{(}\PYG{n+nb+bp}{self}\PYG{p}{,} \PYG{n}{dtype}\PYG{p}{:} \PYG{n}{npt}\PYG{o}{.}\PYG{n}{DTypeLike} \PYG{o}{|} \PYG{k+kc}{None} \PYG{o}{=} \PYG{k+kc}{None}\PYG{p}{)} \PYG{o}{\PYGZhy{}}\PYG{o}{\PYGZgt{}} \PYG{n}{np}\PYG{o}{.}\PYG{n}{ndarray}\PYG{p}{:}
\PYG{n+ne}{\PYGZhy{}\PYGZgt{} }\PYG{l+m+mi}{2070}     \PYG{k}{return} \PYG{n}{np}\PYG{o}{.}\PYG{n}{asarray}\PYG{p}{(}\PYG{n+nb+bp}{self}\PYG{o}{.}\PYG{n}{\PYGZus{}values}\PYG{p}{,} \PYG{n}{dtype}\PYG{o}{=}\PYG{n}{dtype}\PYG{p}{)}

\PYG{n+ne}{ValueError}: could not convert string to float: \PYGZsq{}Iris\PYGZhy{}virginica\PYGZsq{}
\end{sphinxVerbatim}

\end{sphinxuseclass}\end{sphinxVerbatimOutput}

\end{sphinxuseclass}
\end{sphinxuseclass}
\sphinxAtStartPar
\sphinxstyleemphasis{Error? Wait what happened!?!} This error returns a lot of output, but the last line makes things clear:

\sphinxAtStartPar
\sphinxcode{\sphinxupquote{ValueError: could not convert string to float: 'Iris\sphinxhyphen{}virginica'}}

\sphinxAtStartPar
The algorithm expected numbers; it does not know what to do with the flower types (strings). So how do we fix this?


\subsubsection{Processing Categorical Data (the easy way)}
\label{\detokenize{task2_c/example_sup_reg/sup_reg_ex_develop:processing-categorical-data-the-easy-way}}\label{\detokenize{task2_c/example_sup_reg/sup_reg_ex_develop:sup-reg-ex-develop-train-categorical-1}}
\sphinxAtStartPar
One way to fix a problem is to avoid it. You are not required to use all the data \sphinxhyphen{}only some of it. Sometimes choosing the right variables is the real trick. \sphinxhref{https://en.wikipedia.org/wiki/Dimensionality\_reduction}{Dimensionality reduction} is an important part of the data sciences. Here, those flower types DO matter, and it would be best to include that data \sphinxhyphen{}but goal \#1 is to get things working. Improving things is step \#2 and step \#3 and step \#4 and … step \(\# \infty\).

\sphinxAtStartPar
So just to get things rolling, let’s remove the column with the categorical data:

\begin{sphinxuseclass}{cell}
\begin{sphinxuseclass}{tag_scroll-output}\begin{sphinxVerbatimInput}

\begin{sphinxuseclass}{cell_input}
\begin{sphinxVerbatim}[commandchars=\\\{\}]
\PYG{n}{X\PYGZus{}train\PYGZus{}no\PYGZus{}type} \PYG{o}{=} \PYG{n}{X\PYGZus{}train}\PYG{o}{.}\PYG{n}{drop}\PYG{p}{(}\PYG{n}{columns} \PYG{o}{=} \PYG{p}{[}\PYG{l+s+s1}{\PYGZsq{}}\PYG{l+s+s1}{type}\PYG{l+s+s1}{\PYGZsq{}}\PYG{p}{]}\PYG{p}{)}
\PYG{n}{X\PYGZus{}test\PYGZus{}no\PYGZus{}type} \PYG{o}{=}  \PYG{n}{X\PYGZus{}test}\PYG{o}{.}\PYG{n}{drop}\PYG{p}{(}\PYG{n}{columns} \PYG{o}{=} \PYG{p}{[}\PYG{l+s+s1}{\PYGZsq{}}\PYG{l+s+s1}{type}\PYG{l+s+s1}{\PYGZsq{}}\PYG{p}{]}\PYG{p}{)}

\PYG{n}{X\PYGZus{}test\PYGZus{}no\PYGZus{}type}
\end{sphinxVerbatim}

\end{sphinxuseclass}\end{sphinxVerbatimInput}
\begin{sphinxVerbatimOutput}

\begin{sphinxuseclass}{cell_output}
\begin{sphinxVerbatim}[commandchars=\\\{\}]
     sepal\PYGZhy{}width  petal\PYGZhy{}length  petal\PYGZhy{}width
119          2.2           5.0          1.5
128          2.8           5.6          2.1
135          3.0           6.1          2.3
91           3.0           4.6          1.4
112          3.0           5.5          2.1
71           2.8           4.0          1.3
123          2.7           4.9          1.8
85           3.4           4.5          1.6
147          3.0           5.2          2.0
143          3.2           5.9          2.3
127          3.0           4.9          1.8
39           3.4           1.5          0.2
38           3.0           1.3          0.2
93           2.3           3.3          1.0
23           3.3           1.7          0.5
133          2.8           5.1          1.5
30           3.1           1.6          0.2
83           2.7           5.1          1.6
37           3.1           1.5          0.1
41           2.3           1.3          0.3
81           2.4           3.7          1.0
120          3.2           5.7          2.3
43           3.5           1.6          0.6
2            3.2           1.3          0.2
64           2.9           3.6          1.3
62           2.2           4.0          1.0
56           3.3           4.7          1.6
67           2.7           4.1          1.0
49           3.3           1.4          0.2
63           2.9           4.7          1.4
79           2.6           3.5          1.0
54           2.8           4.6          1.5
106          2.5           4.5          1.7
90           2.6           4.4          1.2
145          3.0           5.2          2.3
14           4.0           1.2          0.2
141          3.1           5.1          2.3
51           3.2           4.5          1.5
139          3.1           5.4          2.1
70           3.2           4.8          1.8
97           2.9           4.3          1.3
55           2.8           4.5          1.3
32           4.1           1.5          0.1
104          3.0           5.8          2.2
136          3.4           5.6          2.4
18           3.8           1.7          0.3
108          2.5           5.8          1.8
98           2.5           3.0          1.1
45           3.0           1.4          0.3
68           2.2           4.5          1.5
\end{sphinxVerbatim}

\end{sphinxuseclass}\end{sphinxVerbatimOutput}

\end{sphinxuseclass}
\end{sphinxuseclass}
\sphinxAtStartPar
Now the models will train without error:

\begin{sphinxuseclass}{cell}\begin{sphinxVerbatimInput}

\begin{sphinxuseclass}{cell_input}
\begin{sphinxVerbatim}[commandchars=\\\{\}]
\PYG{n}{linear\PYGZus{}reg\PYGZus{}model}\PYG{o}{.}\PYG{n}{fit}\PYG{p}{(}\PYG{n}{X\PYGZus{}train\PYGZus{}no\PYGZus{}type}\PYG{p}{,} \PYG{n}{y\PYGZus{}train}\PYG{p}{)}
\end{sphinxVerbatim}

\end{sphinxuseclass}\end{sphinxVerbatimInput}
\begin{sphinxVerbatimOutput}

\begin{sphinxuseclass}{cell_output}
\begin{sphinxVerbatim}[commandchars=\\\{\}]
LinearRegression()
\end{sphinxVerbatim}

\end{sphinxuseclass}\end{sphinxVerbatimOutput}

\end{sphinxuseclass}
\sphinxAtStartPar
And the model can make predictions for an entire set:

\begin{sphinxuseclass}{cell}
\begin{sphinxuseclass}{tag_scroll-output}\begin{sphinxVerbatimInput}

\begin{sphinxuseclass}{cell_input}
\begin{sphinxVerbatim}[commandchars=\\\{\}]
\PYG{n}{y\PYGZus{}pred\PYGZus{}no\PYGZus{}type} \PYG{o}{=} \PYG{n}{linear\PYGZus{}reg\PYGZus{}model}\PYG{o}{.}\PYG{n}{predict}\PYG{p}{(}\PYG{n}{X\PYGZus{}test\PYGZus{}no\PYGZus{}type}\PYG{p}{)}
\PYG{n}{y\PYGZus{}pred\PYGZus{}no\PYGZus{}type}
\end{sphinxVerbatim}

\end{sphinxuseclass}\end{sphinxVerbatimInput}
\begin{sphinxVerbatimOutput}

\begin{sphinxuseclass}{cell_output}
\begin{sphinxVerbatim}[commandchars=\\\{\}]
array([[5.98770812],
       [6.42220879],
       [6.81026327],
       [6.3038615 ],
       [6.48153317],
       [5.75587752],
       [6.02106191],
       [6.34587216],
       [6.31716812],
       [6.788514  ],
       [6.23165894],
       [5.01764515],
       [4.57470184],
       [5.07393461],
       [4.87302581],
       [6.48997581],
       [4.88812177],
       [6.34092093],
       [4.885904  ],
       [4.00445291],
       [5.46842817],
       [6.62636673],
       [4.85349432],
       [4.71509986],
       [5.50178197],
       [5.57125107],
       [6.43782043],
       [6.00331976],
       [4.86637251],
       [6.31473613],
       [5.44667891],
       [6.08460761],
       [5.63522521],
       [6.01862993],
       [6.08060051],
       [5.19561829],
       [6.06972588],
       [6.28433001],
       [6.47065854],
       [6.29098332],
       [6.06929744],
       [6.16124571],
       [5.58789409],
       [6.64589822],
       [6.60683523],
       [5.3817326 ],
       [6.61032665],
       [4.89225584],
       [4.57691961],
       [5.58233992]])
\end{sphinxVerbatim}

\end{sphinxuseclass}\end{sphinxVerbatimOutput}

\end{sphinxuseclass}
\end{sphinxuseclass}
\sphinxAtStartPar
Or a single input:

\begin{sphinxuseclass}{cell}\begin{sphinxVerbatimInput}

\begin{sphinxuseclass}{cell_input}
\begin{sphinxVerbatim}[commandchars=\\\{\}]
\PYG{c+c1}{\PYGZsh{} The model was trained with a dataframe, so you can only predict on dataframes}
\PYG{c+c1}{\PYGZsh{} Recall we removed the petal type, and we are predicting the sepal\PYGZhy{}length}
\PYG{n}{column\PYGZus{}names\PYGZus{}short} \PYG{o}{=} \PYG{p}{[}\PYG{l+s+s1}{\PYGZsq{}}\PYG{l+s+s1}{sepal\PYGZhy{}width}\PYG{l+s+s1}{\PYGZsq{}}\PYG{p}{,} \PYG{l+s+s1}{\PYGZsq{}}\PYG{l+s+s1}{petal\PYGZhy{}length}\PYG{l+s+s1}{\PYGZsq{}}\PYG{p}{,} \PYG{l+s+s1}{\PYGZsq{}}\PYG{l+s+s1}{petal\PYGZhy{}width}\PYG{l+s+s1}{\PYGZsq{}}\PYG{p}{]}

\PYG{c+c1}{\PYGZsh{} Creates a dataframe from a single element for input. This avoids a warning for missing feature names. }
\PYG{c+c1}{\PYGZsh{}Alternatively, you can use print(linear\PYGZus{}reg\PYGZus{}model.predict([[3.2, 1.3, .2]])) without error. }
\PYG{n}{input\PYGZus{}df} \PYG{o}{=} \PYG{n}{pd}\PYG{o}{.}\PYG{n}{DataFrame}\PYG{p}{(}\PYG{n}{np}\PYG{o}{.}\PYG{n}{array}\PYG{p}{(}\PYG{p}{[}\PYG{p}{[}\PYG{l+m+mf}{3.2}\PYG{p}{,} \PYG{l+m+mf}{1.3}\PYG{p}{,} \PYG{l+m+mf}{.2}\PYG{p}{]}\PYG{p}{]}\PYG{p}{)}\PYG{p}{,} \PYG{n}{columns} \PYG{o}{=} \PYG{n}{column\PYGZus{}names\PYGZus{}short}\PYG{p}{)}

\PYG{n+nb}{print}\PYG{p}{(}\PYG{n}{linear\PYGZus{}reg\PYGZus{}model}\PYG{o}{.}\PYG{n}{predict}\PYG{p}{(}\PYG{n}{input\PYGZus{}df}\PYG{p}{)}\PYG{p}{)}
\end{sphinxVerbatim}

\end{sphinxuseclass}\end{sphinxVerbatimInput}
\begin{sphinxVerbatimOutput}

\begin{sphinxuseclass}{cell_output}
\begin{sphinxVerbatim}[commandchars=\\\{\}]
[[4.71509986]]
\end{sphinxVerbatim}

\end{sphinxuseclass}\end{sphinxVerbatimOutput}

\end{sphinxuseclass}
\begin{sphinxadmonition}{note}{Note:}
\sphinxAtStartPar
Your model can only predict on data simliar to what it was trained with. Since this model was trained with a dataframe, a matching new dataframe, ‘input\_df’ was created to predict a single input. Alternatively, we could have converted the original data to an array using \sphinxcode{\sphinxupquote{ravel()}} or \sphinxcode{\sphinxupquote{.values}}  (see the {\hyperref[\detokenize{task2_c/example_sup_class/sup_class_ex-develop:sup-class-ex-develop-train}]{\sphinxcrossref{\DUrole{std,std-ref}{previous example}}}}).
\end{sphinxadmonition}


\subsubsection{Accuracy Analysis (for Regression) part 1}
\label{\detokenize{task2_c/example_sup_reg/sup_reg_ex_develop:accuracy-analysis-for-regression-part-1}}\label{\detokenize{task2_c/example_sup_reg/sup_reg_ex_develop:sup-reg-ex-develop-accuracy}}
\sphinxAtStartPar
Now that the model works, we can work on improving it. But first we’ll need metrics so we can tell if we’re making progress. As we are trying to predict a continuous number, even the very best model will have errors in almost every prediction (if not, it’s almost certainly \sphinxhref{https://en.wikipedia.org/wiki/Overfitting}{overfitted}). Whereas measuring the success of our classification model was a simple ratio, here we need a way to measure how much those predictions deviate from actual values. See sklearn’s list of \sphinxhref{https://scikit-learn.org/stable/modules/model\_evaluation.html}{metrics and scoring} for regression.

\sphinxAtStartPar
To get things started, we’ll use the \sphinxstyleemphasis{mean squared error} (MSE), a popular metric for evaluating regression models. Regression metrics are covered in more depth in the {[}Regression Accuracy Analysis{]}(sup\_reg\_ex: develop: accuracy) section.

\begin{sphinxuseclass}{cell}\begin{sphinxVerbatimInput}

\begin{sphinxuseclass}{cell_input}
\begin{sphinxVerbatim}[commandchars=\\\{\}]
\PYG{k+kn}{from} \PYG{n+nn}{sklearn}\PYG{n+nn}{.}\PYG{n+nn}{metrics} \PYG{k+kn}{import} \PYG{n}{mean\PYGZus{}squared\PYGZus{}error}

\PYG{n}{mean\PYGZus{}squared\PYGZus{}error}\PYG{p}{(}\PYG{n}{y\PYGZus{}test}\PYG{p}{,} \PYG{n}{y\PYGZus{}pred\PYGZus{}no\PYGZus{}type}\PYG{p}{)}
\end{sphinxVerbatim}

\end{sphinxuseclass}\end{sphinxVerbatimInput}
\begin{sphinxVerbatimOutput}

\begin{sphinxuseclass}{cell_output}
\begin{sphinxVerbatim}[commandchars=\\\{\}]
0.12264182555541722
\end{sphinxVerbatim}

\end{sphinxuseclass}\end{sphinxVerbatimOutput}

\end{sphinxuseclass}
\sphinxAtStartPar
What does this mean? The closer the MSE is to 0, the better. However, this value is not given in terms of the dependent variable, and MSE values are not comparable across use cases, i.e., comparing an MSE from your project to that of a different model is not comparing “apples to apples.” A “good” MSE will depend on your data and project needs.

\sphinxAtStartPar
This value can be used to determine if tweaks improve the results. For example, reviewing the {\hyperref[\detokenize{task2_c/example_sup_class/sup_class_ex-process:sup-class-ex-descriptive-methods-and-visualizations}]{\sphinxcrossref{\DUrole{std,std-ref}{visualizations of this dataset}}}}, we might expect that the regression coefficients (numbers determining the lines directions) should be positive, and try \sphinxcode{\sphinxupquote{LinearRegression(positive = True)}}.

\begin{sphinxuseclass}{cell}\begin{sphinxVerbatimInput}

\begin{sphinxuseclass}{cell_input}
\begin{sphinxVerbatim}[commandchars=\\\{\}]
\PYG{n}{linear\PYGZus{}reg\PYGZus{}model2} \PYG{o}{=} \PYG{n}{LinearRegression}\PYG{p}{(}\PYG{n}{positive} \PYG{o}{=} \PYG{k+kc}{True}\PYG{p}{)}
\PYG{n}{linear\PYGZus{}reg\PYGZus{}model2}\PYG{o}{.}\PYG{n}{fit}\PYG{p}{(}\PYG{n}{X\PYGZus{}train\PYGZus{}no\PYGZus{}type}\PYG{p}{,} \PYG{n}{y\PYGZus{}train}\PYG{p}{)}
\PYG{n}{y\PYGZus{}pred\PYGZus{}no\PYGZus{}type} \PYG{o}{=} \PYG{n}{linear\PYGZus{}reg\PYGZus{}model2}\PYG{o}{.}\PYG{n}{predict}\PYG{p}{(}\PYG{n}{X\PYGZus{}test\PYGZus{}no\PYGZus{}type}\PYG{p}{)}

\PYG{n}{mean\PYGZus{}squared\PYGZus{}error}\PYG{p}{(}\PYG{n}{y\PYGZus{}test}\PYG{p}{,} \PYG{n}{y\PYGZus{}pred\PYGZus{}no\PYGZus{}type}\PYG{p}{)}
\end{sphinxVerbatim}

\end{sphinxuseclass}\end{sphinxVerbatimInput}
\begin{sphinxVerbatimOutput}

\begin{sphinxuseclass}{cell_output}
\begin{sphinxVerbatim}[commandchars=\\\{\}]
0.10302108975537984
\end{sphinxVerbatim}

\end{sphinxuseclass}\end{sphinxVerbatimOutput}

\end{sphinxuseclass}
\sphinxAtStartPar
And \(0.103 < 0.122\).

\sphinxstepscope


\subsection{Regression Model Development (part 2)}
\label{\detokenize{task2_c/example_sup_reg/sup_reg_ex_develop-2:regression-model-development-part-2}}\label{\detokenize{task2_c/example_sup_reg/sup_reg_ex_develop-2:sup-reg-ex-develop-2}}\label{\detokenize{task2_c/example_sup_reg/sup_reg_ex_develop-2::doc}}
\begin{sphinxuseclass}{cell}
\begin{sphinxuseclass}{tag_hide-input}
\end{sphinxuseclass}
\end{sphinxuseclass}

\subsubsection{Processing Categorical Data (the right way)}
\label{\detokenize{task2_c/example_sup_reg/sup_reg_ex_develop-2:processing-categorical-data-the-right-way}}
\sphinxAtStartPar
In the previous section, we avoided the additional complexity of processing categorical data by simply removing it. While this sped things along, it also dropped potentially valuable insight from our analysis. Now that the code is working, we’ll rebuild our models using that categorical data \sphinxhyphen{}the \sphinxcode{\sphinxupquote{type}} feature.

\begin{sphinxuseclass}{cell}\begin{sphinxVerbatimInput}

\begin{sphinxuseclass}{cell_input}
\begin{sphinxVerbatim}[commandchars=\\\{\}]
\PYG{n}{df\PYGZus{}sample} \PYG{o}{=} \PYG{n}{df}\PYG{o}{.}\PYG{n}{sample}\PYG{p}{(}\PYG{n}{n}\PYG{o}{=}\PYG{l+m+mi}{10}\PYG{p}{,} \PYG{n}{random\PYGZus{}state} \PYG{o}{=} \PYG{l+m+mi}{152}\PYG{p}{)}
\PYG{c+c1}{\PYGZsh{} df\PYGZus{}sample\PYGZus{}highlight = pd.concat([df\PYGZus{}sample.iloc[:5,:], df\PYGZus{}sample.iloc[\PYGZhy{}5:,:]]).style.format().set\PYGZus{}properties(subset=[\PYGZsq{}type\PYGZsq{}], **\PYGZob{}\PYGZsq{}background\PYGZhy{}color\PYGZsq{}: \PYGZsq{}yellow\PYGZsq{}\PYGZcb{})}

\PYG{c+c1}{\PYGZsh{} function definition}
\PYG{k}{def} \PYG{n+nf}{highlight\PYGZus{}cols}\PYG{p}{(}\PYG{n}{s}\PYG{p}{)}\PYG{p}{:}
    \PYG{n}{color} \PYG{o}{=} \PYG{l+s+s1}{\PYGZsq{}}\PYG{l+s+s1}{null}\PYG{l+s+s1}{\PYGZsq{}}
    \PYG{k}{if} \PYG{n}{s} \PYG{o}{==} \PYG{l+s+s1}{\PYGZsq{}}\PYG{l+s+s1}{Iris\PYGZhy{}virginica}\PYG{l+s+s1}{\PYGZsq{}}\PYG{p}{:} \PYG{n}{color} \PYG{o}{=} \PYG{l+s+s1}{\PYGZsq{}}\PYG{l+s+s1}{limegreen}\PYG{l+s+s1}{\PYGZsq{}}
    \PYG{k}{elif} \PYG{n}{s} \PYG{o}{==} \PYG{l+s+s1}{\PYGZsq{}}\PYG{l+s+s1}{Iris\PYGZhy{}setosa}\PYG{l+s+s1}{\PYGZsq{}}\PYG{p}{:} \PYG{n}{color} \PYG{o}{=} \PYG{l+s+s1}{\PYGZsq{}}\PYG{l+s+s1}{lightblue}\PYG{l+s+s1}{\PYGZsq{}}
    \PYG{k}{elif} \PYG{n}{s} \PYG{o}{==} \PYG{l+s+s1}{\PYGZsq{}}\PYG{l+s+s1}{Iris\PYGZhy{}versicolor}\PYG{l+s+s1}{\PYGZsq{}}\PYG{p}{:} \PYG{n}{color} \PYG{o}{=} \PYG{l+s+s1}{\PYGZsq{}}\PYG{l+s+s1}{orange}\PYG{l+s+s1}{\PYGZsq{}}
    \PYG{c+c1}{\PYGZsh{} color = \PYGZsq{}red\PYGZsq{} if s == \PYGZsq{}Iris\PYGZhy{}virginica\PYGZsq{} or \PYGZsq{}blue\PYGZsq{} if s == \PYGZsq{}Iris\PYGZhy{}setosa\PYGZsq{}}
    \PYG{k}{return} \PYG{l+s+s1}{\PYGZsq{}}\PYG{l+s+s1}{background\PYGZhy{}color: }\PYG{l+s+si}{\PYGZpc{} s}\PYG{l+s+s1}{\PYGZsq{}} \PYG{o}{\PYGZpc{}} \PYG{n}{color}
  
\PYG{c+c1}{\PYGZsh{} highlighting the cells}
\PYG{n}{df\PYGZus{}sample}\PYG{o}{.}\PYG{n}{style}\PYG{o}{.}\PYG{n}{applymap}\PYG{p}{(}\PYG{n}{highlight\PYGZus{}cols}\PYG{p}{)}
\end{sphinxVerbatim}

\end{sphinxuseclass}\end{sphinxVerbatimInput}
\begin{sphinxVerbatimOutput}

\begin{sphinxuseclass}{cell_output}
\begin{sphinxVerbatim}[commandchars=\\\{\}]
\PYGZlt{}pandas.io.formats.style.Styler at 0x22f492828d0\PYGZgt{}
\end{sphinxVerbatim}

\end{sphinxuseclass}\end{sphinxVerbatimOutput}

\end{sphinxuseclass}
\sphinxAtStartPar
We have three mutually exclusive flower types, equally distributed, in this feature:

\begin{sphinxuseclass}{cell}
\begin{sphinxuseclass}{tag_remove-output}\begin{sphinxVerbatimInput}

\begin{sphinxuseclass}{cell_input}
\begin{sphinxVerbatim}[commandchars=\\\{\}]
\PYG{n}{df}\PYG{o}{.}\PYG{n}{groupby}\PYG{p}{(}\PYG{l+s+s1}{\PYGZsq{}}\PYG{l+s+s1}{type}\PYG{l+s+s1}{\PYGZsq{}}\PYG{p}{)}\PYG{o}{.}\PYG{n}{size}\PYG{p}{(}\PYG{p}{)}\PYG{o}{.}\PYG{n}{plot}\PYG{p}{(}\PYG{n}{kind}\PYG{o}{=}\PYG{l+s+s1}{\PYGZsq{}}\PYG{l+s+s1}{pie}\PYG{l+s+s1}{\PYGZsq{}}\PYG{p}{,}\PYG{n}{colors} \PYG{o}{=} \PYG{p}{[}\PYG{l+s+s1}{\PYGZsq{}}\PYG{l+s+s1}{lightblue}\PYG{l+s+s1}{\PYGZsq{}}\PYG{p}{,} \PYG{l+s+s1}{\PYGZsq{}}\PYG{l+s+s1}{orange}\PYG{l+s+s1}{\PYGZsq{}}\PYG{p}{,} \PYG{l+s+s1}{\PYGZsq{}}\PYG{l+s+s1}{limegreen}\PYG{l+s+s1}{\PYGZsq{}}\PYG{p}{]}\PYG{p}{)}\PYG{p}{;}
\end{sphinxVerbatim}

\end{sphinxuseclass}\end{sphinxVerbatimInput}

\end{sphinxuseclass}
\end{sphinxuseclass}
\begin{sphinxuseclass}{cell}
\begin{sphinxuseclass}{tag_remove-input}\begin{sphinxVerbatimOutput}

\begin{sphinxuseclass}{cell_output}
\begin{sphinxVerbatim}[commandchars=\\\{\}]
\PYGZlt{}IPython.core.display.HTML object\PYGZgt{}
\end{sphinxVerbatim}

\end{sphinxuseclass}\end{sphinxVerbatimOutput}

\end{sphinxuseclass}
\end{sphinxuseclass}
\sphinxAtStartPar
Recall, a coding {[}error was returned{]}(sup\_reg\_ex: develop: train) after inputting this data directly into \sphinxcode{\sphinxupquote{linear\_reg\_model.fit}}. This occurred because the algorithm did not how to process categorical independent variables. This isn’t a problem when using dependent categorical variables for classification models as in our {\hyperref[\detokenize{task2_c/example_sup_class/sup_class_ex-develop:sup-class-ex-develop}]{\sphinxcrossref{\DUrole{std,std-ref}{classification example}}}}. Those algorithms are written to expect dependent categorical variables \sphinxhyphen{}as they always classify categories.

\sphinxAtStartPar
But algorithms like numbers and there are many instances when ML models can only interpret numerical data. Furthermore, how categories should be represented requires an understanding of the data. Something the algorithm doesn’t have. Thus \sphinxstyleemphasis{feature encoding}, processing data into numerical form, is an essential data analytical skill. To do this properly you should understand your data before preceding.

\sphinxAtStartPar
For example, we could simply re\sphinxhyphen{}label the types as follows:
\begin{equation*}
\begin{split} 
  \text{Iris-setosa} \rightarrow 1 \\
  \text{Iris-versicolor} \rightarrow 2 \\
  \text{Iris-virginica} \rightarrow 3 \\
\end{split}
\end{equation*}
\sphinxAtStartPar
and hand this off to the algorithm. While this would fix the coding error, any mathematical interpretation of this re\sphinxhyphen{}labeling would be meaningless, e.g., \sphinxcode{\sphinxupquote{Iris\sphinxhyphen{}setosa}} is not twice as much as \sphinxcode{\sphinxupquote{Iris\sphinxhyphen{}versicolor}}, nor does \sphinxcode{\sphinxupquote{setosa}} + \sphinxcode{\sphinxupquote{versicolor}} = \sphinxcode{\sphinxupquote{virginica}} \sphinxhyphen{}the type is just a name. We call this type of categorical data \sphinxstyleemphasis{nominal.} Categories with an inherent order, e.g., grades, pay grades, bronze\sphinxhyphen{}silver\sphinxhyphen{}gold, etc., are called \sphinxstyleemphasis{ordinal.} But that doesn’t apply here either. A flower either is an \sphinxcode{\sphinxupquote{Iris\sphinxhyphen{}setosa}} OR it isn’t. Each type is similarly binary so we can interpret \sphinxstyleemphasis{each unique type as a unique feature}, with a 1 or 0, indicating whether the category applies or not.

\sphinxAtStartPar
Most machine learning libraries are equipped with built\sphinxhyphen{}in preprocessing functions; see the available options in the docs: \sphinxhref{https://scikit-learn.org/stable/modules/classes.html\#module-sklearn.preprocessing}{sklearn.preprocessing}. For simplicity, we’ll start with Pandas’ built\sphinxhyphen{}in \sphinxhref{https://pandas.pydata.org/docs/reference/api/pandas.get\_dummies.html}{get\_dummies}. However, it will often be best to use functions specifically written for your model’s library, such as sklearn’s \sphinxhref{https://scikit-learn.org/stable/modules/generated/sklearn.preprocessing.OneHotEncoder.html\#sklearn.preprocessing.OneHotEncoder}{OneHotEncoder}.

\begin{sphinxuseclass}{cell}\begin{sphinxVerbatimInput}

\begin{sphinxuseclass}{cell_input}
\begin{sphinxVerbatim}[commandchars=\\\{\}]
\PYG{n}{after\PYGZus{}pd\PYGZus{}dummy} \PYG{o}{=} \PYG{n}{pd}\PYG{o}{.}\PYG{n}{get\PYGZus{}dummies}\PYG{p}{(}\PYG{n}{df\PYGZus{}sample}\PYG{p}{)}
\end{sphinxVerbatim}

\end{sphinxuseclass}\end{sphinxVerbatimInput}

\end{sphinxuseclass}
\begin{sphinxuseclass}{cell}
\begin{sphinxuseclass}{tag_hide-input}\begin{sphinxVerbatimOutput}

\begin{sphinxuseclass}{cell_output}
\end{sphinxuseclass}\end{sphinxVerbatimOutput}

\end{sphinxuseclass}
\end{sphinxuseclass}
\sphinxAtStartPar
This process is called \sphinxhref{https://neptune.ai/blog/vectorization-techniques-in-nlp-guide\#:~:text=In\%20Machine\%20Learning\%2C\%20vectorization\%20is\%20a\%20step\%20in,train\%20on\%2C\%20by\%20converting\%20text\%20to\%20numerical\%20vectors}{vectorization}. Now we can include \sphinxcode{\sphinxupquote{type}} in the training and testing of a model:

\begin{sphinxuseclass}{cell}\begin{sphinxVerbatimInput}

\begin{sphinxuseclass}{cell_input}
\begin{sphinxVerbatim}[commandchars=\\\{\}]
\PYG{k+kn}{from} \PYG{n+nn}{sklearn}\PYG{n+nn}{.}\PYG{n+nn}{model\PYGZus{}selection} \PYG{k+kn}{import} \PYG{n}{train\PYGZus{}test\PYGZus{}split}
\PYG{k+kn}{from} \PYG{n+nn}{sklearn}\PYG{n+nn}{.}\PYG{n+nn}{linear\PYGZus{}model} \PYG{k+kn}{import} \PYG{n}{LinearRegression} 
\PYG{k+kn}{from} \PYG{n+nn}{sklearn}\PYG{n+nn}{.}\PYG{n+nn}{metrics} \PYG{k+kn}{import} \PYG{n}{mean\PYGZus{}squared\PYGZus{}error}

\PYG{n}{X} \PYG{o}{=} \PYG{n}{pd}\PYG{o}{.}\PYG{n}{get\PYGZus{}dummies}\PYG{p}{(}\PYG{n}{X}\PYG{p}{)}
\PYG{n}{X\PYGZus{}train}\PYG{p}{,} \PYG{n}{X\PYGZus{}test}\PYG{p}{,} \PYG{n}{y\PYGZus{}train}\PYG{p}{,} \PYG{n}{y\PYGZus{}test} \PYG{o}{=} \PYG{n}{train\PYGZus{}test\PYGZus{}split}\PYG{p}{(}\PYG{n}{X}\PYG{p}{,} \PYG{n}{y}\PYG{p}{,} \PYG{n}{test\PYGZus{}size}\PYG{o}{=}\PYG{l+m+mf}{0.333}\PYG{p}{,} \PYG{n}{random\PYGZus{}state}\PYG{o}{=}\PYG{l+m+mi}{41}\PYG{p}{)}
\PYG{n}{linear\PYGZus{}reg\PYGZus{}model\PYGZus{}types} \PYG{o}{=} \PYG{n}{LinearRegression}\PYG{p}{(}\PYG{p}{)}
\PYG{n}{linear\PYGZus{}reg\PYGZus{}model\PYGZus{}types}\PYG{o}{.}\PYG{n}{fit}\PYG{p}{(}\PYG{n}{X\PYGZus{}train}\PYG{p}{,}\PYG{n}{y\PYGZus{}train}\PYG{p}{)}
\PYG{n}{y\PYGZus{}pred} \PYG{o}{=} \PYG{n}{linear\PYGZus{}reg\PYGZus{}model\PYGZus{}types}\PYG{o}{.}\PYG{n}{predict}\PYG{p}{(}\PYG{n}{X\PYGZus{}test}\PYG{p}{)}

\PYG{n}{sme} \PYG{o}{=} \PYG{n}{mean\PYGZus{}squared\PYGZus{}error}\PYG{p}{(}\PYG{n}{y\PYGZus{}test}\PYG{p}{,} \PYG{n}{y\PYGZus{}pred}\PYG{p}{)}
\PYG{n+nb}{print}\PYG{p}{(}\PYG{l+s+s1}{\PYGZsq{}}\PYG{l+s+s1}{MSE using types is :}\PYG{l+s+s1}{\PYGZsq{}} \PYG{o}{+} \PYG{n+nb}{str}\PYG{p}{(}\PYG{n}{sme}\PYG{p}{)}\PYG{p}{)}
\end{sphinxVerbatim}

\end{sphinxuseclass}\end{sphinxVerbatimInput}
\begin{sphinxVerbatimOutput}

\begin{sphinxuseclass}{cell_output}
\begin{sphinxVerbatim}[commandchars=\\\{\}]
MSE using types is :0.12921261168601364
\end{sphinxVerbatim}

\end{sphinxuseclass}\end{sphinxVerbatimOutput}

\end{sphinxuseclass}
\sphinxAtStartPar
When including the flower types, our linear model has a mean squared error of \(\approx 0.129\). Recall from the previous section without the flower types, we had a MSE of abot \(0.123\).

\sphinxAtStartPar
So did it get worse? No. Our data changed and you can’t simply compare MSE’s from different cases. Converting \sphinxcode{\sphinxupquote{types}} to numerical features, added three dimensions to the previous example. Moving from three to a *six\sphinxhyphen{}*dimensional space. Consider the change from just two to three dimensions, e.g., \(2^{2}=4\) to \(2^{3}=8\). Adding dimensions can radically increase the volume of space making the available data relatively sparse \sphinxhyphen{}what’s known as the \sphinxhref{https://en.wikipedia.org/wiki/Curse\_of\_dimensionality}{Curse of Dimensionality}. \sphinxhyphen{}However, that’s not the full story here. While, both \sphinxcode{\sphinxupquote{petal\sphinxhyphen{}length}} and \sphinxcode{\sphinxupquote{petal\sphinxhyphen{}width}} appear positively correlated with \sphinxcode{\sphinxupquote{sepal\sphinxhyphen{}length}}for \sphinxcode{\sphinxupquote{versicolor}} and \sphinxcode{\sphinxupquote{virginica}} it does \sphinxstyleemphasis{not} for \sphinxcode{\sphinxupquote{setosa}} (blue below),

\begin{sphinxuseclass}{cell}
\begin{sphinxuseclass}{tag_remove-output}\begin{sphinxVerbatimInput}

\begin{sphinxuseclass}{cell_input}
\begin{sphinxVerbatim}[commandchars=\\\{\}]
\PYG{k+kn}{import} \PYG{n+nn}{seaborn} \PYG{k}{as} \PYG{n+nn}{sns}

\PYG{c+c1}{\PYGZsh{}correlogram}
\PYG{n}{sns}\PYG{o}{.}\PYG{n}{pairplot}\PYG{p}{(}\PYG{n}{df}\PYG{p}{,}\PYG{n}{x\PYGZus{}vars} \PYG{o}{=} \PYG{p}{[}\PYG{l+s+s1}{\PYGZsq{}}\PYG{l+s+s1}{petal\PYGZhy{}length}\PYG{l+s+s1}{\PYGZsq{}}\PYG{p}{,}\PYG{l+s+s1}{\PYGZsq{}}\PYG{l+s+s1}{petal\PYGZhy{}width}\PYG{l+s+s1}{\PYGZsq{}}\PYG{p}{]}\PYG{p}{,} \PYG{n}{y\PYGZus{}vars}\PYG{o}{=}\PYG{p}{[}\PYG{l+s+s1}{\PYGZsq{}}\PYG{l+s+s1}{sepal\PYGZhy{}length}\PYG{l+s+s1}{\PYGZsq{}}\PYG{p}{]}\PYG{p}{,} \PYG{n}{hue}\PYG{o}{=}\PYG{l+s+s1}{\PYGZsq{}}\PYG{l+s+s1}{type}\PYG{l+s+s1}{\PYGZsq{}}\PYG{p}{)}\PYG{p}{;}
\end{sphinxVerbatim}

\end{sphinxuseclass}\end{sphinxVerbatimInput}

\end{sphinxuseclass}
\end{sphinxuseclass}
\begin{sphinxuseclass}{cell}
\begin{sphinxuseclass}{tag_remove-input}\begin{sphinxVerbatimOutput}

\begin{sphinxuseclass}{cell_output}
\begin{sphinxVerbatim}[commandchars=\\\{\}]
\PYGZlt{}IPython.core.display.HTML object\PYGZgt{}
\end{sphinxVerbatim}

\end{sphinxuseclass}\end{sphinxVerbatimOutput}

\end{sphinxuseclass}
\end{sphinxuseclass}
\sphinxAtStartPar
The drop in accuracy is in part a limitation of our simple linear model trying to make use of a feature pulling the model in the wrong direction. Making the most out of your data involves a mix of understanding the data and the applied algorithm(s) (which don’t understand anything). Using three different linear models yields better results:

\begin{sphinxuseclass}{cell}\begin{sphinxVerbatimInput}

\begin{sphinxuseclass}{cell_input}
\begin{sphinxVerbatim}[commandchars=\\\{\}]
\PYG{k+kn}{from} \PYG{n+nn}{sklearn}\PYG{n+nn}{.}\PYG{n+nn}{model\PYGZus{}selection} \PYG{k+kn}{import} \PYG{n}{train\PYGZus{}test\PYGZus{}split}
\PYG{k+kn}{from} \PYG{n+nn}{sklearn}\PYG{n+nn}{.}\PYG{n+nn}{linear\PYGZus{}model} \PYG{k+kn}{import} \PYG{n}{LinearRegression} 
\PYG{k+kn}{from} \PYG{n+nn}{sklearn}\PYG{n+nn}{.}\PYG{n+nn}{metrics} \PYG{k+kn}{import} \PYG{n}{mean\PYGZus{}squared\PYGZus{}error}


\PYG{n}{df\PYGZus{}dummy} \PYG{o}{=} \PYG{n}{pd}\PYG{o}{.}\PYG{n}{get\PYGZus{}dummies}\PYG{p}{(}\PYG{n}{df}\PYG{p}{)}
\PYG{n}{df\PYGZus{}dummy}
\PYG{n}{df\PYGZus{}s} \PYG{o}{=} \PYG{n}{df\PYGZus{}dummy}\PYG{o}{.}\PYG{n}{loc}\PYG{p}{[}\PYG{n}{df\PYGZus{}dummy}\PYG{p}{[}\PYG{l+s+s1}{\PYGZsq{}}\PYG{l+s+s1}{type\PYGZus{}Iris\PYGZhy{}setosa}\PYG{l+s+s1}{\PYGZsq{}}\PYG{p}{]} \PYG{o}{==} \PYG{l+m+mi}{1}\PYG{p}{]}\PYG{o}{.}\PYG{n}{drop}\PYG{p}{(}\PYG{n}{columns}\PYG{o}{=}\PYG{p}{[}\PYG{l+s+s1}{\PYGZsq{}}\PYG{l+s+s1}{type\PYGZus{}Iris\PYGZhy{}versicolor}\PYG{l+s+s1}{\PYGZsq{}}\PYG{p}{,} \PYG{l+s+s1}{\PYGZsq{}}\PYG{l+s+s1}{type\PYGZus{}Iris\PYGZhy{}virginica}\PYG{l+s+s1}{\PYGZsq{}}\PYG{p}{]}\PYG{p}{)}
\PYG{n}{df\PYGZus{}v} \PYG{o}{=} \PYG{n}{df\PYGZus{}dummy}\PYG{o}{.}\PYG{n}{loc}\PYG{p}{[}\PYG{n}{df\PYGZus{}dummy}\PYG{p}{[}\PYG{l+s+s1}{\PYGZsq{}}\PYG{l+s+s1}{type\PYGZus{}Iris\PYGZhy{}versicolor}\PYG{l+s+s1}{\PYGZsq{}}\PYG{p}{]} \PYG{o}{==} \PYG{l+m+mi}{1}\PYG{p}{]}\PYG{o}{.}\PYG{n}{drop}\PYG{p}{(}\PYG{n}{columns}\PYG{o}{=}\PYG{p}{[}\PYG{l+s+s1}{\PYGZsq{}}\PYG{l+s+s1}{type\PYGZus{}Iris\PYGZhy{}setosa}\PYG{l+s+s1}{\PYGZsq{}}\PYG{p}{,} \PYG{l+s+s1}{\PYGZsq{}}\PYG{l+s+s1}{type\PYGZus{}Iris\PYGZhy{}virginica}\PYG{l+s+s1}{\PYGZsq{}}\PYG{p}{]}\PYG{p}{)}
\PYG{n}{df\PYGZus{}g} \PYG{o}{=} \PYG{n}{df\PYGZus{}dummy}\PYG{o}{.}\PYG{n}{loc}\PYG{p}{[}\PYG{n}{df\PYGZus{}dummy}\PYG{p}{[}\PYG{l+s+s1}{\PYGZsq{}}\PYG{l+s+s1}{type\PYGZus{}Iris\PYGZhy{}virginica}\PYG{l+s+s1}{\PYGZsq{}}\PYG{p}{]} \PYG{o}{==} \PYG{l+m+mi}{1}\PYG{p}{]}\PYG{o}{.}\PYG{n}{drop}\PYG{p}{(}\PYG{n}{columns}\PYG{o}{=}\PYG{p}{[}\PYG{l+s+s1}{\PYGZsq{}}\PYG{l+s+s1}{type\PYGZus{}Iris\PYGZhy{}setosa}\PYG{l+s+s1}{\PYGZsq{}}\PYG{p}{,} \PYG{l+s+s1}{\PYGZsq{}}\PYG{l+s+s1}{type\PYGZus{}Iris\PYGZhy{}versicolor}\PYG{l+s+s1}{\PYGZsq{}}\PYG{p}{]}\PYG{p}{)}

\PYG{k}{def} \PYG{n+nf}{line\PYGZus{}regression\PYGZus{}pipe}\PYG{p}{(}\PYG{n}{df\PYGZus{}list}\PYG{p}{)}\PYG{p}{:}
    \PYG{k}{for} \PYG{n}{df} \PYG{o+ow}{in} \PYG{n}{df\PYGZus{}list}\PYG{p}{:}
        \PYG{n}{X} \PYG{o}{=} \PYG{n}{df}\PYG{o}{.}\PYG{n}{drop}\PYG{p}{(}\PYG{n}{columns}\PYG{o}{=}\PYG{p}{[}\PYG{l+s+s1}{\PYGZsq{}}\PYG{l+s+s1}{sepal\PYGZhy{}length}\PYG{l+s+s1}{\PYGZsq{}}\PYG{p}{]}\PYG{p}{)} \PYG{c+c1}{\PYGZsh{}indpendent variables}
        \PYG{n}{y} \PYG{o}{=} \PYG{n}{df}\PYG{p}{[}\PYG{p}{[}\PYG{l+s+s1}{\PYGZsq{}}\PYG{l+s+s1}{sepal\PYGZhy{}length}\PYG{l+s+s1}{\PYGZsq{}}\PYG{p}{]}\PYG{p}{]}\PYG{o}{.}\PYG{n}{copy}\PYG{p}{(}\PYG{p}{)} \PYG{c+c1}{\PYGZsh{}dependent variables}
        \PYG{c+c1}{\PYGZsh{}split the variable sets into training and testing subsets}
        \PYG{n}{X\PYGZus{}train}\PYG{p}{,} \PYG{n}{X\PYGZus{}test}\PYG{p}{,} \PYG{n}{y\PYGZus{}train}\PYG{p}{,} \PYG{n}{y\PYGZus{}test} \PYG{o}{=} \PYG{n}{train\PYGZus{}test\PYGZus{}split}\PYG{p}{(}\PYG{n}{X}\PYG{p}{,} \PYG{n}{y}\PYG{p}{,} \PYG{n}{test\PYGZus{}size}\PYG{o}{=}\PYG{l+m+mf}{0.333}\PYG{p}{,} \PYG{n}{random\PYGZus{}state}\PYG{o}{=}\PYG{l+m+mi}{41}\PYG{p}{)}
        \PYG{n}{linear\PYGZus{}reg\PYGZus{}model\PYGZus{}a} \PYG{o}{=} \PYG{n}{LinearRegression}\PYG{p}{(}\PYG{p}{)}
        \PYG{n}{linear\PYGZus{}reg\PYGZus{}model\PYGZus{}a}\PYG{o}{.}\PYG{n}{fit}\PYG{p}{(}\PYG{n}{X\PYGZus{}train}\PYG{p}{,}\PYG{n}{y\PYGZus{}train}\PYG{p}{)}
        \PYG{n}{y\PYGZus{}pred} \PYG{o}{=} \PYG{n}{linear\PYGZus{}reg\PYGZus{}model\PYGZus{}a}\PYG{o}{.}\PYG{n}{predict}\PYG{p}{(}\PYG{n}{X\PYGZus{}test}\PYG{p}{)}
        \PYG{n}{sme} \PYG{o}{=} \PYG{n}{mean\PYGZus{}squared\PYGZus{}error}\PYG{p}{(}\PYG{n}{y\PYGZus{}test}\PYG{p}{,} \PYG{n}{y\PYGZus{}pred}\PYG{p}{)}
        \PYG{n+nb}{print}\PYG{p}{(}\PYG{l+s+s1}{\PYGZsq{}}\PYG{l+s+s1}{MSE of }\PYG{l+s+s1}{\PYGZsq{}} \PYG{o}{+} \PYG{n}{df}\PYG{o}{.}\PYG{n}{columns}\PYG{p}{[}\PYG{l+m+mi}{4}\PYG{p}{]} \PYG{o}{+} \PYG{l+s+s1}{\PYGZsq{}}\PYG{l+s+s1}{ is :}\PYG{l+s+s1}{\PYGZsq{}} \PYG{o}{+} \PYG{n+nb}{str}\PYG{p}{(}\PYG{n}{sme}\PYG{p}{)} \PYG{p}{)}

\PYG{n}{df\PYGZus{}list} \PYG{o}{=} \PYG{p}{[}\PYG{n}{df\PYGZus{}s}\PYG{p}{,} \PYG{n}{df\PYGZus{}v}\PYG{p}{,} \PYG{n}{df\PYGZus{}g}\PYG{p}{]}
\PYG{n}{line\PYGZus{}regression\PYGZus{}pipe}\PYG{p}{(}\PYG{n}{df\PYGZus{}list}\PYG{p}{)}
\end{sphinxVerbatim}

\end{sphinxuseclass}\end{sphinxVerbatimInput}
\begin{sphinxVerbatimOutput}

\begin{sphinxuseclass}{cell_output}
\begin{sphinxVerbatim}[commandchars=\\\{\}]
MSE of type\PYGZus{}Iris\PYGZhy{}setosa is :0.06604867484155268
MSE of type\PYGZus{}Iris\PYGZhy{}versicolor is :0.09862408497975977
MSE of type\PYGZus{}Iris\PYGZhy{}virginica is :0.0802189108860733
\end{sphinxVerbatim}

\end{sphinxuseclass}\end{sphinxVerbatimOutput}

\end{sphinxuseclass}
\sphinxAtStartPar
The goal of this section was to illustrate how to incorporate independent categorical variables. Though it starts with converting strings to numbers, how those conversations are done is important to consider. Moreover, the introduction of more features can impact both the computational efficiency and accuracy of the model.

\sphinxstepscope

\begin{sphinxuseclass}{cell}
\begin{sphinxuseclass}{tag_hide-input}
\end{sphinxuseclass}
\end{sphinxuseclass}

\subsection{Accuracy Analysis}
\label{\detokenize{task2_c/example_sup_reg/sup_reg_ex_accuracy:accuracy-analysis}}\label{\detokenize{task2_c/example_sup_reg/sup_reg_ex_accuracy:sup-reg-ex-develop-accuracy}}\label{\detokenize{task2_c/example_sup_reg/sup_reg_ex_accuracy::doc}}
\sphinxAtStartPar
In the \DUrole{xref,myst}{Accuracy Analysis section of part D} of your project’s documentation, you will need to define and discuss how the metric for measuring the success of your application’s algorithm.

\sphinxAtStartPar
As we are trying to predict a continuous number, even the very best model will have errors in almost every prediction (if not, it’s almost certainly \sphinxhref{https://en.wikipedia.org/wiki/Overfitting}{overfitted}). Whereas measuring the success of our classification model was a simple ratio, here we need a way to measure how much those predictions deviate from actual values. See sklearn’s list of \sphinxhref{https://scikit-learn.org/stable/modules/model\_evaluation.html}{metrics and scoring} for regression. Which metric is best depends on the needs of your project. However, any accuracy metric appropriate to your model will be accepted.

\sphinxAtStartPar
To get some examples started, we’ll use the data (with \sphinxcode{\sphinxupquote{type}} and model from the previous sections. Recall, the model predicts a number for \sphinxcode{\sphinxupquote{sepal\sphinxhyphen{}length}}.

\begin{sphinxuseclass}{cell}\begin{sphinxVerbatimInput}

\begin{sphinxuseclass}{cell_input}
\begin{sphinxVerbatim}[commandchars=\\\{\}]
\PYG{k+kn}{from} \PYG{n+nn}{sklearn}\PYG{n+nn}{.}\PYG{n+nn}{linear\PYGZus{}model} \PYG{k+kn}{import} \PYG{n}{LinearRegression} 

\PYG{n}{linear\PYGZus{}reg\PYGZus{}model} \PYG{o}{=} \PYG{n}{LinearRegression}\PYG{p}{(}\PYG{p}{)}
\PYG{n}{linear\PYGZus{}reg\PYGZus{}model}\PYG{o}{.}\PYG{n}{fit}\PYG{p}{(}\PYG{n}{X\PYGZus{}train}\PYG{p}{,}\PYG{n}{y\PYGZus{}train}\PYG{p}{)}
\PYG{n}{y\PYGZus{}pred} \PYG{o}{=} \PYG{n}{linear\PYGZus{}reg\PYGZus{}model}\PYG{o}{.}\PYG{n}{predict}\PYG{p}{(}\PYG{n}{X\PYGZus{}test}\PYG{p}{)}
\end{sphinxVerbatim}

\end{sphinxuseclass}\end{sphinxVerbatimInput}

\end{sphinxuseclass}

\subsubsection{Mean Squared Error}
\label{\detokenize{task2_c/example_sup_reg/sup_reg_ex_accuracy:mean-squared-error}}\label{\detokenize{task2_c/example_sup_reg/sup_reg_ex_accuracy:sup-reg-ex-develop-accuracy-mse}}
\sphinxAtStartPar
The mean square error (MSE) measures the average of the squares of errors (difference between predicted and actual values).
\begin{equation*}
\begin{split}\text{MSE} = \frac{1}{n} \sum^{n}_{i=1} (Y_i - \hat{Y}_i)^{2}\end{split}
\end{equation*}
\sphinxAtStartPar
Where \(Y_i\) and \(\bar{Y}_i\) are the \(i^{\text{th}}\) actual and predicted values respectively.

\sphinxAtStartPar
In addition to giving positive values, squaring in the MSE emphasizes larger differences. Which can be good or bad depending on your needs. If your data has many or very large outliers, consider removing outliers or using the \sphinxhref{https://scikit-learn.org/stable/modules/generated/sklearn.metrics.mean\_absolute\_error.html\#sklearn.metrics.mean\_absolute\_error}{mean absolute error}. See the \sphinxhref{https://scikit-learn.org/stable/modules/generated/sklearn.metrics.mean\_squared\_error.html\#sklearn.metrics.mean\_squared\_error}{sklearn MSE docs} for more info and examples.

\sphinxAtStartPar
Applying the MSE to the \sphinxstyleemphasis{test} data, we have:

\begin{sphinxuseclass}{cell}\begin{sphinxVerbatimInput}

\begin{sphinxuseclass}{cell_input}
\begin{sphinxVerbatim}[commandchars=\\\{\}]
\PYG{k+kn}{from} \PYG{n+nn}{sklearn}\PYG{n+nn}{.}\PYG{n+nn}{metrics} \PYG{k+kn}{import} \PYG{n}{mean\PYGZus{}squared\PYGZus{}error}

\PYG{n}{mean\PYGZus{}squared\PYGZus{}error}\PYG{p}{(}\PYG{n}{y\PYGZus{}test}\PYG{p}{,} \PYG{n}{y\PYGZus{}pred}\PYG{p}{)}
\end{sphinxVerbatim}

\end{sphinxuseclass}\end{sphinxVerbatimInput}
\begin{sphinxVerbatimOutput}

\begin{sphinxuseclass}{cell_output}
\begin{sphinxVerbatim}[commandchars=\\\{\}]
0.12921261168601364
\end{sphinxVerbatim}

\end{sphinxuseclass}\end{sphinxVerbatimOutput}

\end{sphinxuseclass}

\paragraph{MSE explanation and example in 2D}
\label{\detokenize{task2_c/example_sup_reg/sup_reg_ex_accuracy:mse-explanation-and-example-in-2d}}\label{\detokenize{task2_c/example_sup_reg/sup_reg_ex_accuracy:sup-reg-ex-develop-accuracy-mse-example}}
\sphinxAtStartPar
For illustration purposes, we’ll use just the \sphinxstyleemphasis{petal\sphinxhyphen{}length} to predict the \sphinxstyleemphasis{sepal\sphinxhyphen{}length} with linear regression on 15 random values. For a multi\sphinxhyphen{}variable case, see the {[}example below{]}(sup\_reg\_ex: develop: accuracy: MSE\_2).

\sphinxAtStartPar
The regression line looks like this:

\begin{sphinxuseclass}{cell}
\begin{sphinxuseclass}{tag_hide-input}
\begin{sphinxuseclass}{tag_remove-output}
\end{sphinxuseclass}
\end{sphinxuseclass}
\end{sphinxuseclass}
\begin{sphinxuseclass}{cell}
\begin{sphinxuseclass}{tag_remove-input}\begin{sphinxVerbatimOutput}

\begin{sphinxuseclass}{cell_output}
\begin{sphinxVerbatim}[commandchars=\\\{\}]
\PYGZlt{}IPython.core.display.HTML object\PYGZgt{}
\end{sphinxVerbatim}

\end{sphinxuseclass}\end{sphinxVerbatimOutput}

\end{sphinxuseclass}
\end{sphinxuseclass}
\sphinxAtStartPar
The error squared looks like:

\begin{sphinxuseclass}{cell}
\begin{sphinxuseclass}{tag_hide-input}
\begin{sphinxuseclass}{tag_remove-output}
\end{sphinxuseclass}
\end{sphinxuseclass}
\end{sphinxuseclass}
\begin{sphinxuseclass}{cell}
\begin{sphinxuseclass}{tag_remove-input}\begin{sphinxVerbatimOutput}

\begin{sphinxuseclass}{cell_output}
\begin{sphinxVerbatim}[commandchars=\\\{\}]
\PYGZlt{}IPython.core.display.HTML object\PYGZgt{}
\end{sphinxVerbatim}

\end{sphinxuseclass}\end{sphinxVerbatimOutput}

\end{sphinxuseclass}
\end{sphinxuseclass}
\sphinxAtStartPar
Increasing the number of variables uses the same concept only the regression line becomes multi\sphinxhyphen{}dimensional. For example, additionally, including ‘sepal\sphinxhyphen{}length’ and ‘petal\sphinxhyphen{}length’ creates a \sphinxstyleemphasis{4\sphinxhyphen{}dimensional} line. So it’s a little hard to visualize. Using the squares of the errors is standard but ME is sometimes easier for non\sphinxhyphen{}technical audiences to understand.


\subparagraph{MSE for Multivariable Regression}
\label{\detokenize{task2_c/example_sup_reg/sup_reg_ex_accuracy:mse-for-multivariable-regression}}\label{\detokenize{task2_c/example_sup_reg/sup_reg_ex_accuracy:sup-reg-ex-develop-accuracy-mse-2}}
\sphinxAtStartPar
The case above presents Regression as most people are introduced to the concept \sphinxhyphen{}with two variables. Adding more variables does not change how MSE is computed, and the predicted values are still one\sphinxhyphen{}dimensional. However, visualizing the regression line can be \sphinxhref{https://stats.stackexchange.com/questions/73320/how-to-visualize-a-fitted-multiple-regression-model}{interesting} when there are more than two independent variables. One approach is to simply plot predicted and true values as paired data points. While some information about the model and relationship is lost, it does clearly show the model’s accuracy.

\begin{sphinxuseclass}{cell}
\begin{sphinxuseclass}{tag_remove-output}
\begin{sphinxuseclass}{tag_hide-input}
\end{sphinxuseclass}
\end{sphinxuseclass}
\end{sphinxuseclass}
\begin{sphinxuseclass}{cell}
\begin{sphinxuseclass}{tag_remove-input}\begin{sphinxVerbatimOutput}

\begin{sphinxuseclass}{cell_output}
\begin{sphinxVerbatim}[commandchars=\\\{\}]
\PYGZlt{}IPython.core.display.HTML object\PYGZgt{}
\end{sphinxVerbatim}

\end{sphinxuseclass}\end{sphinxVerbatimOutput}

\end{sphinxuseclass}
\end{sphinxuseclass}

\subparagraph{\protect\(R^{2}\protect\), \protect\(r^{2}\protect\), \protect\(R\protect\), and \protect\(r\protect\)}
\label{\detokenize{task2_c/example_sup_reg/sup_reg_ex_accuracy:r-2-r-2-r-and-r}}\label{\detokenize{task2_c/example_sup_reg/sup_reg_ex_accuracy:sup-reg-ex-develop-accuracy-r}}
\begin{sphinxadmonition}{warning}{Warning:}
\sphinxAtStartPar
These metrics are for \sphinxstyleemphasis{linear} models only! While they can be computed for any model, the results are not necessarily valid and can be misleading. For more details see \sphinxhref{https://statisticsbyjim.com/regression/r-squared-invalid-nonlinear-regression/}{\(R^{2}\) is Not Valid for Nonlinear Regression} and this \sphinxhref{https://www.ncbi.nlm.nih.gov/pmc/articles/PMC2892436/}{paper}.

\sphinxAtStartPar
What’s a linear model? See \sphinxhref{https://statisticsbyjim.com/regression/difference-between-linear-nonlinear-regression-models/}{Difference between a linear and nonlinear model} and sklearn’s \sphinxhref{https://scikit-learn.org/stable/modules/linear\_model.html}{Linear Models}
\end{sphinxadmonition}

\sphinxAtStartPar
These metrics (as with linear regression) have long been used in statistics before being adopted by the ML field. While ML uses these tools with a focus on results, statistics is a science so uses them differently. The overlap is the cause of some confusion and inconsistencies in the ML jargon. Here we’ll quickly try to clarify some of that. If working on the data as a science, we recommend the \sphinxhref{https://docs.scipy.org/doc/scipy/reference/main\_namespace.html}{scipy library}.
\begin{itemize}
\item {} 
\sphinxAtStartPar
\sphinxstyleemphasis{\(r\):} the \sphinxstyleemphasis{(sample) correlation coefficient} is a statistic metric measuring the strength and direction of a \sphinxstyleemphasis{linear} relationship between two variables, e.g., \(X\) and \(Y\); sometimes denoted \(R\). There are different types (usually \sphinxhref{https://en.wikipedia.org/wiki/Pearson\_correlation\_coefficient}{Pearson} when not specified) all having values in \([-1,1]\), with \(\pm 1\) indicating the strongest possible negative/positive relationship and \(0\) no relationship. Confusingly (maybe sloppily), sometimes \(r\) denotes the non\sphinxhyphen{}multiple correlation between \(Y\) and \(\hat{Y}\) (values fitted by the model), in which case \(r\geq 0\) (see the {[}image above{]}(sup\_reg\_ex: develop: accuracy: MSE\_example)).

\item {} 
\sphinxAtStartPar
\sphinxstyleemphasis{\(R\)}: the \sphinxstyleemphasis{coefficient of multiple correlation} or \sphinxstyleemphasis{correlation coefficient} measuring the Pearson correlation between observed, \(Y\), and predicted values, \(\hat{Y}\). \(R=\sqrt{R^{2}}\) and \(R\geq 0\) (assuming the model has an intercept) with higher values indicating better predictability of a linear model (as shown in the {[}image above{]}(sup\_reg\_ex: develop: accuracy: MSE\_2)).

\item {} 
\sphinxAtStartPar
\sphinxstyleemphasis{\(R^{2}\):} the \sphinxstyleemphasis{coefficient of determination} measures the proportion of the variation in the dependent variable predictable (by the linear model) from the independent variable(s), i.e., a metric measuring goodness of fit. The most general definition is: \(R^{2}= 1 -\frac{SS_{\text{res}}}{SS_{\text{tot}}}\) where \(SS_{\text{res}}=\sum_{i}(y_i-\hat{y}_i)^2\) (the \sphinxhref{https://en.wikipedia.org/wiki/Residual\_sum\_of\_squares}{residual sum of squares}) and \(SS_{\text{tot}}=\sum_{i}(y_i-\bar{y})^2\) (the \sphinxhref{https://en.wikipedia.org/wiki/Total\_sum\_of\_squares}{total sum of squares}). The former sums squared differences between actual and model output, and the latter sums squared differences between actual outputs and the mean. When \(SS_{\text{tot}}<SS_{\text{res}}\), i.e., the mean is a better predictor than the linear model, it’s possible for \(R^{2}\) to be negative.

\item {} 
\sphinxAtStartPar
\sphinxstyleemphasis{\(r^{2}\):} \(r^{2}\) is not \sphinxstyleemphasis{necessarily} \(R^{2}\). For simple linear regression, \sphinxhref{https://stats.stackexchange.com/questions/99669/the-equivalence-of-sample-correlation-and-r-statistic-for-simple-linear-regressi}{\(R^{2}=r^{2}\)}, but when the model lacks an intercept \sphinxhref{https://stats.stackexchange.com/questions/134167/is-there-any-difference-between-r2-and-r2\#:~:text=In\%20the\%20case\%20of\%20simple\%20linear\%20regression\%20specifically\%2C,this\%20means\%20that\%20R\%20\%3D\%20\%7C\%20r\%20\%7C.}{things get more complicated}, and unfortunately there are some inconsistencies in the notation.

\end{itemize}

\sphinxAtStartPar
\(R^{2}\) and \(r\) have can be more intuitively understood compared to the \sphinxhref{https://en.wikipedia.org/wiki/Mean\_squared\_error}{mean squared error} and similar metrics, e.g., \sphinxhref{https://en.wikipedia.org/wiki/Mean\_absolute\_error}{MAE}, \sphinxhref{https://en.wikipedia.org/wiki/Mean\_absolute\_percentage\_error}{MAPE}, and \sphinxhref{https://en.wikipedia.org/wiki/Root-mean-square\_deviation}{RMSE} which have arbitrary. \(r\) determines if variables are related, but since ML is more often concerned with how well a model predicts, \(R^{2}\) is more frequently used.

\begin{sphinxuseclass}{cell}\begin{sphinxVerbatimInput}

\begin{sphinxuseclass}{cell_input}
\begin{sphinxVerbatim}[commandchars=\\\{\}]
\PYG{k+kn}{from} \PYG{n+nn}{sklearn}\PYG{n+nn}{.}\PYG{n+nn}{metrics} \PYG{k+kn}{import} \PYG{n}{r2\PYGZus{}score}
\PYG{c+c1}{\PYGZsh{} https://scikit\PYGZhy{}learn.org/stable/modules/generated/sklearn.metrics.r2\PYGZus{}score.html\PYGZsh{}sklearn.metrics.r2\PYGZus{}score}

\PYG{n}{r2\PYGZus{}score}\PYG{p}{(}\PYG{n}{y\PYGZus{}test}\PYG{p}{,} \PYG{n}{y\PYGZus{}pred}\PYG{p}{)}
\end{sphinxVerbatim}

\end{sphinxuseclass}\end{sphinxVerbatimInput}
\begin{sphinxVerbatimOutput}

\begin{sphinxuseclass}{cell_output}
\begin{sphinxVerbatim}[commandchars=\\\{\}]
0.76202887109925
\end{sphinxVerbatim}

\end{sphinxuseclass}\end{sphinxVerbatimOutput}

\end{sphinxuseclass}
\sphinxAtStartPar
\(76.2\%\). Is that good? Here, as with most ML models, our goal is predicting the dependent variable. To demonstrate how well our model fits to actual results (it’s \sphinxstyleemphasis{goodness of fit}), the closer to \(1\) the better. But what’s close enough? That depends on how precise you need to be. Some fields demand a standard as high as \(95\%\). However, as a percentage of the dependent variable variation explained by the independent variables, \sphinxcode{\sphinxupquote{0.76202}} is a big chunk. Here’s a rough, totally unofficial guideline:


\begin{savenotes}\sphinxattablestart
\centering
\begin{tabulary}{\linewidth}[t]{|T|T|}
\hline
\sphinxstyletheadfamily 
\sphinxAtStartPar
\sphinxstylestrong{\(R^{2}\)}
&\sphinxstyletheadfamily 
\sphinxAtStartPar
Interpetaion
\\
\hline
\sphinxAtStartPar
\(\geq 0.75\)
&
\sphinxAtStartPar
Significant variance explained!
\\
\hline
\sphinxAtStartPar
\((0.75, .5]\)
&
\sphinxAtStartPar
Good amount explained.
\\
\hline
\sphinxAtStartPar
\((0.5, .25]\)
&
\sphinxAtStartPar
Meh, some explained.
\\
\hline
\sphinxAtStartPar
\((0.25, 0)\)
&
\sphinxAtStartPar
It’s still better than nothing!
\\
\hline
\sphinxAtStartPar
\(\leq 0\)
&
\sphinxAtStartPar
You’d be better off using the mean.
\\
\hline
\end{tabulary}
\par
\sphinxattableend\end{savenotes}

\sphinxAtStartPar
Conveniently, I’ve defined my model’s performance as a great success 😃. But in reality this very subjective and depends on the situation, field, and data. See this discussion, \sphinxhref{https://stats.stackexchange.com/questions/154755/is-there-any-rule-of-thumb-to-classify-r2-as-small-medium-or-large-effect-si}{is there any rule of thumb for classifying \(R^2\)}. Modeling complex and interesting situations will naturally be more difficult and should be held to different standard. In the behavioral sciences, the following guidelines for measuring the effect size are popular (as found {[}\hyperlink{cite.resources:id5}{Coh13}{]}):


\begin{savenotes}\sphinxattablestart
\centering
\begin{tabulary}{\linewidth}[t]{|T|T|}
\hline
\sphinxstyletheadfamily 
\sphinxAtStartPar
\sphinxstylestrong{\(R^{2}\)}
&\sphinxstyletheadfamily 
\sphinxAtStartPar
Effect
\\
\hline
\sphinxAtStartPar
\(\geq 0.26\)
&
\sphinxAtStartPar
Large
\\
\hline
\sphinxAtStartPar
\(0.13\)
&
\sphinxAtStartPar
Medium
\\
\hline
\sphinxAtStartPar
\(0.02\)
&
\sphinxAtStartPar
Small
\\
\hline
\end{tabulary}
\par
\sphinxattableend\end{savenotes}

\sphinxAtStartPar
As any \(R^{2} > 0\) provides a predictive value better than the mean, any positive \(R^{2}\) can be argued to have value.


\subparagraph{Margin of Error}
\label{\detokenize{task2_c/example_sup_reg/sup_reg_ex_accuracy:margin-of-error}}
\sphinxAtStartPar
The best way to measure your model’s success depends on what you’re trying to do. Say my flower customers want to predict \sphinxcode{\sphinxupquote{sepal\sphinxhyphen{}length}} (I have no idea why), but any prediction within \sphinxcode{\sphinxupquote{0.5}} of an inch would be acceptable. That is a “good” prediction, would be any falling with the dashed lines:

\begin{sphinxuseclass}{cell}
\begin{sphinxuseclass}{tag_hide-input}
\begin{sphinxuseclass}{tag_remove-output}
\end{sphinxuseclass}
\end{sphinxuseclass}
\end{sphinxuseclass}
\begin{sphinxuseclass}{cell}
\begin{sphinxuseclass}{tag_remove-input}\begin{sphinxVerbatimOutput}

\begin{sphinxuseclass}{cell_output}
\begin{sphinxVerbatim}[commandchars=\\\{\}]
\PYGZlt{}IPython.core.display.HTML object\PYGZgt{}
\end{sphinxVerbatim}

\end{sphinxuseclass}\end{sphinxVerbatimOutput}

\end{sphinxuseclass}
\end{sphinxuseclass}
\begin{sphinxuseclass}{cell}\begin{sphinxVerbatimInput}

\begin{sphinxuseclass}{cell_input}
\begin{sphinxVerbatim}[commandchars=\\\{\}]
\PYG{k}{def} \PYG{n+nf}{percent\PYGZus{}within\PYGZus{}moe}\PYG{p}{(}\PYG{n}{actual}\PYG{p}{,} \PYG{n}{predict}\PYG{p}{,} \PYG{n}{margin}\PYG{p}{)}\PYG{p}{:}
    \PYG{n}{total\PYGZus{}correct} \PYG{o}{=} \PYG{l+m+mi}{0}
    \PYG{n}{predict} \PYG{o}{=} \PYG{n}{predict}\PYG{o}{.}\PYG{n}{flatten}\PYG{p}{(}\PYG{p}{)} 
    \PYG{n}{actual} \PYG{o}{=} \PYG{n}{actual}\PYG{p}{[}\PYG{l+s+s1}{\PYGZsq{}}\PYG{l+s+s1}{sepal\PYGZhy{}length}\PYG{l+s+s1}{\PYGZsq{}}\PYG{p}{]}\PYG{o}{.}\PYG{n}{tolist}\PYG{p}{(}\PYG{p}{)}
    \PYG{k}{for} \PYG{n}{y\PYGZus{}actual}\PYG{p}{,} \PYG{n}{y\PYGZus{}predict} \PYG{o+ow}{in} \PYG{n+nb}{zip}\PYG{p}{(}\PYG{n}{actual}\PYG{p}{,}\PYG{n}{predict}\PYG{p}{)}\PYG{p}{:}
        \PYG{k}{if} \PYG{n+nb}{abs}\PYG{p}{(}\PYG{n}{y\PYGZus{}actual} \PYG{o}{\PYGZhy{}} \PYG{n}{y\PYGZus{}predict}\PYG{p}{)} \PYG{o}{\PYGZlt{}} \PYG{n}{margin}\PYG{p}{:}
            \PYG{n}{total\PYGZus{}correct} \PYG{o}{=} \PYG{n}{total\PYGZus{}correct}\PYG{o}{+}\PYG{l+m+mi}{1}
    \PYG{n}{percent\PYGZus{}correct} \PYG{o}{=} \PYG{n}{total\PYGZus{}correct}\PYG{o}{/}\PYG{n+nb}{len}\PYG{p}{(}\PYG{n}{actual}\PYG{p}{)}
    \PYG{k}{return} \PYG{n}{percent\PYGZus{}correct}

\PYG{n}{percent\PYGZus{}within\PYGZus{}moe}\PYG{p}{(}\PYG{n}{y\PYGZus{}test}\PYG{p}{,} \PYG{n}{y\PYGZus{}pred}\PYG{p}{,} \PYG{l+m+mf}{.5}\PYG{p}{)}
\end{sphinxVerbatim}

\end{sphinxuseclass}\end{sphinxVerbatimInput}
\begin{sphinxVerbatimOutput}

\begin{sphinxuseclass}{cell_output}
\begin{sphinxVerbatim}[commandchars=\\\{\}]
0.84
\end{sphinxVerbatim}

\end{sphinxuseclass}\end{sphinxVerbatimOutput}

\end{sphinxuseclass}
\sphinxAtStartPar
So \(84\%\) of predicted test values were within \(\pm .5\) of being correct. In addition to being easy to interpret, this approach can be modified to meet the specific needs of your project. A stock exchange prediction model, for example, predicting within a range or not overpredicting (i.e. predicting a profit when a loss occurs) might matter more than the MSE. It also has the added flexibility of being able to define success.

\begin{sphinxuseclass}{cell}\begin{sphinxVerbatimInput}

\begin{sphinxuseclass}{cell_input}
\begin{sphinxVerbatim}[commandchars=\\\{\}]
\PYG{n}{percent\PYGZus{}within\PYGZus{}moe}\PYG{p}{(}\PYG{n}{y\PYGZus{}test}\PYG{p}{,} \PYG{n}{y\PYGZus{}pred}\PYG{p}{,} \PYG{l+m+mf}{1.5}\PYG{p}{)}
\end{sphinxVerbatim}

\end{sphinxuseclass}\end{sphinxVerbatimInput}
\begin{sphinxVerbatimOutput}

\begin{sphinxuseclass}{cell_output}
\begin{sphinxVerbatim}[commandchars=\\\{\}]
1.0
\end{sphinxVerbatim}

\end{sphinxuseclass}\end{sphinxVerbatimOutput}

\end{sphinxuseclass}

\subsubsection{Train and test other models}
\label{\detokenize{task2_c/example_sup_reg/sup_reg_ex_accuracy:train-and-test-other-models}}\begin{itemize}
\item {} 
\sphinxAtStartPar
See \sphinxhref{https://scikit-learn.org/stable/supervised\_learning.html\#supervised-learning}{sklearn’s regression library}

\item {} 
\sphinxAtStartPar
See \sphinxhref{https://scikit-learn.org/stable/modules/model\_evaluation.html\#regression-metrics}{sklearn’s regression metrics}

\end{itemize}

\sphinxstepscope


\chapter{Task 2: The Documentation}
\label{\detokenize{task2_doc/task2_doc:task-2-the-documentation}}\label{\detokenize{task2_doc/task2_doc::doc}}
\begin{sphinxadmonition}{warning}{Warning:}
\sphinxAtStartPar
On 9/25/2023, C964 was updated to a new version (SIM3). Though the rubric and task directions were reworded, \sphinxstylestrong{the actual requirements and their assessment criteria are unchanged.}
\end{sphinxadmonition}

\begin{sphinxadmonition}{note}{Note:}
\sphinxAtStartPar
The content here for parts A, B, and D aligns with the latest version of the \sphinxhref{https://westerngovernorsuniversity-my.sharepoint.com/:w:/g/personal/jim\_ashe\_wgu\_edu/ERGxhsNfkbhEutlkXVFITMQBPOmWlkVx1p5H0UisvnBesg}{Task 2 template}. Following the new template meets all the documentation requirements while being more succinct and clear. We recommend using the template for both SIM2 and SIM3. However, using the \sphinxhref{https://westerngovernorsuniversity-my.sharepoint.com/:w:/g/personal/jim\_ashe\_wgu\_edu/EcklZjLXTB5EpDS4BVYc8SEBhT3VHy3s\_9lZSIZ5aH6Q5w}{previous Task 2 template} is still acceptable. Currently, the examples (including most of in the archives) follow the old template.
\end{sphinxadmonition}


\section{Task 2 Documentation Template}
\label{\detokenize{task2_doc/task2_doc:task-2-documentation-template}}\label{\detokenize{task2_doc/task2_doc:task2-doc}}
\sphinxAtStartPar
After completing the application (part C), write your documentation (parts A, B, and D) following the \sphinxstylestrong{C964 Task 2 documentation template}:
\begin{quote}

\sphinxAtStartPar
\sphinxhref{https://westerngovernorsuniversity-my.sharepoint.com/:w:/g/personal/jim\_ashe\_wgu\_edu/ERGxhsNfkbhEutlkXVFITMQBPOmWlkVx1p5H0UisvnBesg}{Task 2 documentation template}
\end{quote}

\begin{sphinxadmonition}{warning}{Warning:}
\sphinxAtStartPar
\sphinxstyleemphasis{The rubric is currently under development.} If you find the rubric difficult to understand, we recommend first referring to the guidelines on this webpage and the \sphinxhref{https://westerngovernorsuniversity-my.sharepoint.com/:w:/g/personal/jim\_ashe\_wgu\_edu/ERGxhsNfkbhEutlkXVFITMQBPOmWlkVx1p5H0UisvnBesg}{Task 2 template}.
\end{sphinxadmonition}

\sphinxAtStartPar
To gauge the level of detail evaluators typically expect, review these examples:


\section{Example Documents}
\label{\detokenize{task2_doc/task2_doc:example-documents}}\label{\detokenize{task2_doc/task2_doc:task2-doc-examples}}
\sphinxAtStartPar
The examples found here are of actual passing tasks 1 and 2, \sphinxstyleemphasis{flaws and all}. To best represent what might be accepted, we’ve made no corrections. See the {[}excellence archives{]}((resources:examples:archive) to review better projects.

\sphinxAtStartPar
These examples (and most of those in archives) were written using the \sphinxhref{https://westerngovernorsuniversity-my.sharepoint.com/:w:/g/personal/jim\_ashe\_wgu\_edu/EcklZjLXTB5EpDS4BVYc8SEBhT3VHy3s\_9lZSIZ5aH6Q5w}{previous Task 2 template}. In the coming months, we will add more recent examples following the new template.

\begin{sphinxuseclass}{sd-tab-set}
\begin{sphinxuseclass}{sd-tab-item}\subsubsection*{Task 2 documentation example A}

\begin{sphinxuseclass}{sd-tab-content}\begin{quote}

\sphinxAtStartPar
\sphinxhref{https://github.com/ashejim/C964/blob/main/resources/example\_task2-a.pdf}{\sphinxincludegraphics{{example_task2-a}.png}}
\end{quote}

\sphinxAtStartPar
Also see: \sphinxhref{https://github.com/ashejim/C964/blob/main/resources/example\_task1-a.pdf}{Task 1 example A}

\end{sphinxuseclass}
\end{sphinxuseclass}
\begin{sphinxuseclass}{sd-tab-item}\subsubsection*{Task 2 documentation example B}

\begin{sphinxuseclass}{sd-tab-content}\begin{quote}

\sphinxAtStartPar
\sphinxhref{https://github.com/ashejim/C964/blob/main/resources/example\_task2-b.pdf?raw=true}{\sphinxincludegraphics{{example_task2-b}.png}}
\end{quote}

\sphinxAtStartPar
Also see: \sphinxhref{https://github.com/ashejim/C964/blob/main/resources/example\_task1-b.pdf}{Task 1 example B}

\end{sphinxuseclass}
\end{sphinxuseclass}
\end{sphinxuseclass}


\sphinxAtStartPar
 Pass {\hyperref[\detokenize{task1:task1}]{\sphinxcrossref{\DUrole{std,std-ref}{Task 1}}}}.

\sphinxAtStartPar
 Finish \DUrole{xref,myst}{Task 2 part C}.



\sphinxAtStartPar
 Data is used to create an ML model.

\sphinxAtStartPar
 The user input can provide input and the ML model is applied to that input.

\sphinxAtStartPar
 Three images are included.

\sphinxAtStartPar
 The code runs without errors.



\sphinxAtStartPar
 Write task 2 part D.



\sphinxAtStartPar
 Write the \DUrole{xref,myst}{User Guide}.

\sphinxAtStartPar
 Write the \DUrole{xref,myst}{machine Learning} section.

\sphinxAtStartPar
 Write the \DUrole{xref,myst}{Validation}

\sphinxAtStartPar
 Write the remaining \DUrole{xref,myst}{part D} sections,



\sphinxAtStartPar
 Write part B.

\sphinxAtStartPar
 Write part A.

\sphinxAtStartPar
 Following APA guidelines, \DUrole{xref,myst}{check grammar and sources}, export as a \sphinxstyleemphasis{single} pdf, and submit.



\begin{sphinxadmonition}{tip}{Tip:}
\sphinxAtStartPar
Sections are assessed independently against the rubric requirements, i.e., when evaluating a section, the evaluator will check for the fulfillment of the requirements within that section. They \sphinxstyleemphasis{don’t} assess writing style. You can reuse content from other sections within the document or C951 task 3 as needed. It’s not about writing something fun to read \sphinxhyphen{}it’s about demonstrating that the requirements are met.
\begin{enumerate}
\sphinxsetlistlabels{\arabic}{enumi}{enumii}{}{.}%
\item {} 
\sphinxAtStartPar
Follow the \sphinxhref{https://westerngovernorsuniversity-my.sharepoint.com/:w:/g/personal/jim\_ashe\_wgu\_edu/ERGxhsNfkbhEutlkXVFITMQBPOmWlkVx1p5H0UisvnBesg}{Task 2 template}

\item {} 
\sphinxAtStartPar
Submit parts A, B, and D as a \sphinxstyleemphasis{single} .pdf file.

\item {} 
\sphinxAtStartPar
Part D is what matters \sphinxhyphen{}particularly the \sphinxstyleemphasis{Machine Learning}, \sphinxstyleemphasis{Validation}, and \sphinxstyleemphasis{User Guide} sections. Parts A and B need to be completed but have few qualitative requirements.

\end{enumerate}
\end{sphinxadmonition}


\section{FAQ}
\label{\detokenize{task2_doc/task2_doc:faq}}\label{\detokenize{task2_doc/task2_doc:task2doc-faq}}

\subsection{What version am I enrolled in? SIM2 or SIM3? Which resources should I use?}
\label{\detokenize{task2_doc/task2_doc:what-version-am-i-enrolled-in-sim2-or-sim3-which-resources-should-i-use}}
\sphinxAtStartPar
To see which version you are enrolled, go to the ‘Assessments’ section of your C964 COS page:
\sphinxincludegraphics{{C964_sim3-1}.png}

\sphinxAtStartPar
For both versions, SIM2 and SIM3, we recommend using the most recent version of the \sphinxhref{https://westerngovernorsuniversity-my.sharepoint.com/:w:/g/personal/jim\_ashe\_wgu\_edu/ERGxhsNfkbhEutlkXVFITMQBPOmWlkVx1p5H0UisvnBesg}{Task 2 template} and following the advice on this webpage.


\subsection{Are there length requirements for the documentation?}
\label{\detokenize{task2_doc/task2_doc:are-there-length-requirements-for-the-documentation}}
\sphinxAtStartPar
No. What you see in the examples and template are just guidelines. The individual evaluator determines what qualifies as “sufficient detail,” which further varies depending on the project and writing style. If you feel you’ve met the requirements, simply move on to the next section. Upon submission, it will pass, or they will request more details. In the latter case, you can then focus on revising the more narrow scope as directed by the evaluator’s comments, which is generally more efficient than overworking the entire project.


\subsection{I’ve completed the task 2 documentation. Should I send it to my course instructor for review?}
\label{\detokenize{task2_doc/task2_doc:i-ve-completed-the-task-2-documentation-should-i-send-it-to-my-course-instructor-for-review}}
\sphinxAtStartPar
If you have specific questions or concerns \sphinxhyphen{}yes. However, in most cases, it’s best just to submit. What suffices as “sufficient detail” is highly subjective. We can always tell you to add more, but if you’ve done your best to fulfill the requirements, submit it and let them tell you which (if any) parts need revision. At best, it passes; at worst, we address the issues cited by the evaluator \sphinxhyphen{}and then it passes. Responding to the more narrow focus of the evaluator’s comments is generally easier than overworking the entire project.

\sphinxAtStartPar
You have \sphinxstyleemphasis{unlimited} submissions but limited time. And, typically this is the best and most efficient approach.


\subsection{I only have a Linux or Mac machine. Will evaluators be able to run my code?”}
\label{\detokenize{task2_doc/task2_doc:i-only-have-a-linux-or-mac-machine-will-evaluators-be-able-to-run-my-code}}
\sphinxAtStartPar
It is preferable to submit an app which can run on both on Windows and Mac, e.g., Jupyter Notebook, Python project, webpage, etc. The ‘User Guide’ in part D only needs to have instructions for Windows as technically (and unfortunately), we are a “Windows” university,

\sphinxAtStartPar
However, being Windows\sphinxhyphen{}compatible is \sphinxstyleemphasis{nowhere specifically required} in the C964 rubric, and doing so would be a little silly for a computer science program. That said, WGU evaluators are only issued Windows 10 machines, and they may have difficulty running a Linux or Mac app without special instructions. Therefore, for projects that must be run from a non\sphinxhyphen{}Windows machine, we recommend that the \DUrole{xref,myst}{user guide} provide explicit instructions for a Windows 10 user to run your code, such as using a {[}virtual \sphinxhref{https://ubuntu.com/tutorials/how-to-run-ubuntu-desktop-on-a-virtual-machine-using-virtualbox\#1-overview}{machine}, a remote machine, or using a \sphinxhref{https://ubuntu.com/tutorials/install-ubuntu-on-wsl2-on-windows-10\#1-overview}{Linux subsystem}.


\subsection{How many submission attempts am I allowed?}
\label{\detokenize{task2_doc/task2_doc:how-many-submission-attempts-am-i-allowed}}
\sphinxAtStartPar
You have unlimited submissions (as with all WGU performance assessments). Furthermore, a project requiring multiple submissions is not precluded from being given an excellence award. However, do attempt to fully meet \sphinxstyleemphasis{each} requirement as submissions falling significantly short of the minimum requirements may be \sphinxstyleemphasis{locked} from further submissions without instructor approval. Moreover, such submissions do not receive meaningful evaluator comments.




\subsection{My task two was returned for \sphinxstyleemphasis{sources} stating:}
\label{\detokenize{task2_doc/task2_doc:my-task-two-was-returned-for-sources-stating}}\begin{quote}

\sphinxAtStartPar
“In\sphinxhyphen{}text citations could not be found for portions of the task that have been quoted or paraphrased…”
\end{quote}

\sphinxAtStartPar
\sphinxstylestrong{What does this comment mean?}

\sphinxAtStartPar
It indicates they could not find a matching in\sphinxhyphen{}text citation for every source on your reference list. Check that each reference has a match following APA style, e.g., (Author, year), and remove any references without matches. Use the \sphinxhref{https://support.microsoft.com/en-us/office/create-a-bibliography-citations-and-references-17686589-4824-4940-9c69-342c289fa2a5}{MS Word Reference Tool} to create, manage references, and avoid such errors.


\subsection{Can I use my C951 task 3? Should I use it?}
\label{\detokenize{task2_doc/task2_doc:can-i-use-my-c951-task-3-should-i-use-it}}
\sphinxAtStartPar
You can use anything you’ve written for C964, including copying verbatim from C951 task 3. If it’s convenient, feel free to do it. But at best, the time saved is little. At worst, you might get bogged down trying to work on two projects simultaneously and going with an unnecessarily complex C964 topic.

\sphinxAtStartPar
Here are some points to consider:
\begin{itemize}
\item {} 
\sphinxAtStartPar
C951.3 is just a written project, typically around five pages (I’m guessing; ask your C951 instructor), and can be completed in a single afternoon. Comparatively, C964 requires a working machine learning application and accompanying documentation, typically around 20 pages.

\item {} 
\sphinxAtStartPar
C951.3 only relates to parts A and B of C964.2. These parts are just a framework for communicating the project to a general audience and almost always pass. Furthermore, they’ll have to be at least partially rewritten anyway. Parts C and D of C964 are what evaluators care about, but C951.3 has no corresponding parts C and D.

\item {} 
\sphinxAtStartPar
Rewriting content C951.3 for a different C964 topic will take little additional work.

\item {} 
\sphinxAtStartPar
As it’s just a written project, students often pick a complex topic for C951.3. But then they feel pressured to use the same complex topic for C964 and struggle with creating the app.

\item {} 
\sphinxAtStartPar
Trying to comprehend two projects at once is just more difficult.

\end{itemize}

\sphinxAtStartPar
Whatever you do for C964 can meet the requirements of C951 task 3. If you have plenty of time, completing C964 first might be a good option.


\section{The rubric is confusing}
\label{\detokenize{task2_doc/task2_doc:the-rubric-is-confusing}}\label{\detokenize{task2_doc/task2_doc:task1-faq-confusingrubric}}
\sphinxAtStartPar
\sphinxstyleemphasis{We do not advise directly following the official rubric for C964.} Follow the guidelines found on this webpage and the \sphinxhref{https://westerngovernorsuniversity-my.sharepoint.com/:w:/g/personal/jim\_ashe\_wgu\_edu/ERGxhsNfkbhEutlkXVFITMQBPOmWlkVx1p5H0UisvnBesg}{Task 2 template}. Because of the ambiguity of the official rubric, following this template is helpful (almost necessary) in aligning your documentation with the rubric outcomes. So while following the template format is not technically required, it is highly recommended.

\sphinxAtStartPar
Preserve the template’s section titles, and order, and submit all four parts as a single document (preferably a \sphinxcode{\sphinxupquote{.pdf}}). With a long, complicated document, aligning content to competencies presents a challenge. Don’t make things difficult for the evaluator by spreading the content over several documents in an unfamiliar format.


\section{Questions, comments, or suggestions?}
\label{\detokenize{task2_doc/task2_doc:questions-comments-or-suggestions}}


\sphinxstepscope


\section{Task 2 Part D}
\label{\detokenize{task2_doc/task2_doc_d:task-2-part-d}}\label{\detokenize{task2_doc/task2_doc_d::doc}}\phantomsection\label{\detokenize{task2_doc/task2_doc_d:task2-doc-d}}
\sphinxAtStartPar
Part D of the documentation explains the technical details of your project and should target the computer science subject matter experts. Hence, of the documentation, part D is most valued and scrutinized by the evaluators. Of part D, the \DUrole{xref,myst}{User Guide}, \DUrole{xref,myst}{Machine Learning}, and \DUrole{xref,myst}{Validation} sections are the most important. As such, we recommend completing these sections first. Follow the \sphinxhref{https://westerngovernorsuniversity-my.sharepoint.com/:w:/g/personal/jim\_ashe\_wgu\_edu/ERGxhsNfkbhEutlkXVFITMQBPOmWlkVx1p5H0UisvnBesg}{Task 2 Template}.


\subsection{User Guide}
\label{\detokenize{task2_doc/task2_doc_d:user-guide}}\label{\detokenize{task2_doc/task2_doc_d:task2-doc-d-user-guide}}
\sphinxAtStartPar
The evaluator might go here first. Following your instructions, they will verify your app functions as intended and simultaneously gain context of its purpose through demonstration.

\sphinxAtStartPar
Include an enumerated (steps 1, 2, 3, etc.) guide on how to execute and use your application. The evaluator will play the “client” role and follow your instructions as provided. So you should include instructions for downloading and installing any necessary software or libraries (they’ll have most of these installed anyway). Your application will be considered “user\sphinxhyphen{}friendly” if the evaluator successfully executes and uses your application on a Windows 10 machine following your instructions.
\begin{itemize}
\item {} 
\sphinxAtStartPar
\sphinxstylestrong{Provide examples.} Examples not only make the instructions easier to follow (hence easier to pass) but also demonstrate how the app solves the organizational need. When applicable provide sample input or input which can be copied and pasted.

\item {} 
\sphinxAtStartPar
\sphinxstylestrong{Provide visual aides.} Screenshots or even a Panotpo video would all be great ways in aiding the evaluator’s understanding of your project.

\item {} 
\sphinxAtStartPar
\sphinxstylestrong{Test.} Create a new environment or borrow a friend’s computer, follow your instructions verbatim, and verify that everything works as expected. It’s easy to overlook small details.

\end{itemize}

\begin{sphinxShadowBox}
\sphinxstylesidebartitle{No Windows machine?}

\sphinxAtStartPar
\sphinxstyleemphasis{Nowhere is Windows specifically required} in the C964 rubric. However, WGU evaluators are issued Windows 10 machines, and they may have difficulty running a Linux or Mac app without special instructions. Therefore, we recommend that the \DUrole{xref,myst}{User Guide} provide explicit instructions for a Windows 10 user to run your code, such as using a \sphinxhref{https://ubuntu.com/tutorials/how-to-run-ubuntu-desktop-on-a-virtual-machine-using-virtualbox\#1-overview}{virtual machine}, remote machine, or using a \sphinxhref{https://ubuntu.com/tutorials/install-ubuntu-on-wsl2-on-windows-10\#1-overview}{Linux subsystem}. You should also provide a note when submitting to Assessments and alerting your course instructor. See the \DUrole{xref,myst}{Task 2 part C FAQ} for more details.

\phantomsection\label{\detokenize{task2_doc/task2_doc_d:task2-doc-d-user-guide-examples}}\end{sphinxShadowBox}

\begin{sphinxuseclass}{sd-sphinx-override}
\begin{sphinxuseclass}{sd-cards-carousel}
\begin{sphinxuseclass}{sd-card-cols-4}
\begin{sphinxuseclass}{sd-card}
\begin{sphinxuseclass}{sd-sphinx-override}
\begin{sphinxuseclass}{sd-mb-3}
\begin{sphinxuseclass}{sd-shadow-sm}
\begin{sphinxuseclass}{sd-card-body}
\noindent\sphinxincludegraphics[height=100\sphinxpxdimen]{{user_guide1}.png}

\end{sphinxuseclass}
\end{sphinxuseclass}
\end{sphinxuseclass}
\end{sphinxuseclass}
\end{sphinxuseclass}
\begin{sphinxuseclass}{sd-card}
\begin{sphinxuseclass}{sd-sphinx-override}
\begin{sphinxuseclass}{sd-mb-3}
\begin{sphinxuseclass}{sd-shadow-sm}
\begin{sphinxuseclass}{sd-card-body}
\noindent\sphinxincludegraphics[height=100\sphinxpxdimen]{{user_guide2}.png}

\end{sphinxuseclass}
\end{sphinxuseclass}
\end{sphinxuseclass}
\end{sphinxuseclass}
\end{sphinxuseclass}
\begin{sphinxuseclass}{sd-card}
\begin{sphinxuseclass}{sd-sphinx-override}
\begin{sphinxuseclass}{sd-mb-3}
\begin{sphinxuseclass}{sd-shadow-sm}
\begin{sphinxuseclass}{sd-card-body}
\noindent\sphinxincludegraphics[height=100\sphinxpxdimen]{{user_guide3}.png}

\end{sphinxuseclass}
\end{sphinxuseclass}
\end{sphinxuseclass}
\end{sphinxuseclass}
\end{sphinxuseclass}
\begin{sphinxuseclass}{sd-card}
\begin{sphinxuseclass}{sd-sphinx-override}
\begin{sphinxuseclass}{sd-mb-3}
\begin{sphinxuseclass}{sd-shadow-sm}
\begin{sphinxuseclass}{sd-card-body}
\noindent\sphinxincludegraphics[height=100\sphinxpxdimen]{{user_guide4}.png}

\end{sphinxuseclass}
\end{sphinxuseclass}
\end{sphinxuseclass}
\end{sphinxuseclass}
\end{sphinxuseclass}
\begin{sphinxuseclass}{sd-card}
\begin{sphinxuseclass}{sd-sphinx-override}
\begin{sphinxuseclass}{sd-mb-3}
\begin{sphinxuseclass}{sd-shadow-sm}
\begin{sphinxuseclass}{sd-card-body}
\noindent\sphinxincludegraphics[height=100\sphinxpxdimen]{{user_guide5}.png}

\end{sphinxuseclass}
\end{sphinxuseclass}
\end{sphinxuseclass}
\end{sphinxuseclass}
\end{sphinxuseclass}
\end{sphinxuseclass}
\end{sphinxuseclass}
\end{sphinxuseclass}

\subsection{Machine Learning}
\label{\detokenize{task2_doc/task2_doc_d:machine-learning}}\label{\detokenize{task2_doc/task2_doc_d:task2-doc-d-ml}}
\sphinxAtStartPar
This section should describe the development of your ML application justifying and explaining decisions made in the process. Explain the \sphinxstyleemphasis{what}, \sphinxstyleemphasis{how}, and \sphinxstyleemphasis{why} of the machine learning model and its application.
\begin{itemize}
\item {} 
\sphinxAtStartPar
\sphinxstyleemphasis{What?} Outline what your product does and then explain in detail what machine learning does in solving the proposed problem. Describe the algorithms, libraries, and other tools used to develop the machine learning model.

\item {} 
\sphinxAtStartPar
\sphinxstyleemphasis{How?} Outline your application’s implementation plan and then explain in detail how the machine learning portion was developed (or trained) and improved.

\item {} 
\sphinxAtStartPar
\sphinxstyleemphasis{Why?} Thoroughly justify development decisions. Address your algorithm(s) as a  good choice, why your training process was appropriate, etc.

\end{itemize}


\subsection{Validation}
\label{\detokenize{task2_doc/task2_doc_d:validation}}\label{\detokenize{task2_doc/task2_doc_d:task2-doc-d-validation}}
\sphinxAtStartPar
In this section, discuss how you assessed the accuracy or success of the ML application(s). In most cases, this means providing an appropriate \sphinxstyleemphasis{metric} for assessing accuracy OR providing a development plan for obtaining such a metric in the future. Most libraries have builtins for this; see \sphinxhref{https://scikit-learn.org/stable/modules/model\_evaluation.html}{sklearn metrics}.

\begin{sphinxadmonition}{note}{Note:}
\sphinxAtStartPar
There is \sphinxstylestrong{no} minimal accuracy requirement. At most, evaluators will assess the appropriateness of the metric (or planned metric).
\end{sphinxadmonition}


\subsubsection{For Supervised Classification Methods}
\label{\detokenize{task2_doc/task2_doc_d:for-supervised-classification-methods}}\label{\detokenize{task2_doc/task2_doc_d:task2d-accuracy-super}}
\sphinxAtStartPar
The metric for measuring a supervised classification model’s accuracy is straightforward. We use \sphinxhref{https://scikit-learn.org/stable/modules/generated/sklearn.metrics.accuracy\_score.html\#sklearn.metrics.accuracy\_score}{the ratio of correct to total predictions}:
\begin{equation*}
\begin{split}\text{Accuracy}=\frac{\text{correct predictions}}{\text{total predictions}}\end{split}
\end{equation*}
\sphinxAtStartPar
Though no minimal accuracy is required, your model should perform better than randomly selecting categories, e.g., the model predicting 1 of 3 flower types should perform better than \(\frac{1}{3} = 33.3\bar{3}\%\).


\subsubsection{For Supervised Regression Methods}
\label{\detokenize{task2_doc/task2_doc_d:for-supervised-regression-methods}}
\sphinxAtStartPar
As regression models estimate continuous values, they rarely exactly match actual values. Thus success of the model is measured by how closely the model fits the data. Common metrics include mean square error (MSE), mean absolute error (MAE), and mean absolute percentage error (MAPE). Statistical metrics such as the correlation coefficient or (more commonly) the coefficient of determination, \(R^{2}\), can be used. See \sphinxhref{https://scikit-learn.org/stable/modules/model\_evaluation.html\#regression-metrics}{sklearn’s regression metric documentation}. As above there is no performance benchmark, but the model should at predict at least as well as thaking the mean.


\subsubsection{For Unsupervised Methods}
\label{\detokenize{task2_doc/task2_doc_d:for-unsupervised-methods}}
\sphinxAtStartPar
Depending on the method, metrics might similarly be used for \sphinxstyleemphasis{unsupervised models}, such as \sphinxhref{https://scikit-learn.org/stable/auto\_examples/cluster/plot\_kmeans\_silhouette\_analysis.html}{Silhouette coefficients} for KMeans clustering. Alternatively (and typically), a future development plan for measuring the accuracy of your unsupervised method can be used.


\subsubsection{For Reinforced Learning Methods}
\label{\detokenize{task2_doc/task2_doc_d:for-reinforced-learning-methods}}
\sphinxAtStartPar
Reinforced learning methods seek to optimize an outcome, e.g., the C950 delivery app seeks to minimize miles driven. The better this outcome, the better your algorithm.


\subsection{Solution Summary}
\label{\detokenize{task2_doc/task2_doc_d:solution-summary}}
\sphinxAtStartPar
Summarize the problem and solution. Describe how the application (Part C) supports a solution to the problem presented in parts A and B.


\subsection{Data Summary}
\label{\detokenize{task2_doc/task2_doc_d:data-summary}}
\sphinxAtStartPar
Provide the source of the rwa data, how the data was collected, or how it was simulated. Include a description of how data was processed and managed throughtout the application development life cycle: design, development, maintenance, or others.


\subsection{Visualizations}
\label{\detokenize{task2_doc/task2_doc_d:visualizations}}
\sphinxAtStartPar
Identify the location of at least three unique imgaes in part C. Part or all of those images can also be included in this section. Tehcnically, the visualizations are required to be part of the application. However, often this is not desirable, say when the app is intended to customer facing. So it is allowable to submit the code provindg the visualization separate from the main application code. Recall, images can be generated by the code or inserted as static images.

\sphinxstepscope


\section{Task 2 parts A and B}
\label{\detokenize{task2_doc/task2_doc_a_and_b:task-2-parts-a-and-b}}\label{\detokenize{task2_doc/task2_doc_a_and_b:task2ab}}\label{\detokenize{task2_doc/task2_doc_a_and_b::doc}}
\sphinxAtStartPar
The purpose of parts A and B is to demonstrate competency in communicating your technical solution to different less technical audiences. Follow the \sphinxhref{https://westerngovernorsuniversity-my.sharepoint.com/:w:/g/personal/jim\_ashe\_wgu\_edu/ERGxhsNfkbhEutlkXVFITMQBPOmWlkVx1p5H0UisvnBesg}{Task 2 Template}.
\begin{itemize}
\item {} 
\sphinxAtStartPar
Part A \sphinxhyphen{} Convince senior leadership to approve your project.

\item {} 
\sphinxAtStartPar
Part B \sphinxhyphen{} Implementation plan for IT leadership (but not CS experts) or middle management.

\end{itemize}

\sphinxAtStartPar
Part D, explains the CS\sphinxhyphen{}related aspects of your project. With the exception of the \DUrole{xref,myst}{User Guide}, all sections should target the computer science subject\sphinxhyphen{}matter experts. This is \sphinxstyleemphasis{not} a business project, and neither does the rubric have any qualitative business criteria for parts A and B. As such, parts A and B aspects such as budgets, methodology, and planning parts are not assessed rigorously from a business perspective.


\subsection{Part A: Letter of Transmittal}
\label{\detokenize{task2_doc/task2_doc_a_and_b:part-a-letter-of-transmittal}}\label{\detokenize{task2_doc/task2_doc_a_and_b:task2-parta}}
\sphinxAtStartPar
Write a letter convincing senior leadership to approve your project \sphinxhyphen{}a brief cover letter (suggested length 1\sphinxhyphen{}2 pages) describing the problem, how the application (part C) applies to the problem, the practical benefits to the organization, and a brief implementation plan. Include all artifacts typical of a professional (business) letter, e.g., subject line, date, greeting, signature, etc. \sphinxstylestrong{Write everything in the future tense.}

\sphinxAtStartPar
The letter should be concise and target a non\sphinxhyphen{}technical audience. Include the following:
\begin{itemize}
\item {} 
\sphinxAtStartPar
A summary of the problem.

\item {} 
\sphinxAtStartPar
A proposed solution centering around your application.

\item {} 
\sphinxAtStartPar
How the proposed solution benefits the organization.

\item {} 
\sphinxAtStartPar
A summary of the costs, timeline, data, and any ethical concerns (if relevant).

\item {} 
\sphinxAtStartPar
Your relevant expertise.

\end{itemize}


\subsection{Part B: Project Proposal}
\label{\detokenize{task2_doc/task2_doc_a_and_b:part-b-project-proposal}}
\sphinxAtStartPar
The project proposal should target your client’s IT leadership or middle management. This audience may be IT professionals but have limited computer science expertise. Use appropriate industry jargon and sufficient technical details to describe the proposed project and its application. Remember, you’re establishing the technical context for your project and how it will be implemented for the client. \sphinxstylestrong{Write everything in the future tense.}


\subsubsection{Project Summary}
\label{\detokenize{task2_doc/task2_doc_a_and_b:project-summary}}
\sphinxAtStartPar
Include the following:
\begin{itemize}
\item {} 
\sphinxAtStartPar
A description of the problem.

\item {} 
\sphinxAtStartPar
A summary of the client and their needs as related to the problem.

\item {} 
\sphinxAtStartPar
Descriptions of all deliverables. For example, the finished application and a user guide.

\item {} 
\sphinxAtStartPar
A summary justifying how the application will benefit the client.

\end{itemize}


\subsubsection{Data Summary}
\label{\detokenize{task2_doc/task2_doc_a_and_b:data-summary}}
\sphinxAtStartPar
Include the following:
\begin{itemize}
\item {} 
\sphinxAtStartPar
The source of the raw data, how the data will be collected, or how it will be simulated.

\item {} 
\sphinxAtStartPar
A description of how data will be processed and managed throughout the application development life cycle: design, development, maintenance, or others.

\item {} 
\sphinxAtStartPar
A justification of why the data meets the needs of the project. If relevant, describe how data anomalies, e.g., outliers, incomplete data, etc., will be handled.

\item {} 
\sphinxAtStartPar
A list of any ethical or legal concerns regarding the data and how these concerns will be addressed. If there are no concerns, explain why.

\end{itemize}


\subsubsection{Implementation}
\label{\detokenize{task2_doc/task2_doc_a_and_b:implementation}}
\sphinxAtStartPar
Include the following:
\begin{itemize}
\item {} 
\sphinxAtStartPar
A description of an industry\sphinxhyphen{}standard methodology to be used.

\item {} 
\sphinxAtStartPar
An outline of the project’s implementation plan. The outline can focus on the project’s development as a whole; or it may focus on only the implementation of the machine learning solution.

\end{itemize}


\subsubsection{Timeline}
\label{\detokenize{task2_doc/task2_doc_a_and_b:timeline}}
\sphinxAtStartPar
Provide a projected timeline, including projected start dates and end dates for each milestone. Though not strictly required, a table is encouraged:


\begin{savenotes}\sphinxattablestart
\centering
\begin{tabulary}{\linewidth}[t]{|T|T|T|T|}
\hline
\sphinxstyletheadfamily 
\sphinxAtStartPar
Milestone or Deliverable
&\sphinxstyletheadfamily 
\sphinxAtStartPar
Duration
&\sphinxstyletheadfamily 
\sphinxAtStartPar
Projected Start Date
&\sphinxstyletheadfamily 
\sphinxAtStartPar
Projected End date
\\
\hline
\sphinxAtStartPar
Some milestones
&
\sphinxAtStartPar
7 days
&
\sphinxAtStartPar
7/23/2022
&
\sphinxAtStartPar
7/30/2022
\\
\hline
\sphinxAtStartPar
Some deliverables
&
\sphinxAtStartPar
14 days
&
\sphinxAtStartPar
7/16/2022
&
\sphinxAtStartPar
7/30/2022
\\
\hline
\sphinxAtStartPar
\(\vdots\)
&
\sphinxAtStartPar
\(\vdots\)
&
\sphinxAtStartPar
\(\vdots\)
&
\sphinxAtStartPar
\(\vdots\)
\\
\hline
\end{tabulary}
\par
\sphinxattableend\end{savenotes}

\begin{sphinxadmonition}{note}{Note:}
\sphinxAtStartPar
\sphinxstyleemphasis{All} dates must be in the future. Part B is a proposal.
\end{sphinxadmonition}


\subsubsection{Evaluation Plan}
\label{\detokenize{task2_doc/task2_doc_a_and_b:evaluation-plan}}
\sphinxAtStartPar
Include the following:
\begin{itemize}
\item {} 
\sphinxAtStartPar
A description of the verification method(s) to be used at each stage of development.

\item {} 
\sphinxAtStartPar
A description of the validation method to be used upon completion.

\end{itemize}

\sphinxAtStartPar
\sphinxstylestrong{Verification} is testing that your product meets its specifications and requirements. This can include tests, inspections, or methods applying to the code or model. For the latter approaches to avoid overfitting could be included. Verification checks that the product is built correctly.

\sphinxAtStartPar
\sphinxstylestrong{Validation} is testing how well the machine learning model performs.


\subsubsection{Resources and Costs}
\label{\detokenize{task2_doc/task2_doc_a_and_b:resources-and-costs}}
\sphinxAtStartPar
Include an itemized list of all resources and costs:
\begin{itemize}
\item {} 
\sphinxAtStartPar
Itemize hardware and software costs.

\item {} 
\sphinxAtStartPar
Itemize estimated labor time and costs.

\item {} 
\sphinxAtStartPar
Itemize all estimated application costs, e.g., deployment, hosting, maintenance, etc.

\end{itemize}

\sphinxstepscope


\section{Task 2 final touches}
\label{\detokenize{task2_doc/task2_doc_finish:task-2-final-touches}}\label{\detokenize{task2_doc/task2_doc_finish::doc}}

\subsection{Submitting Task 2, the document (parts A, B, and D)}
\label{\detokenize{task2_doc/task2_doc_finish:submitting-task-2-the-document-parts-a-b-and-d}}\label{\detokenize{task2_doc/task2_doc_finish:task2-doc-finish-how-to-submit}}\begin{enumerate}
\sphinxsetlistlabels{\arabic}{enumi}{enumii}{}{.}%
\item {} 
\sphinxAtStartPar
Use the \sphinxhref{https://westerngovernorsuniversity-my.sharepoint.com/:w:/g/personal/jim\_ashe\_wgu\_edu/ERGxhsNfkbhEutlkXVFITMQBPOmWlkVx1p5H0UisvnBesg}{Task 2 Template}.

\item {} 
\sphinxAtStartPar
Check Grammar!!

\item {} 
\sphinxAtStartPar
Check adherence to APA formatting, particularly citations. Preserve the formatting of the template and use the MS reference tool, and this shouldn’t be a problem.

\item {} 
\sphinxAtStartPar
Export parts A, B, and D as a \sphinxstyleemphasis{single} .pdf file.

\item {} 
\sphinxAtStartPar
\sphinxstyleemphasis{Always upload the document} separately from the code. It should be easily, identified.

\end{enumerate}


\subsection{Submitting Task 2, the code (part C)}
\label{\detokenize{task2_doc/task2_doc_finish:submitting-task-2-the-code-part-c}}\label{\detokenize{task2_doc/task2_doc_finish:task2-doc-finish-how-to-submit-code}}
\sphinxAtStartPar
Three things are required:
\begin{enumerate}
\sphinxsetlistlabels{\arabic}{enumi}{enumii}{}{.}%
\item {} 
\sphinxAtStartPar
Evaluators can access everything needed to recreate your application. This includes access to data, source files, and all software requirements.

\item {} 
\sphinxAtStartPar
For any materials modifiable after submission, e.g., webpage links, hosted notebooks, etc., you must upload the sources files.

\item {} 
\sphinxAtStartPar
Uploaded files cannot exceed 200 MB in total.

\end{enumerate}

\sphinxAtStartPar
The second requirement is to ensure there is a static record for the Evaluation team and Assessment’s records. But you can still expect evaluators to view links representing the polished finished application. When submitting, consider making it as easy as possible to review your application \sphinxhyphen{}webpages and hosted notebooks are a great way to do this and always appreciated by evaluators, and are encouraged \sphinxhyphen{}just submit the source files as directed below.

\begin{sphinxadmonition}{note}{Note:}
\sphinxAtStartPar
You may receive a warning when submitting .ipynb (Jupyter), Python, or other source files. Submitting these files is permitted, and the warning can be ignored.
\end{sphinxadmonition}


\subsubsection{What if all or part of the application is online?}
\label{\detokenize{task2_doc/task2_doc_finish:what-if-all-or-part-of-the-application-is-online}}
\sphinxAtStartPar
If some or all of your application is online, then you should submit the necessary link(s) AND the minimal source files needed to recreate the project locally. For example,
\begin{itemize}
\item {} 
\sphinxAtStartPar
If the application is a Jupyter Notebook hosted on Google Colab, submit the Colab link AND the .ipynb and any necessary data files.

\item {} 
\sphinxAtStartPar
If the application is a webpage, submit the web link AND the necessary .html files.

\item {} 
\sphinxAtStartPar
If the trained model is stored in a cloud drive, provide the link AND the source code needed to train the model.

\end{itemize}

\sphinxAtStartPar
It is acceptable to have code import data or submit links provide the source can’t be modified, such as links from \sphinxhref{http://Kaggle.com}{Kaggle.com}. Never use Google Drive, as WGU policy forbids WGU employees from using it. Use of Google Colab is acceptable, but upload the source code as directed above.


\subsubsection{What if the project files exceed 200 MB?}
\label{\detokenize{task2_doc/task2_doc_finish:what-if-the-project-files-exceed-200-mb}}
\sphinxAtStartPar
Projects over 200 MB typically result from large datasets or saved models.


\paragraph{For large data sets}
\label{\detokenize{task2_doc/task2_doc_finish:for-large-data-sets}}\begin{itemize}
\item {} 
\sphinxAtStartPar
If the data source doesn’t belong to you, your code can import the data directly (preferred) or instruct the evaluator to download the data.

\item {} 
\sphinxAtStartPar
You can upload a smaller subset of the data. Your documentation and presentation of the application can still be based on the full dataset.

\end{itemize}


\paragraph{For large saved models}
\label{\detokenize{task2_doc/task2_doc_finish:for-large-saved-models}}\begin{itemize}
\item {} 
\sphinxAtStartPar
Check to see if the serialization can be compressed. For example, \sphinxhref{https://joblib.readthedocs.io/en/latest/}{\sphinxcode{\sphinxupquote{joblib}}} has a \sphinxcode{\sphinxupquote{compress}} argument. However, most serialization methods use a relatively compact method by default.

\item {} 
\sphinxAtStartPar
Upload all the source code and provide instructions for the evaluator to train the model locally. If training are very time or resource consuming, provide a link to download the model for the evaluator’s convenience.

\end{itemize}

\sphinxAtStartPar
If none of the above work, please contact your course instructor so they can help work out a solution.


\subsection{Grammar and Sources}
\label{\detokenize{task2_doc/task2_doc_finish:grammar-and-sources}}

\subsubsection{Grammar}
\label{\detokenize{task2_doc/task2_doc_finish:grammar}}\label{\detokenize{task2_doc/task2_doc_finish:task2-doc-finish-grammar}}
\begin{sphinxadmonition}{warning}{Warning:}
\sphinxAtStartPar
Grammar and sources are the \sphinxstyleemphasis{most common} reason for returned submissions! Even minor grammatical errors can prevent a submission from passing.
\end{sphinxadmonition}

\sphinxAtStartPar
After focusing on the content, check your grammar using \sphinxhref{https://www.grammarly.com/}{Grammarly.com} \sphinxincludegraphics{{icon-grammarly}.png} (it’s what the evaluators use). Style is not assessed (blue and green), but even a few grammar mistakes will prevent competency in \sphinxstyleemphasis{Professional Communication}. The free side has been sufficient, but if using the online app, you sometimes need to wait before mistakes are caught.

\begin{sphinxadmonition}{warning}{Warning:}
\sphinxAtStartPar
Students have reported missed mistakes when using the Google doc Grammarly extension. Therefore, we advise that you copy and paste it into the online app or purchase the premium version for MS Word.
\end{sphinxadmonition}


\subsubsection{Sources}
\label{\detokenize{task2_doc/task2_doc_finish:sources}}\label{\detokenize{task2_doc/task2_doc_finish:task2-doc-finish-sources}}
\sphinxAtStartPar
For sources, you should follow \sphinxhref{https://apastyle.apa.org/style-grammar-guidelines}{APA guidelines}. Avoid errors by using the \sphinxhref{https://support.microsoft.com/en-us/office/create-a-bibliography-citations-and-references-17686589-4824-4940-9c69-342c289fa2a5}{MS Word Reference Tool} to create and manage references.
\begin{itemize}
\item {} 
\sphinxAtStartPar
No references are required.

\item {} 
\sphinxAtStartPar
All sources on the reference page must have a matching in\sphinxhyphen{}text citation, e.g., (Author, year)

\item {} 
\sphinxAtStartPar
To cite sources used for code, you should include the referfences as code comments within the source code.

\end{itemize}

\sphinxstepscope


\chapter{C964 Resources}
\label{\detokenize{resources:c964-resources}}\label{\detokenize{resources::doc}}





\section{Examples}
\label{\detokenize{resources:examples}}\label{\detokenize{resources:resources-examples}}
\sphinxAtStartPar
The examples found here are of actual passing tasks 1 and 2, \sphinxstyleemphasis{flaws and all}. To best represent what might be accepted, we’ve made no corrections. See the excellence archives to review better projects.

\sphinxAtStartPar
These examples (and most of those in archives) were written using the \sphinxhref{https://westerngovernorsuniversity-my.sharepoint.com/:w:/g/personal/jim\_ashe\_wgu\_edu/EcklZjLXTB5EpDS4BVYc8SEBhT3VHy3s\_9lZSIZ5aH6Q5w}{previous Task 2 template}. In the coming months, we will add more recent examples following the new template.

\sphinxAtStartPar
\sphinxstylestrong{Example A:}
\begin{itemize}
\item {} 
\sphinxAtStartPar
\sphinxhref{https://github.com/ashejim/C964/blob/main/resources/example\_task1-a.pdf}{task 1 ex. A}

\item {} 
\sphinxAtStartPar
\sphinxhref{https://github.com/ashejim/C964/blob/main/resources/example\_task2-a.pdf}{task 2 ex. A} (part C not available)

\end{itemize}

\sphinxAtStartPar
\sphinxstylestrong{Example B:}
\begin{itemize}
\item {} 
\sphinxAtStartPar
\sphinxhref{https://github.com/ashejim/C964/blob/main/resources/example\_task1-b.pdf}{task 1 ex. B}

\item {} 
\sphinxAtStartPar
\sphinxhref{https://github.com/ashejim/C964/blob/main/resources/example\_task2-b.pdf}{task 2 ex. B} (part C not available)

\end{itemize}
\phantomsection\label{\detokenize{resources:resources-examples-archive}}
\sphinxAtStartPar
\sphinxstylestrong{WGU Capstone Excellence Archive}

\sphinxAtStartPar
The \sphinxhref{https://westerngovernorsuniversity.sharepoint.com/sites/capstonearchives/excellence/Pages/UndergraduateInformation.aspx}{Capstone Excellence Archive} includes a wide range of completed projects. Reviewing them may help get ideas, provide inspiration, and understand the requirements. However, keep in mind that they all are \sphinxstyleemphasis{above and beyond} the requirements. Therefore, don’t use these as examples of what’s needed to meet the requirements. For a more down\sphinxhyphen{}to\sphinxhyphen{}earth example of what’s required, see the {\hyperref[\detokenize{resources:resources-examples}]{\sphinxcrossref{\DUrole{std,std-ref}{examples}}}} above.

\begin{sphinxadmonition}{tip}{Tip:}
\sphinxAtStartPar
Currently, most of these examples (including those in the archives) follow the old template. However, the advice on this site aligns with the latest version of the \sphinxhref{https://westerngovernorsuniversity-my.sharepoint.com/:w:/g/personal/jim\_ashe\_wgu\_edu/ERGxhsNfkbhEutlkXVFITMQBPOmWlkVx1p5H0UisvnBesg}{Task 2 template} as the new template meets all the documentation requirements while being more succinct and clear. We recommend using the new template for both SIM2 and SIM3.
\end{sphinxadmonition}


\section{Task 1 Resources}
\label{\detokenize{resources:task-1-resources}}\label{\detokenize{resources:resources-task1}}

\subsection{Topic Approval Form Template}
\label{\detokenize{resources:topic-approval-form-template}}\label{\detokenize{resources:resources-task1-task1example}}\begin{quote}

\sphinxAtStartPar
\sphinxhref{https://westerngovernorsuniversity-my.sharepoint.com/:w:/g/personal/jim\_ashe\_wgu\_edu/EaH8yexFJjhDp5hnrcAZeKoB6XxU9r8Z5IH1QqVLmVu87w?e=OwRtpe}{\sphinxincludegraphics{{C964_t1_approval}.png}}
\end{quote}


\subsection{Excellence Archive}
\label{\detokenize{resources:excellence-archive}}\label{\detokenize{resources:resources-task1-excellence}}
\sphinxAtStartPar
The Capstone Excellence Archive includes a wide range of completed projects. When reviewing archived capstones, keep in mind that they all are, by definition, above and beyond the requirements. Therefore, do not use these as examples of what’s needed to meet the requirements. For a more down\sphinxhyphen{}to\sphinxhyphen{}earth example of what’s required, see the {\hyperref[\detokenize{resources:resources-examples}]{\sphinxcrossref{\DUrole{std,std-ref}{examples}}}} above.


\subsection{Topic Approval Form Example}
\label{\detokenize{resources:topic-approval-form-example}}\begin{quote}

\sphinxAtStartPar
\sphinxhref{https://westerngovernorsuniversity-my.sharepoint.com/:b:/g/personal/jim\_ashe\_wgu\_edu/ESVtyYRIzQhNoLL37S7prkQB9uGuuljjUbCoKaVe0h47Xg?e=dHkn6n}{\sphinxincludegraphics{{C964_t1_example}.png}}
\end{quote}


\subsection{Waiver Form}
\label{\detokenize{resources:waiver-form}}\begin{quote}

\sphinxAtStartPar
\sphinxhref{https://westerngovernorsuniversity-my.sharepoint.com/:w:/g/personal/jim\_ashe\_wgu\_edu/ESLuMNRuDjpCrKvqWaC6cywB4I97WEPdk5MRZRq4LfmFhQ}{\sphinxincludegraphics{{C964_waiver}.png}}
\end{quote}


\subsection{Data}
\label{\detokenize{resources:data}}\label{\detokenize{resources:resources-task1-data}}
\sphinxAtStartPar
Any open source data set is freely available for use.
\begin{itemize}
\item {} 
\sphinxAtStartPar
\sphinxhref{https://www.kaggle.com/datasets}{Kaggle.com}

\item {} 
\sphinxAtStartPar
\sphinxhref{https://datasetsearch.research.google.com/}{Google Dataset Search}

\item {} 
\sphinxAtStartPar
\sphinxhref{https://data.gov/}{Data.gov}

\item {} 
\sphinxAtStartPar
More \sphinxhref{https://careerfoundry.com/en/blog/data-analytics/where-to-find-free-datasets/}{here} and \sphinxhref{https://medium.com/analytics-vidhya/top-100-open-source-datasets-for-data-science-cd5a8d67cc3d}{here}

\item {} 
\sphinxAtStartPar
Simulated data

\end{itemize}


\section{Task 2 Part C (the app) Resources}
\label{\detokenize{resources:task-2-part-c-the-app-resources}}\label{\detokenize{resources:resources-task2examplec}}

\subsection{Coding}
\label{\detokenize{resources:coding}}\begin{itemize}
\item {} 
\sphinxAtStartPar
{\hyperref[\detokenize{ci_other:ci-other}]{\sphinxcrossref{\DUrole{std,std-ref}{CS and C964 faculty}}}} available for math, data analytics, and coding\sphinxhyphen{}related questions.

\item {} 
\sphinxAtStartPar
\sphinxhref{https://westerngovernorsuniversity.sharepoint.com/sites/ProgrammingCenter}{WGU Programming Center} provides support to all WGU students in learning basic programming concepts, R, JavaScript, and Python.

\end{itemize}


\subsection{Videos, guides, and Tutorials}
\label{\detokenize{resources:videos-guides-and-tutorials}}\label{\detokenize{resources:resources-task2-videos}}

\subsubsection{Getting Started on the Capstone}
\label{\detokenize{resources:getting-started-on-the-capstone}}\label{\detokenize{resources:resources-task2c-videos-ml-sup-reg-class}}



\subsubsection{ML Supervised Classification \& Regression Overview}
\label{\detokenize{resources:ml-supervised-classification-regression-overview}}



\subsubsection{ML Supervised Learning Classification Coding Example}
\label{\detokenize{resources:ml-supervised-learning-classification-coding-example}}\label{\detokenize{resources:resources-task2c-videos-ml-sup-class-code}}



\subsubsection{Student Presentation}
\label{\detokenize{resources:student-presentation}}


\sphinxAtStartPar
This presentation was intended for a general audience.


\subsubsection{Convolution Neural Networks (CNN)}
\label{\detokenize{resources:convolution-neural-networks-cnn}}

\paragraph{CNN Videos}
\label{\detokenize{resources:cnn-videos}}\begin{itemize}
\item {} 
\sphinxAtStartPar
\sphinxhref{https://youtu.be/r5nXYc2wYvI?feature=shared}{MIT: CNNs Convolutional Neural Networks \sphinxhyphen{} Deep Learning in Life Sciences}. This is a great lecture explaining the concepts and terms of a CNNs. The focus is not coding, but understanding what the code does.

\item {} 
\sphinxAtStartPar
\sphinxhref{https://www.youtube.com/watch?v=qFJeN9V1ZsI}{Keras with TensorFlow Course \sphinxhyphen{} Python Deep Learning and Neural Networks for Beginners Tutorial}

\end{itemize}


\paragraph{CNN Tutorials}
\label{\detokenize{resources:cnn-tutorials}}
\sphinxAtStartPar
CNNs Coding Tutorials:
\begin{itemize}
\item {} 
\sphinxAtStartPar
From the \sphinxhref{https://www.tensorflow.org/tutorials/quickstart/beginner}{TensorFlow docs}
\begin{itemize}
\item {} 
\sphinxAtStartPar
\sphinxhref{https://www.tensorflow.org/tutorials/quickstart/beginner}{TensorFlow Quickstart}

\item {} 
\sphinxAtStartPar
\sphinxhref{https://www.tensorflow.org/tutorials/images/cnn}{CNN tutorial}

\end{itemize}

\item {} 
\sphinxAtStartPar
\sphinxhref{https://machinelearningmastery.com/tensorflow-tutorial-deep-learning-with-tf-keras/}{TensorFlow Tutorial: Get Started in Deep Learning with tf.keras}

\item {} 
\sphinxAtStartPar
\sphinxhref{https://www.geeksforgeeks.org/convolutional-neural-network-cnn-in-tensorflow/}{GeeksforGeeks}

\item {} 
\sphinxAtStartPar
\sphinxhref{https://www.datacamp.com/tutorial/tensorflow-tutorial}{DataCamp}

\end{itemize}


\section{Task 2 Parts A, B, \& C (the documetation) Resources}
\label{\detokenize{resources:task-2-parts-a-b-c-the-documetation-resources}}\label{\detokenize{resources:resources-task2doc}}
\sphinxAtStartPar
Part C is your submitted application. See all our available resources here: {\hyperref[\detokenize{task2_c/task2_part_c:task2-part-c}]{\sphinxcrossref{\DUrole{std,std-ref}{task 2 part C}}}}


\subsection{Task 2 Parts A, B, and D Template \& Part C Guide}
\label{\detokenize{resources:task-2-parts-a-b-and-d-template-part-c-guide}}\label{\detokenize{resources:resources-task2doc-doctemplate}}
\sphinxAtStartPar
Write your proposal following \sphinxstylestrong{C964 Task 2 documentation template}:
\begin{quote}

\sphinxAtStartPar
\sphinxhref{https://westerngovernorsuniversity-my.sharepoint.com/:w:/g/personal/jim\_ashe\_wgu\_edu/ERGxhsNfkbhEutlkXVFITMQBPOmWlkVx1p5H0UisvnBesg?rtime=3q\_Efs-u2kg}{\sphinxincludegraphics{{template_task2}.png}}
\end{quote}



\sphinxAtStartPar
Preserve the template’s section titles, and order, and submit all four parts as a single document (preferably a pdf). With a long, complicated document, aligning content to competencies presents a challenge. Don’t make things difficult for the evaluator by spreading the content over several documents in an unfamiliar format.


\subsection{Task 2 Parts A, B, and D Examples}
\label{\detokenize{resources:task-2-parts-a-b-and-d-examples}}\label{\detokenize{resources:resources-task2doc-docexample}}
\sphinxAtStartPar
\sphinxstylestrong{Example A:}
\begin{itemize}
\item {} 
\sphinxAtStartPar
\sphinxhref{https://github.com/ashejim/C964/blob/main/resources/example\_task2-a.pdf}{task 2 ex. A} (part C not available)

\end{itemize}

\sphinxAtStartPar
\sphinxstylestrong{Example B:}
\begin{itemize}
\item {} 
\sphinxAtStartPar
\sphinxhref{https://github.com/ashejim/C964/blob/main/resources/example\_task2-b.pdf}{task 2 ex. B} (part C not available)

\end{itemize}

\begin{sphinxadmonition}{note}{Note:}
\sphinxAtStartPar
The examples found here are of actual passing tasks 1 and 2, \sphinxstyleemphasis{flaws and all}. To best represent what might be accepted, we’ve made no corrections. See the excellence archives to review better projects.
\end{sphinxadmonition}


\subsubsection{WGU Capstone Excellence Archives}
\label{\detokenize{resources:wgu-capstone-excellence-archives}}\label{\detokenize{resources:resources-examples-excellent-archives}}
\sphinxAtStartPar
The \sphinxhref{https://westerngovernorsuniversity.sharepoint.com/sites/capstonearchives/excellence/Pages/UndergraduateInformation.aspx}{Capstone Excellence Archives} include a wide range of completed projects. Reviewing them may help get ideas, provide inspiration, and understand the requirements. However, keep in mind that they all are \sphinxstyleemphasis{above and beyond} the requirements. Therefore, don’t use these as examples of what’s needed to meet the requirements. For a more down\sphinxhyphen{}to\sphinxhyphen{}earth example of what’s required, see the {\hyperref[\detokenize{resources:resources-examples}]{\sphinxcrossref{\DUrole{std,std-ref}{examples}}}} above.


\section{General Resources}
\label{\detokenize{resources:general-resources}}

\subsection{Student Resources}
\label{\detokenize{resources:student-resources}}\label{\detokenize{resources:resources-gen-student-resources}}\begin{itemize}
\item {} 
\sphinxAtStartPar
\sphinxhref{https://resource-hub.wgu.edu/}{Student Resource Hub}

\item {} 
\sphinxAtStartPar
\sphinxhref{https://my.wgu.edu/success-centers/writing-center}{Writing Center}

\item {} 
\sphinxAtStartPar
\sphinxhref{https://cm.wgu.edu/t5/Writing-Center-Knowledge-Base/tkb-p/C770\_kb}{Writing Center Knowledge Base}

\item {} 
\sphinxAtStartPar
\sphinxhref{https://my.wgu.edu/success-centers/math-center}{WGU Math Center}. The Math Center is \sphinxhref{https://www.wolframalpha.com/input/?i=e\%5E\%282*pi*i\%29}{\(e^{2\pi i}\)}!

\item {} 
\sphinxAtStartPar
\sphinxhref{https://westerngovernorsuniversity.sharepoint.com/sites/ProgrammingCenter}{WGU Programming Center}

\end{itemize}


\subsection{Grammarly.com}
\label{\detokenize{resources:grammarly-com}}\label{\detokenize{resources:resources-gen-grammarly}}
\sphinxAtStartPar
Check your grammar using \sphinxhref{https://www.grammarly.com/}{Grammarly.com} \sphinxincludegraphics{{icon-grammarly}.png} (it’s what the evaluators use). Style is not assessed (blue and green), but even a few grammar mistakes will prevent competency in \sphinxstyleemphasis{Professional Communication}. The free side has been sufficient, but if using the online app, you sometimes need to wait before mistakes are caught.

\begin{sphinxadmonition}{warning}{Warning:}
\sphinxAtStartPar
Students have reported missed mistakes when using the Google doc Grammarly extension. Therefore, we advise copying content directly into the app or purchasing the premium version and checking grammar in MS Word.
\end{sphinxadmonition}


\subsection{APA}
\label{\detokenize{resources:apa}}\label{\detokenize{resources:resources-gen-apa}}
\sphinxAtStartPar
Your paper should adhere to \sphinxhref{https://cm.wgu.edu/t5/Writing-Center-Knowledge-Base/I-Need-Help-with-APA-Style/ta-p/33524}{APA guidelines}. Avoid reference errors by using the \sphinxhref{https://support.microsoft.com/en-us/office/create-a-bibliography-citations-and-references-17686589-4824-4940-9c69-342c289fa2a5}{MS Word Reference Tool} to create and manage references.


\subsection{Libraries}
\label{\detokenize{resources:libraries}}\label{\detokenize{resources:resources-gen-libraries}}\begin{itemize}
\item {} 
\sphinxAtStartPar
\sphinxhref{https://wgu.libguides.com/friendly.php?s=library}{WGU’s library}

\item {} 
\sphinxAtStartPar
\sphinxhref{https://scholar.google.com/}{google.scholar.com}

\end{itemize}

\begin{sphinxadmonition}{tip}{Tip:}
\sphinxAtStartPar
You can search \sphinxhref{https://wgu.libguides.com/friendly.php?s=library}{WGU’s library} and other open\sphinxhyphen{}source libraries using \sphinxhref{https://scholar.google.com/}{google.scholar.com} Go to >’Google.scholar>setting>libraires>’ and then add WGU and other libraries.


\end{sphinxadmonition}


\subsection{Bibliography}
\label{\detokenize{resources:bibliography}}
\sphinxstepscope


\chapter{Course Faculty}
\label{\detokenize{ci_page:course-faculty}}\label{\detokenize{ci_page:cipage}}\label{\detokenize{ci_page::doc}}




\sphinxAtStartPar
The purpose of the capstone is to showcase knowledge and skills accumulated throughout your BSCS degree program. Helping you integrate your accumulative expertise into a passing capstone is the primary responsibility of your assigned C964 course instructor. However, the capstone was designed by WGU  with the expectation that the necessary skills have already been mastered. So questions regarding those skills, e.g., coding, debugging, mathematics, and machine learning; may be outside the scope of your assigned course instructor’s expertise (the capstone faculty team supports \sphinxstyleemphasis{all} capstones in the undergraduate IT college).

\begin{sphinxShadowBox}
\sphinxstylesidebartitle{Which courses are these skills covered?}

\sphinxAtStartPar
See the \DUrole{xref,myst}{course map}, but there are major gaps. Noteably, machine learning, data processing, and inferential statistics. Tangential skills such as vector calculus and probability are also missing. BUT the true asset of a computer \sphinxstyleemphasis{science} graduate, is the ability to adapt, learn, and problem solve. Something you’ve done plenty of and should expect to do your entire career.
\end{sphinxShadowBox}

\sphinxAtStartPar
In cases when that happens, your C964 instructor will help direct you to the correct resource. But given the broad range of approaches, languages, and libraries available to use, it should not be expected to find an available WGU expert on everything. We recommend reviewing available resources \sphinxstyleemphasis{before} investing time into your project and being aware that WGU resources might be limited when choosing tools outside the scope of those listed.


\section{Who to contact?}
\label{\detokenize{ci_page:who-to-contact}}
\begin{sphinxuseclass}{sd-sphinx-override}
\begin{sphinxuseclass}{sd-cards-carousel}
\begin{sphinxuseclass}{sd-card-cols-2}
\begin{sphinxuseclass}{sd-card}
\begin{sphinxuseclass}{sd-sphinx-override}
\begin{sphinxuseclass}{sd-m-3}
\begin{sphinxuseclass}{sd-shadow-sm}
\begin{sphinxuseclass}{sd-card-hover}
\begin{sphinxuseclass}{sd-card-header}
\begin{sphinxuseclass}{bg-light}
\begin{sphinxuseclass}{text-center}
\sphinxAtStartPar
\sphinxstylestrong{C964 Faculty}

\end{sphinxuseclass}
\end{sphinxuseclass}
\end{sphinxuseclass}
\begin{sphinxuseclass}{sd-card-body}
\begin{sphinxuseclass}{text-center}
\noindent\sphinxincludegraphics[height=100\sphinxpxdimen]{{virtual_meeting1}.jpg}

\sphinxAtStartPar
Understanding the project
Revising returned projects
Topic approval
Planning \& resources

\end{sphinxuseclass}
\end{sphinxuseclass}
\begin{sphinxuseclass}{sd-card-footer}
\sphinxAtStartPar
C964 CI page 

\end{sphinxuseclass}\sphinxhref{./ci\_c964.html}{}
\end{sphinxuseclass}
\end{sphinxuseclass}
\end{sphinxuseclass}
\end{sphinxuseclass}
\end{sphinxuseclass}
\begin{sphinxuseclass}{sd-card}
\begin{sphinxuseclass}{sd-sphinx-override}
\begin{sphinxuseclass}{sd-m-3}
\begin{sphinxuseclass}{sd-shadow-sm}
\begin{sphinxuseclass}{sd-card-hover}
\begin{sphinxuseclass}{sd-card-header}
\begin{sphinxuseclass}{bg-light}
\begin{sphinxuseclass}{text-center}
\sphinxAtStartPar
\sphinxstylestrong{BSCS, Math, \& Software Faculty}

\end{sphinxuseclass}
\end{sphinxuseclass}
\end{sphinxuseclass}
\begin{sphinxuseclass}{sd-card-body}
\begin{sphinxuseclass}{text-center}
\noindent\sphinxincludegraphics[height=100\sphinxpxdimen]{{debug1}.jpg}

\sphinxAtStartPar
Debugging Python and Java 
Data processing 
Data Analytics 
Statistics, Math, \& AI

\end{sphinxuseclass}
\end{sphinxuseclass}
\begin{sphinxuseclass}{sd-card-footer}
\sphinxAtStartPar
BSCS, Math, \& Software CI page 

\end{sphinxuseclass}\sphinxhref{./ci\_other.html}{}
\end{sphinxuseclass}
\end{sphinxuseclass}
\end{sphinxuseclass}
\end{sphinxuseclass}
\end{sphinxuseclass}
\end{sphinxuseclass}
\end{sphinxuseclass}
\end{sphinxuseclass}
\sphinxAtStartPar
Always practice professional communication:
\begin{itemize}
\item {} 
\sphinxAtStartPar
Use your WGU email (we may not receive emails from outside WGU).

\item {} 
\sphinxAtStartPar
Provide a subject, your capstone course (we support all IT college capstones), and your program mentor’s name (if not in your signature).

\item {} 
\sphinxAtStartPar
Clearly state your questions or requests. When contacting non\sphinxhyphen{}C964 faculty, provide context and limit questions to the scope of their expertise.

\end{itemize}

\sphinxstepscope


\section{C964 Course Faculty}
\label{\detokenize{ci_c964:c964-course-faculty}}\label{\detokenize{ci_c964:ci-c964}}\label{\detokenize{ci_c964::doc}}




\sphinxAtStartPar
The capstone was designed to showcase knowledge and skills accumulated throughout the BSCS degree program. Helping you integrate your accumulative expertise into a passing capstone is the primary responsibility of your assigned C964 course instructor. They will actively monitor your progress and offer one\sphinxhyphen{}on\sphinxhyphen{}one advice. They will discuss your progress, help you find answers to content questions, and give you tools to navigate the capstone successfully. Throughout your engagement with this course, it is expected that you continue communicating with your program mentor. Your program mentor will help you set weekly study goals, recommend specific learning materials, and tell you what to expect in this course and how it aligns with your program’s competencies.

\sphinxAtStartPar
If your assigned course instructor is out of the office, you should contact \sphinxhref{mailto:ugcapstoneit@wgu.edu?cc=your\%20course\%20instructor\&subject=C964:\%20capstone}{ugcapstoneit@wgu.edu}. Whether emailing \sphinxhref{mailto:ugcapstoneit@wgu.edu?cc=my\%20course\%20instructor\&subject=C964:\%20capstone}{ugcapstoneit@wgu.edu} or your CI directly, always practice professional communication:
\begin{itemize}
\item {} 
\sphinxAtStartPar
Use your WGU email (we may not receive emails from outside WGU).

\item {} 
\sphinxAtStartPar
Provide a subject, your capstone course (we support all IT college capstones), and your program mentor’s name (if not in your signature).

\item {} 
\sphinxAtStartPar
Clearly state your questions or requests.

\end{itemize}


\subsection{Candice Allen}
\label{\detokenize{ci_c964:candice-allen}}\label{\detokenize{ci_c964:ci-c964-cis}}
\begin{sphinxShadowBox}
\sphinxstylesidebartitle{Contact Info}

\sphinxAtStartPar
📞 (385) 428\sphinxhyphen{}5987 
📧 \sphinxhref{mailto:ugcapstoneit@wgu.edu?cc=candice.allen@wgu.edu\&subject=C769\%20capstone\%20question\%20for\%20Candice\%20Allen}{ugcapstoneit@wgu.edu} (include Candice’s name in the subject line) 
📅 \sphinxhref{https://timetrade.com/app/wgu-mentoring/workflows/WGU100/schedule/?resourceId=00530000006rcrYAAQ\&locationId=course\_mentoring\&appointmentTypeGroupId=CM\&questionId\_\_course\_code=C964}{Schedule an Appointment} 


\end{sphinxShadowBox}

\sphinxhref{https://timetrade.com/app/wgu-mentoring/workflows/WGU100/schedule/?resourceId=00530000006rcrYAAQ\&locationId=course\_mentoring\&appointmentTypeGroupId=CM\&questionId\_\_course\_code=C964}{{\hspace*{\fill}\sphinxincludegraphics[height=200\sphinxpxdimen]{{candice_allen-a}.jpg}\hspace*{\fill}}}

\sphinxAtStartPar
Candice Allen has worked for WGU for over ten years. Before coming to WGU she taught a mixture of IT fundamental courses and business courses at post\sphinxhyphen{}secondary colleges in Northwest Indiana and online. She graduated with her bachelor’s degree from Purdue University and has two master’s degrees (IDOL and MSIS). She attended Capella University and AIU to obtain her master’s degree. Currently, she is a course instructor for the College of IT.


\subsection{Jim Ashe}
\label{\detokenize{ci_c964:jim-ashe}}
\begin{sphinxShadowBox}
\sphinxstylesidebartitle{Contact Info}

\sphinxAtStartPar
📞  (385) 428\sphinxhyphen{}4209 
📧 \sphinxhref{mailto:jim.ashe@wgu.edu?subject=C964\%20capstone\%}{jim.ashe@wgu.edu} 
📅 \sphinxhref{https://timetrade.com/app/wgu-mentoring/workflows/WGU100/schedule/?resourceId=005a000000CAi7dAAD\&locationId=course\_mentoring\&appointmentTypeGroupId=CM\&questionId\_\_course\_code=C964}{Schedule an Appointment} 


\end{sphinxShadowBox}

\sphinxhref{https://timetrade.com/app/wgu-mentoring/workflows/WGU100/schedule/?resourceId=005a000000CAi7dAAD\&locationId=course\_mentoring\&appointmentTypeGroupId=CM\&questionId\_\_course\_code=C964}{{\hspace*{\fill}\sphinxincludegraphics[height=200\sphinxpxdimen]{{jim_ashe-a}.jpg}\hspace*{\fill}}}

\sphinxAtStartPar
Dr. James Ashe is a dedicated maths, computer science, and statistics teacher with over 20 years of experience teaching in an online, large university, small college, HBCU, and community college setting. In 2016 he joined WGU to help students in the newly created Math Center. In 2018 Jim joined IT to contribute to the new Computer Science program. Currently, he supports the IT, data analytics, and computer science capstone. Though his research studied abstract objects, experimentation, creating examples, and producing necessitated a lot of coding. It was here that he developed a love for programming and computer science. He has a Ph.D. and MS in mathematics from the University of Tennessee and a BS in history with a minor in art from East Tennessee State University. Jim and his wife reside in Asheville, NC, with their four children, four cats, a dog, and an undetermined number of chickens. In his nonexistent spare time, he enjoys reading and kayaking.


\subsection{Tawnya Lee}
\label{\detokenize{ci_c964:tawnya-lee}}
\begin{sphinxShadowBox}
\sphinxstylesidebartitle{Contact Info}

\sphinxAtStartPar
📞 (385) 428\sphinxhyphen{}4052 x4052  
📧 \sphinxhref{mailto:tawnya.lee@wgu.edu?subject=C769\%20capstone\&body=Your\%20name\%20and\%20question\%20here.\%20We\%20can\%20only\%20respond\%20to\%20messages\%20from\%20a\%20valid\%20WGU\%20email\%20address.\%20\%0A\%0ADegree\%20program\%3A\%20\%0AProgram\%20Mentor\%3A\%20\%0A}{tawnya.lee@wgu.edu}
📅 \sphinxhref{https://timetrade.com/app/wgu-mentoring/workflows/WGU100/schedule/?resourceId=005a0000009HkMlAAK\&locationId=course\_mentoring\&appointmentTypeGroupId=CM\&questionId\_\_course\_code=C769}{Schedule an Appointment}


\end{sphinxShadowBox}

\noindent{\hspace*{\fill}\sphinxincludegraphics[height=200\sphinxpxdimen]{{tawnya_lee}.jpg}\hspace*{\fill}}

\sphinxAtStartPar
Tawnya has been with WGU 2013. Prior to WGU, Tawnya worked for over 12 years as a web developer, coding in .NET and PHP. She also ran her own web design business, serving small to medium\sphinxhyphen{}sized businesses. Education is her passion She is committed to seeing every student achieve their goal of earning their degree. She has two Master’s, one in Science and another in Instructional Design. She is a member of Delta Mu Delta.


\subsection{Charlie Paddock}
\label{\detokenize{ci_c964:charlie-paddock}}
\begin{sphinxShadowBox}
\sphinxstylesidebartitle{Contact Info}

\sphinxAtStartPar
📞 (385) 428\sphinxhyphen{}1858 x1858  
📧 \sphinxhref{mailto:charles.paddock@wgu.edu?subject=C964\%20capstone\%}{charles.paddock@wgu.edu} 
📅 \sphinxhref{https://timetrade.com/app/wgu-mentoring/workflows/WGU100/schedule/?resourceId=00530000006rd6fAAA\&locationId=course\_mentoring\&appointmentTypeGroupId=CM\&questionId\_\_course\_code=C964}{Schedule an Appointment}


\end{sphinxShadowBox}

\sphinxhref{https://timetrade.com/app/wgu-mentoring/workflows/WGU100/schedule/?resourceId=00530000006rd6fAAA\&locationId=course\_mentoring\&appointmentTypeGroupId=CM\&questionId\_\_course\_code=C964}{{\hspace*{\fill}\sphinxincludegraphics[height=200\sphinxpxdimen]{{charlie_paddock-b}.jpg}\hspace*{\fill}}}

\sphinxAtStartPar
Dr. Charles Paddock received his BS in Business from the University of New Orleans. Following a tour of duty in the Air Force, he returned to school and received his MBA concentrating in Systems and Operations Management and Ph.D. in Management Information Systems, both from the University of Houston. Charles previously taught at Arizona State University and The University of Nevada, Las Vegas, and worked in the Research Centers of both Universities as a consultant to their corporate members. Charles joined WGU in February of 2006 and now lives in Salt Lake City, UT.

\sphinxstepscope


\section{BSCS, Software, and other Course Faculty}
\label{\detokenize{ci_other:bscs-software-and-other-course-faculty}}\label{\detokenize{ci_other::doc}}



\phantomsection\label{\detokenize{ci_other:ci-other}}
\begin{sphinxadmonition}{warning}{Warning:}
\sphinxAtStartPar
Faculty not assigned to you, in particular non\sphinxhyphen{}C964 faculty,  are happy to help any WGU student. But students assigned to them take priority.  Follow the guidelines below, and understand their assistance is a courtesy and that response times may be delayed (particularly towards the end of the month).
\end{sphinxadmonition}


\subsection{Better questions get better answers}
\label{\detokenize{ci_other:better-questions-get-better-answers}}\label{\detokenize{ci_other:ci-other-better-questions-get-better-answers}}
\sphinxAtStartPar
The fix might need an understanding of several intertwined layers, of your code structure, the language, libraries, or data. You’ve already spent hours working on it. Don’t expect someone else to quickly diagnose and fix the issue from a quick live code section or a series of screenshots. Remember, asking for help means asking for someone’s time. Faculty on this page enjoy helping students (they’ve volunteered for the additional work), but you need to make it as easy as possible for them to do so.

\sphinxAtStartPar
\sphinxstylestrong{Coding\sphinxhyphen{}related questions should be emailed following these guidelines:}
\begin{itemize}
\item {} 
\sphinxAtStartPar
Clearly state the problem you are trying to fix.

\item {} 
\sphinxAtStartPar
Describe how to recreate the problem.

\item {} 
\sphinxAtStartPar
Provide everything needed to reproduce the error. Preferably a minimal working example. Do \sphinxstyleemphasis{NOT} rely on screenshots.

\item {} 
\sphinxAtStartPar
Describe what you’ve tried already; some \sphinxhref{https://www.freecodecamp.org/news/what-is-debugging-how-to-debug-code/}{debugging tips}.

\item {} 
\sphinxAtStartPar
Start with the simplest error. Focus on one problem at a time.

\end{itemize}

\sphinxAtStartPar
\sphinxstylestrong{Additionally, always practice professional communication:}
\begin{itemize}
\item {} 
\sphinxAtStartPar
Use your WGU email (we may not receive emails from outside WGU).

\item {} 
\sphinxAtStartPar
Provide a subject, your capstone course (we support all IT college capstones), and your program mentor’s name (if not in your signature).

\item {} 
\sphinxAtStartPar
Clearly state your question, provide context, and do not assume the faculty on this page knows the requirements of your project. Restrict the scope of your question to their area of expertise.

\end{itemize}


\subsection{WGU Programming Center}
\label{\detokenize{ci_other:wgu-programming-center}}
\sphinxAtStartPar
The \sphinxhref{https://westerngovernorsuniversity.sharepoint.com/sites/ProgrammingCenter}{WGU Programming Center} provides support to all WGU students in learning basic programming concepts, R, JavaScript, and Python.

\sphinxhref{https://westerngovernorsuniversity.sharepoint.com/sites/ProgrammingCenter}{{\hspace*{\fill}\sphinxincludegraphics[height=200\sphinxpxdimen]{{WGU_programming_center_python}.png}\hspace*{\fill}}}


\subsection{WGU Math Center}
\label{\detokenize{ci_other:wgu-math-center}}
\sphinxAtStartPar
The \sphinxhref{https://my.wgu.edu/success-centers/math-center}{Math Center} provides a supportive environment to help you improve your mathematical skills and achieve your academic goals.

\sphinxhref{https://my.wgu.edu/success-centers/math-center}{{\hspace*{\fill}\sphinxincludegraphics[height=200\sphinxpxdimen]{{math-center}.svg}\hspace*{\fill}}}


\subsection{Jim Ashe, PhD}
\label{\detokenize{ci_other:jim-ashe-phd}}\label{\detokenize{ci_other:ci-other-cis}}
\begin{sphinxShadowBox}
\sphinxstylesidebartitle{Contact Info}

\sphinxAtStartPar
📞  (385) 428\sphinxhyphen{}4209 
📧 \sphinxhref{mailto:jim.ashe@wgu.edu?subject=C964\%20capstone\%}{jim.ashe@wgu.edu} 
📅 \sphinxhref{https://timetrade.com/app/wgu-mentoring/workflows/WGU100/schedule/?resourceId=005a000000CAi7dAAD\&locationId=course\_mentoring\&appointmentTypeGroupId=CM\&questionId\_\_course\_code=C964}{Schedule an Appointment} 


\end{sphinxShadowBox}
\begin{itemize}
\item {} 
\sphinxAtStartPar
Mathematics and Statistics

\item {} 
\sphinxAtStartPar
Data Analytics

\item {} 
\sphinxAtStartPar
Python

\end{itemize}

\sphinxhref{https://timetrade.com/app/wgu-mentoring/workflows/WGU100/schedule/?resourceId=005a000000CAi7dAAD\&locationId=course\_mentoring\&appointmentTypeGroupId=CM\&questionId\_\_course\_code=C964}{{\hspace*{\fill}\sphinxincludegraphics[height=200\sphinxpxdimen]{{jim_ashe-a}.jpg}\hspace*{\fill}}}



\sphinxAtStartPar
Dr. James Ashe is a dedicated maths, computer science, and statistics teacher with over 20 years of experience teaching in an online, large university, small college, HBCU, and community college setting. In 2016 he joined WGU to help students in the newly created Math Center. In 2018 Jim joined IT to contribute to the new Computer Science program. Currently, he supports the IT, data analytics, and computer science capstone. Though his research studied abstract objects, experimentation, creating examples, and producing necessitated a lot of coding. It was here that he developed a love for programming and computer science. He has a Ph.D. and MS in mathematics from the University of Tennessee and a BS in history with a minor in art from East Tennessee State University. Jim and his wife reside in Asheville, NC, with their four children, four cats, a dog, and an undetermined number of chickens. In his nonexistent spare time, he enjoys reading and kayaking.


\subsection{Amy Antonucci, PhD}
\label{\detokenize{ci_other:amy-antonucci-phd}}
\begin{sphinxShadowBox}
\sphinxstylesidebartitle{Contact Info}

\sphinxAtStartPar
📞  (385) 428\sphinxhyphen{}7197 
📧 \sphinxhref{mailto:amy.antonucci@wgu.edu?subject=C964\%20capstone\%20related\%20question}{amy.antonucci@wgu.edu} 
📅 \sphinxhref{https://timetrade.com/app/wgu-mentoring/workflows/WGU100/schedule/?locationId=course\_mentoring\&appointmentTypeGroupId=CM\&resourceId=005a000000B2XzeAAF\&questionId\_\_course\_code=C964}{Schedule an Appointment} 


\end{sphinxShadowBox}

\sphinxhref{https://timetrade.com/app/wgu-mentoring/workflows/WGU100/schedule/?locationId=course\_mentoring\&appointmentTypeGroupId=CM\&resourceId=005a000000B2XzeAAF\&questionId\_\_course\_code=C964}{{\hspace*{\fill}\sphinxincludegraphics[height=200\sphinxpxdimen]{{wonderwoman1}.png}\hspace*{\fill}}}
\begin{itemize}
\item {} 
\sphinxAtStartPar
Debugging Java and Python

\item {} 
\sphinxAtStartPar
Data Analytics

\end{itemize}

\sphinxAtStartPar
Amy has been with WGU since January of 2015.  Before joining WGU, she worked as an adjunct college instructor for 10 years teaching computer science classes, and as a programmer for 10 years before that.  She has a BS from Penn State University in Computer Science, and two Masters’s Degrees, both in Computer Science. She also has a Ph.D. in Information Systems from Nova Southeastern University. She is currently working on an MS in Cybersecurity and Information Assurance from WGU. Her interests include Norway and its culture and language, reading fantasy and science fiction, playing board games, and loving her dog and cats.


\subsection{Mark Denchy, PhD}
\label{\detokenize{ci_other:mark-denchy-phd}}
\begin{sphinxShadowBox}
\sphinxstylesidebartitle{Contact Info}

\sphinxAtStartPar
📧 \sphinxhref{mailto:mark.denchy@wgu.edu?subject=C964\%20capstone\%20related\%20question}{mark.denchy@wgu.edu} 
📅 \sphinxhref{https://scheduling.wgu.edu/wgu-mentoring/workflows/WGU100/schedule/?locationId=course\_mentoring\&appointmentTypeGroupId=CM\&resourceId=0053x00000FoVG6AAN\&ch=emailsignature\&questionId\_\_course\_code=C964}{Schedule an Appointment} 


\end{sphinxShadowBox}

\sphinxhref{https://scheduling.wgu.edu/wgu-mentoring/workflows/WGU100/schedule/?locationId=course\_mentoring\&appointmentTypeGroupId=CM\&resourceId=0053x00000FoVG6AAN\&ch=emailsignature\&questionId\_\_course\_code=C964}{{\hspace*{\fill}\sphinxincludegraphics[height=200\sphinxpxdimen]{{superman1}.jpg}\hspace*{\fill}}}
\begin{itemize}
\item {} 
\sphinxAtStartPar
Debugging Java and Python

\item {} 
\sphinxAtStartPar
Data Analytics

\end{itemize}

\sphinxAtStartPar
Holding bachelor’s and master’s degrees in Business Administration and Economic Leadership, Mark has enjoyed teaching as an adjunct professor for the past six years, covering Computer Science and Business\sphinxhyphen{}related courses at both the undergraduate and graduate levels. Currently, Mark is working on his Ph.D. in Information Systems with a focus on Internet of Things (IoT) device interaction.

\sphinxAtStartPar
Mark comes to WGU with a global background spanning thirty years in commercial software engineering and currently leading the Software Modernization efforts for the top Pharmaceutical Automation company worldwide. He is also a certified ScrumMaster.

\sphinxAtStartPar
On teaching, Mark cites he “…fiercely believes that everyone deserves the opportunity of advancing their education, in a supportive and nurturing environment. This approach sets the stage for the success of the next generation of students.”

\sphinxstepscope


\chapter{Help Support this Website}
\label{\detokenize{support_this_page:help-support-this-website}}\label{\detokenize{support_this_page::doc}}
\sphinxAtStartPar
This website is a collaborative work in progress by the C964 course faculty. We are continually trying to improve this page to better help students. How can you help? Feedback! Tell us what we’re doing right and what needs improvement. With your help, we can build this into the best possible resource for helping students. Here’s how you can contribute:
\begin{itemize}
\item {} 
\sphinxAtStartPar
Leave a {\hyperref[\detokenize{support_this_page:support-comments}]{\sphinxcrossref{\DUrole{std,std-ref}{comment below}}}} and/or our \sphinxhref{https://github.com/ashejim/C964}{⭐ repo} (you’ll need a GitHub account).

\item {} 
\sphinxAtStartPar
Go to your C964 COS page and click ‘Course Feedback.’
\begin{quote}


\end{quote}

\sphinxAtStartPar
This creates a ticket for our product development team, and is appropriate for giving feedback on the official rubric or learning resource.

\item {} 
\sphinxAtStartPar
For corrections or suggestions to THIS webpage, open an issue in the repo, make a comment below, or contact \sphinxhref{mailto:ugcapstoneit@wgu.edu\&subject=C964\%20website\%20feedback\&body=Your\%20feedback\%20here.\%20Thank\%20you\%21}{jim.ashe@wgu.edu}

\item {} 
\sphinxAtStartPar
📧 Email \sphinxhref{mailto:ugcapstoneit@wgu.edu?cc=betsey.stadelmann@wgu.edu\%3Bdave.huff@wgu.edu\&subject=C964\%20website\%20feedback\&body=Your\%20feedback\%20here.\%20Thank\%20you\%21}{people who value your input}

\end{itemize}


\section{Questions, comments, or suggestions?}
\label{\detokenize{support_this_page:questions-comments-or-suggestions}}\label{\detokenize{support_this_page:support-comments}}


\begin{sphinxthebibliography}{Coh13}
\bibitem[Coh13]{resources:id5}
\sphinxAtStartPar
Jacob Cohen. \sphinxstyleemphasis{Statistical power analysis for the behavioral sciences}. Academic press, 2013.
\bibitem[Wik18]{resources:id2}
\sphinxAtStartPar
Wikipedia. Travelling salesman problem solved with simulated annealing. 2018. {[}Online; accessed August 4, 2023{]}. URL: \sphinxurl{https://commons.wikimedia.org/wiki/File:Travelling\_salesman\_problem\_solved\_with\_simulated\_annealing.gif}.
\bibitem[xkcda]{resources:id4}
\sphinxAtStartPar
xkcd. Code quality. {[}Online; accessed August 11, 2023{]}. URL: \sphinxurl{https://xkcd.com/844/}.
\bibitem[xkcdb]{resources:id3}
\sphinxAtStartPar
xkcd. Code quality. {[}Online; accessed August 11, 2023{]}. URL: \sphinxurl{https://xkcd.com/1513/}.
\end{sphinxthebibliography}







\renewcommand{\indexname}{Index}
\printindex
\end{document}